\begin{flushright} {\tiny {\color{gray} pair\_q1p0.tex}} \end{flushright}
%~~~~~~~~~~~~~~~~~~~~~~~~~~~~~~~~~~~~~~~~~~~~~~~~~~~~~~~~~~~~~~~~~~~~~~~~~~~~~~~~~~~~~~~~~~~~~~~~~~


\begin{minipage}{0.48\textwidth}
\begin{center}
\begin{flushright} {\tiny {\color{gray} (tikz\_q1p0.tex)}} \end{flushright}
%~~~~~~~~~~~~~~~~~~~~~~~~~~~~~~~~~~~~~~~~~~~~~~~~~~~~~~~~~~~~~~~~~~~~~~~~~~~~~~~~~~~~~~~~~~~~~~~~~~

\begin{tikzpicture}
%\draw[fill=gray!23,gray!23](0,0) rectangle (5,5);
%\draw[step=0.5cm,gray,very thin] (0,0) grid (4,4); %background grid
\draw[thick] (1,1) -- (3,1.2) -- (2.7,3) -- (1.1,3.1) -- cycle;  
\node[] at (0.8,0.8) {0};
\node[] at (3.2,1) {1};
\node[] at (2.9,3.1) {2};
\node[] at (0.9,3.2) {3};
\draw[violet] (1.9,2.075) circle (4pt);
\draw[black,fill=teal] (1,1)   circle (2pt);
\draw[black,fill=teal] (3,1.2)  circle (2pt);
\draw[black,fill=teal] (2.7,3)  circle (2pt);
\draw[black,fill=teal] (1.1,3.1) circle (2pt);
\draw[black,fill=teal] (3.1,0.2) circle (2pt); 
\node[] at (3.4,0.2) {$\vec\upnu$};
\draw[violet] (4.1,0.2) circle (4pt); 
\node[] at (4.4,0.2) {$p$};
\node[] at (2.5,4.5) {4 vel. nodes, 1 press. node};
\end{tikzpicture}

\end{center}
\end{minipage}
\begin{minipage}{0.48\textwidth}
\begin{center}

\begin{tikzpicture}
%\draw[fill=gray!23,gray!23](0,0) rectangle (5,5);
%\draw[step=0.25cm,gray,very thin] (0,0) grid (5,4); %background grid
\draw[thick] (1,0.5) -- (3.25,0.75) -- (3,3) -- (0.5,2.5) -- cycle; %1-2-6-5
\draw[thick] (3.25,0.75) -- (4,1.5) -- (4.25,3.75) -- (3,3) -- cycle; %2-3-7-6
\draw[thick] (0.5,2.5) -- (3,3) -- (4.25,3.75) -- (1.75,3.5) -- cycle; %5-6-7-4
\draw[thin]   (1,0.5) -- (2,1.75) -- (1.75,3.5) -- (0.5,2.5)   --cycle; % 1-0-4-5 
\draw[thin] (2,1.75) -- (4,1.5); 
%\node[] at (0.8,0.8) {0};
%\node[] at (3.2,1) {1};
%\node[] at (2.9,3.1) {2};
%\node[] at (0.9,3.2) {3};
\draw[violet] (2.5,2.) circle (4pt);
\draw[black,fill=teal] (1,0.5)   circle (2pt);
\draw[black,fill=teal] (3.25,0.75)   circle (2pt);
\draw[black,fill=teal] (3,3)   circle (2pt);
\draw[black,fill=teal] (0.5,2.5)   circle (2pt);
\draw[black,fill=teal] (1.75,3.5)  circle (2pt);
\draw[black,fill=teal] (4.25,3.75)  circle (2pt);
\draw[black,fill=teal] (4,1.5) circle (2pt);
\draw[black,fill=teal] (2,1.75) circle (2pt);
\draw[black,fill=teal] (3.1,0.2) circle (2pt); 
\node[] at (3.4,0.2) {$\vec\upnu$};
\draw[violet] (4.1,0.2) circle (4pt); 
\node[] at (4.4,0.2) {$p$};
\node[] at (2.5,4.5) {8 vel. nodes, 1 press. node};
\end{tikzpicture}

\end{center}
\end{minipage}

However simple it may look, the \index{general}{$Q_1 \times P_0$} element is 
one of the hardest elements to analyze and many questions are still open about its properties. 
The element does not satisfy the inf-sup condition \cite[p211]{hugh}. 
In Gresho \& Sani \cite{grsa} it is labeled as follows: slightly unstable but highly usable. 

The $Q_1 \times P_0$ mixed approximation is the lowest order conforming approximation 
method defined on a rectangular grid. It also happens to be the most famous example 
of an unstable mixed approximation method.
\cite[p235]{elsw}.
\textcite{boni84} (1984) and \textcite{boni85} (1985) show that it is not stable.

This element is discussed in Fortin (1981) \cite{fort81}, Fortin \& Fortin (1985) \cite{fofo85} 
and in Pitk\"aranta \& Saarinen (1985) \cite{pisa85} in the context of multigrid use.

This element is plagued by so-called pressure checkerboard modes which
have been thoroughly analysed \cite{grsi94}, \cite{chpc95}, \cite{sagl81a,sagl81b}.
These can be filtered out \cite{chpc95}. Smoothing techniques are also discussed in \cite{legs79}, 
and explained in Section~\ref{psmoothing}.

\Literature: Fortin \& Boivin (1990) \cite{fobo90}, Gresho \& Lee (1985) \cite{grle85},
Le Tallec \& Ruas (1986) \cite{leru86}, Oden \& Jacquotte (1984) \cite{odja84}
