\begin{flushright} {\tiny {\color{gray} weakseeds.tex}} \end{flushright}

{\sl This section was mostly written by I. van Zelst with some input by S. Buiter}. 
\index{contributors}{I. van Zelst}

Numerical models that investigate dynamics of the lithosphere and upper mantle always
start from an initial geometry with a set of prescribed mechanical and thermal conditions. 
This initial setup is usually a more-or-less standard representation of the
lithosphere and asthenosphere, as defined from compilations of geological and geophysical 
observations and laboratory measurements. Deformation is driven by internal buoyancy
forces and/or velocity or stress boundary conditions. However, unless an 
intrinsically unstable setup is defined or boundary conditions are discontinuous, 
deformation may take long model time to localize (up to millions of years). 
This is because these models need to build up numerical disturbance to create starting 
points for the deformation. In such models, deformation may in the first stages be
accommodated by pure shear extension or shortening \cite{pybf00,moql07}. 

To avoid this long starting phase and, in addition, exert some control 
over the initial location of deformation (preferably away from the boundaries), modelers 
use different approaches to initiate and localize deformation.

One manner to localize deformation is by discontinuous boundary conditions, such 
as the so-called S-point velocity discontinuity at the bottom of the system 
(or the tip of a basal sheet) which is used in both numerical 
\cite{brbe95,elfb95,will99a,bemh00,bube06,thfb08,brya10}
and analogue studies \cite{bube06,mime00}.
These models are usually on the scale of the (upper-) crust. 
S-point models are less flexible than upper-mantle scale models as they do not include
feedback relations between deformation and the basal velocity field.
Models of extension of continental lithosphere often use 'seeds' to initiate
extension. Such seeds are usually small regions that are weaker than the surrounding crust
and lithosphere. The use of seeds can be justified by considering the fact that in
nature continental lithosphere is hardly ever (if at all) homogeneous in composition 
and stratification. In addition, extension often occurs in regions of former
convergence, such as the opening of the North Atlantic Ocean that largely followed the old 
sutures of the Iapetus and Rheic Oceans \cite{wils66}. Analogues for numerical seeds 
can therefore be found in inherited faults, inherited crustal thickness changes, and/or 
plumes impacting the lithosphere. However, this immediately points out a problem with 
single-seed models as orogenic inheritance and mantle upwellings may be expected to 
occur over larger areas than a seed of some hundreds of meters to a few kilometers 
in width and height.

A literature survey shows that seeds in previous numerical studies differ in shape, size, 
orientation, mechanical and thermal properties, and depth in the models. 
Three types of weak seeds can be identified that have been 
used in previous models of (continental) extension:

%.................................................
\paragraph{Seeding through thermal effects}
A weak region can be achieved by a temperature anomaly in the crust or lithosphere \cite{bupo99},
which is created by directly imposing a temperature difference, by assigning high radiogenic 
heat production, or modeling a thermal upwelling in the mantle below.
The elevated temperature reduces viscosity values for models with a temperature-dependent 
viscosity.
An advantage of using an imposed temperature anomaly is that it will dissipate with time, 
thus reducing the impact on later model stages \cite{hani03}.
Examples of thermal anomalies used to initiate extension are
an elevated temperature at the base of the crust \cite{hani03},
an elevated temperature at the base of the lithosphere 
(100$\rm^\circ$ in \cite{bupo01}, 
up to 200$\rm^\circ$ in \cite{brau13}),
a temperature anomaly imposed from the base of the lithosphere to the middle
crust \cite{chld92},
and a 10\si{\milli\watt\per\square\metre} perturbation in basal heat flow \cite{frbr01}.
In \cite{bupo01} the rifting is initiated by means of 
a thermal perturbation placed at the bottom of the mantle lithosphere with a maximum temperature $T_{2}$ 
exponentially which decays from the center to $T_{1}$ on the left and $T_{3}$ on the right. 

%.................................................
\paragraph{Seeding by mechanical inhomogeneity} 
A seed may be composed of a material with a lower rheological strength than the 
surroundings.
A weak seed may, for example, have a lower imposed viscosity 
\cite{lemm08,kaus10,mishin11}, 
a lower value for angle of internal friction 
\cite{pybf02,kapo06,thie11,grpy13,chbe13}, 
a lower value for cohesion \cite{alht11}, or a lower value for density \cite{tibb08}. 

The seed may also be assigned different material properties, as, for example,
a Von Mises seed in a frictional plastic material \cite{hube07}.
A frequently used approach is to assume that a region has already accumulated
strain, leading to strain-weakening 
\cite{labp00,hubb05,peso08,alht11,alht12,knak13,alhf13}. 
Previous studies have used a variety of shapes and sizes for weak seeds. 
Examples are square seeds, fault-shaped weak inclusions, and rectangular seeds with different 
aspect ratios: \cite{hubb05} use a $6\times 3$ km seed, while the weak seed 
of \cite{hube07} has a size of $12\times 10$ km. 

Instead on confining the seed to a geometrically simple region, randomly distributed seeds 
have also been used \cite{thie11}, \cite{thsh14}, albeit for compression. 

\paragraph{Seeding through geometrical discontinuity} 
A seed is created by an abrupt variation in the thickness of the crust and/or lithosphere.
A locally thinned crust could be thought to be caused by a previous rifting phase,
whereas a thicker crust could represent preceding mountain building.
Such crustal thickness variations effect not only mechanical strength, but may also
impose a thermal anomaly.
Burg \& Schmalholz \cite{busc08} implemented a Gaussian shaped mohorovi\v{c}i\'c discontinuity of 250m 
height as a representation of the weak zone resulting in a slightly thinner crust.
A step change in crustal thickness alters the symmetry of the domain. 
Chenin \etal \cite{chsm20} implement a sinusoidal perturbation of the Moho.

%\paragraph{Seeding through numerical noise}
%include initiation on numerical noise (models of Moresi, Tirel) 

Only few studies have investigated how different methods of implementing a weak zone 
can affect the results of a model. 
\cite{dyrm07} found that a single seed produces a symmetric narrow rift, an initial shear
zone tends to produce an asymmetric rift, and multiple seeds promote a wide rift. 
Note however that this behavior will be affected by rheological stratification, as not
all systems can evolve in a wide rift mode \cite{hubb05,buhb08}.
\cite{dyrm07} found that a seed needs to be 10 times weaker than 
the surrounding material in order to localize strain. 
To initiate shear bands with a Coulomb dip angle (45 $\pm\phi$/2, where $\phi$ is the
angle of internal friction), seeds need to be well resolved (5-10 elements, \cite{kaus10}). 

This variety in the shape, size, orientation, mechanical and thermal properties, 
and depth of the seed(s) begs the question if these different approaches to
initiate extension could have an effect on model evolution? 
As such variations might not be
removed by subsequent deformation stages, the initiation effects could propagate into
later model evolution. 
In addition, weak seeds introduce a weakness into the extensional system that may 
potentially be long-lasting. 
For instance, seeds with a weakness defined by material or strain-weakened properties 
stay in the model and may control deformation also in later model stages. 
The heat associated with thermal seeds will diffuse away, but additional 
heat has been introduced into the initial system and the setup will therefore differ from 
models with mechanically weak seeds. 

