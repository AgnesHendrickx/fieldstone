\begin{flushright} {\tiny {\color{gray} dgfem1D\_ssdiff.tex}} \end{flushright}
%~~~~~~~~~~~~~~~~~~~~~~~~~~~~~~~~~~~~~~~~~~~~~~~~~~~~~~~~~~~~~~~~~~~~~~~~~~~~~~~~~~~~~~~~~~~~~~~~~~

Let us start simple with the 1D steady state heat conduction problem in 1D, given by the following 
equation:
\begin{equation}
\frac{d^2T}{dx^2}=0 \qquad T(x=0)=0 \qquad T(x=1)=1 \qquad \text{on} \quad x\in[0,1]
\end{equation}
Although this equation is usually solved as is with its second-order derivative, it can also 
be written in a mixed form, using the heat flux $q$ (a scalar in 1D):
\begin{eqnarray}
-\frac{dq}{dx} &=&0   \nn\\
q-\frac{dT}{dx}&=&0 \qquad x\in[0,1]
\end{eqnarray}
and the boundary conditions remain unchanged. 

We apply the standard approach to establish the weak forms of these two first-order ODEs, and we do so 
on an element $e$ bound by nodes $k$ and $k+1$ with coordinates $x_k$ and $x_{k+1}$
\begin{eqnarray}
-\int_{x_k}^{x_{k+1}} \frac{dq}{dx} \tilde{f}(x) dx = -\left[q \tilde{f} \right]_{x_k}^{x_{k+1}} 
+ \int_{x_k}^{x_{k+1}} \frac{d\tilde{f}}{dx} q(x) dx &=& 0
\label{eq:dg1}\\
\int_{x_k}^{x_{k+1}}  \left( q-\frac{dT}{dx} \right) \overline{f}(x) dx
=
\int_{x_k}^{x_{k+1}}  q(x) \overline{f}(x) dx
-\left[ T \overline{f}  \right]_{x_k}^{x_{k+1}} + \int_{x_k}^{x_{k+1}} \frac{d\overline{f}}{dx} T(x) dx 
&=& 0
\label{eq:dg2}
\end{eqnarray}
where $\tilde{f}$ and $\overline{f}$ are test functions.
We now must examine the term between square brackets. 
Inside the element, the test functions $\tilde{f}$ and $\overline{f}$ are well defined polynomials
and we coin:
\begin{eqnarray}
\tilde{f}_k^+&=&\tilde{f}(x_k^+)\\
\tilde{f}_{k+1}^-&=&\tilde{f}(x_{k+1}^-)\\
\overline{f}_k^+&=&\overline{f}(x_k^+)\\
\overline{f}_{k+1}^-&=&\overline{f}(x_{k+1}^-)
\end{eqnarray}
Concerning $q$ and $T$, we will for now  give them values $\hat{q}_k$ and $\hat{T}_k$ at node $k$
and $\hat{q}_{k+1}$ and $\hat{T}_{k+1}$ at node $k+1$, and we will specify the hat quantities as follows:

\begin{eqnarray}
\hat{T}_k &=&
\left\{
\begin{array}{ll}
T_k^-   & k=1 \\ 
\frac{1}{2}(T_k^-+T_k^+) + {\cal C} (T_k^- - T_k^+) & k=2,...N-1\\
T_k^+    & k=N \\ 
\end{array}
\right. \nonumber\\
\hat{q}_k &=&
\left\{
\begin{array}{ll}
q_k^+ -{\cal E} (T_k^--T_k^+)  & k=1 \\ 
\frac{1}{2}(q_k^+ + q_k^-) - {\cal E} (T_k^- - T_k^+) - {\cal C}(q_k^- - q_k^+) & k=2,...N-1\\
q_k^- -{\cal E} (T_k^--T_k^+)    & k=N \\ 
\end{array}
\right.
\end{eqnarray}
where $N$ is the number of nodes and where ${\cal C}$ and ${\cal E}$ are two constants. 

{\color{green} Discuss the meaning/values of these!}

\begin{remark}
Note that $\hat{T}_k=T_1^-$ on the left boundary is consistent with 
$\hat{T}_k=\frac{1}{2}(T_k^-+T_k^+) + {\cal C} (T_k^- - T_k^+)$ provided $T_1^-=T_1^+$.
The same goes for the right boundary, and the same reasoning applies for the heat flux terms $\hat{q}_k$. 
\end{remark}

Inside an element bounded by nodes $k$ and $k+1$, 
the temperature $T$ and heat flux $q$ are interpolated over an isoparametric linear element:
\[
T_h(x) = \bN_k(x) T_k^+ + \bN_{k+1}(x)T_{k+1}^-
\]
\[
q_h(x) = \bN_k(x) q_k^+ + \bN_{k+1}(x)q_{k+1}^-
\]
As in the (Continuous/Standard) Galerkin case of section~\ref{sec:diff1D}, the test functions are taken to 
be the basis functions, and in this case for both temperature and flux. 

There are four unknowns ${\color{red} q_k^+}$, ${\color{red} q_{k+1}^-}$, 
${\color{red}T_k^+}$ and ${\color{red}T_{k+1}^-}$ per element. 
All other $q$ and $T$ quantities in the above/following equations will need to find their way to the rhs. 

\begin{itemize}
\item Eq.~\ref{eq:dg1} becomes:
\begin{eqnarray}
0 &=&
-\hat{q}_{k+1} \tilde{f}(x_{k+1}^-)
+\hat{q}_k     \tilde{f}(x_k^+)
+ \int_{x_k}^{x_{k+1}} \frac{d\tilde{f}}{dx} q_h(x) dx  \nonumber\\
&=&-\hat{q}_{k+1} \tilde{f}_{k+1}^-
+\hat{q}_k     \tilde{f}_k^+
+ \int_{x_k}^{x_{k+1}} \frac{d\tilde{f}}{dx} ( \bN_k(x) q_k^+ + \bN_{k+1}(x)q_{k+1}^- ) dx \nonumber\\
&=& -\hat{q}_{k+1} \tilde{f}_{k+1}^-
+\hat{q}_k     \tilde{f}_k^+
+ \int_{x_k}^{x_{k+1}} \frac{d\tilde{f}}{dx} \bN_k(x) dx \cdot q_k^+ 
+ \int_{x_k}^{x_{k+1}} \frac{d\tilde{f}}{dx} \bN_{k+1}(x) dx \cdot q_{k+1}^- 
\end{eqnarray}

\begin{itemize}
\item We take $\tilde{f}=\bN_k$ and by vertue of the properties of basis functions $\bN$ we have: 
\begin{eqnarray}
\tilde{f}_k^+&=&\tilde{f}(x_k^+) = \bN_k(x_k^+)=1 \nn\\
\tilde{f}_{k+1}^-&=&\tilde{f}(x_{k+1}^-)   = \bN_k(x_{k+1}^+)=0 \nn
\end{eqnarray}
so that 
\begin{eqnarray}
0 
&=& \hat{q}_k   
+ \int_{x_k}^{x_{k+1}} \frac{d\bN_k}{dx} \bN_k(x) dx \cdot q_k^+ 
+ \int_{x_k}^{x_{k+1}} \frac{d\bN_k}{dx} \bN_{k+1}(x) dx \cdot q_{k+1}^- \nn\\ 
&=& 
\frac{1}{2}({\color{red}q_k^+} + q_k^-) - {\cal E} (T_k^- - {\color{red}T_k^+}) 
- {\cal C}(q_k^- - {\color{red}q_k^+}) \nn\\
&+& \int_{x_k}^{x_{k+1}} \frac{d\bN_k}{dx} \bN_k dx \cdot {\color{red}q_k^+ }
+ \int_{x_k}^{x_{k+1}} \frac{d\bN_k}{dx} \bN_{k+1} dx \cdot {\color{red} q_{k+1}^-} \label{eq:dg4}
\end{eqnarray}

\item We take $\tilde{f}=\bN_{k+1}$ and likewise:
\begin{eqnarray}
\tilde{f}_k^+&=&\tilde{f}(x_k^+) = \bN_{k+1}(x_k^+)=0 \nn\\
\tilde{f}_{k+1}^-&=&\tilde{f}(x_{k+1}^-)   = \bN_{k+1}(x_{k+1}^+)=1 \nn
\end{eqnarray}
so that 
\begin{eqnarray}
0 
&=& -\hat{q}_{k+1} 
+ \int_{x_k}^{x_{k+1}} \frac{d\bN_{k+1}}{dx} \bN_k(x) dx \cdot q_k^+ 
+ \int_{x_k}^{x_{k+1}} \frac{d\bN_{k+1}}{dx} \bN_{k+1}(x) dx \cdot q_{k+1}^- \nn\\ 
&=& - \left[
\frac{1}{2}(q_{k+1}^+ + {\color{red}q_{k+1}^-}) - {\cal E} ({\color{red}T_{k+1}^-} 
- T_{k+1}^+) -{\cal C}({\color{red}q_{k+1}^-} - q_{k+1}^+) 
\right]\nn\\
&+& \int_{x_k}^{x_{k+1}} \frac{d\bN_{k+1}}{dx} \bN_k dx \cdot {\color{red}q_k^+} 
+ \int_{x_k}^{x_{k+1}} \frac{d\bN_{k+1}}{dx} \bN_{k+1}dx \cdot {\color{red}q_{k+1}^-}  \label{eq:dg6}
\end{eqnarray}



\end{itemize}

and finally 
\begin{eqnarray}
&&\int_{x_k}^{x_{k+1}} 
\left(
\begin{array}{cc}
\frac{d\bN_k}{dx} \bN_k     & \frac{d\bN_k}{dx} \bN_{k+1} \\
\frac{d\bN_{k+1}}{dx} \bN_k & \frac{d\bN_{k+1}}{dx} \bN_{k+1}
\end{array}
\right)dx  \cdot
\left(
\begin{array}{c}
    {\color{red}q_k^+}  \\
    {\color{red}q_{k+1}^-}
\end{array}
\right)
+\left(
\begin{array}{c}
     ({\cal C}+\frac{1}{2})  {\color{red}q_k^+}  \\
     ({\cal C}-\frac{1}{2})  {\color{red}q_{k+1}^-} 
\end{array}
\right)+\left(
\begin{array}{c}
     {\cal E}    {\color{red}T_k^+}  \\
     {\cal E}    {\color{red}T_{k+1}^-} 
\end{array}
\right) \nn \\
&&= 
\left(
\begin{array}{c}
     ({\cal C}-\frac{1}{2}) q_k^-  \\
     ({\cal C}+\frac{1}{2}) q_{k+1}^+ 
\end{array}
\right)
+ \left(
\begin{array}{c}
     {\cal E}   T_k^-  \\
     {\cal E}   T_{k+1}^+
\end{array}
\right)  
\label{eq:dgq1}
\end{eqnarray}



\newpage
\item Eq.~\ref{eq:dg2} becomes:
\begin{eqnarray}
0&=&
-[ T \overline{f}  ]_{x_k}^{x_{k+1}} 
+ \int_{x_k}^{x_{k+1}}  q_h(x) \overline{f}(x) dx
+ \int_{x_k}^{x_{k+1}} \frac{d\overline{f}}{dx} T_h(x) dx  
\nn\\
&=&
-\hat{T}_{k+1} \overline{f}_{k+1}^- 
+\hat{T}_k     \overline{f}_{k}^+ 
+ \int_{x_k}^{x_{k+1}}  q_h(x) \overline{f}(x) dx
+ \int_{x_k}^{x_{k+1}} \frac{d\overline{f}}{dx} T_h(x) dx  \nn
%&=&
%-\left( \frac{1}{2}({\color{red}T_{k+1}^-}+ T_{k+1}^+) 
%+ {\cal C} ({\color{red} T_{k+1}^-} - T_{k+1}^+) \right) \overline{f}_{k+1}^-
%+\left( \frac{1}{2}(T_k^-+{\color{red}T_k^+}) + {\cal C} (T_k^- - {\color{red}T_k^+}) \right) \overline{f}_{k}^+ \nn\\
%&& + \int_{x_k}^{x_{k+1}} ( N_k(x) {\color{red}q_k^+} + N_{k+1}(x) {\color{red}q_{k+1}^-}    ) \overline{f}(x) dx
%+ \int_{x_k}^{x_{k+1}} \frac{d\overline{f}}{dx} ( N_k(x) {\color{red}T_k^+} + N_{k+1}(x) {\color{red}T_{k+1}^-}  ) dx \nn\
\end{eqnarray}


\begin{itemize}
\item We take $\overline{f}=\bN_k$: 
\begin{eqnarray}
\overline{f}_k^+&=&\overline{f}(x_k^+)= \bN_k(x_k^+) =1 \nn\\
\overline{f}_{k+1}^-&=&\overline{f}(x_{k+1}^-) = \bN_k(x_{k+1}^-) = 0\nn
\end{eqnarray}
so that 
\begin{eqnarray}
0
&=& \hat{T}_k     
+ \int_{x_k}^{x_{k+1}}  q_h(x) \bN_k dx
+ \int_{x_k}^{x_{k+1}} \frac{d\bN_k}{dx} T_h(x) dx  \nn\\
&=& 
\frac{1}{2}(T_k^-+{\color{red}T_k^+}) + {\cal C} (T_k^- - {\color{red}T_k^+}) \nn\\
&+& \int_{x_k}^{x_{k+1}}  (\bN_k(x) {\color{red}q_k^+} + \bN_{k+1}(x) {\color{red}q_{k+1}^-}) \bN_k dx
+ \int_{x_k}^{x_{k+1}} \frac{d\bN_k}{dx} (\bN_k(x) {\color{red}T_k^+} + \bN_{k+1}(x) {\color{red}T_{k+1}^-})   dx 
\label{eq:dg5}
\end{eqnarray}

\item We take $\overline{f}=\bN_{k+1}$:
\begin{eqnarray}
\overline{f}_k^+&=&\overline{f}(x_k^+) = \bN_{k+1}(x_k^+) = 0 \nn\\
\overline{f}_{k+1}^-&=&\overline{f}(x_{k+1}^-) = \bN_{k+1}(x_{k+1}^-) =1 \nn
\end{eqnarray}
so that 
\begin{eqnarray}
0
&=& -\hat{T}_{k+1} 
+ \int_{x_k}^{x_{k+1}}  q_h(x) \bN_{k+1} dx
+ \int_{x_k}^{x_{k+1}} \frac{d\bN_{k+1} }{dx} T_h(x) dx  \nn\\
&=&-\left[\frac{1}{2}({\color{red}T_{k+1}^-}+T_{k+1}^+) + {\cal C}({\color{red}T_{k+1}^-} -T_{k+1}^+)\right]\\
&+& \int_{x_k}^{x_{k+1}} (\bN_k(x) {\color{red}q_k^+} + \bN_{k+1}(x) {\color{red}q_{k+1}^-})  \bN_{k+1} dx
+ \int_{x_k}^{x_{k+1}} \frac{d\bN_{k+1} }{dx} (\bN_k(x) {\color{red} T_k^+} + \bN_{k+1}(x) {\color{red}T_{k+1}^-}) dx 
\label{eq:dg7}
\end{eqnarray}



 
\end{itemize}

and finally 

\begin{eqnarray}
&& \int_{x_k}^{x_{k+1}} 
\left(
\begin{array}{cc}
 \bN_k \bN_k     & \bN_k \bN_{k+1} \\
 \bN_{k+1} \bN_k & \bN_{k+1} \bN_{k+1}
\end{array}
\right)dx
\left(
\begin{array}{c}
    {\color{red}q_k^+}  \\
    {\color{red}q_{k+1}^-}
\end{array}
\right)+
\int_{x_k}^{x_{k+1}} 
\left(
\begin{array}{cc}
\frac{d\bN_k}{dx} \bN_k     & \frac{d\bN_k}{dx} \bN_{k+1} \\
\frac{d\bN_{k+1}}{dx} \bN_k & \frac{d\bN_{k+1}}{dx} \bN_{k+1}
\end{array}
\right)dx
\left(
\begin{array}{c}
 {\color{red}T_k^+}  \\
{\color{red}T_{k+1}^-} 
\end{array}
\right) \nn \\
&&\qquad+ \left(
\begin{array}{c}
     (\frac{1}{2}-{\cal C}) {\color{red}T_k^+}  \\
     -({\cal C}+\frac{1}{2}){\color{red}T_{k+1}^-} 
\end{array}
\right)
= \left(
\begin{array}{c}
     -({\cal C}+\frac{1}{2})  T_k^- \\
     (\frac{1}{2}-{\cal C})  T_{k+1}^+ 
\end{array}
\right) 
\label{eq:dgT1}
\end{eqnarray}





\end{itemize}


\newpage
We will also use the results obtained in Appendix~\ref{app:mm} for 1D linear elements:
\begin{eqnarray}
{\bm M}^e
&=&\int_{\Omega_e} \vec{\bN}^T \vec{\bN} dV
=\int_{\Omega_e} 
\left(
\begin{array}{cc}
 \bN_k \bN_k & \bN_k \bN_{k+1} \\
 \bN_{k+1} \bN_k & \bN_{k+1}\bN_{k+1}
\end{array}
\right)
dV 
= \frac{h}{2} \frac{1}{3} 
\left(
\begin{array}{cc}
2  & 1 \\
1 & 2
\end{array}
\right)
= 
\frac{h}{6}
\left(
\begin{array}{cc}
2  & 1 \\
1 & 2
\end{array}
\right)
\nn\\
{\bm K}^e &=&
\int_{\Omega_e} 
\left(
\begin{array}{cc}
\frac{d\bN_k}{dx} \bN_k     & \frac{d\bN_k}{dx} \bN_{k+1} \\
\frac{d\bN_{k+1}}{dx} \bN_k & \frac{d\bN_{k+1}}{dx} \bN_{k+1}
\end{array}
\right)
dV
=
\frac{1}{2}
\left(
\begin{array}{cc}
-1  & -1 \\
1 & 1
\end{array}
\right)
\end{eqnarray}


\noindent
Filling this into equations (\ref{eq:dgq1}) and (\ref{eq:dgT1}), gives 
\begin{eqnarray}
{\bm K}^e \cdot 
\left( 
\begin{array}{cc}
    {\color{red}q_k^+}  \\
    {\color{red}q_{k+1}^-}
\end{array}
\right)+ 
\left(
\begin{array}{cc}
     ({\cal C}+\frac{1}{2})  {\color{red}q_k^+}  \\
     ({\cal C}-\frac{1}{2})  {\color{red}q_{k+1}^-} 
\end{array}
\right)+\left(
\begin{array}{cc}
     {\cal E}    {\color{red}T_k^+}  \\
     {\cal E}    {\color{red}T_{k+1}^-} 
\end{array}
\right) 
&=& 
\left(
\begin{array}{cc}
     ({\cal C}-\frac{1}{2}) q_k^-  \\
     ({\cal C}+\frac{1}{2}) q_{k+1}^+ 
\end{array}
\right)
+ \left(
\begin{array}{cc}
     {\cal E}   T_k^-  \\
     {\cal E}   T_{k+1}^+
\end{array}
\right)  
\nn
\\
{\bm M}^e \cdot
\left(
\begin{array}{cc}
    {\color{red}q_k^+}  \\
    {\color{red}q_{k+1}^-}
\end{array}
\right)+
{\bm K}^e \cdot
\left(
\begin{array}{cc}
 {\color{red}T_k^+}  \\
{\color{red}T_{k+1}^-} 
\end{array}
\right) 
+ \left(
\begin{array}{cc}
     (\frac{1}{2}-{\cal C}) {\color{red}T_k^+}  \\
     -({\cal C}+\frac{1}{2}){\color{red}T_{k+1}^-} 
\end{array}
\right)
&=& \left(
\begin{array}{cc}
     -({\cal C}+\frac{1}{2})  T_k^- \\
     (\frac{1}{2}-{\cal C})  T_{k+1}^+ 
\end{array}
\right) 
\end{eqnarray}
which becomes
\begin{eqnarray}
\left(
\begin{array}{cc}
  {\cal C}  &  -\frac{1}{2}\\
   \frac{1}{2}  &  {\cal C}
\end{array}
\right)
\left(
\begin{array}{cc}
     {\color{red}{q_k^+}}  \\
     {\color{red}{q_{k+1}^-}}
\end{array}
\right)
+
\left(
\begin{array}{cc}
  {\cal E} &  0\\
   0  &  {\cal E}
\end{array}
\right)
\left(
\begin{array}{cc}
     {\color{red}{T_k^+}}  \\
     {\color{red}{T_{k+1}^-}}
\end{array}
\right)
&=&
\left(
\begin{array}{cc}
      ({\cal C}-\frac{1}{2})q_k^-+{\cal{E}} T_k^-  \\
      ({\cal C}+\frac{1}{2})q_{k+1}^++{\cal{E}} T_{k+1}^+
\end{array}
\right)
\nn 
\\
\frac{h}{6}
\left(
\begin{array}{cc}
  2  &  1\\
   1 & 2
\end{array}
\right)
\left(
\begin{array}{cc}
     {\color{red}{q_k^+}}  \\
     {\color{red}{q_{k+1}^-}}
\end{array}
\right)
+
\left(
\begin{array}{cc}
  -{\cal C} &  -\frac{1}{2}\\
   \frac{1}{2}  &  -{\cal C}
\end{array}
\right)
\left(
\begin{array}{cc}
     {\color{red}{T_k^+}}  \\
     {\color{red}{T_{k+1}^-}}
\end{array}
\right)
&=&
\left(
\begin{array}{cc}
     -(\frac{1}{2}+{\cal{C}}) T_k^-  \\
      (\frac{1}{2}-{\cal{C}}) T_{k+1}^+
\end{array}
\right)\nn \
\end{eqnarray}



Combining these equations gives the expression for the linear system for the 
element under consideration:
\begin{eqnarray}
\boxed{
\left(
\begin{array}{cccc}
\frac{h}{3}    &  \frac{h}{6} & -{\cal C}   &-\frac{1}{2} \\
\frac{h}{6}    &  \frac{h}{3} & \frac{1}{2} & -{\cal C} \\
{\cal C}    & -\frac{1}{2} & {\cal E} & 0\\
\frac{1}{2} & {\cal C} & 0 & {\cal E}
\end{array}
\right) \left(
\begin{array}{c}
     \color{red}{q_k^+}  \\
     \color{red}{q_{k+1}^-} \\
     \color{red}{T_k^+} \\
     \color{red}{T_{k+1}^-}
\end{array}
\right)=\left(
\begin{array}{c}
     -(\frac{1}{2}+{\cal{C}}) T_k^-  \\
      (\frac{1}{2}-{\cal{C}}) T_{k+1}^+ \\
     -(\frac{1}{2}-{\cal{C}})q_k^-+{\cal{E}} T_k^-  \\
      (\frac{1}{2}+{\cal{C}})q_{k+1}^++{\cal{E}} T_{k+1}^+
\end{array}
\right)
}  \label{eq:dgsyst1D}
\end{eqnarray}

\paragraph{Left boundary}
Special care must be taken for the two elements on the boundaries of the domain. 
On the left, we have 
\begin{eqnarray}
\hat{q}_1 &=& q_1^+ -{\cal E} (T_1^--T_1^+) \nn  \\
\hat{T}_1 &=& T_1^- \nn
\end{eqnarray}
Eq.~(\ref{eq:dg4}) becomes:
\begin{eqnarray}
0 
&=& \hat{q}_k   
+ \int_{x_k}^{x_{k+1}} \frac{d\bN_k}{dx} \bN_k(x) dx \cdot q_k^+ 
+ \int_{x_k}^{x_{k+1}} \frac{d\bN_k}{dx} \bN_{k+1}(x) dx \cdot q_{k+1}^- \nn\\ 
&=& 
{\color{red}q_1^+} - {\cal E} (T_k^- - {\color{red}T_1^+}) 
+ \int_{x_k}^{x_{k+1}} \frac{d\bN_k}{dx} \bN_k dx \cdot {\color{red}q_1^+ }
+ \int_{x_k}^{x_{k+1}} \frac{d\bN_k}{dx} \bN_{k+1} dx \cdot {\color{red} q_{2}^-}
\end{eqnarray}
Eq.~(\ref{eq:dg5}) becomes:
\begin{eqnarray}
0
&=& \hat{T}_1     
+ \int_{x_k}^{x_{k+1}}  q_h(x) \bN_k dx
+ \int_{x_k}^{x_{k+1}} \frac{d\bN_k}{dx} T_h(x) dx  \nn\\
&=& 
T_1^-  
+ \int_{x_k}^{x_{k+1}}  (\bN_k(x) {\color{red}q_1^+} + \bN_{k+1}(x) {\color{red}q_{2}^-}) \bN_k dx
+ \int_{x_k}^{x_{k+1}} \frac{d\bN_k}{dx} (\bN_k(x) {\color{red}T_1^+} + \bN_{k+1}(x) {\color{red}T_{2}^-})   dx 
\end{eqnarray}
Eq.~(\ref{eq:dgsyst1D}) then becomes:
\begin{eqnarray}
\left(
\begin{array}{cccc}
\frac{h}{3}    &  \frac{h}{6} & -\frac{1}{2}   &-\frac{1}{2} \\
\frac{h}{6}    &  \frac{h}{3} & \frac{1}{2} & -{\cal C} \\
\frac{1}{2}    & -\frac{1}{2} & {\cal E} & 0\\
\frac{1}{2} & {\cal C} & 0 & {\cal E}
\end{array}
\right) \left(
\begin{array}{c}
     \color{red}{q_1^+}  \\
     \color{red}{q_{2}^-} \\
     \color{red}{T_1^+} \\
     \color{red}{T_{2}^-}
\end{array}
\right)=\left(
\begin{array}{c}
     -T_1^-  \\
      (\frac{1}{2}-{\cal{C}}) T_{k+1}^+ \\
     {\cal{E}} T_k^-  \\
      (\frac{1}{2}+{\cal{C}})q_{k+1}^++{\cal{E}} T_{k+1}^+
\end{array}
\right)
\label{eq:dgsyst1Dleft}
\end{eqnarray}








\paragraph{Right boundary} The element is composed of nodes $N-1$ and $N$. The fluxes are 
\begin{eqnarray}
\hat{q}_N &=& q_N^+ -{\cal E} (T_N^--T_N^+) \nn  \\
\hat{T}_N &=& T_N^- \nn
\end{eqnarray}
Eq.~\eqref{eq:dg6} becomes:
\begin{eqnarray}
0 
&=& -\hat{q}_{N} 
+ \int_{x_k}^{x_{k+1}} \frac{d\bN_{k+1}}{dx} \bN_k(x) dx \cdot q_{N-1}^+ 
+ \int_{x_k}^{x_{k+1}} \frac{d\bN_{k+1}}{dx} \bN_{k+1}(x) dx \cdot q_{N}^- \nn\\ 
&=& - \left[ q_N^+ -{\cal E} (T_N^--T_N^+) \right] 
+ \int_{x_k}^{x_{k+1}} \frac{d\bN_{k+1}}{dx} \bN_k dx \cdot {\color{red}q_{N-1}^+} 
+ \int_{x_k}^{x_{k+1}} \frac{d\bN_{k+1}}{dx} \bN_{k+1}dx \cdot {\color{red}q_{N}^-} 
\end{eqnarray}
Eq.~\eqref{eq:dg7} becomes:
\begin{eqnarray}
0
&=& -\hat{T}_{N} 
+ \int_{x_k}^{x_{k+1}}  q_h(x) \bN_{k+1} dx
+ \int_{x_k}^{x_{k+1}} \frac{d\bN_{k+1} }{dx} T_h(x) dx  \nn\\
&=&-T_N^- + \int_{x_k}^{x_{k+1}} (\bN_k(x) {\color{red}q_{N-1}^+} + \bN_{k+1}(x) {\color{red}q_{N}^-})  \bN_{k+1} dx
+ \int_{x_k}^{x_{k+1}} \frac{d\bN_{k+1} }{dx} (\bN_k(x) {\color{red} T_{N-1}^+} + \bN_{k+1}(x) {\color{red}T_{N}^-}) dx
\nn 
\end{eqnarray}
Eq.~\eqref{eq:dgsyst1D} then becomes:
\begin{eqnarray}
\left(
\begin{array}{cccc}
\frac{h}{3}    &  \frac{h}{6} & -{\cal C}   &-\frac{1}{2} \\
\frac{h}{6}    &  \frac{h}{3} & \frac{1}{2} & \frac{1}{2} \\
{\cal C}    & -\frac{1}{2} & {\cal E} & 0\\
\frac{1}{2} & -\frac{1}{2} & 0 & {\cal E}
\end{array}
\right) \left(
\begin{array}{c}
     \color{red}{q_{N-1}^+}  \\
     \color{red}{q_{N}^-} \\
     \color{red}{T_{N-1}^+} \\
     \color{red}{T_{N}^-}
\end{array}
\right)=\left(
\begin{array}{cc}
     -(\frac{1}{2}+{\cal{C}}) T_{N-1}^-  \\
      T_{N}^+ \\
     -(\frac{1}{2}-{\cal{C}})q_{N-1}^-+{\cal{E}} T_{N-1}^-  \\
      {\cal{E}} T_{N}^+
\end{array}
\right)
\label{eq:dgsyst1Dright}
\end{eqnarray}


\paragraph{Solving strategies} Following Li \cite{li06}, there are three main strategies:
\begin{itemize}
\item Successive substitution: all the variables are initialized to zero. 
Eq.~\eqref{eq:dgsyst1Dleft} is solved
to obtain the data for the first element, where boundary conditions are specified.
Then Eq.~\eqref{eq:dgsyst1D} is used for all interior elements.
Finally Eq.~\eqref{eq:dgsyst1Dright} is used for the last element.
This procedure is carried out until all fields have converged.  

\item Global assembly: this approach is identical to the one 
we have taken so far with the continuous Galerkin Finite Element method. 
We form a large global matrix and then solve the linear system 
to obtain the solution. The disadvantage of this approach lies in the size 
of the generated marix: each node counts 4 dofs so the assembled matrix 
will be about 4 times as big as in the standard FE case. 

\item Elimination then asssembly: one can first eliminate the variable $q$
and solve for the temperature $T$ only. This speedis up the
calculations, but also increases the bandwidth of the element matrix. Li \cite{li06}
states: "Further comparison shows that the saving in CPU time for solving T alone is less
significant than the $q-T$ iterative solution, in particular, for 3-D problems."


Eq.~(\ref{eq:dgsyst1D}) can be rewritten:
\begin{equation}
\left(
\begin{array}{cc}
{\bm M}_e & {\bm C}_1 \\
{\bm C}_2 & {\bm E} 
\end{array}
\right)
\cdot
\left(
\begin{array}{c}
\vec{q} \\ \vec{T} 
\end{array}
\right)
=
\left(
\begin{array}{c}
\vec{f} \\ \vec{g} 
\end{array}
\right)
\end{equation}
The unknown of the original ODE is temperature so this is the quantity we are after. 
The first line of the matrix can be written 
\[
 {\bm M}_e \cdot \vec{q}+ {\bm C}_1 \cdot \vec{T} = \vec{f} 
\]
or,
\[
\vec{q} =   {\bm M}_e^{-1} \cdot (\vec{f} -{\bm C}_1 \cdot \vec{T} )
\]
The second line of the matrix is 
\[
{\bm C}_2 \cdot \vec{q} + {\bm E} \cdot \vec{T} =  \vec{g}
\]
and we then replace $\vec{q}$ by the expression above:
\[
{\bm C}_2 \cdot [ {\bm M}_e^{-1}\cdot (\vec{f} -{\bm C}_1\cdot  \vec{T} ) ]
 + {\bm E}\cdot \vec{T} =  \vec{g}
\]
or, 
\[
- {\bm C}_2\cdot {\bm M}_e^{-1}\cdot {\bm C}_1 \cdot \vec{T}  
 + {\bm E}\cdot \vec{T} 
=  \vec{g} - {\bm C}_2\cdot  {\bm M}_e^{-1}\cdot \vec{f}
\]
and finally
\[
[ {\bm E} -  {\bm C}_2 \cdot {\bm M}_e^{-1} \cdot {\bm C}_1 ]\cdot  \vec{T}
=  \vec{g} - {\bm C}_2 \cdot {\bm M}_e^{-1} \cdot \vec{f}
\]
\end{itemize}
Note that the matrix will still be twice as big than in the standard FEM case since 
each node counts two temperature dofs. 


\paragraph{Choice of ${\cal C}$ and ${\cal E}$} Li \cite{li06}
shows satisfying results for ${\cal E}=4$ and ${\cal C}=-1/2,0,1/2$ or 
${\cal E}=0$ and ${\cal C}=1,4,10$.


\begin{remark}
Aside from the shear complexity of the above derivations as compared to those
for the SG method, the fact that we have two tuning parameters ${\cal E}$
and ${\cal C}$ is a real drawback.
\end{remark}




