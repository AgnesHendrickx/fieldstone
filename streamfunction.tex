\index{Stream Function}

\Literature \cite{giju98}\cite{scja81}\cite{chyu84}\cite{chri84}\cite{hayu94}\cite{olwh97}

\subsubsection{In Cartesian coordinates}

The Stream function (commonly denoted by $\Phi$ or $\Psi$) approach is a useful approach in fluid dynamics as it 
can provide relatively quick solutions to 2D incompressible flow problems.
Lines of constant $\Phi$ are called stream lines and give a useful representation of the flow.
The definition of the stream function is such that 
\begin{eqnarray}
u &=& -\frac{\partial \Phi}{\partial y} \\
v &=& \frac{\partial \Phi}{\partial x} 
\end{eqnarray}
It then follows that the velocity field based on the above equations 
automatically fulfills the continuity equation:
\[
\vec\nabla\cdot\vec\upnu 
= \frac{\partial u}{\partial x}+\frac{\partial v}{\partial y}
= -\frac{\partial^2 \Phi}{\partial x \partial y}+\frac{\partial^2 \Phi}{\partial x\partial y}
=0
\]
The stream function can also be substituted into the (constant viscosity) Stokes equation $-\vec\nabla p + \eta \Delta \vec\upnu=\vec 0$:
\begin{eqnarray}
-\frac{\partial p}{\partial x} 
- \eta \left( \frac{\partial^3 \Phi}{\partial^2 x\partial y} + \frac{\partial^3 \Phi}{\partial^3 y} \right) &=& 0 \\
-\frac{\partial p}{\partial y} 
- \eta \left( 
\frac{\partial^3 \Phi}{\partial^3 x}
+ 
\frac{\partial^3 \Phi}{\partial x\partial^2 y}  \right) &=& 0
\end{eqnarray}
We can now eliminate the pressure term by taking the partial derivative of the 
first equation with respect to $y$ and the partial derivative of the second one 
with respect to $x$, and substracting both. We get:
\begin{equation}
\frac{\partial^4\Phi}{\partial x^4}+
\frac{\partial^4\Phi}{\partial x^2 \partial y^2}+
\frac{\partial^4\Phi}{\partial y^4}=0
\end{equation}
or, 
\begin{equation}
\left(
\frac{\partial^2}{\partial x^2}+
\frac{\partial^2}{\partial y^2}
\right)
\left(
\frac{\partial^2}{\partial x^2}+
\frac{\partial^2}{\partial y^2}
\right)
\Phi = 0
\end{equation}
or, 
\[
\vec\nabla^2 \vec\nabla^2 \Phi = \vec\nabla^4 \Phi = 0
\]
which is known as the Biharmonic operator.

\index{Biharmonic Operator} 


\subsubsection{In Cylindrical coordinates}

TODO

VERIFY THOSE! minus signs ?
\[
\upnu_r=\frac{1}{r}\frac{\partial \Phi}{\partial \theta} 
\]
\[
\upnu_\theta=-\frac{\partial \Phi}{\partial r} 
\]
