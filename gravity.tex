







What follows on this page is an unfinished attempt to link spherical harmonics with 
my 2018 paper. 

We start from the Poisson equation for the gravity potential:
\begin{equation}
\Delta U = 4\pi {\cal G} \rho(\vec{r})
\end{equation}
As a consequence, inside a domain where $\rho=0$, the equation becomes $\Delta U=0$.

Let us assume that the spherical coordinates are appropriate for the problem at hand, and that 
the potential can be decomposed as follows:
\[
U(r,\theta,\phi) = U_r(r) U\bot(\theta,\phi)
\]
The full Laplacian operator in spherical coordinates is given 
by\footnote{\url{https://en.wikipedia.org/wiki/Laplace_operator}}:
\[
\Delta U 
= 
\underbrace{\frac{1}{r^2} \frac{\partial }{\partial r}\left(r^2 \frac{\partial U}{\partial r}\right)}_{\Delta_r}
+
\underbrace{
\frac{1}{r^2 \sin\theta} \frac{\partial }{\partial \theta} \left(\sin\theta \frac{\partial U}{\partial \theta} \right) 
+
\frac{1}{r^2 \sin^2\theta} \frac{\partial^2 U }{\partial \phi^2}
}_{\Delta_\bot}
\]
we then have:
\[
(\Delta_r + \Delta_\bot)(U_r U_\bot)=0
\]
i.e., 
\[
U_\bot \Delta_r U_r + U_r \Delta_\bot U_\bot=0
\]
Assuming $U_\bot=\sum_l\sum_m U_{lm}Y_{lm}$, knowing that spherical 
harmonics functions verify
\[
r^2 \Delta_\bot Y_l^m(\theta,\phi) = -l(l+1) Y_l^m (\theta,\phi)
\]
and assuming for now that the problem at hand is 1st degree (l=1), then 
\[
\Delta_\bot Y_l^m(\theta,\phi) = -\frac{2}{r^2} Y_l^m (\theta,\phi)
\]
and then
\[
\Delta_r U_r - U_r \frac{2}{r^2}=0
\]
make a link with my 2018 paper. 






\newpage
In spherical coordinates, the Laplacian is given by
\[
\Delta = 
\frac{1}{r^2} \frac{\partial }{\partial r} \left( r^2 \frac{\partial }{\partial r} \right)
+
\frac{1}{r^2 \sin^2 \theta} \frac{\partial}{\partial \theta} \left( \sin\theta \frac{\partial }{\partial \theta} \right)
+
\frac{1}{r^2 \sin^2 \theta} \frac{\partial^2}{\partial \phi^2}
\]
We wish to solve Laplace's equation $\Delta T(r,\theta,\phi)=0$ using the method 
of separation of variables:
\[
T(r,\theta,\phi) = R(r) \Theta(\theta) \Phi(\phi)
\]
We can insert this decomposition into the Laplace equation and multiply it by $r^2/R\Theta\Phi$ to obtain
\[
\frac{1}{R} \frac{d}{dr} \left( r^2 \frac{dR}{dr} \right) 
+ 
\frac{1}{\Theta} \frac{1}{\sin \theta} \frac{d}{d\theta} \left( \sin \theta \frac{d\Theta}{d\theta} \right)
+
\frac{1}{\Phi} \frac{1}{\sin^2 \theta} \frac{d^2\Phi}{d\phi^2}
=
0
\]
For reasons that will become clear later, the separation constant  is taken to be $-m^2$:
\begin{eqnarray}
\frac{1}{\Phi} \frac{d^2\Phi}{d\phi^2} &=& -m^2  \label{eq:spha2} \\
-\frac{\sin\theta}{\Theta} \frac{d}{d\theta} \left( \sin \theta \frac{d\Theta}{d\theta} \right)
- \frac{\sin^2\theta}{R} \frac{d}{dr} \left( r^2 \frac{dR}{dr} \right) &=& -m^2  \label{eq:spha2}
\end{eqnarray}
The first equation yields
\[
\Phi(\phi) = 
\left\{
\begin{array}{c}
e^{im\phi} \\ e^{-im\phi}
\end{array}
\right.
\qquad\qquad 
\text{for} \; m=0,1,2,3,...
\]
Note that $m$ must be an integer since $\phi$ is a periodic variable and $\Phi(\phi + 2\pi) = \Phi(\phi)$. 
In the case of $m=0$, the general solution is $\Phi(\phi) = a\phi + b$, but we must choose $a=0$ to
be consistent with $\Phi(\phi + 2\pi) = \Phi(\phi)$. Hence in the case of $m = 0$, only one solution is
allowed.

Eq.~\eqref{eq:spha2} can now be recast in the following form:
\begin{equation}
 \frac{1}{R} \frac{d}{dr} \left( r^2 \frac{dR}{dr} \right) 
= 
-\frac{1}{\Theta}\frac{1}{\sin\theta} \frac{d}{d\theta} \left( \sin \theta \frac{d\Theta}{d\theta} \right) 
+\frac{m^2}{\sin^2\theta} \label{eq:spha3}
\end{equation}
where the separation variable at this step is denoted by $l(l + 1)$ for reasons that will shortly
become clear.
The resulting radial equation is 
\[
\frac{1}{R} \frac{d}{dr} \left( r^2 \frac{dR}{dr} \right)  = l(l+1)
\]
or, 
\[
r^2 \frac{d^2R}{dr^2} + 2 r \frac{dR}{dr} - l(l+1)R =0
\]
The solution is of the form $R=r^s$. To determine the exponent $s$, 
we insert this solution back into the above ODE. The end result is
\[
s(s+1)=l(l+1) \qquad \Rightarrow \qquad s=l \; \text{or} \; s=-l-1
\]
or, 
\[
R(r) = 
\left\{
\begin{array}{c}
r^l \\ r^{-(l+1)}
\end{array}
\right.
\]
Eq.~\eqref{eq:spha3} also yields:
\[
\frac{1}{\sin\theta} \frac{d}{d\theta} \left( \sin\theta \frac{d\Theta}{d\theta} \right) 
+
\left[ l(l+1) - \frac{m^2}{\sin^2\theta} \right] \Theta = 0
\]
One can then carry out the following change of variables $x=\cos\theta$ and $y=\Theta(\theta)$ so that 
the above equation reduces to:
\[
(1-x^2) \frac{d^2 y}{d x^2} - 2x \frac{dy}{dx} + 
\left[ l(l+1)-\frac{m^2}{\sin^2\theta} \right] y =0
\]
This equation is the differential equation for associated Legendre 
polynomials\footnote{\url{https://en.wikipedia.org/wiki/Associated_Legendre_polynomials}}.
\index{general}{Associated Legendre polynomials}
We then have
\[
y=P_l^m(x) \qquad \text{for} \; l=0,1,2,3,... \quad \text{and} \; m=-l,-l+1,...0,...l=1,l
\]
and 
\[
\boxed{
P_l^m(x)= \frac{(-1)^m}{2^l \; l!} (1-x^2)^{m/2}
\frac{d^{l+m}}{dx^{l+m}} (x^2-1)^l
}
\]
with $m \geq 0$ and $l\geq 0$.
The first few polynomials are
\begin{eqnarray}
P_0^0(\cos\theta)    &=& 1 \nn\\ \nn\\ 
P_1^{-1}(\cos\theta) &=& \frac12 \sin\theta \nn\\ 
P_1^{0}(\cos\theta)  &=&  \cos\theta \nn\\ 
P_1^{+1}(\cos\theta) &=& -\sin\theta \nn\\ \nn\\
P_2^{-2}(\cos\theta) &=& \frac18 \sin^2\theta \nn\\
P_2^{-1}(\cos\theta) &=& \frac12\sin\theta\cos\theta \nn\\
P_2^{0}(\cos\theta)  &=& \frac12(3\cos^2\theta -1) \nn\\
P_2^{+1}(\cos\theta) &=& -3 \sin\theta\cos\theta \nn\\
P_2^{+2}(\cos\theta) &=& 3\sin^2\theta \nn
\end{eqnarray}





In our case the differential equation for the associated Legendre polynomials, given above, depends
on $m^2$ and is therefore not sensitive to the sign of $m$.
Consequently, $P_l^m(x)$ and $P_l^{-m}(x)$ must be equivalent solutions and 
hence proportional to each other, and one can show that
\begin{equation}
P_l^{-m}(\cos\theta) = (-1)^m\frac{(l-m)!}{(l+m)} P_l^m(\cos\theta)
\label{eq:spha4}
\end{equation}
Combining all the results obtained above, we have found that the general solution to
Laplace’s equation is of the form
\begin{mdframed}[backgroundcolor=blue!5]
\[
T(r,\theta,\phi) = 
\left\{
\begin{array}{c}
r^l \\ r^{-(l+1)}
\end{array}
\right\}
P_l^m(\cos\theta) 
\left\{
\begin{array}{c}
e^{im\phi} \\ e^{-im\phi}
\end{array}
\right\}
\]
\end{mdframed}
where $l=0,1,2,3,...$ and $m=-l,-l+1,...,l-1,l$.

When solving the Laplace’s equation in spherical coordinates, it is traditional
to introduce the spherical harmonics, $Y_l^m(\theta,\phi)$:
\begin{equation}
Y_l^m(\theta,\phi) = (-1)^m \sqrt{\frac{2l+1}{4\pi} \frac{(l-m)!}{(l+m)!}} P_l^m(\cos\theta) e^{im\phi}
\qquad 
\textrm{for} \; l=0,1,2,3,... \; \textrm{and} \; m=-l,-l+1,...,l-1,l
\label{eq:spha5}
\end{equation}
The phase factor ($-1$) , introduced originally by Condon and Shortley, is convenient for
applications in quantum mechanics. Note that Eq.~\eqref{eq:spha4} implies that
\[
Y_l^{-m} (\theta, \phi) = (-1)^m Y_l^m (\theta,\phi)^* 
\]
where the star means complex conjugation.

The normalization factor in Eq.~\eqref{eq:spha5} has been
chosen such that the spherical harmonics are normalized to one. In particular, these func-
tions are orthonormal and complete. The orthonormality relation is given by:
\[
\int Y_l^m(\theta,\phi) Y_{l'}^{m'}(\theta,\phi) d\Omega = \delta_{ll'} \delta_{mm'}
\]
where $d\Omega = \sin\theta d\theta d\phi$ is the differential solid angle in spherical coordinates.

It is important to note that there are different normalisations for spherical harmonics.
In this document we choose:
\begin{equation}
\boxed{
Y_l^m(\theta,\phi) = \sqrt{\frac{2l+1}{4\pi} \frac{(l-m)!}{(l+m)!}} P_l^m(\cos\theta) e^{im\phi}
}
\qquad 
\textrm{for} \; l=0,1,2,3,... \; \textrm{and} \; m=-l,-l+1,...,l-1,l
\label{eq:spha6}
\end{equation}
which is Eq.(7.8.1) in the Schubert, Turcotte \& Olson book \cite{scto01}. 
In this case the $(-1)^m$ is inside the $P_m^m$.
The first few spherical harmonics are shown below in the real representation (i.e.cusing $\cos m\phi$  instead of $e^{i m \phi}$) \footnote{\url{https://en.wikipedia.org/wiki/Table_of_spherical_harmonics}} \footnote{\url{https://mathworld.wolfram.com/SphericalHarmonic.html}}:
\begin{eqnarray}
Y_0^0(\theta,\phi)    &=& \sqrt{\frac{1}{4\pi}} \nn\\ \nn\\
Y_1^{-1}(\theta,\phi) &=& \sqrt{\frac{3}{8\pi}} \cos\phi\sin\theta\nn\\
Y_1^{0 }(\theta,\phi) &=& \sqrt{\frac{3}{4\pi}} \cos\theta \nn\\
Y_1^{+1}(\theta,\phi) &=& -\sqrt{\frac{3}{8\pi}} \cos\phi\sin\theta \nn\\ \nn\\
Y_2^{-2}(\theta,\phi) &=& \sqrt{\frac{15}{32\pi}} \cos(2\phi) \sin^2\theta \nn\\ 
Y_2^{-1}(\theta,\phi) &=& \sqrt{\frac{15}{8\pi}} \cos\phi \sin\theta\cos\theta \nn\\ 
Y_2^{ 0}(\theta,\phi) &=& \sqrt{\frac{5}{16\pi}} (3\cos^2\theta -1) \nn\\
Y_2^{+1}(\theta,\phi) &=& -\sqrt{\frac{15}{8\pi}} \cos\phi \sin\theta\cos\theta \nn\\
Y_2^{+2}(\theta,\phi) &=& \sqrt{\frac{15}{32\pi}} \cos(2\phi) \sin^2\theta \nn
\end{eqnarray}
\todo[inline]{replace those my complex ones !}


Another normalisation is sometimes used:  
\begin{equation}
Y_l^m(\theta,\phi) = \sqrt{(2l+1) \frac{(l-m)!}{(l+m)!}} P_l^m(\cos\theta) e^{im\phi}
\qquad 
\textrm{for} \; l=0,1,2,3,... \; \textrm{and} \; m=-l,-l+1,...,l-1,l
\label{eq:spha6}
\end{equation}
with
\[
\frac{1}{4\pi}
\int Y_l^m(\theta,\phi) Y_{l'}^{m'}(\theta,\phi) d\Omega = \delta_{ll'} \delta_{mm'}
\]





\begin{remark}
In \cite{zhmt08} the authors use a normalized associated Legendre
polynomial that is related to the associated Legendre polynomial $P_l^m$ as:
\[
p_{lm}(\theta,\phi) = \sqrt{\frac{2l+1}{2\pi(1+\delta_{m0})} \frac{(l-m)!}{(l+m)!}} P_l^m(\cos\theta)
\]
Note the absence of the $(-1)^m$ term and the presence of the kronecker delta in the denominator.
\end{remark}



\Literature: SHTools: Tools for Working with Spherical Harmonics \cite{wime18}
