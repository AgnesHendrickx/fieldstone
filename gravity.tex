WORK in PROGRESS. DUH.

We start from the Poisson equation for the gravity potential:
\begin{equation}
\Delta U = 4\pi \rho {\cal G}
\end{equation}
As a consequence, inside a domain where $\rho=0$, the equation becomes $\Delta U=0$.

Let us assume that the spherical coordinates are appropriate for the problem at hand, and that 
the potential can be decomposed as follows:
\[
U(r,\theta,\phi) = U_r(r) U\bot(\theta,\phi)
\]
The full Laplacian operator in spherical coordinates is given 
by\footnote{\url{https://en.wikipedia.org/wiki/Laplace_operator}}:
\[
\Delta U 
= 
\underbrace{\frac{1}{r^2} \frac{\partial }{\partial r}\left(r^2 \frac{\partial U}{\partial r}\right)}_{\Delta_r}
+
\underbrace{
\frac{1}{r^2 \sin\theta} \frac{\partial }{\partial \theta} \left(\sin\theta \frac{\partial U}{\partial \theta} \right) 
+
\frac{1}{r^2 \sin^2\theta} \frac{\partial^2 U }{\partial \phi^2}
}_{\Delta_\bot}
\]
we then have:
\[
(\Delta_r + \Delta_\bot)(U_r U_\bot)=0
\]
i.e., 
\[
U_\bot \Delta_r U_r + U_r \Delta_\bot U_\bot=0
\]
Assuming $U_\bot=\sum_l\sum_m U_{lm}Y_{lm}$, knowing that spherical 
harmonics functions verify
\[
r^2 \Delta_\bot Y_l^m(\theta,\phi) = -l(l+1) Y_l^m (\theta,\phi)
\]
and assuming for now that the problem at hand is 1st degree (l=1), then 
\[
\Delta_\bot Y_l^m(\theta,\phi) = -\frac{2}{r^2} Y_l^m (\theta,\phi)
\]
and then
\[
\Delta_r U_r - U_r \frac{2}{r^2}=0
\]
make a link with my 2018 paper. 
\newpage

In spherical coordinates, the Laplacian is given by
\[
\Delta = 
\frac{1}{r^2} \frac{\partial }{\partial r} \left( r^2 \frac{\partial }{\partial r} \right)
+
\frac{1}{r^2 \sin^2 \theta} \frac{\partial}{\partial \theta} \left( \sin\theta \frac{\partial }{\partial \theta} \right)
+
\frac{1}{r^2 \sin^2 \theta} \frac{\partial^2}{\partial \phi^2}
\]
We wish to solve Laplace's equation $\Delta T(r,\theta,\phi)=0$ using the method 
of separation of variables:
\[
T(r,\theta,\phi) = R(r) \Theta(\theta) \Phi(\phi)
\]
We can insert this decomposition into the Laplace equation and multiply it by $r^2/R\Theta\Phi$ to obtain
\[
\frac{1}{R} \frac{d}{dr} \left( r^2 \frac{dR}{dr} \right) 
+ 
\frac{1}{\Theta} \frac{1}{\sin \theta} \frac{d}{d\theta} \left( \sin \theta \frac{d\Theta}{d\theta} \right)
+
\frac{1}{\Phi} \frac{1}{\sin^2 \theta} \frac{d^2\Phi}{d\phi^2}
=
0
\]
For reasons that will become clear later, the separation constant  is taken to be $-m^2$:
\begin{equation}
\frac{1}{\Phi} \frac{d^2\Phi}{d\phi^2} = -m^2
\label{eq:spha2}
\end{equation}
and 
\begin{equation}
-\frac{\sin\theta}{\Theta} \frac{d}{d\theta} \left( \sin \theta \frac{d\Theta}{d\theta} \right)
- \frac{\sin^2\theta}{R} \frac{d}{dr} \left( r^2 \frac{dR}{dr} \right) = -m^2
\label{eq:spha2}
\end{equation}
The first equation yields
\[
\Phi(\phi) = 
\left\{
\begin{array}{c}
e^{im\phi} \\ e^{-im\phi}
\end{array}
\right.
\qquad\qquad 
\text{for} \; m=0,1,2,3,...
\]
Note that $m$ must be an integer since $\phi$ is a periodic variable and $\Phi(\phi + 2\pi) = \Phi(\phi)$. 
In the case of $m=0$, the general solution is $\Phi(\phi) = a\phi + b$, but we must choose $a=0$ to
be consistent with $\Phi(\phi + 2\pi) = \Phi(\phi)$. Hence in the case of $m = 0$, only one solution is
allowed.

Equation \ref{eq:spha2} can now be recast in the following form:
\[
 \frac{1}{R} \frac{d}{dr} \left( r^2 \frac{dR}{dr} \right) 
= 
-\frac{1}{\Theta}\frac{1}{\sin\theta} \frac{d}{d\theta} \left( \sin \theta \frac{d\Theta}{d\theta} \right) 
+\frac{m^2}{\sin^2\theta} = l(l+1)
\]
where the separation variable at this step is denoted by $l(l + 1)$ for reasons that will shortly
become clear.
The resulting radial equation is 
\[
\frac{1}{R} \frac{d}{dr} \left( r^2 \frac{dR}{dr} \right)  = l(l+1)
\]
or, 
\[
r^2 \frac{d^2R}{dr^2} + 2 r \frac{dR}{dr} - l(l+1)R =0
\]
The solution is of the form $R=r^s$. To determine the exponent $s$, 
we insert this solution back into the above ODE. The end result is
\[
s(s+1)=l(l+1) \qquad \Rightarrow \qquad s=l \; \text{or} \; s=-l-1
\]
or, 
\[
R(r) = 
\left\{
\begin{array}{c}
r^l \\ r^{-(l+1)}
\end{array}
\right.
\]








