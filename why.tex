The Finite Element Method (FEM) is by no means the only method 
to solve PDEs in geodynamics, nor is it necessarily the best one.
Other methods are employed very succesfully, such as the Finite Difference 
Method (FDM), the Finite Volume Method (FVM), and to a lesser extent
the Discrete Element Method (DEM) \cite{egho07,egsc07,funi14}, 
or the Element Free Galerkin Method (EFGM) \cite{hans03}.
I have been using FEM since 2008 and I do not have real 
experience to speak of in FVM or FDM so I concentrate in this book 
on what I know best. 

The first papers I could find showcasing the FEM in geodynamics are oisted hereafter:
\cite{gart78}, 
\cite{anbr80}
\cite{engl82}
\cite{baum85}
\cite{enho86}
\cite{zupa86}
\cite{zupf86}
\cite{brau93}
\cite{brau94}
\cite{brbe95}
\cite{yowo95}
\cite{dusa96}.

