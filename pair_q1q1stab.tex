\begin{flushright} {\tiny {\color{gray} pair\_q1q1stab.tex}} \end{flushright}
%~~~~~~~~~~~~~~~~~~~~~~~~~~~~~~~~~~~~~~~~~~~~~~~~~~~~~~~~~~~~~~~~~~~~~~~~~~~~~~~~~~~~~~~~~~~~~~~~~~

\begin{minipage}[t]{0.5\textwidth}

\begin{center}
\begin{tikzpicture}
%\draw[fill=gray!23,gray!23](0,0) rectangle (5,5);
%\draw[step=0.5cm,gray,very thin] (0,0) grid (4,4); %background grid
\draw[thick] (1,1) -- (3,1.2) -- (2.7,3) -- (1.1,3.1) -- cycle;  
\node[] at (0.8,0.8) {0};
\node[] at (3.2,1)   {1};
\node[] at (2.9,3.1) {2};
\node[] at (0.9,3.2) {3};
\draw[black,fill=teal] (1,1)     circle (2pt); \draw[violet] (1,1) circle (4pt);
\draw[black,fill=teal] (3,1.2)   circle (2pt); \draw[violet] (3,1.2) circle (4pt);
\draw[black,fill=teal] (2.7,3)   circle (2pt); \draw[violet] (2.7,3) circle (4pt);
\draw[black,fill=teal] (1.1,3.1) circle (2pt); \draw[violet] (1.1,3.1) circle (4pt);
\draw[black,fill=teal] (3.1,0.2) circle (2pt); 
\node[] at (3.4,0.2) {$\vec\upnu$};
\draw[violet] (4.1,0.2) circle (4pt); 
\node[] at (4.4,0.2) {$p$};
\node[] at (2.5,4.5) {4 vel. nodes, 4 press. nodes};
\end{tikzpicture}\\
\end{center}

\end{minipage}
\begin{minipage}[t]{0.5\textwidth}

\begin{center}
\begin{tikzpicture}
\draw[fill=gray!23,gray!23](0,0) rectangle (5,5);
%\draw[step=0.25cm,gray,very thin] (0,0) grid (5,4); %background grid
\draw[thick] (1,0.5) -- (3.25,0.75) -- (3,3) -- (0.5,2.5) -- cycle; %1-2-6-5
\draw[thick] (3.25,0.75) -- (4,1.5) -- (4.25,3.75) -- (3,3) -- cycle; %2-3-7-6
\draw[thick] (0.5,2.5) -- (3,3) -- (4.25,3.75) -- (1.75,3.5) -- cycle; %5-6-7-4
\draw[thin]   (1,0.5) -- (2,1.75) -- (1.75,3.5) -- (0.5,2.5)   --cycle; % 1-0-4-5 
\draw[thin] (2,1.75) -- (4,1.5); 
%\node[] at (0.8,0.8) {0};
%\node[] at (3.2,1) {1};
%\node[] at (2.9,3.1) {2};
%\node[] at (0.9,3.2) {3};
\draw (1,0.5) circle (4pt);
\draw (3.25,0.75) circle (4pt);
\draw (4,1.5) circle (4pt);
\draw (2,1.75) circle (4pt);
\draw (0.5,2.5) circle (4pt);
\draw (3,3) circle (4pt);
\draw (4.25,3.75) circle (4pt);
\draw (1.75,3.5) circle (4pt);
\draw[black,fill=black] (1,0.5)   circle (2pt);
\draw[black,fill=black] (3.25,0.75)   circle (2pt);
\draw[black,fill=black] (3,3)   circle (2pt);
\draw[black,fill=black] (0.5,2.5)   circle (2pt);
\draw[black,fill=black] (1.75,3.5)  circle (2pt);
\draw[black,fill=black] (4.25,3.75)  circle (2pt);
\draw[black,fill=black] (4,1.5) circle (2pt);
\draw[black,fill=black] (2,1.75) circle (2pt);
\draw[black,fill=black] (3.1,0.2) circle (2pt); \node[] at (3.4,0.2) {$\vec\upnu$};
\draw (4.1,0.2) circle (4pt); \node[] at (4.4,0.2) {$p$};
\node[] at (2.5,4.5) {8 vel. nodes, 8 press. nodes};
\end{tikzpicture}\\
\end{center}

\end{minipage}

The $Q_1\times Q_1$ element is not LBB-stable but it can be stabilised. Despite
some applications in geodnamics (it is used in \textcite{bugs09} (2009) 
and \textcite{busa13} (2013)), it is not appropriate for buoyancy-driven flows, 
as shown in \textcite{thba21}.

See \textcite{nosi01} (2001) for a fourier analysis of the normal 
and stablised (a la \textcite{hufb86} (1986)) $Q_1\times Q_1$ element.
Stabilisation is worked out out in \textcite{dobo04} (2004), \textcite{bodg06} (2006), 
and \textcite{bodo06} (2006).

$Q_1\times P_0$-stab. Pro: stabilisation can be switched off; Con: stabilisation for deformed elements? 
problem near boundaries: incomplete stencil? choice of parameter $\beta$.

$Q_1\times Q_1$-stab. Pro: easier to implement than $Q_1\times P_0$-stab, stabilisation local to element, easier when elements are not rectangular, no free parameter; Con: stabilisation cannot be switched off.

\Literature: \textcite{shry78,temr92,tezd92,grcc95,idsn95,knto00,fros07,lihc09}. 
See \textcite{brlu09} for a review of local projection stabilisation for incompressible flow problems. 

This unstable pair is also used in ice sheet modelling \textcite{heah18} , \textcite{zhjg11}, 
\textcite{zwgg07}. A $P_1\times P_1$ version of it is used in \textcite{kahp20} (2020).
