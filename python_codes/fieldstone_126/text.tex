See \textcite{grfr03}

Two main relations are used, expressing fluid mass conservation
\[
\frac{\partial (\rho \phi)}{\partial t} + \vec\nabla\cdot (\rho \vec{\upnu}) = 0
\]
and Darcy's law
\[
\vec\upnu = {\bm K} \cdot (\vec{\nabla} P + \rho \vec{g})
\]
where $\rho$ is the density of the fluid, $\phi$ is the
connected mean porosity of the rock, $\vec{\upnu}$ is the Darcy flow
rate (fluid particle velocity times porosity), ${\bm K}$ is the
permeability tensor, $P$ is the total pore pressure and $\vec{g}$ is
the gravity.

First, during the evolution, we consider that the density 
$\rho = \rho' + \rho_m$ will evolve in time of an amount $\rho'$ 
around its local mean value $\rho_m$ , and finally, we rewrite the fluid mass
conservation by neglecting the terms $\rho'$ compared to $\rho_m$.

TODO

Second, we consider that the total pore pressure $P$ can
be decomposed as $P = p + P_{hydro}$ such that $P_{hydro}$ is the
hydrostatic pressure verifying $\vec\nabla P_{hydro} + \rho_m \vec{g} = \vec{0}$. Third, we
consider that the principal directions of the permeability
tensor K are the axes of the fault frame x, z, which means
that the anisotropy of the permeability takes the natural
direction of the fault. This last assumption seems reasonable
if the fault has a constant straight and vertical shape. By
applying these three simplifications, we rewrite the fluid
mass conservation and Darcy’s law in the following
simplified form:


\[
\phi \frac{\partial \rho'}{\partial t}
+ \rho_m \frac{\partial \phi}{\partial t} + \vec\nabla \cdot (\rho_m \vec\upnu) = f
\]
with $\vec{\upnu}$ the Darcy flow rate (minus sign in grfr03??)
\[
\vec\upnu = - {\bm K} \cdot \vec{\nabla} p
\]
and
\[
\phi \frac{\partial \rho'}{\partial t}
=
\rho_m C_f  \phi \frac{\partial p}{\partial t}
\]
so 
\[
\rho_m C_f  \phi \frac{\partial p}{\partial t}
+ \rho_m \frac{\partial \phi}{\partial t} 
- \vec\nabla \cdot (\rho_m {\bm K} \cdot \vec{\nabla} p ) ) = f
\]

We assume an isotropic permeability so that the tensor now simply writes:
\[
K = K_0 \phi^3
\]
The porosity $\phi$ is given by:


\[
\underbrace{
\left( \int_\Omega \vec{\bN}^T \rho_m C_f \phi \vec{\bN} dV  \right)
}_{\bm M}
 \cdot \frac{\partial \vec{\cal P}}{\partial t}
+
\underbrace{
\left( \int_\Omega {\bm B}^T \cdot \rho_m {\bm K} \cdot {\bm B} dV \right)
}_{{\bm K}_d}
 \cdot \vec{\cal P}
=
\int_\Omega \vec{\bN}^T f dV 
-
\int_\Omega \vec{\bN}^T   \rho_m \frac{\partial \phi}{\partial t} dV
\]


\[
\frac{\partial \phi}{\partial t} = \alpha \dot{\varepsilon}_e
\]

\[
\frac{\partial f}{\partial t} = \varphi f \dot{\varepsilon}_e
\qquad
\Rightarrow
\qquad
f^{t+\delta t} = f^t (1 + \varphi  \dot{\varepsilon}_e \delta t)
\]


characteristic time
\[
t_c 
= \frac{\phi C_f h^2}{K} 
= \frac{\phi^{-2} C_f h^2}{K_0} 
\]


TODO:
- $f$ constant, derive analytical soltuion, compare with ss results

critical time of dacry <<<< tectonics time step   -> solve ss darcy ?! only doable bc dphi/dt is freezed when darcy solved

- dphi/dt remains constant=0 during a tectonic time step, dt-> dt_tect ~1e5 year. while solving Darcy, phi is fixed, so we can compute ss.

same thing for \dot{f} equation: during Darcy f is fixed. incrementing f takes place before Darcy is called.  

- cutoff on porosity - done


