The setup is as follows: a 2D square of elastic material of size $L$ is 
subjected to the following boundary conditions: free slip on the sides, no slip at the 
bottom and free at the top. It has a density $\rho$ and is placed is a gravity 
field ${\bm g}=-g {\bm e}_y$.
For an isotropic elastic medium the stress tensor is given by:
\[
{\bm \sigma} = \lambda ({\bm \nabla}\cdot{\bm u}) {\bm 1} + 2 \mu {\bm \varepsilon}
\]
where $\lambda$ is the Lam{\'e} parameter and $\mu$ is the shear modulus.
The displacement field is ${\bm u}=(0,u_y(y))$ because of symmetry reasons 
(we do not expect any of the dynamic quantities to depend on the $x$ coordinate and 
also expect the horizontal displacement to be exactly zero).
The velocity divergence is then ${\bm \nabla}\cdot{\bm u} = \partial u_y/\partial y$
and the strain tensor:
\[
{\bm \varepsilon}
=
\left(
\begin{array}{cc}
0 & 0 \\
0 & \frac{\partial u_y}{\partial y}
\end{array}
\right)
\]
so that the stress tensor is:

\[
{\bm \sigma} =
\left(
\begin{array}{cc}
\lambda \frac{\partial u_y}{\partial y} &  0 \\
0 & (\lambda + 2 \mu) \frac{\partial u_y}{\partial y}
\end{array}
\right)
\]

\[
{\bm \nabla}\cdot {\bm \sigma} =
(\partial_x \quad \partial_y)\cdot 
\left(
\begin{array}{cc}
\lambda \frac{\partial u_y}{\partial y} &  0 \\
0 & (\lambda + 2 \mu) \frac{\partial u_y}{\partial y}
\end{array}
\right)
=
\left(
\begin{array}{c}
0 \\
(\lambda + 2 \mu) \frac{\partial^2 u_y}{\partial y^2}
\end{array}
\right)
=
\left(
\begin{array}{c}
0 \\
\rho g
\end{array}
\right)
\]
so that the vertical displacement is then given by:
\[
u_y(y) = \frac{1}{2} \frac{\rho g}{\lambda + 2 \mu} y^2 + \alpha y + \beta 
\] 
where $\alpha$ and $\beta$ are two integration constants.
We need now to use the two boundary conditions: the first one states that the displacement
is zero at the bottom, i.e. $u_y(y=0)=0$ which immediately implies $\beta=0$.
The second states that the stress at the top is zero (free surface), which implies that 
$\partial u_y/\partial y (y=L)=0$ which allows us to compute $\alpha$.
Finally:
\[
u_y(y) = \frac{\rho g}{\lambda + 2 \mu} (\frac{y^2}{2}-L y) 
\] 
The pressure is given by
\[
p=-(\lambda + \frac{2}{3} \mu) {\bm \nabla}\cdot{\bm u}
= (\lambda + \frac{2}{3} \mu)  \frac{\rho g}{\lambda + 2 \mu} (L -y)
= \frac{\lambda + \frac{2}{3} \mu}{\lambda + 2 \mu} \rho g (L-y)  
= \frac{1 + \frac{2 \mu}{3 \lambda} }{1 + 2 \mu/\lambda} \rho g (L-y)  
\]
In the incompressible limit, the poisson ratio is $\nu \sim 0.5$. 
Materials are characterised by a finite Young's modulus $E$, which is related to 
$\nu$ and $\lambda$:
\[
\lambda=\frac{E \nu}{(1+\nu)(1-2\nu)}
\quad\quad
\mu=\frac{E}{2(1+\nu)}
\]
It is then clear that for incompressible parameters $\lambda$ becomes 
infinite while $\mu$ remains finite. In that case the pressure 
then logically converges to the well known formula:
\[
p=\rho g (L-y)
\]


