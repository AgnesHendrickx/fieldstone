
{\sl This fieldstone was developed in collaboration with Arie van den Berg}.

The system is a layer of fluid between $y=0$ and $y=1$, with boundary conditions $T(x,y=0)=1$ and $T(x,y=1)=0$, characterized by $\rho$, $c_p$, $k$, $\eta_0$. The Rayleigh number of the system is 
\[
\text{Ra}= \frac{\rho_0 g_0 \alpha \Delta T h^3}{\eta_0 \kappa}
\]
We have $\Delta T=1$, $h=1$ and choose $\kappa=1$ so that the Rayleigh number simplifies to
$\text{Ra}= \rho_0 g_0 \alpha /\eta_0$.

The Stokes equation is $\vec \nabla \cdot \bm \sigma + \vec b = \vec 0$ with $\vec b=\rho \vec g$. 
Then the components of the this equation on the $x$- and $y-$axis are:
\begin{eqnarray}
(\vec \nabla \cdot \bm \sigma)_x &=& - \rho \vec g \cdot \vec e_x = 0\\ 
(\vec \nabla \cdot \bm \sigma)_y &=& - \rho \vec g \cdot \vec e_y = \rho g_0
\end{eqnarray}
since $\vec g$ and $\vec e_y$ are in opposite directions ($\vec g = - g_0 \vec e_y$, with $g_0>0$).
The stream function formulation of the incompressible isoviscous Stokes equation is then
\[
\nabla^4 \Psi= \frac{g_0}{\eta_0}  \frac{\partial \rho}{\partial x} 
\]
Assuming a linearised density field with regards to temperature $\rho(T)=\rho_0 (1-\alpha T)$
we have 
\[
\frac{\partial \rho}{\partial x} 
=
-\rho_0 \alpha \frac{\partial T}{\partial x} 
\]
and then 
\begin{equation}
\boxed{
\nabla^4 \Psi= -\frac{\rho_0 g_0 \alpha}{\eta_0} g \frac{\partial T}{\partial x} 
= -Ra \frac{\partial T}{\partial x} 
}
\end{equation}
For small perturbations of the conductive state $T_0(y)=1-y$ we define the temperature perturbation $T_1(x,y)$ such that 
\[
T(x,y)=T_0(y)+T_1(x,y)
\]
The temperature perturbation $T_1$ satisfies the homogeneous boundary conditions $T_1(x,y=0)=0$ and
$T_1(x,y=1)=0$.
The temperature equation is
\[
\rho c_p \frac{DT}{Dt}
=\rho c_p \left( \frac{\partial T}{\partial t} + {\vec \upnu}\cdot {\vec \nabla} T \right) 
=\rho c_p \left( \frac{\partial T_0+T_1}{\partial t} + {\vec \upnu}\cdot {\vec \nabla} (T_0+T_1) \right) 
= k \Delta (T_0+T_1)
\]
and can be simplified as follows:
\[
\rho c_p \left( \frac{\partial T_1}{\partial t} + {\vec \upnu}\cdot {\vec \nabla} T_0 \right) 
= k \Delta T_1
\]
since $T_0$ does not depend on time, $\Delta T_0=0$ and we assume the nonlinear term ${\vec \upnu}\cdot {\vec \nabla} T_1 $ to be second order (temperature perturbations and coupled velocity changes are assumed to be small).
Using the relationship between velocity and stream function
$v_y=-\partial_x \Psi$ we have ${\bm v}\cdot {\bm \nabla} T_0 = -v_y = \partial_x \Psi$ and since $\kappa =k/\rho c_p=1$ we get 
\begin{equation}
\boxed{
\frac{\partial T_1}{\partial t} - \kappa \Delta T_1 = -\frac{\partial \Psi}{\partial x}
}
\end{equation}
%We also have [{\color{red} prove}]
%\[
%{\bm \nabla}^4 \Psi = -Ra \frac{\partial T_1}{\partial %x}
%\]
Looking at these equations, we immediately think about a separation of variables approach to solve these
equations. Both equations showcase the Laplace operator $\Delta$, and the eigenfunctions of the biharmonic operator and the Laplace operator are the same. 
We then pose that $\Psi$ and $T_1$ can be written:
\begin{eqnarray}
\Psi(x,y,t) &=& A_\Psi \exp(pt)\exp(\pm i k_x x) \exp(\pm i k_y y)= A_\Psi E_\psi(x,y,t) \\
T_1(x,y,t) &=& A_T \exp(pt) \exp(\pm i k_x x) \exp(\pm i k_y y)=A_T E_T(x,y,t) 
\end{eqnarray}
where $k_x$ and $k_y$ are the horizontal and vertical wave number respectively.
Note that we then have
\[
\nabla^2 \Psi = -(k_x^2+k_y^2) \Psi
\quad\quad
\nabla^2 T_1 = -(k_x^2+k_y^2) T_1
\]
The boundary conditions on $T_1$, coupled with a choice of a real function for the $x$ dependence yields:
\[
E_T(x,y,t) = \exp(pt) \cos (k_x x) \sin (n\pi y).
\]

{\color{red} from here onwards check for minus signs!}

The velocity boundary conditions are $v_y(x,y=0)=0$ and $v_y(x,y=1)=0$ which imposes conditions on $\partial \Psi/\partial x$ and we find that we can use the same $y$ dependence as for $T_1$. 
Choosing again for a real function for the $x$ dependence yields:
\[
E_\Psi(x,y,t) = \exp(pt) \sin(k_x x) \sin(n\pi z)
\]
We then have
\begin{eqnarray}
\Psi(x,y,t) &=& A_\Psi \exp(pt)  \sin(k_x x) \sin(n\pi z)   = A_\Psi E_\psi(x,y,t) \\
T_1(x,y,t)  &=& A_T \exp(pt)  \cos(k_x x) \sin(n\pi z)   = A_T E_T(x,y,t) 
\end{eqnarray}
In what follows we simplify notations: $k=k_x$. Then the two PDEs become:
\begin{eqnarray}
p T_1  + \kappa (k^2 + n^2 \pi^2 )  
-  k A_\Psi \exp(pt) \cos(k_x x) \sin(n\pi z) = k A_\Psi E_\theta 
\end{eqnarray}
\begin{eqnarray}
-Ra A_T \cos(kx) \sin(n\pi z) + \kappa (k^2 + n^2 \pi^2 )^2 A_\Psi = -Ra A_T E_\Psi 
+  \kappa (k^2 + n^2 \pi^2 )^2A_\Psi = 0
\end{eqnarray}
These equations must then be verified for all ...
which leads to write:
\[
\left(
\begin{array}{cc}
p + (k^2+n^2\pi^2) & -k \\
-Ra \; k & (k^2+n^2\pi^2)^2 
\end{array}
\right)
\left(
\begin{array}{c}
A_\theta \\ A_\Psi
\end{array}
\right)
=
\left(
\begin{array}{c}
0 \\ 0
\end{array}
\right)
\]
The determinant of such system must be nul otherwise there is only a trivial solution to the problem, i.e. $A_\theta=0$ and $A_\Psi=0$ which is not helpful. CHECK/REPHRASE
\[
D= [p + (k^2+n^2\pi^2)](k^2+n^2\pi^2)^2 - Ra \; k^2 =0
\]
or, 
\[
p = \frac{Ra \; k^2 -(k^2+n^2\pi^2)^3 }{ (k^2+n^2\pi^2)^2}
\]



The coefficient $p$ determines the stability of the system: if it is negative, the system is stable and both $\Psi$ and $T_1$ will decay to zero (return to conductive state). If $p=0$, then the system is meta-stable, and if $p>0$ then the system is unstable and the perturbations will grow. 
The threshold is then $p=0$ and the solution of the above system is 

