\begin{flushright} {\tiny {\color{gray} time\_discretisation.tex}} \end{flushright}
%~~~~~~~~~~~~~~~~~~~~~~~~~~~~~~~~~~~~~~~~~~~~~~~~~~~~~~~~~~~~~~~~~~~~~~~~~~~~~~~~~~~~~~~~~~~~~~~~~~



\index{general}{Forward Euler}
\index{general}{Backward Euler}
\index{general}{Crank-Nicolson}

Essentially we have to solve a PDE of the type:
\[
\frac{\partial T}{\partial t} = {\cal F}(\vec \upnu,T,\vec\nabla T,\Delta T)
\]
with ${\cal F}=\frac{1}{\rho C_p}(-\vec\upnu\cdot\vec\nabla T + \vec\nabla\cdot k\vec\nabla T)$.
The (explicit) forward Euler method is:
\[
\frac{T^{n+1}-T^n}{\delta t} = {\cal F}^n(T,\vec\nabla T,\Delta T)
\]
The (implicit) backward Euler method is:
\[
\frac{T^{n+1}-T^n}{\delta t} = {\cal F}^{n+1}(T,\vec\nabla T,\Delta T)
\]
and the (implicit) Crank-Nicolson algorithm is:
\[
\frac{T^{n+1}-T^n}{\delta t} = 
\frac{1}{2}
\left[
{\cal F}^{n}(T,\vec\nabla T,\Delta T)
+
{\cal F}^{n+1}(T,\vec\nabla T,\Delta T)
\right]
\]
where the superscript $n$ indicates the time step.
The Crank-Nicolson is obviously based on the trapezoidal rule, with second-order convergence in time.


In what follows, I omit the superscript on the mass matrix to simplify notations: ${\bm M}^\uptheta={\bm M}$.
In terms of Finite Elements, these become:
\begin{itemize}
\item Explicit Forward euler:
\[
\frac{1}{\delta t} ({\bm M}^{n+1} \cdot \vec T^{n+1}  -{\bm M}^n \cdot \vec T^{n} )
=
-({\bm K}_a^n+{\bm K}^n_d) \cdot \vec T^{n}
\]
or, 
\[
\boxed{
{\bm M}^{n+1} \cdot \vec T^{n+1}
= \left(  {\bm M}^n  - ({\bm K}_a^n+{\bm K}_d^n) \delta t \right)\cdot \vec T^{n} 
}
\]

\item Implicit Backward euler:
\[
\frac{1}{\delta t} ({\bm M}^{n+1} \cdot \vec T^{n+1}  -{\bm M}^n \cdot \vec T^{n} )
= -({\bm K}_a^{n+1}+{\bm K}_d^{n+1}) \cdot \vec T^{n+1}
\]
or, 
\begin{equation}
\boxed{
\left( {\bm M}^{n+1} +({\bm K}_a^{n+1}+{\bm K}_d^{n+1})\delta t \right) \cdot \vec T^{n+1}
=
{\bm M}^n \cdot \vec T^{n} 
}
\label{eq:hte_ibe}
\end{equation}

\item Crank-Nicolson

\[
\frac{1}{\delta t} \left({\bm M}^{n+1} \cdot \vec T^{n+1}  -{\bm M}^n \cdot \vec T^{n} \right)
= 
\frac{1}{2}
\left[
-({\bm K}_a^{n+1}+{\bm K}_d^{n+1}) \cdot \vec T^{n+1}
-({\bm K}_a^{n}+{\bm K}_d^{n}) \cdot \vec T^{n}
\right]
\]
or,
\[
\boxed{
\left( {\bm M}^{n+1} +({\bm K}_a^{n+1}+{\bm K}_d^{n+1})\frac{\delta t}{2} \right) \cdot \vec T^{n+1}
= \left(  {\bm M}^n  - ({\bm K}_a^n+{\bm K}_d^n) \frac{\delta t}{2} \right)\cdot \vec T^{n} 
}
\]

Note that in benchmarks where the domain/grid does not deform, the coefficients do not change in space
and the velocity field is constant in time, or in practice out of convenience, the ${\bm K}$  and ${\bm M}$ 
matrices do not change and the r.h.s. can be constructed with the same matrices as the FE matrix.

\end{itemize}




\index{general}{BDF-2}
\paragraph{The Backward differentiation formula} (see for instance Hairer \& Wanner \cite{hawa91} 
or Wikipedia\footnote{\url{https://en.wikipedia.org/wiki/Backward_differentiation_formula}}. 
See also step-31 of deal.II\footnote{\url{https://www.dealii.org/current/doxygen/deal.II/step_31.html}}.
The second-order BDF (or BDF-2) as shown in Kronbichler \etal (2012) \cite{krhb12} 
is as follows: it is a finite-difference 
quadratic interpolation approximation of the $\partial T/\partial t$ term which involves
$t^n$, $t^{n-1}$ and $t^{n-2}$:
\begin{equation}
\frac{\partial T}{\partial t}(t^n) \simeq
\frac{1}{\delta t_n} \left( \frac{2\delta t_n + \delta t_{n-1}}{\delta t_n+\delta t_{n-1} } T^n  
- \frac{\delta t_n +\delta t_{n-1}}{\delta t_{n-1}} T^{n-1}
+ \frac{\delta t_n^2}{\delta t_{n-1}(\delta t_n+\delta t_{n-1})} T^{n-2}
\right)
\end{equation}
where $\delta t_n=t^n-t^{n-1}$. We also then have the approximation
\[
T^n  
\simeq T^{n-1} + \delta t_n \frac{\partial T}{\partial t}
\simeq T^{n-1} + \delta t_n \frac{T^{n-1}-T^{n-2}}{\delta t_{n-1}} 
= \left(1 + \frac{\delta t_n}{\delta t_{n-1}} \right) T^{n-1} + \frac{\delta t_n}{\delta t_{n-1}} T^{n-2}
\]

Starting again from 
${\bm M}^\uptheta \cdot \dot{\vec T} + ({\bm K}_a + {\bm K}_d) \cdot \vec T = \vec 0$,
we write 
\[
{\bm M}^\uptheta \cdot 
\frac{1}{\delta t_n} \left( \frac{2\delta t_n + \delta t_{n-1}}{\delta t_n+\delta t_{n-1} } \vec T^n  
- \frac{\delta t_n +\delta t_{n-1}}{\delta t_{n-1}} \vec T^{n-1}
+ \frac{\delta t_n^2}{\delta t_{n-1}(\delta t_n+\delta t_{n-1})} \vec T^{n-2} \right)
+ ({\bm K}_a + {\bm K}_d) \cdot \vec T^n = \vec 0
\]
and finally:
\begin{equation}
\left[ \frac{2\delta t_n + \delta t_{n-1}}{\delta t_n+\delta t_{n-1} }
{\bm M}^\uptheta + \delta t_n({\bm K}_a + {\bm K}_d) \right]
 \cdot \vec T^n =
 \frac{\delta t_n +\delta t_{n-1}}{\delta t_{n-1}} {\bm M}^\uptheta \cdot \vec T^{n-1}
- \frac{\delta t_n^2}{\delta_{n-1}(\delta t_n+\delta t_{n-1})} {\bm M}^\uptheta \cdot \vec T^{n-2}
\label{eq:bdf2}
\end{equation}
For practical reasons one may wish to bring the advection term to the rhs (i.e. fully implicit)
so that the matrix is symmetric. 
In this case the equation becomes 
\[
\left[
\frac{2\delta t_n + \delta t_{n-1}}{\delta t_n+\delta t_{n-1} }
{\bm M}^\uptheta
+ \delta t_n {\bm K}_d
\right]
 \cdot \vec T^n =
 \frac{\delta t_n +\delta t_{n-1}}{\delta t_{n-1}} {\bm M}^\uptheta \cdot \vec T^{n-1}
- \frac{\delta t_n^2}{\delta_{n-1}(\delta t_n+\delta t_{n-1})} {\bm M}^\uptheta \cdot \vec T^{n-2}
- \delta t_n {\bm K}_a^\star \cdot \vec T^{n,\star} 
\]
with 
\[
(\cdot)^\star = \left(1 + \frac{\delta t_n}{\delta t_{n-1}} \right) (\cdot)^{n-1} 
+ \frac{\delta t_n}{\delta t_{n-1}} (\cdot )^{n-2}
\]
which denotes the extrapolation of a quantity to time $n$.
Be aware that the ${\bm K}_a^\star$ matrix contains the velocity $\vec\upnu^\star$.

Note that if all timesteps are equal, i.e. $\delta t_n=\delta t_{n-1}=\delta t$, Eq.~\eqref{eq:bdf2} becomes:
\[
\left[
\frac{3}{2}
{\bm M}^\uptheta
+ \delta t({\bm K}_a + {\bm K}_d)
\right]
 \cdot \vec T^n =
{\bm M}^\uptheta \cdot \left(2 \vec T^{n-1} - \frac{1}{2} \vec T^{n-2} \right)
\]
or, 
\[
\left[
{\bm M}^\uptheta
+ \frac{2}{3}\delta t({\bm K}_a + {\bm K}_d)
\right]
 \cdot \vec T^n =
{\bm M}^\uptheta \cdot \left( \frac{4}{3} \vec T^{n-1} - \frac{1}{3} \vec T^{n-2} \right)
\]

When the timestep $\delta t$ is kept constant (which may be a bad idea with regards to the CFL condition),
the backward differenciation formula family of implicit methods
for the integration of ODEs are simplified. 
The BDF-1 is simply the backward Euler method as seen above:
\[
T^{n+1}-T^n=\delta t\;  {\cal F}^{n+1}
\]
The BDF-2 is given by 
\[
T^{n+2} - \frac{4}{3}T^{n+1} +\frac{1}{3} T^n = \frac{2}{3} \delta t\;  {\cal F}^{n+2}
\]
The BDF-3 is given by 
\[
T^{n+3} - \frac{18}{11}T^{n+2} +\frac{9}{11} T^{n+1} -\frac{2}{11}T^n = \frac{6}{11} \delta t \; {\cal F}^{n+3}
\]
The BDF-4 is given by 
\[
T^{n+4}-\frac{48}{25}T^{n+1}+\frac{36}{25}T^{n+1}-\frac{16}{25}T^{n+1}+\frac{3}{25}T^n 
= \frac{12}{25}\delta t\;  {\cal F}^{n+4}
\]
Each BDF-$s$ method achieves order $s$.





