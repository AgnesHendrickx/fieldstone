\index{general}{Angular Velocity} 
\index{general}{Angular Momentum} 
\index{general}{Moment of Inertia}

When free slip boundary conditions are prescribed in an annulus or
hollow sphere geometry there exists a rotational nullspace, or in other words there exists
a tangential velocity field ('pure rotation') which, 
if added or subtracted to the solution, genrates a solution which is still the solution of the PDEs. 

As in the pressure normalisation case (see section \ref{ss_pnorm}), the solution is simple:
\begin{enumerate}
\item fix the tangential velocity at {\it one} node on a boundary, and solve the sytem (the nullspace 
has been removed)
\item post-process the solution to have the velocity field fulfill the required conditions, i.e.
either a zero net angular momentum or a zero net angular velocity of the domain. 
\end{enumerate}

\begin{remark}
In \aspect{} this is available under the option 
"Remove nullspace = angular momentum" and "Remove nullspace = net rotation".
The "angular momentum" option removes a rotation such that the net angular momentum is zero.
The "net rotation" option removes the net rotation of the domain.
\end{remark}

%____________________________________
\paragraph{Angular momentum approach}

In order to remove the angular momentum, we search for a rotation
vector ${\vec \omega}$ such that
\begin{equation}
\int_\Omega \rho[{\vec r} \times ({\vec v}-{\vec \omega} \times {\vec r})] \; dV= \vec 0
\end{equation}

The angular momentum of a rigid body can be obtained from the sum 
of the angular momentums of the particles forming the 
body\footnote{\url{http://www.kwon3d.com/theory/moi/iten.html}}:
\begin{eqnarray}
\vec H 
&=& \sum_i \vec L_i\\
&=& \sum_i \vec r_i \times m_i \vec v_i\\
&=& \sum_i \vec r_i \times m_i (\vec \omega_i \times \vec r_i)\\
&=& \sum_i m_i 
\left(
\begin{array}{ccc}
\sum_i m_i(y_i^2+z_i^2) & -\sum_i m_i x_iy_i & -\sum_i m_i x_i z_i \\
-\sum_i m_i x_iy_i & \sum_i m_i(x_i^2+z_i^2) & -\sum_i m_i y_i z_i \\
-\sum_i m_i x_i z_i & -\sum_i m_i y_i z_i & \sum_i m_i(x_i^2+y_i^2)
\end{array}
\right)
\cdot
\left(
\begin{array}{c}
\omega_x \\ \omega_y \\ \omega_z
\end{array}
\right)
\end{eqnarray}
In the continuum limit, we have:
\begin{equation}
{\vec H} = \int_\Omega \rho(\vec r) \, {\vec r} \times {\vec v}\; dV
\end{equation}
and the $3\times3$ moment of inertia tensor $\bm I$
(also called inertia tensor) is given by\footnote{\url{https://en.wikipedia.org/wiki/Moment\_of\_inertia}}
\begin{equation}
{\bm I}= 
\int_\Omega \rho(\vec r) [\vec r\cdot\vec r \; \bm 1 - \vec r \times \vec r  ] dV
\end{equation}
so that the above equation writes:
$
{\vec H}={\bm I}\cdot {\vec \omega}
$
and then ${\vec \omega}={\bm I}^{-1} \cdot {\vec H}$.

Ultimately, at each velocity node a rotation about the rotation 
vector ${\vec \omega}$ is then subtracted from the velocity 
solution \cite[eq. 26]{zhmt08}:
\begin{equation}
\vec v_{new} = \vec v_{old} - \vec \omega \times \vec r 
\end{equation}


%____________________________________
%\paragraph{Angular velocity approach}

%The angular velocity\footnote{\url{https://en.wikipedia.org/wiki/Angular_velocity }}
% vector is given by $\vec\omega = \frac{\vec r\times \vec v}{r^2}$
%so that the volume-averaged angular velocity of the cylindrical shell is:
%\begin{equation}
%\vec {\omega} = \frac{1}{|\Omega|} \int_\Omega \frac{{\vec r}\times {\vec v}}{r^2} dV
%\end{equation}


%...............................
\subsubsection{Three dimensions}

The angular momentum vector is given by:
\begin{equation}
\vec H 
= \int_\Omega \rho(\vec r) \left( 
\begin{array}{c} 
yw-zv \\ zu-xw \\ xv-yu 
\end{array} \right) d\vec r
= 
\left(\begin{array}{c} 
\int_\Omega \rho(\vec r) (yw-zv) d\vec r\\
\int_\Omega \rho(\vec r) (zu-xw) d\vec r\\
\int_\Omega \rho(\vec r) (xv-yu) d\vec r
\end{array} \right)
= 
\left( 
\begin{array}{c} 
H_x \\ H_y \\ H_z
\end{array} \right)
\end{equation}
while the inertia tensor for a continuous body is given 
by
\begin{eqnarray}
\bm I
&=&\int_\Omega \rho(\vec r) [\vec r\cdot\vec r \; \bm 1 - \vec r \times \vec r  ] d\vec r \\
&=&\int_\Omega \rho(\vec r) 
\left[
\left(
\begin{array}{ccc}
x^2+y^2+z^2 & 0 & 0 \\
0 & x^2+y^2+z^2 & 0 \\
0 & 0 & x^2+y^2+z^2
\end{array}
\right)
- 
\left(
\begin{array}{ccc}
xx & xy & xz \\
yx & yy & yz \\
zx & zy & zz 
\end{array}
\right)
\right] 
d\vec r \\
&=&\int_\Omega \rho(\vec r) 
\left(
\begin{array}{ccc}
y^2+z^2 & -xy & -xz \\
-yx & x^2+z^2 & -yz \\
-zx & -zy & x^2+y^2 
\end{array}
\right)
d\vec r \\
&=&
\left(
\begin{array}{ccc}
\int_\Omega \rho(\vec r) (y^2+z^2) d\vec r & 
-\int_\Omega \rho(\vec r) xy d\vec r & 
-\int_\Omega \rho(\vec r) xz d\vec r \\\\
-\int_\Omega \rho(\vec r) yx d\vec r & 
\int_\Omega \rho(\vec r) (x^2+z^2) d\vec r & 
-\int_\Omega \rho(\vec r) yz d\vec r \\\\
-\int_\Omega \rho(\vec r) zx d\vec r & 
-\int_\Omega \rho(\vec r) zy d\vec r & 
\int_\Omega \rho(\vec r) (x^2+y^2) d\vec r 
\end{array}
\right)\\
&=&
\left(
\begin{array}{ccc}
I_{xx} & I_{xy} & I_{xz} \\
I_{yx} & I_{yy} & I_{yz} \\
I_{zx} & I_{zy} & I_{zz} 
\end{array}
\right)
\end{eqnarray}


%-----------------------------
\subsubsection{Two dimensions}

In two dimensions, flow is taking place in the $(x,y)$ plane. 
This means that $\vec r=(x,y,0)$ and $\vec v=(u,v,0)$ are coplanar, 
and therefore that $\vec \omega$ is perpendicular to the plane.
We have then
\begin{equation}
\vec H = \int_\Omega \rho(\vec r) \left( 
\begin{array}{c} 
0 \\ 0 \\ xv-yu 
\end{array} \right) d\vec r
= 
\left(\begin{array}{c} 
0 \\ 0 \\
\int_\Omega \rho(\vec r) (xv-yu) d\vec r
\end{array} \right)
\end{equation}
and 
\begin{equation}
\bm I
=
\left(
\begin{array}{ccc}
I_{xx} & I_{xy} & I_{xz} \\
I_{yx} & I_{yy} & I_{yz} \\
I_{zx} & I_{zy} & I_{zz} 
\end{array}
\right)
=
\left(
\begin{array}{ccc}
I_{xx} & I_{xy} & 0 \\
I_{yx} & I_{yy} & 0 \\
0      & 0      & I_{zz} 
\end{array}
\right)
\end{equation}
since $I_{xz}=I_{yz}=0$ as $z=0$, and with 
$I_{xx}=\int_\Omega \rho(\vec r) y^2 d\vec r$ and 
$I_{yy}=\int_\Omega \rho(\vec r) x^2 d\vec r$.
The solution to ${\bm I}\cdot \vec \omega = \vec H$ can be easily obtained 
(see Appendix \ref{sec:inv3x3}):
\begin{eqnarray}
\omega_x
&=&
\frac{1}{det(\bm I)}
\left| 
\begin{array}{ccc}
0 & I_{xy} & 0 \\
0 & I_{yy} & 0 \\
H_3 & 0 & I_{zz} 
\end{array}
\right| = 0 \\ \nonumber\\
\omega_y
&=&
\frac{1}{det(\bm I)}
\left| 
\begin{array}{ccc}
I_{xx} & 0 & 0 \\
I_{yx} & 0 & 0 \\
0 & H_z & I_{zz} 
\end{array}
\right| = 0 \\ \nonumber\\ 
\omega_z
&=&
\frac{1}{det(\bm I)}
\left| 
\begin{array}{ccc}
I_{xx} & I_{xy} & 0\\
I_{yx} & I_{yy} & 0\\
0 & 0 & H_z
\end{array}
\right| \\
&=& \frac{1}{det(\bm I)} \left( I_{xx}I_{yy}H_z - I_{yx}I_{xy}H_z \right) \\
&=& \frac{1}{det(\bm I)} \left( I_{xx}I_{yy} - I_{yx}I_{xy} \right) H_z 
\end{eqnarray}
with $det(\bm I)=I_{xx}I_{yy}I_{zz}-I_{yx}I_{xy}I_{zz}=(I_{xx}I_{yy}-I_{yx}I_{xy})I_{zz}$ and then
\[
\omega_z
=\frac{ ( I_{xx}I_{yy} - I_{yx}I_{xy} ) H_z}{(I_{xx}I_{yy}-I_{yx}I_{xy})I_{zz}}
=\frac{ H_z}{I_{zz}}
=\frac{ \int_\Omega \rho(\vec r) (xv-yu) d\vec r }{ \int_\Omega \rho(\vec r) (x^2+y^2) d\vec r  }
\]

Concretely, this means that in 2D one does not need to solve the system ${\bm I}\cdot \vec \omega = \vec H$
since only $\omega_z$ is not zero.

%Likewise, the volume-averaged angular velocity is then simply:
%\begin{equation}
%\omega_z = \frac{1}{|\Omega|}\int_\Omega \frac{xv-yu}{r^2}d\vec r
%\end{equation}
Then, since $\vec{r}=(x,y,z)$ and $\vec{\omega}=(0,0,\omega_z)$: 
\begin{equation}
\vec \upnu_{new}(\vec{r}) = \vec \upnu_{old} - \vec \omega \times \vec r 
=\left(
\begin{array}{c}
u_{old} - (-\omega_z y) \\
v_{old} - (\omega_z x)\\
0 \\
\end{array}
\right)
\end{equation}















