\label{sec_mixed}

What follows is formulated in 2D as the extension to 3D is 
rather trivial. Also the flow is assumed to be incompressible, 
isoviscous, isothermal. 

The methodology to derive the discretised equations of the mixed system is 
quite similar to the one we have used in the case of the penalty formulation.
The big difference comes from the fact that we are now solving for both 
velocity and pressure at the same time, and that we therefore must solve 
the mass and momentum conservation equations together.
As before, velocity inside an element is given by 
\begin{equation}
{\vec \upnu}({\vec r})=\sum_{i=1}^{m_v} N_i^\upnu({\vec r})\;  {\vec \upnu}_i
\label{mixed01}
\end{equation}
where $N_i^{v}$ are the polynomial basis functions for the velocity,
and the summation runs over the $m_v$ nodes composing the element.
A similar expression is used for pressure:
\begin{equation}
p({\vec r})=\sum_{i=1}^{m_p} N_i^p({\vec r}) \; p_i
\label{mixed02}
\end{equation}
Note that the velocity is a vector of size while pressure (and temperature)
is a scalar. There are then $ndof_v$ velocity degrees of freedom per node
and $ndof_p$ pressure degrees of freedom.
It is also very important to remember that the numbers of 
velocity nodes and pressure nodes for a given element 
are more often than not different and that velocity and pressure
nodes need not be colocated. Indeed, unless 
co-called 'stabilised elements' are used, we have $m_v>m_p$, which 
means that the polynomial order of the velocity field is higher than 
the polynomial order of the pressure field (usually by value 1).

\todo[inline]{insert here link(s) to manual and literature} 

Other notations are sometimes used for Eqs.(\ref{mixed01},\ref{mixed02}):
\begin{equation}
u({\vec r}) = \vec{N}^\upnu \cdot \vec{u}
\quad\quad\quad\quad
v({\vec r}) = \vec{N}^\upnu \cdot \vec{v}
\quad\quad\quad\quad
p({\vec r}) = \vec{N}^p \cdot \vec{p}
\end{equation} 
where ${\vec \upnu}=(u,v)$ and $\vec{N}^\upnu$ is the vector containing all basis functions evaluated at location ${\vec r}$:
\begin{eqnarray}
\vec{N}^v &=& \left( N_1^\upnu({\vec r}),  N_2^\upnu({\vec r}),  N_3^\upnu({\vec r}), \dots  N_{m_v}^\upnu({\vec r}) \right) \\
\vec{N}^p &=& \left( N_1^p({\vec r}),  N_2^p({\vec r}),  N_3^p({\vec r}), \dots  N_{m_p}^p({\vec r}) \right)
\end{eqnarray}
and with 
\begin{eqnarray}
\vec{u} &=& \left( u_1,  u_2,  u_3, \dots  u_{m_v} \right) \\
\vec{v} &=& \left( v_1,  v_2,  v_3, \dots  v_{m_v} \right) \\
\vec{p} &=& \left( p_1,  p_2,  p_3, \dots  p_{m_p} \right) 
\end{eqnarray}
We will now establish the weak form of the momentum conservation equation. 
We start again from 
\begin{eqnarray}
{\vec \nabla}\cdot {\bm \sigma} + {\vec b} &=& {\vec 0} \\
{\vec \nabla}\cdot {\vec v} &=& 0
\end{eqnarray}
For the $N_i^\upnu$'s and $N_i^p$ 'regular enough', we can write:
\begin{eqnarray}
\int_{\Omega_e} N_i^\upnu {\vec \nabla}\cdot {\bm \sigma} d\Omega + \int_{\Omega_e} N_i^\upnu  {\vec b} \; d\Omega 
&=& \vec 0 \\
\int_{\Omega_e} N_i^p {\vec \nabla}\cdot {\vec v} d\Omega &=& 0
\end{eqnarray}
We can integrate by parts and drop the surface term\footnote{We will come back to this at a later stage}:
\begin{eqnarray}
\int_{\Omega_e} {\vec \nabla } N_i^\upnu \cdot {\bm \sigma} d\Omega &=& \int_{\Omega_e} N_i^\upnu  {\vec b} d\Omega \\
\int_{\Omega_e} N_i^p {\vec \nabla}\cdot {\vec v} d\Omega &=& 0
\end{eqnarray}
or, 
\begin{equation}
\int_{\Omega_e} 
\left(
\begin{array}{ccc}
\frac{\partial N_i^\upnu}{\partial x} & 0 & \frac{\partial N_i^\upnu}{\partial y} \\  \\
0 & \frac{\partial N_i^\upnu}{\partial y} &  \frac{\partial N_i^\upnu}{\partial x}  
\end{array}
\right)
\cdot
\left(
\begin{array}{c}
\sigma_{xx}\\
\sigma_{yy}\\
\sigma_{xy}\\
\end{array}
\right)
d\Omega = \int_{\Omega_e} N_i^\upnu {\vec b} d\Omega 
\end{equation}
As before (see section XXX) the above equation can ultimately be written:
\begin{equation}
\int_{\Omega_e} {\bm B}^T \cdot 
\left(
\begin{array}{c}
\sigma_{xx}\\
\sigma_{yy}\\
\sigma_{xy}\\
\end{array}
\right)
d\Omega
=
\int_{\Omega_e} {\vec N}_b d\Omega 
\end{equation}
We have previously established that the strain rate 
vector $\vec{\dot \varepsilon}$ is:
\begin{equation}
\vec{\dot\varepsilon}=
\left(
\begin{array}{c}
\frac{\partial u}{\partial x} \\ \\
\frac{\partial v}{\partial y} \\ \\
\frac{\partial u}{\partial y} + \frac{\partial v}{\partial x} \\
\end{array}
\right)
=
\underbrace{
\left(
\begin{array}{ccccccccccc}
\frac{\partial N_1^\upnu}{\partial x} & 0 & 
\frac{\partial N_2^\upnu}{\partial x} & 0 & 
\frac{\partial N_3^\upnu}{\partial x} & 0 & \dots & 
\frac{\partial N_{m_v}^\upnu}{\partial x} & 0
\\  \\
0 & \frac{\partial N_1^\upnu}{\partial y} & 
0 & \frac{\partial N_2^\upnu}{\partial y} &
0 & \frac{\partial N_3^\upnu}{\partial y} & \dots & 
0 & \frac{\partial N_{m_v}^\upnu}{\partial x} 
\\ \\
\frac{\partial N_1^\upnu}{\partial y} &  \frac{\partial N_1^\upnu}{\partial x} &  
\frac{\partial N_2^\upnu}{\partial y} &  \frac{\partial N_2^\upnu}{\partial x} & 
\frac{\partial N_3^\upnu}{\partial y} &  \frac{\partial N_3^\upnu}{\partial x} &   \dots &  
\frac{\partial N_{m_v}^\upnu}{\partial y} &  \frac{\partial N_{m_v}^\upnu}{\partial x}  
\end{array}
\right) 
}_{\bm B}
\cdot
\underbrace{
\left(
\begin{array}{c}
u_1 \\ v_1 \\ u_2 \\ v_2 \\ u_3 \\ v_3 \\ \dots \\ u_{m_v} \\ v_{m_v}
\end{array}
\right)
}_{\vec V}
\end{equation}
or, $\vec{\dot \varepsilon}={\bm B}\cdot {\vec V}$ where ${\bm B}$ is the gradient 
matrix and ${\vec V}$ is the vector of all vector degrees of freedom for the 
element. The matrix ${\bm B}$ is then of size $3\times m_v$ and the vector
${\vec V}$ is $m_v$ long.
we have 
\begin{eqnarray}
\sigma_{xx}&=&-p + 2\eta \dot\varepsilon_{xx} \\
\sigma_{yy}&=&-p + 2\eta \dot\varepsilon_{yy} \\
\sigma_{xy}&=& \hspace{5.5mm} + 2\eta \dot\varepsilon_{xy} 
\end{eqnarray}
so
\begin{equation}
\vec{\sigma} 
=-\left( 
\begin{array}{c}
1 \\ 1 \\ 0 
\end{array}
\right) p+ 2 \eta \vec{\dot\varepsilon}
=
- \left(
\begin{array}{c}
1 \\ 1 \\ 0 
\end{array}
\right)
\vec{N^p} \cdot {\vec P}  + 
{\bm C} \cdot  {\bm B}\cdot {\vec V}
\end{equation}
Let us define matrix ${\bm N}^p$ of size $3\times m_p$:
\begin{equation}
{\bm N}^p=
\left(
\begin{array}{c}
1 \\ 1 \\ 0
\end{array}
\right)
\vec{N^p} 
=
\left(
\begin{array}{c}
\vec{N^p} \\
\vec{N^p} \\
0
\end{array}
\right)
\end{equation}
so that
\begin{equation}
\vec{\sigma} 
= - {\bm N}^p
 \cdot {\vec P}  + 
{\bm C} \cdot  {\bm B}\cdot {\vec V}
\end{equation}
finally
\begin{equation}
\int_{\Omega_e} {\bm B}^T \cdot 
[
- {\bm N}^p  \cdot {\vec P}  + {\bm C} \cdot  {\bm B}\cdot {\vec V}
]
d\Omega
=
\int_{\Omega_e} {\bm N}_b d\Omega 
\end{equation}
or,
\begin{equation}
\underbrace{\left(-\int_{\Omega_e} {\bm B}^T \cdot 
{\bm N}^p  
d\Omega \right)}_{\G} \cdot {\vec P} 
+
\underbrace{
\left(
\int_{\Omega_e} {\bm B}^T \cdot 
{\bm C} \cdot  {\bm B}
d\Omega
\right)}_{\K}
\cdot {\vec V}
=
\underbrace{\int_{\Omega_e} {\vec N}_b d\Omega }_{\vec f}
\end{equation}
where the matrix $\K$ is of size $(m_v*ndof_v \times m_v*ndof_v)$, 
and matrix ${\G}$ is of size $(m_v*ndof_v \times m_p*ndof_p)$.
Turning now to the mass conservation equation:
\begin{eqnarray}
0&=&\int_{\Omega_e} \vec{N}^p {\vec \nabla}\cdot {\vec v} \; d\Omega \nonumber\\
&=& \int_{\Omega_e} \vec{N}^p \sum_{i=1}^{m_v} 
\left( \frac{\partial N_i^\upnu}{\partial x} u_i + \frac{\partial N_i^\upnu}{\partial y} v_i \right)  
d\Omega \nonumber\\
&=& 
\int_{\Omega_e} 
\left(
\begin{array}{c}
N_1^p \left(\sum\limits_{i=1}^{m_v} \frac{\partial N_i^\upnu }{\partial x} u_i +
\sum\limits_{i=1}^{m_v} \frac{\partial N_i^\upnu }{\partial x} v_i \right) \\
N_2^p \left(\sum\limits_{i=1}^{m_v} \frac{\partial N_i^\upnu }{\partial x} u_i +
\sum\limits_{i=1}^{m_v} \frac{\partial N_i^\upnu }{\partial x} v_i \right) \\
N_3^p \left(\sum\limits_{i=1}^{m_v} \frac{\partial N_i^\upnu }{\partial x} u_i +
\sum\limits_{i=1}^{m_v} \frac{\partial N_i^\upnu }{\partial x} v_i \right) \\
\dots \\
N_{m_p}^p \left(\sum\limits_{i=1}^{m_v} \frac{\partial N_i^\upnu }{\partial x} u_i +
\sum\limits_{i=1}^{m_v} \frac{\partial N_i^\upnu }{\partial x} v_i \right) \\
\end{array}
\right) d \Omega \nonumber \\  %%%%%%%%%%%%%%%%%%%%%%%%%%
&=& 
\int_{\Omega_e} 
\left(
\begin{array}{ccc}
{N}_1^p & {N}_1^p & 0 \\
{N}_2^p & {N}_2^p & 0 \\
{N}_3^p & {N}_3^p & 0 \\
\dots & \dots & \dots \\
{N}_{m_p}^p & {N}_{m_p}^p & 0 
\end{array}
\right)
\cdot
\left(
\begin{array}{c}
\sum\limits_{i=1}^{m_v} \frac{\partial N_i^\upnu}{\partial x} u_i \\ 
\sum\limits_{i=1}^{m_v} \frac{\partial N_i^\upnu}{\partial x} v_i \\
\sum\limits_{i=1}^{m_v} \frac{\partial N_i^\upnu}{\partial x} v_i +
\sum\limits_{i=1}^{m_v} \frac{\partial N_i^\upnu}{\partial x} u_i 
\end{array}
\right) d\Omega \nonumber\\ %%%%%%%%%%%%%%%%%%%%%%%%%%
&=& 
\int_{\Omega_e} 
\left(
\begin{array}{ccc}
{N}_1^p & {N}_1^p & 0 \\
{N}_2^p & {N}_2^p & 0 \\
{N}_3^p & {N}_3^p & 0 \\
\dots & \dots & \dots \\
{N}_{m_p}^p & {N}_{m_p}^p & 0 
\end{array}
\right)
\cdot
\left(
\begin{array}{ccccccccccc}
\frac{\partial N_1^v}{\partial x} & 0 & 
\frac{\partial N_2^v}{\partial x} & 0 & 
\frac{\partial N_3^v}{\partial x} & 0 & \dots & 
\frac{\partial N_{m_v}^v}{\partial x} & 0
\\  \\
0 & \frac{\partial N_1^v}{\partial y} & 
0 & \frac{\partial N_2^v}{\partial y} &
0 & \frac{\partial N_3^v}{\partial y} & \dots & 
0 & \frac{\partial N_{m_v}^v}{\partial x} 
\\ \\
\frac{\partial N_1^v}{\partial y} &  \frac{\partial N_1^v}{\partial x} &  
\frac{\partial N_2^v}{\partial y} &  \frac{\partial N_2^v}{\partial x} & 
\frac{\partial N_3^v}{\partial y} &  \frac{\partial N_3^v}{\partial x} &   \dots &  
\frac{\partial N_{m_v}^v}{\partial y} &  \frac{\partial N_{m_v}^v}{\partial x}  
\end{array}
\right) 
\cdot
\left(
\begin{array}{c}
u_1 \\ v_1 \\ u_2 \\ v_2 \\ \dots \\ u_{m_v} \\ v_{m_v}
\end{array}
\right)
d\Omega  \nonumber \\
&=& 
\left(\int {\bm N}^p \cdot {\bm B} d\Omega \right) \cdot \vec{V} \nonumber\\
&=& -\G_e^T \cdot {\vec V}
\end{eqnarray}

\todo[inline]{say something about minus sign?}

Ultimately we obtain the following system for each element:
\[
\left(
\begin{array}{cc}
\K_e & \G_e \\
\G_e^T & 0
\end{array}
\right)
\cdot
\left(
\begin{array}{c}
\vec{V} \\ \vec{P} 
\end{array}
\right)
=
\left(
\begin{array}{c}
\vec{f}_e \\ 0 
\end{array}
\right)
\]
Such a matrix is then generated for each element and then must me assembled into the 
global F.E. matrix. 








