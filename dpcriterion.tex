The von Mises yield criterion is not suitable for modelling the yielding of frictional material as it does not include the effect of mean stress as observed in experiments. To overcome this limitation, Drucker and Prager (1952) proposed a revised function for frictional materials.

The Drucker-Prager yield criterion has the function form
\[
f(J_1, J_2) = 0 
\]
This criterion is most often used for concrete where both normal and shear stresses can determine failure. The Drucker-Prager yield criterion may be expressed as

\begin{equation}
F^{DP}=\sqrt{J_2} - (\alpha J_1 + k) \label{dpcriterion} 
\end{equation}

Since the Drucker-Prager yield surface is a smooth version of the Mohr-Coulomb yield surface, it 
is often expressed in terms of the cohesion $c$ and the angle of internal friction $\phi$ that are used 
to describe the Mohr-Coulomb yield surface. One then wishes to relate $\alpha$ and $k$ to $c$ and $\phi$:
$\alpha= fct(\phi)$ and $k= fct(c,\phi) $.


%%%%%%%%%%%%%%%%%%%%%%%%%%%%%%%%%%
\paragraph{two-dimensional space}

By choosing 
\[
k=c \cos \phi \quad\quad\quad \alpha=\frac{1}{2}\sin \phi
\]
we can make the Drucker-Prager yield criterion coincide with the Mohr-Coulomb yield criterion.


%%%%%%%%%%%%%%%%%%%%%%%%%%%%%%%%%%
\paragraph{three-dimensional space}

Let us first assume that the Drucker-Prager yield surface circumscribes the Mohr-Coulomb yield surface such that the two surfaces coincide at $\theta=\tfrac{\pi}{3}$. 
The expression for the Mohr-Coulomb yield criterion is

\[
F^{MC,3D} = -\frac{1}{3}J_1 \sin \phi  + \sqrt{J_2} ( \cos \theta - \frac{1}{\sqrt{3}} \sin \theta  \sin \phi ) - c \cos \phi 
\]
Taking $\theta=\pi/6$ yields:
\begin{eqnarray}
F^{MC,3D} 
&=& -\frac{1}{3}J_1 \sin \phi  + \sqrt{J_2} ( \frac{\sqrt{3}}{2} - \frac{1}{\sqrt{3}} \frac{1}{2}  \sin \phi ) - c \cos \phi  \nn\\
&=& -\frac{1}{3}J_1 \sin \phi  + \frac{1}{2\sqrt{3}}\sqrt{J_2} ( 3 -  \sin \phi ) - c \cos \phi  \nn\\
&=& -\frac{1}{3}J_1 \sin \phi  + \frac{1}{6}\sqrt{J_2} \sqrt{3}( 3 -  \sin \phi ) - c \cos \phi  \nn\\
&=& \frac{\sqrt{3}(3-\sin\phi)}{6} \left( -  \frac{2 \sin \phi}{\sqrt{3}(3-\sin\phi)}   J_1  + \sqrt{J_2}  - \frac{6c \cos \phi}{\sqrt{3}(3-\sin\phi)} \right) \nn
\end{eqnarray}
The constant in front of the brackets (which always strictly positive) does not matter since we look at the sign of $F$.
Comparing with 
\[
F^{DP}=\sqrt{J_2} - (\alpha J_1 + k) \label{dpcriterion} 
\]
we naturally set 
\[
\alpha =\frac{2 \sin \phi}{\sqrt{3}(3-\sin\phi)}
\quad\quad\quad  
k =  \frac{6c \cos \phi}{\sqrt{3}(3-\sin\phi)} 
\]

\begin{mdframed}[backgroundcolor=blue!5]
\[
F^{DP,3D} =  -  \frac{2 \sin \phi}{\sqrt{3}(3-\sin\phi)}   J_1  + \sqrt{J_2}  - \frac{6c \cos \phi}{\sqrt{3}(3-\sin\phi)} 
\]
\end{mdframed}






\vspace{1.3cm}
TO VERIFY
If the Drucker-Prager surface inscribes the Mohr-Coulomb surface, then matching the two surfaces at $\theta=0$  gives
\[
k = \cfrac{6~c~\cos\phi}{\sqrt{3}(3-\sin\phi)} ~;~~ \alpha = \cfrac{2~\sin\phi}{\sqrt{3}(3-\sin\phi)} 
\]
In the principal stress space, the Drucker-Prager surface has the form of a circular cone, whilst the von Mises yield surface is an infinitely long cylinder.

%\begin{center}
%\includegraphics[width=0.6\textwidth]{RHEOLOGY/viscoplasticity/dpcriterion.pdf}
%\end{center}



