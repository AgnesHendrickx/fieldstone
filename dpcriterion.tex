The von Mises yield criterion is not suitable for modelling the yielding of frictional material 
as it does not include the effect of mean stress as observed in experiments. To overcome this 
limitation, Drucker and Prager \cite{drpr52} proposed a revised function for frictional materials.

The Drucker-Prager yield criterion has the function form
\begin{equation}
F^{\text{\tiny DP}}({\bm \sigma})=F \left( {\cal I}_1({\bm \sigma}), {\cal I}_2({\bm \tau}) \right) = 0 
\end{equation}
This criterion is most often used for concrete where both normal and shear stresses 
can determine failure. The Drucker-Prager yield criterion may be expressed as
\begin{equation}
F^{\text{\tiny DP}}= \sqrt{{\cal I}_2({\bm \tau})} + \alpha {\cal I}_1({\bm \sigma}) + k =0  \label{dpcriterion} 
\end{equation}
Using the parameters $\sigma_m$, $\tau_m$, $a=-\sqrt{3}\tan\theta$, ${\cal I}_1({\bm \sigma})$ 
and ${\cal I}_2({\bm \tau})$ of Section~\ref{sec:altinv} we have
\begin{eqnarray}
F^{\text{\tiny DP}}
&=&  \sqrt{{\cal I}_2({\bm \tau})} + \alpha {\cal I}_1({\bm \sigma}) + k \nn\\
&=& \sqrt{\frac{\tau_m^2}{3}(a^2+3)} + \alpha (3\sigma_m-a\tau_m) + k \nn\\ 
&=& \tau_m \sqrt{(a^2/3+1)} + \alpha (3\sigma_m+\tau_m\sqrt{3}\tan\theta ) + k    \qquad ({\rm since }\; \tau_m>0)\nn\\ 
&=& \tau_m \sqrt{\tan^2\theta+1} + \alpha (3\sigma_m+\tau_m\sqrt{3}\tan\theta ) + k  \nn\\
&=& \tau_m \sqrt{ \frac{1}{\cos^2\theta} } + \alpha (3\sigma_m+\tau_m\sqrt{3}\tan\theta ) + k  \nn\\
&=& \tau_m \frac{1}{\cos\theta} +\alpha (3\sigma_m+\tau_m\sqrt{3}\tan\theta ) + k  \qquad ({\rm since }\; \cos\theta >0)\nn
\end{eqnarray}
$F=0$ then leads to write
\begin{eqnarray}
\tau_m  + (3 \alpha \sigma_m+k)\cos\theta  + \tau_m \alpha \sqrt{3}\sin\theta  &=&0 \nn\\
\Rightarrow \qquad \tau_m(1 + \alpha \sqrt{3}\sin\theta)  + (3 \alpha \sigma_m+k)\cos\theta &=&0 \nn
\end{eqnarray}
and finally
\[
\tau_m = -\frac{(3 \alpha \sigma_m+k)\cos\theta}{1 + \alpha \sqrt{3}\sin\theta}
= -\frac{3 \alpha \cos\theta}{1 + \alpha \sqrt{3}\sin\theta} \sigma_m 
-\frac{k\cos\theta}{1 + \alpha \sqrt{3}\sin\theta}
\]
\begin{remark}
This is the same equation as Eq. 19 of Wojciechowski \cite{wojc18} but with $\theta \rightarrow -\theta$. 
\end{remark}

\vspace{.5cm}

The Mohr-Coulomb yield criterion writes  (see Eq.~(\ref{eq:mccrit}))
\[
\tau_m = -\sigma_m \sin\phi + c \cos\phi
\]
so that equating both expressions of $\tau_m$ for the Drucker-Prager 
and Mohr-Coulomb criteria leads to:
\begin{eqnarray}
-\frac{3 \alpha \cos\theta}{1 + \alpha \sqrt{3}\sin\theta} &=& -\sin\phi \label{eq:qq1}\\
-\frac{k\cos\theta}{1 + \alpha \sqrt{3}\sin\theta} &=& c \cos\phi \label{eq:qq2}
\end{eqnarray}
Eq.~(\ref{eq:qq1}) yields
\[
3 \alpha \cos\theta = \sin\phi (1 + \alpha \sqrt{3}\sin\theta) 
\]
\[
\Rightarrow \qquad 3 \alpha \cos\theta - \alpha \sqrt{3}\sin\theta \sin\phi = \sin\phi 
\]
and finally 
\[
\boxed{
\alpha(\phi) =  \frac{\sin\phi}{ 3 \cos\theta - \sqrt{3}\sin\theta \sin\phi}
}
\]
Inserting this into Eq.~(\ref{eq:qq2}):
\begin{eqnarray}
- k \cos\theta 
&=& c \cos \phi \left(1 +\alpha \sqrt{3} \sin\theta \right)  \nn\\
&=& c \cos \phi \left(1 + \frac{\sin\phi}{ 3 \cos\theta - \sqrt{3}\sin\theta \sin\phi}  \sqrt{3} \sin\theta\right) \nn\\
&=& c \cos \phi \left(1 + 
\frac{ \sqrt{3}\sin\phi \sin\theta }{ 3 \cos\theta - \sqrt{3}\sin\theta \sin\phi} \right) \nn\\
&=& c \cos \phi \left(
\frac{ 3 \cos\theta - \sqrt{3}\sin\theta \sin\phi}{ 3 \cos\theta - \sqrt{3}\sin\theta \sin\phi} 
+ 
\frac{ \sqrt{3}\sin\phi \sin\theta }{ 3 \cos\theta - \sqrt{3}\sin\theta \sin\phi} \right) \nn\\
&=& c \cos \phi \left(
\frac{ 3 \cos\theta}{ 3 \cos\theta - \sqrt{3}\sin\theta \sin\phi} \right) \nn
\end{eqnarray}
so that 
\[
\boxed{
k(c,\phi) =- \frac{ 3\; c \cos \phi }{ 3 \cos\theta - \sqrt{3}\sin\theta \sin\phi} 
}
\]
%Unsurprisingly we recover the Eqs. 20 and 21 of Wojciechowski \cite{wojc18} by replacing $\theta$ by $-\theta$.

The Drucker-Prager yield criterion which for a given $\theta$ is equal to the Mohr-Coulomb yield is then:
\begin{eqnarray}
F^{\text{\tiny DP}}
&=& \sqrt{{\cal I}_2({\bm \tau})} + \alpha(\phi) {\cal I}_1({\bm \sigma}) + k(c,\phi)  \nn\\
&=& \sqrt{{\cal I}_2({\bm \tau})} 
+ \frac{\sin\phi}{ 3 \cos\theta - \sqrt{3}\sin\theta \sin\phi}  {\cal I}_1({\bm \sigma})  
- \frac{ 3\; c \cos \phi }{ 3 \cos\theta - \sqrt{3}\sin\theta \sin\phi} \nn\\
&=& \sqrt{{\cal I}_2({\bm \tau})} 
- \left[ -\frac{3 \sin\phi}{ 3 \cos\theta - \sqrt{3}\sin\theta \sin\phi}  \frac{{\cal I}_1({\bm \sigma})}{3}
+ \frac{ 3\; c \cos \phi }{ 3 \cos\theta - \sqrt{3}\sin\theta \sin\phi} \right] \label{eq:Fdp}\\
&=& \sqrt{{\cal I}_2({\bm \tau})} 
- \left[ \frac{3\; p \sin\phi}{ 3 \cos\theta - \sqrt{3}\sin\theta \sin\phi} 
+ \frac{ 3\; c \cos \phi }{ 3 \cos\theta - \sqrt{3}\sin\theta \sin\phi} \right] \nn\\
&=& \sqrt{{\cal I}_2({\bm \tau})}  
- \frac{3\; p \sin\phi  + 3\; c \cos \phi }{ 3 \cos\theta - \sqrt{3}\sin\theta \sin\phi} \nn\\ 
&=& \sqrt{{\cal I}_2({\bm \tau})}  
- \frac{p \sin\phi  + c \cos \phi }{  \cos\theta - \frac{1}{\sqrt{3}}\sin\theta \sin\phi} 
\end{eqnarray}
which, when multiplied by $\cos\theta - \frac{1}{\sqrt{3}}\sin\theta \sin\phi$, gives
the Mohr-Coulomb criterion of Eq.~(\ref{eq:mcF}). 

For $\theta=\pi/6$, the DP yield surface {\bf circumscribes} the MC yield surface and Eq.~(\ref{eq:Fdp}) writes:
\begin{eqnarray}
F^{\text{\tiny DP}}
&=& \sqrt{{\cal I}_2({\bm \tau})} 
- \left[ -\frac{3 \sin\phi}{ 3 \sqrt{3}/2 - \sqrt{3}/2 \; \sin\phi}  \frac{{\cal I}_1({\bm \sigma})}{3}
+ \frac{ 3\; c \cos \phi }{ 3 \sqrt{3}/2 - \sqrt{3}/2 \; \sin\phi} \right] \nn\\
&=& \sqrt{{\cal I}_2({\bm \tau})} 
- \left[ -\frac{6 \sin\phi}{\sqrt{3} (3 - \sin\phi) }  \frac{{\cal I}_1({\bm \sigma})}{3}
+ \frac{ 6\; c \cos \phi }{ \sqrt{3}(3 - \sin\phi)} \right] \nn\\
&=& \sqrt{{\cal I}_2({\bm \tau})} 
- \frac{ 6 p \sin \phi + 6\; c \cos \phi }{ \sqrt{3}(3 - \sin\phi)} \label{eq:dpc}
\end{eqnarray}
which is the formula used in Glerum et al (2018) \cite{gltf18}.

For $\theta=-\pi/6$, the DP yield surface {\bf middle circumscribes} the MC yield surface and Eq.~(\ref{eq:Fdp}) writes:
\begin{eqnarray}
F^{\text{\tiny DP}}
&=& \sqrt{{\cal I}_2({\bm \tau})} 
- \left[ -\frac{3 \sin\phi}{ 3 \sqrt{3}/2 + \sqrt{3}/2 \sin\phi}  \frac{{\cal I}_1({\bm \sigma})}{3}
+ \frac{ 3\; c \cos \phi }{ 3 \sqrt{3}/2 + \sqrt{3}/2 \sin\phi} \right] \nn\\
&=& \sqrt{{\cal I}_2({\bm \tau})} 
- \left[ -\frac{6 \sin\phi}{\sqrt{3} (3 + \sin\phi) }  \frac{{\cal I}_1({\bm \sigma})}{3}
+ \frac{ 6\; c \cos \phi }{ \sqrt{3}(3 + \sin\phi}) \right] \nn\\
&=& \sqrt{{\cal I}_2({\bm \tau})} 
- \frac{6 p \sin\phi + 6 c \cos \phi}{\sqrt{3} (3 + \sin\phi) } 
\end{eqnarray}

Another DP formulation which {\bf inscribes} the MC yield surface is found on the 
wikipedia page of the Drucker-Prager yield criterion
\footnote{\url{https://en.wikipedia.org/wiki/Drucker-Prager_yield_criterion}}
(but I have no idea how it is arrived at):
\begin{eqnarray}
F^{\text{\tiny DP}}
&=& \sqrt{{\cal I}_2({\bm \tau})} 
- \left[ -\frac{3 \sin\phi}{\sqrt{9+3\sin^2\phi} }  \frac{{\cal I}_1({\bm \sigma})}{3}
+ \frac{ 3\; c \cos \phi }{ \sqrt{9+3\sin^2\phi} } \right] \nn\\
\end{eqnarray}

The yield surfaces of these three Drucker-Prager formulations are plotted against the Mohr-Coulomb
yield surface in Section~\ref{ss:envelope}. 



%%%%%%%%%%%%%%%%%%%%%%%%%%%%%%%%%%
%\paragraph{two-dimensional space}

%By choosing $k=c \cos \phi $ and $\alpha=\frac{1}{2}\sin \phi$ and seeing that that $p=\frac{1}{2} {\cal I}_1({\bm \sigma})$, we can make the Drucker-Prager yield criterion coincide with the Mohr-Coulomb yield criterion
%so that 
%\begin{equation}
%F^{DP,2D} = \tau_e - (p \sin \phi + c \; \cos \phi) 
%\end{equation}


%%%%%%%%%%%%%%%%%%%%%%%%%%%%%%%%%%
%\paragraph{three-dimensional space}

%Let us first assume that the Drucker-Prager yield surface circumscribes the Mohr-Coulomb yield surface such that the two surfaces coincide at $\theta=\tfrac{\pi}{3}$. 
%The expression for the Mohr-Coulomb yield criterion is

%\[
%F^{MC,3D} = -\frac{1}{3}J_1 \sin \phi  + \sqrt{J_2} ( \cos \theta - \frac{1}{\sqrt{3}} \sin \theta  \sin \phi ) - c \cos \phi 
%\]
%Taking $\theta=\pi/6$ yields:
%\begin{eqnarray}
%F^{MC,3D} 
%&=& -\frac{1}{3}J_1 \sin \phi  + \sqrt{J_2} ( \frac{\sqrt{3}}{2} - \frac{1}{\sqrt{3}} \frac{1}{2}  \sin \phi ) - c \cos \phi  \nn\\
%&=& -\frac{1}{3}J_1 \sin \phi  + \frac{1}{2\sqrt{3}}\sqrt{J_2} ( 3 -  \sin \phi ) - c \cos \phi  \nn\\
%&=& -\frac{1}{3}J_1 \sin \phi  + \frac{1}{6}\sqrt{J_2} \sqrt{3}( 3 -  \sin \phi ) - c \cos \phi  \nn\\
%&=& \frac{\sqrt{3}(3-\sin\phi)}{6} \left( -  \frac{2 \sin \phi}{\sqrt{3}(3-\sin\phi)}   J_1  + \sqrt{J_2}  - \frac{6c \cos \phi}{\sqrt{3}(3-\sin\phi)} \right) \nn
%\end{eqnarray}
%The constant in front of the brackets (which always strictly positive) does not matter since we look at the sign of $F$.
%Comparing with 
%\[
%F^{DP}=\sqrt{J_2} - (\alpha J_1 + k) \label{dpcriterion} 
%\]
%we naturally set 
%\[
%\alpha =\frac{2 \sin \phi}{\sqrt{3}(3-\sin\phi)}
%\quad\quad\quad  
%k =  \frac{6c \cos \phi}{\sqrt{3}(3-\sin\phi)} 
%\]
%and since $p=\frac{1}{3}{\cal I}_1({\bm \sigma})$:
%\begin{mdframed}[backgroundcolor=blue!5]
%\begin{equation}
%F^{DP,3D} = \tau_e  - \left[ \frac{6 \sin \phi}{\sqrt{3}(3-\sin\phi)}   p  + 
%\frac{6c \cos \phi}{\sqrt{3}(3-\sin\phi)}  \right]
%\label{eqdp3D}
%\end{equation}
%\end{mdframed}





%\begin{center}
%\includegraphics[width=0.6\textwidth]{RHEOLOGY/viscoplasticity/dpcriterion.pdf}
%\end{center}


\vspace{1.3cm}
\begin{remark}
Leroy \& Ortiz \cite{leor89} use the Drucker-Prager plasticity model also and match it to the Mohr-Coulomb model in the 
triaxial test and formulate it as follows 
(Their definition of the second invariant of stress contains a 3/2 term):
\begin{eqnarray}
F 
&=& \tau_e \sqrt{3} + \frac{6 \sin\phi}{3-\sin\phi} \left( -p  - \frac{c}{\tan \phi} \right) \nn\\
&=& \tau_e \sqrt{3} - \left( \frac{6 \sin\phi}{3-\sin\phi}  p  + c \frac{6 \cos\phi}{3-\sin\phi} \right) \nn\\
&=& \sqrt{3} \left[ \tau_e  - \left( \frac{6 \sin\phi}{\sqrt{3}(3-\sin\phi)}  p  + c \frac{6 \cos\phi}{\sqrt{3}(3-\sin\phi)} \right)  \right]
\end{eqnarray}
Except for the $\sqrt{3}$ this is identical to Eq.~(\ref{eq:dpc}).
\end{remark}


\newpage
