
The two bubble functions are defined on the reference element $[-1,1]\times [-1,1]$:
\begin{eqnarray}
b^{(1)}(r,s) &=&  (1-r)(1-s)\cdot (1-r^2) (1-s^2) \\  
b^{(2)}(r,s) &=&  \left(1+\frac{r+s}{4}\right) \cdot (1-r^2) (1-s^2) 
\end{eqnarray}
Both bubble functions are exactly one in the middle of the element and exactly zero on the edges
of the element as expected from basis functions.

%In what follows I focus on the two bubble functions $b^{(1)}$ and $b^{(2)}$ in \cite{lami17}.
%When I rewrite these for the reference element $[-1,1]\times[-1,1]$, then $x=(r+1)/2$, $y=(s+1)/2$
%and $1-x=(1-r)/2$ and $1-y=(1-s)/2$.
%\begin{eqnarray}
%b^{(1)}(r,s) 
%&=& 64 \frac{1}{4} (1-r)^2 \frac{1}{4}(1-s)^2 \frac{1}{2} (r+1) \frac{1}{2} (s+1)  \\
%&=&  (1-r)^2 (1-s)^2 (r+1)  (s+1) \\
%&=& (1-r)(1-s) (1-r^2) (1-s^2) \\ \nn\\ 
%b^{(2)}(r,s) 
%&=& 8[1+(r+1)/2+(s+1)/2]\frac{1}{2}(r+1)\frac{1}{2}(s+1)\frac{1}{2}(1-r)\frac{1}{2}(1-s) \\
%&=& \frac{1}{2} [1+(r+1)/2+(s+1)/2] (r+1) (s+1) (1-r) (1-s) \\
%&=& \left(1+\frac{r+s}{4}\right) (1-r^2) (1-s^2) 
%\end{eqnarray}

We then have
\begin{eqnarray}
\frac{\partial b^{(1)}}{\partial r}(r,s) 
&=& (1-s)^2(1+s)[-2(1-r)(1+r)+(1-r)^2]\nn\\
&=& (1-s)^2(1+s)[-2+2r^2 + 1-2r+r^2]\nn\\
&=& (1-s)^2(1+s)[-1-2r+3r^2]\\
\frac{\partial b^{(1)}}{\partial s}(r,s) 
&=& (1-r)^2(1+r)[-1-2s+3s^2 ] \\
\frac{\partial b^{(2)}}{\partial r}(r,s) 
&=& \frac{1}{4} (1-s^2) (1-r^2 + (4+r+s) (-2r)) \nn\\
&=& \frac{1}{4} (1-s^2) (1-8r-3r^2 -2rs) \\
\frac{\partial b^{(2)}}{\partial s}(r,s) 
&=& \frac{1}{4} (1-r^2) (1-s^2 + (4+r+s) (-2s)) \nn\\
&=& \frac{1}{4} (1-r^2) (1-8s-3s^2 -2rs) 
\end{eqnarray}
We postulate that a function $f$ has the following representation 
in the element:
\[
f^h(r,s)=a+br+cs+drs+e \; b(r,s) 
\]
where $b(r,s)$ stands for the bubble function which is of the form $b(r,s)=(1-r^2)(1-s^2)\phi(r,s)$
and $\phi$ is a (bi)-linear function of $r,s$.

We need
\begin{eqnarray}
f^h(r_1,s_1) &=& a-b-c+d  =f_1 \\
f^h(r_2,s_2) &=& a+b-c-d  =f_2 \\
f^h(r_3,s_3) &=& a+b+c+d  =f_3 \\
f^h(r_4,s_4) &=& a-b+c-d  =f_4 \\
f^h(r_5,s_5) &=& a      +e=f_5 
\end{eqnarray}
This can be written as a linear system: 
\[
\left(
\begin{array}{ccccc}
1 &-1 &-1 & 1 &0 \\
1 & 1 &-1 &-1 &0 \\
1 & 1 & 1 & 1 &0 \\
1 &-1 & 1 &-1 &0 \\
1 & 0 & 0 & 0 &1 
\end{array}
\right)
\cdot
\left(
\begin{array}{c}
a \\ b \\ c \\ d \\ e
\end{array}
\right)
=
\left(
\begin{array}{c}
f_1 \\ f_2 \\ f_3 \\ f_4 \\ f_5
\end{array}
\right)
\]
and the solution is then:
\[
\left(
\begin{array}{c}
a \\ b \\ c \\ d \\ e
\end{array}
\right)
=
\frac{1}{4}
\left(
\begin{array}{ccccc}
 1 & 1 &  1 & 1 &0\\
-1 & 1 &  1 &-1 &0\\
-1 &-1 &  1 & 1 &0\\
 1 &-1 &  1 &-1 &0\\
-1 &-1 & -1 &-1 &4
\end{array}
\right)
\cdot
\left(
\begin{array}{c}
f_1 \\ f_2 \\ f_3 \\ f_4 \\ f_5
\end{array}
\right)
\]
or, 
\begin{eqnarray}
a &=& \frac{1}{4}( f_1 + f_2 +f_3 +f_4) \nn\\
b &=& \frac{1}{4}(-f_1 + f_2 +f_3 -f_4) \nn\\
c &=& \frac{1}{4}(-f_1 - f_2 +f_3 +f_4) \nn\\
d &=& \frac{1}{4}( f_1 - f_2 +f_3 -f_4) \nn\\
e &=& \frac{1}{4}(-f_1 - f_2 -f_3 -f_4 + 4f_5) 
\end{eqnarray}
Then 
\begin{eqnarray}
4f^h(r,s)
&=&4 [a+br+cs+drs+e (1-r^2) (1-s^2) \phi(r,s)] \nn\\
&=&  (f_1 + f_2 +f_3 +f_4) \nn\\
&&+ (-f_1 + f_2 +f_3 -f_4)r \nn\\
&&+(-f_1 - f_2 +f_3 +f_4)s \nn\\
&&+ (f_1 - f_2 +f_3 -f_4)rs \nn\\
&&+ (-f_1 - f_2 -f_3 -f_4 + 4f_5) (1-r^2) (1-s^2) \phi(r,s) \nn\\
&=& (1-r-s+rs - b(r,s))f_1 \nn\\
&&+ (1+r-s-rs-b(r,s))f_2 \nn\\
&&+ (1+r+s+rs-b(r,s))f_3 \nn\\
&&+ (1-r+s-rs- b(r,s))f_4 \nn\\
&&+ 4b(r,s) f_5
\end{eqnarray}
or, 
\begin{eqnarray}
f^h(r,s)&=&
\underbrace{\left(\frac{1}{4}(1-r)(1-s)-\frac{1}{4}b(r,s)\right)}_{\bN_1} f_1 + 
\underbrace{\left(\frac{1}{4}(1+r)(1-s)-\frac{1}{4}b(r,s)\right)}_{\bN_2} f_2\\
&+& 
\underbrace{\left(\frac{1}{4}(1+r)(1+s)-\frac{1}{4}b(r,s)\right)}_{\bN_3} f_3 +
\underbrace{\left(\frac{1}{4}(1-r)(1+s)-\frac{1}{4}b(r,s)\right)}_{\bN_4} f_4 \\
&+& \underbrace{b(r,s)}_{\bN_5} f_5
\end{eqnarray}

As in the $P_1^+$ case the resulting basis functions are a combination 
of the regular $Q_1$ basis functions and the bubble.

\begin{itemize}
\item
Zeroth-order consistency check $f(r,s)=C$:
\begin{equation}
f^h(r,s) 
= \sum_{i=1}^5 \bN_i(r,s) f_i \\
= C \sum_{i=1}^5 \bN_i(r,s)  \\
= C
\end{equation}

\item
First-order consistency check $f(r,s)=r$ (or $f(r,s)=s)$:
\begin{eqnarray}
f^h(r,s) 
&=& \sum_{i=1}^5 \bN_i(r,s) f_i \nn\\
&=& \bN_1(r,s) (-1) + \bN_2(r,s) (+1) + \bN_3(r,s) (+1) + \bN_4(r,s) (-1) + \bN_5(r,s) (0) \nn\\
&=& -\bN_1(r,s)+\bN_2(r,s)+\bN_3(r,s)-\bN_4(r,s) \nn\\
&=& r
\end{eqnarray}

\item
Second-order consistency check $f(r,s)=rs$ ($f_1=(-1)(-1)=1$, $f_2=(+1)(-1)=-1$, etc ...)
\begin{eqnarray}
f^h(r,s) 
&=& \sum_{i=1}^5 \bN_i(r,s) f_i \nn\\
&=& \bN_1(r,s) (+1) + \bN_2(r,s) (-1) + \bN_3(r,s) (+1) + \bN_4(r,s) (-1) + \bN_5(r,s) (0) \nn\\
&=& \bN_1-\bN_2+\bN_3-\bN_4 \nn\\
&=& 
\left(\frac{1}{4}(1-r)(1-s)-\frac{1}{4}b(r,s)\right)
-\left(\frac{1}{4}(1+r)(1-s)-\frac{1}{4}b(r,s)\right) \nn\\
&+& 
\left(\frac{1}{4}(1+r)(1+s)-\frac{1}{4}b(r,s)\right)
-\left(\frac{1}{4}(1-r)(1+s)-\frac{1}{4}b(r,s)\right) \nn\\
&=& 
 \frac{1}{4}(1-r)(1-s)
-\frac{1}{4}(1+r)(1-s)
+\frac{1}{4}(1+r)(1+s)
-\frac{1}{4}(1-r)(1+s) \nn\\
&=&
 \frac{1}{2}(-r)(1-s)
+\frac{1}{2}(+r)(1+s) \nn\\
&=& rs
\end{eqnarray}
We find that the basis functions can represent  a bilinear field exactly. 



Consistency check for quadratic terms, i.e. $f(r,s)=r^2$ (or $f(r,s)=s^2$): 
\begin{eqnarray}
f^h(r,s) 
&=& \sum_{i=1}^5 \bN_i(r,s) f_i \nn\\
&=& \bN_1(r,s)\cdot (+1) + \bN_2(r,s)\cdot (+1) + \bN_3(r,s)\cdot (+1) + \bN_4(r,s)\cdot (+1) + \bN_5(r,s)\cdot (0) \nn\\
&=& 
\left(\frac{1}{4}(1-r)(1-s)-\frac{1}{4}b(r,s)\right)
+\left(\frac{1}{4}(1+r)(1-s)-\frac{1}{4}b(r,s)\right) \nn\\
&+& 
\left(\frac{1}{4}(1+r)(1+s)-\frac{1}{4}b(r,s)\right)
+\left(\frac{1}{4}(1-r)(1+s)-\frac{1}{4}b(r,s)\right) \nn\\
&=&
\frac{1}{2}(1-s) + \frac{1}{2}(1+s) -b(r,s) \nn\\
&=& 
1 - b(r,s) 
\end{eqnarray}
We have 
\begin{eqnarray}
\int_{-1}^{+1} \int_{-1}^{+1} (1-b_1(r,s)) dr ds  
&=& \int_{-1}^{+1} \int_{-1}^{+1} [1 - (1-r^2)(1-s^2)(1-r)(1-s) ] dr ds = 20/9 \simeq 2.2222 \nn\\
\int_{-1}^{+1} \int_{-1}^{+1} (1-b_2(r,s,\beta)) dr ds  
&=& \int_{-1}^{+1} \int_{-1}^{+1} [1 - (1-r^2)(1-s^2)(1+\beta(r+s)) ] dr ds = 20/9  \qquad \forall \beta \nn 
\end{eqnarray}
Both bubbles yield the same average. This is not helpful. 

Let us now look at the (root) mean square:
\begin{eqnarray}
\int_{-1}^{+1} \int_{-1}^{+1} (1-b_1(r,s))^2 dr ds &=& 21284/11025 \simeq 1.93052 \nn\\
\int_{-1}^{+1} \int_{-1}^{+1} (1-b_2(r,s,\beta))^2 dr ds &=& \frac{4}{1575} (128 \beta^2  + 623) 
\end{eqnarray}
The problem is that the minimum is reached for $\beta=0$ which is not allowed so 
we cannot choose $\beta$ so as to minimise the error.
For $\beta=0.25$ as used in the paper:
\[
\int_{-1}^{+1} \int_{-1}^{+1} (1-b_2(r,s))^2 dr ds = 2524/1575 \simeq 1.60254 
\]
On the other hand, this means that using the second bubble function does a better job  
at representing square terms ($r^2$, $s^2$) than using the first one. 

\end{itemize}

One can also revisit the second bubble function: in Lamichhane (2017) \cite{lami17}
it is postulated to be defined by 
\begin{eqnarray}
b^{(2)}(r,s) &=& (a+br+cs)(1-r^2)(1-s^2) \qquad abc\neq 0
\end{eqnarray}
on the reference element $[-1,1]\times [-1,1]$. Then the author 
states that 'for simplicity we choose':
\begin{eqnarray}
b^{(2)}(r,s) &=& \frac{1}{4} (4+r+s)(1-r^2)(1-s^2) 
\end{eqnarray}
and that 'the factor 1/4 is used to force the value of the bubble function at
the centroid of the square to be 1'.

Looking closer, we see that forcing the bubble to be 1 in $(r,s)=(0,0)$ does impose
$a=1$ but leaves $b,c$ free, i.e. the bubble is then:
\begin{eqnarray}
b^{(2)}(r,s) &=& (1+br+cs)(1-r^2)(1-s^2) \qquad bc\neq 0
\end{eqnarray}

For symmetry reasons I would be tempted to indeed take $b=c$ but I am then left with 
\begin{eqnarray}
b^{(2)}(r,s) &=&  [1+b(r+s)](1-r^2)(1-s^2) \qquad b\neq 0
\end{eqnarray}
which means that Lamichhane sets $b=c=1/4$ in his paper. 

\underline{Question}: We know that $b=0$ is not allowed, but could it not be 
possible to design an analytical or numerical test or a 
theory to choose an 'optimal' value (in some sense) for $b$?  







