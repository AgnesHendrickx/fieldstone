
\begin{itemize}
\item Darcy flow. redo WAFLE (see http://cedricthieulot.net/wafle.html)
\item carry out critical Rayleigh experiments for various geometries/aspect ratios. Use Arie's notes. 
\item Newton solver
\item Corner flow 
\item elasticity with markers
\item Indentor/punch with stress b.c. ?
\item chunk grid
\item read in crust 1.0 in 2D on chunk
\item compute gravity based on tetrahedra
\item compare Q2 with Q2-serendipity
\item NS a la http://ww2.lacan.upc.edu/huerta/exercises/Incompressible/Incompressible\_Ex2.htm
\item produce example of mckenzie slab temperature
\item write about impose bc on el matrix
\item constraints
\item discontinuous galerkin
\item formatting of code style
\item navier-stokes ? (LUKAS) use dohu matlab code
\item nonlinear poiseuille
\end{itemize}



\begin{itemize}



\item
compositions, marker chain

\item
free-slip bc on annulus and sphere . See for example p540 Gresho and Sani book.

\item
non-linear rheologies (two layer brick spmw16, tosn15) 

\item
Picard vs Newton

\item
periodic boundary conditions

\item
free surface 

\item 
newton method to localise markers

\item
zaleski disk advection

\item
including phase changes (w. R. Myhill)

\item
compute strainrate in middle of element or at quad point for punch?

\item
GEO1442 code 

\item
GEO1442 indenter setup in plane ?

\item
in/out flow on sides for lith modelling

\item
Fehlberg RK advection

\item redo puth17 2 layer experiment

\end{itemize}


https://peterkovesi.com/projects/colourmaps/
velocity: CET-D1A
pressure: CET-D1
divv: CET-L1
density: CET-D3
strainrate: CET-R2




%\newpage
%\subsection{With periodic boundary conditions}
%\subsection{Different Cmat}
%\subsection{Penalty Uzawa formulation}
%\subsection{Powell-Hestenes iterations a la MILAMIN}
%\subsection{With temperature and phase change}
%\subsection{Conformal refinement}
%\subsection{Stress b.c.}
%\subsection{open boundary conditions}
%\subsection{melt generation}

\noindent Problems to solve:

colorscale 

better yet simple matrix storage ?

write Scott about matching compressible2 setup with his paper

deal with large matrices. 
