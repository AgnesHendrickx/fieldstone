\subsection{Cedric To Do/Ideas List}

\begin{itemize}
\item ask Taka to finish/write grain contact benchmark
\item extract know-how and notes about $Q_1 \times P_0$-stab off ELEFANT and make stone(s). mention \cite{lisi12}
\item revisit CBF and produce stone(s)
\item produce stone for Taka's notch
\item priduce stone for cicrle/ellipse hole in place - taka
\item incorporate Jort's DG 
\item build fast/reliable schur complement solver
\item look at Wouter K compgeo thesis again
\item revisit all tikz of elements and add consistent colours
\item revisit diff/disl creep partition stuff
\item in the context of mesh generation on sphere cite \cite{moma19}
\item finish extract info 
\item sort out Aitken method
\item Indentor/punch with stress b.c. ?
\item free-slip bc on annulus and sphere . See for example p540 Gresho and Sani book. find book \cite{deab72}.
also check \cite{ensg82} !!
\item constraints \cite{absh79}
\item illustrate early boundary fitted static meshes with \cite{thar85}
\item \cite{bepo10} spell out the derivation of Jaumann derivative in appendix
\item look at strain-rate softening in \cite{belz02}
\item Material point method \cite{sucs94,susc96,susp07}
\item redo/explore dyn topo bench of \cite{bore19}
\item check \cite{bufm19} for RT0 element use
\item GEO1442 indenter setup in plane ?
\item SIMPLE a la p667 \cite{john16} 
\item implement/monitor div v
\item check the BASIL code by Houseman et al \url{http://homepages.see.leeds.ac.uk/~eargah/basil/}
on which the ELLE code is based \url{http://elle.ws/} 
\item try Anderson acceleration for Uzawa \cite{hoow17} with m=1
\item deformation around rigid particles \cite{ilma93}
\item mesh containing both quadrilaterals and triangles \cite{anbr80}
\item implementation of fault in FEM codes \cite{zhgu94,zhgu95}
\item remove nnp from f14 and other stones
\item check $Q_2 \times Q_{-1}$ element, \cite{grsa} p 697. pressure basis function based at the four 2x2 gauss points.
\item run 3D burstedde bench qith Q1Q1 stab in aspect
\item use 3D benchmark of s75 for s82 !
\item write/implement picard, defect Picard, line search...
\item check aspect manual for slash si units !
\item Write about

  \begin{itemize}
  \item harmonic spectrum , see \cite{ribr99}
  \item write about impose bc on el matrix
  \item write about stream functions 
  \item write section in features about thermo mechanical simulations and how/why we solve vp, then T.
  \item write Scott about matching compressible2 setup with his paper
  \item write about vorticity-velocity method: \cite{gats91,gust93,dehu95,ergq99,amct04,spez87}
  \item write about flexural isostasy \cite{maie12}, bottom Sopale
  \item write about splines as shape functions \cite{chri92}. second or third order shape functions , using extra nodes instead of using more nodes per element. 
  Smaller matrix than Q2 or Q3 but: spline coeffs on nodes are no more unknowns. Plus bc are complicated. Does it work well with visc contrasts ?
  \item write about jacobian tensor and norm for Q1 rectangles. and more ?
  \item write about Courant nb
  \end{itemize}
\end{itemize}

%----------------------------------
\subsection{BSc thesis topics}
\begin{itemize} 
\item Darcy flow. redo WAFLE (see http://cedricthieulot.net/wafle.html)
\item chunk grid. benchmark of busa13?
\item compute gravity based on tetrahedra
\item Slab detachment + diff elements + Aitken
\end{itemize}

%----------------------------------
\subsection{GR thesis topics}
\begin{itemize} 
\item create stone for layeredflow 
\item re-incorporate WK inversion as stone with python Stokes solver
\item implement Navier-Stokes eqs a la http://ww2.lacan.upc.edu/huerta/exercises/Incompressible/Incompressible\_Ex2.htm
\item FSSA implementation, derivation, everywhere in domain. 
\item numerical viscosimeter \cite{batt84}
\item advection a la van Hunen
\item W.Klessens MD report idea, onset of convection
\item folding experiments a la Frehner MSc/PhD thesis and papers 
\item defomrtaion around a rising diapir modeled by creeping flow past a sphere, analogue \cite{crud88}
\end{itemize}

%----------------------------------
\subsection{MSc thesis topics}
\begin{itemize} 
\item implement Newton solver in a stone for power-law rheologies. Use existing fortran code. Test it thoroughly.
\item write surface processes code
\item surface tension see \cite{reddybook2}p28-29 - see ibuprofem, \cite{dett04} 
\item elasticity with markers
\item navier-stokes ? (LUKAS) use dohu matlab code
\item pure shear deformation of inclusions \cite{trla00}
\item redo Buck and Sokoutis benchmark for continental convergence \cite{buso94}
\item redo topo and geoid calculations a la \cite{king09}
\item propagator matrix ? what is it ? \cite{ribe18} 
\Literature \cite{haoc78,haoc81,riha84,zhon96,como97,mohc98,zhzu00,lezh08,leha08,mofm07,mibb09,fope91,lizh13,bugo94} 
\item redo erosion solutions by \cite{cull60} 
\item redo convection at high Ra 2D \cite{scan85}
\item redo very early FE paper 1971 \cite{stbe71}
\item redo early 3D subd \cite{zhgu96}
\item redo improved method of Nusselt number calculation \cite{hohr87}
\item redo magma chamber studies \cite{cuwi14,gehn18}
\item implement multigrid
\item adjoint methods in geodynamics \cite{bugs09,ghbu16,hobo14,isks07,ligs17,wahg15,wama09,wosp14}.
Also see {\tt johnson\_Notes on Adjoint Methods.pdf} and {\tt bradley-PDE-constrained optimization and the adjoint method.pdf} 
\end{itemize}

with ASPECT:

\begin{itemize}
\item redo early compressional orogen study by Beaumont \cite{bequ94}
\item redo extension 3D Allken papers + \cite{poay84,katl95} 
\item redo Travis study \cite{trab90} which is close to Blankenbach \cite{blbc89}. Note that \cite{maie12} looks at kinetic energy for \cite{trab90} 
\end{itemize}


%-------------------------------
\subsection{Open questions}

\begin{itemize}
\item what does it mean to have a negative pressure ? should we threshold it when computing yield strength ? 
\item Why pressure bc works for Q2Q1 but not Q1P0 ??
\item is there a formal definition of shape fct, trial fct, basis fct vs test fct 
\item is there an efficient manner to start from (say) quadrilateral mesh defined by corners only, choose an 
order $n$ corresponding to $Q_n \times Q_{n-1}$ and generate nodes without doubles?
\end{itemize}





