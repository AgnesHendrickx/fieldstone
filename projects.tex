To do:

\begin{itemize}
\item revisit CBF and produce stone(s)
\item incorporate Sverre's gravity work
\item produce stone for Taka's notch
\item incorporate Jort's DG 
\item build fast/reliable schur complement solver\
\item look at Wouter K compgeo thesis again
\item revisit all tikz of elements and add consistent colours
\item revisit diff/disl creep partition stuff
\end{itemize}



%----------------------------------
\subsection{BSc thesis topics}
\begin{itemize} 
\item Darcy flow. redo WAFLE (see http://cedricthieulot.net/wafle.html)
\item chunk grid
\item compute gravity based on tetrahedra
\end{itemize}

%----------------------------------
\subsection{GR thesis topics}
\begin{itemize} 
\item create stone for layeredflow 
\end{itemize}

%----------------------------------
\subsection{MSc thesis topics}
\begin{itemize} 
\item implement Newton solver in a stone for power-law rheologies. Use existing fortran code. Test it thoroughly.
\item write surface processes code
\item surface tension see \cite{reddybook2}p28-29 - see ibuprofem 
\item elasticity with markers
\item navier-stokes ? (LUKAS) use dohu matlab code
\item pure shear deformation of inclusions \cite{trla00}
\item redo Buck and Sokoutis benchmark for continental convergence \cite{buso94}
\item redo topo and geoid calculations a la \cite{king09}
\item propagator matrix ? what is it ? \cite{ribe18} 
\Literature \cite{haoc78,haoc81,riha84,zhon96,como97,mohc98,zhzu00,lezh08,leha08,mofm07,mibb09,fope91,lizh13,bugo94} 
\item redo erosion solutions by \cite{cull60} 
\item redo convection at high Ra 2D \cite{scan85}
\item redo very early FE paper 1971 \cite{stbe71}
\item redo early 3D subd \cite{zhgu96}
\item redo improved method of Nusselt number calculation \cite{hohr87}
\item redo magma chamber studies \cite{cuwi14,gehn18}
\end{itemize}

with ASPECT:

\begin{itemize}
\item redo early compressional orogen study by Beaumont \cite{bequ94}
\item redo extension 3D Allken papers + \cite{poay84,katl95} 
\item redo Travis study \cite{trab90} which is close to Blankenbach \cite{blbc89}. Note that \cite{maie12} looks at kinetic energy for \cite{trab90} 
\end{itemize}




%-------------------------------
\subsection{Write about}

\begin{itemize}
\item harmonic spectrum , see \cite{ribr99}
\item write about impose bc on el matrix
\item write about stream functions 
\item write section in features about thermo mechanical simulations and how/why we solve vp, then T.
\item write Scott about matching compressible2 setup with his paper
\item write about vorticity-velocity method: \cite{gats91,gust93,dehu95,ergq99,amct04,spez87}
\item write about flexural isostasy \cite{maie12}, bottom Sopale
\item write about splines as shape functions \cite{chri92}. second or third order shape functions , using extra nodes instead of using more nodes per element. Smaller matrix than Q2 or Q3 but: spline coeffs on nodes are no more unknowns. Plus bc are complicated. Does it work well with visc contrasts ?
\item write about jacobian tensor and norm for Q1 rectangles. and more ?
\end{itemize}



%-------------------------------
\subsection{Open questions}

what does it mean to have a negative pressure ? should we threshold it when computing yield strength ? 



\begin{itemize} 
\item free-slip bc on annulus and sphere . See for example p540 Gresho and Sani book. find book \cite{deab72}.
also check \cite{ensg82} !!
\item constraints \cite{absh79}
\item Finish nonlinear cavity case5.
\item in the context of mesh generation on sphere cite \cite{moma19}
\item illustrate early boundary fitted static meshes with \cite{thar85}
\item \cite{bepo10} spell out the derivation of Jaumann derivative in appendix
\item look at strain-rate softening in \cite{belz02}
\item Material point method \cite{sucs94,susc96,susp07}
\item redo/explore dyn topo bench of \cite{bore19}
\end{itemize}

\begin{itemize} 
\item Indentor/punch with stress b.c. ?
\item read in crust 1.0 in 2D on chunk
\item NS a la http://ww2.lacan.upc.edu/huerta/exercises/Incompressible/Incompressible\_Ex2.htm
\item including phase changes (w. R. Myhill)
\item GEO1442 indenter setup in plane ?
\item redo puth17 2 layer experiment
\item SIMPLE a la p667 \cite{john16} 
\item implement/monitor div v
\item shape fct, trial fct, basis fct vs test fct doc
%\item Flow over a backward-facing step \cite{grdn}
%\item Flow past an obstacle \cite{grdn}
%\item Zalesak disc \cite{basd08}
\item lukas' 2D and 3D benchmark
\item ROTATING disc
\item check the BASIL code by Houseman et al \url{http://homepages.see.leeds.ac.uk/~eargah/basil/}
on which the ELLE code is based \url{http://elle.ws/} 
\item try Anderson acceleration for Uzawa \cite{hoow17} with m=1
\item $Q_1^+ \times P_0$ Look at fort81 , rota87b and vadv03 , begt92
\item check \cite{bufm19} for RT0 element use
\item deformation around rigid particles \cite{ilma93}
\item mesh containing both quadrilaterals and triangles \cite{anbr80}
\item redo/adapt bsc thesis with inversion on stokes sphere
\item implementation of fault in FEM codes \cite{zhgu94,zhgu95}
\item van keken instantane MINI elt
\item check $Q_2 \times Q_{-1}$ element, \cite{grsa} p 697. pressure basis function based at the four 2x2 gauss points.
\end{itemize}



<Cells>
<DataArray type=”Int32” Name=”connectivity” .../>
<DataArray type=”Int32” Name=”offsets” .../>
<DataArray type=”UInt8” Name=”types” .../>
</Cells>



