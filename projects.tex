
\begin{itemize}
%..............................
\item Bsc thesis
\begin{itemize} 
\item Darcy flow. redo WAFLE (see http://cedricthieulot.net/wafle.html)
\item chunk grid
\end{itemize}
%..............................
\item MSc guided research/thesis
\begin{itemize} 
\item surface tension see \cite{reddybook2}p28-29 - see ibuprofem 
\item elasticity with markers
\item navier-stokes ? (LUKAS) use dohu matlab code
\item pure shear deformation of inclusions \cite{trla00}
\item redo Buck and Sokoutis benchmark for continental convergence \cite{buso94}
\item redo topo and geoid calculations a la \cite{king09}
\item redo Travis study \cite{trab90} which is close to Blankenbach \cite{blbc89}. Note that \cite{maie12} looks at kinetic energy for \cite{trab90} 
\item propagator matrix ? what is it ? \cite{ribe18} \mscthesis \index{general}{MSc Thesis}  
\Literature \cite{haoc78,haoc81,riha84,zhon96,como97,mohc98,zhzu00,lezh08,leha08,mofm07,mibb09,fope91,lizh13,bugo94} 
\item redo early compressional orogen study by Beaumont \cite{bequ94}
\item redo erosion solutions by \cite{cull60} \mscthesis \index{general}{MSc Thesis} 
\item redo extension 3D Allken papers + \cite{poay84,katl95} 
\item redo convection at high Ra 2D \cite{scan85}
\item redo very early FE paper 1971 \cite{stbe71}
\item redo early 3D subd \cite{zhgu96}
\item redo improved method of Nusselt number calculation \cite{hohr87}
\end{itemize}
%..............................
\item Miscellaneous /to do
\begin{itemize} 
\item write about impose bc on el matrix
\item free-slip bc on annulus and sphere . See for example p540 Gresho and Sani book. find book \cite{deab72}.
also check \cite{ensg82} !!
\item constraints \cite{absh79}
\item formatting of code style
\item Finish nonlinear cavity case5.
\item write about stream functions 
\item create stone for layeredflow (see folder one up)
\item in the context of mesh generation on sphere cite \cite{moma19}
\item illustrate early boundary fitted static meshes with \cite{thar85}
\item \cite{bepo10} spell out the derivation of Jaumann derivative in appendix
\item look at strain-rate softening in \cite{belz02}
\item write section in features about thermo mechanical simulations and how/why we solve vp, then T.
\item Material point method \cite{sucs94,susc96,susp07}
\item redo/explore dyn topo bench of \cite{bore19}
\end{itemize}


\item carry out critical Rayleigh experiments for various geometries/aspect ratios. Use Arie's notes. 
\item Indentor/punch with stress b.c. ?
\item read in crust 1.0 in 2D on chunk
\item compute gravity based on tetrahedra
\item NS a la http://ww2.lacan.upc.edu/huerta/exercises/Incompressible/Incompressible\_Ex2.htm
\item write Scott about matching compressible2 setup with his paper
\item including phase changes (w. R. Myhill)
\item GEO1442 indenter setup in plane ?
\item redo puth17 2 layer experiment
\item SIMPLE a la p667 \cite{john16} 
\item implement/monitor div v
\item shape fct, trial fct, basis fct vs test fct doc
\item write/draw the whole FEM process for a 4x3 grid for compgeo
%\item Flow over a backward-facing step \cite{grdn}
%\item Flow past an obstacle \cite{grdn}
%\item Zalesak disc \cite{basd08}
\item lukas' 2D and 3D benchmark
\item ROTATING disc
\item cylindrical footing on (elasto)-viscous medium - analytical solution, Haskell, etc ...
\item check the BASIL code by Houseman et al \url{http://homepages.see.leeds.ac.uk/~eargah/basil/}
on which the ELLE code is based \url{http://elle.ws/} 
\item try Anderson acceleration for Uzawa \cite{hoow17} with m=1
\item $Q_1^+ \times P_0$ Look at fort81 , rota87b and vadv03
\item check \cite{bufm19} for RT0 element use
\item deformation around rigid particles \cite{ilma93}
\item write about flexural isostasy \cite{maie12}, bottom Sopale
\item mesh containing both quadrilaterals and triangles \cite{anbr80}
\item redo/adapt bsc thesis with inversion on stokes sphere
\item implementation of fault in FEM codes \cite{zhgu94,zhgu95}
\item \url{https://en.wikipedia.org/wiki/Bernstein_polynomial}
\item cvi \cite{pukp16},\cite{mcna11}
\item van keken instantane MINI elt
\item check $Q_2 \times Q_{-1}$ element, \cite{grsa} p 697. pressure basis function based at the four 2x2 gauss points.
\item look at condition number of $\K$ block for Q1P0 and Q2Q1 as a function of resolution. Insert results in section about why q1p0 should not be used. 
\item write about splines as shape functions \cite{chri92}. second or third order shape functions , using extra nodes instead of using more nodes per element. Smaller matrix than Q2 or Q3 but: spline coeffs on nodes are no more unknowns. Plus bc are complicated. Does it work well with visc contrasts ?

\item write about vorticity-velocity method: \cite{gats91,gust93,dehu95,ergq99}

\end{itemize}

 <Cells>
      <DataArray type=”Int32” Name=”connectivity” .../>
      <DataArray type=”Int32” Name=”offsets” .../>
      <DataArray type=”UInt8” Name=”types” .../>
    </Cells>


open questions:
what does it mean to have a negative pressure ? should we threshold it when computing yield strength ? 


velocity-based

geometry:
element type:
mixed or penalty: 
number of dimensions:
physics: 
solver:
matrix storage: 
compressible (Y/N): 
analytical benchmark (Y/N): 
numerical benchmark (Y/N):
error convergence (Y/N):
thermo-mechanically coupled (Y/N):
time-stepping (Y/N):
non-linear (Y/N)
mesher (Y/N)

displacement-based stones


stream function stones 

other stones


