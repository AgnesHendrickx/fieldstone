
\begin{itemize}
%..............................
\item Bsc thesis
\begin{itemize} 
\item Darcy flow. redo WAFLE (see http://cedricthieulot.net/wafle.html)
\item nonlinear poiseuille
\item Fehlberg RK advection
\item implement mms5 \ref{mms5}, mms7 \ref{mms7}
\item chunk grid
\end{itemize}
%..............................
\item MSc guided research/thesis
\begin{itemize} 
\item Newton solver
\item surface tension see \cite{reddybook2}p28-29 - see ibuprofem 
\item elasticity with markers
\item navier-stokes ? (LUKAS) use dohu matlab code
\item lev hager 2008 RT instability with anisotropic visc
\item discontinuous galerkin
\item free surface \cite{dumy16} 
\item redo benchmarks convection of \cite{trab90}
\item pure shear deformation of inclusions \cite{trla00}
\item anisotropic heat conduction\cite[p121]{reddybook2}, \cite[p143]{reddybook2}, section\ref{sec:anisotropic}
\item sinking sphere in bingham or herschel-bulkley fluid \cite{bemj04} 
\item redo Buck and Sokoutis benchmark for continental convergence \cite{buso94}
\item redo topo and geoid calculations a la \cite{king09}
\item redo Travis study \cite{trab90} which is close to Blankenbach \cite{blbc89}.
\end{itemize}
%..............................
\item Miscellaneous /to do
\begin{itemize} 
\item write about impose bc on el matrix
\item free-slip bc on annulus and sphere . See for example p540 Gresho and Sani book. find book \cite{deab72}.
also check \cite{ensa82} !!
\item mention Lattice-Boltzmann in geosciences \cite{hupc08}
\item constraints \cite{absh79}
\item formatting of code style
\item Finish nonlinear cavity case5.
\item write about stream functions 
\item create stone for layeredflow (see folder one up)
\item in the context of mesh generation on sphere cite \cite{moma19}
\item about CMR: \cite{vaks15},\cite{kott05}
\item symmetric vs gradient formulation of Stokes
\item add \cite{devv00a,dadh07} to list of papers doing vankeken bench
\item illustrate early boundary fitted static meshes with \cite{thar85,boww89,whbw92}
\item visco-elastic flow past a cylinder in a channel \cite{bepo10}
\item \cite{bepo10} spell out the derivation of Jaumann derivative in appendix
\item look at strain-rate softening in \cite{belz02}
\end{itemize}


\item carry out critical Rayleigh experiments for various geometries/aspect ratios. Use Arie's notes. 
\item Indentor/punch with stress b.c. ?
\item read in crust 1.0 in 2D on chunk
\item compute gravity based on tetrahedra
\item NS a la http://ww2.lacan.upc.edu/huerta/exercises/Incompressible/Incompressible\_Ex2.htm
\item zaleski disk advection
\item write Scott about matching compressible2 setup with his paper
\item compositions, marker chain
\item non-linear rheologies (two layer brick spmw16, tosn15) 
\item including phase changes (w. R. Myhill)
\item GEO1442 indenter setup in plane ?
\item redo puth17 2 layer experiment
\item SIMPLE a la p667 \cite{john16} 
\item implement/monitor div v
\item shape fct, trial fct, basis fct vs test fct doc
\item Delaunay triangulation, Voronoi, stripack
\item write/draw the whole FEM process for a 4x3 grid for compgeo
%\item Flow over a backward-facing step \cite{grdn}
%\item Flow past an obstacle \cite{grdn}
%\item Dam break problem \cite{grdn,hini81,moeb99,basd08}
%\item Zalesak disc \cite{basd08}
\item lukas' 2D and 3D benchmark
\item ROTATING disc
\item cylindrical footing on (elasto)-viscous medium - analytical solution, Haskell, etc ...
\item check the BASIL code by Houseman et al \url{http://homepages.see.leeds.ac.uk/~eargah/basil/}
on which the ELLE code is based \url{http://elle.ws/} 
\item try Anderson acceleration for Uzawa \cite{hoow17} with m=1
\item $Q_1^+ \times P_0$ Look at fort81 , rota87b and vadv03
\item check \cite{bufm19} for RT0 element use
\item deformation around rigid particles \cite{ilma93}
\item write about flexural isostasy \cite{maie12}, bottom Sopale
\item \cite{maie12} looks at kinetic energy for \cite{trab90} 
\end{itemize}

 <Cells>
      <DataArray type=”Int32” Name=”connectivity” .../>
      <DataArray type=”Int32” Name=”offsets” .../>
      <DataArray type=”UInt8” Name=”types” .../>
    </Cells>


Why do I have to promise where I am going while I am not there yet?

You can't google something you don't know exists.

You can be correct or you can get stuff done

open questions:
what does it mean to have a negative pressure ? should we threshold it when computing yield strength ? 
