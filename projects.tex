
\begin{itemize}
\item Darcy flow. redo WAFLE (see http://cedricthieulot.net/wafle.html)
\item carry out critical Rayleigh experiments for various geometries/aspect ratios. Use Arie's notes. 
\item Newton solver
\item Corner flow 
\item elasticity with markers
\item Indentor/punch with stress b.c. ?
\item chunk grid
\item read in crust 1.0 in 2D on chunk
\item compute gravity based on tetrahedra
\item compare Q2 with Q2-serendipity
\item NS a la http://ww2.lacan.upc.edu/huerta/exercises/Incompressible/Incompressible\_Ex2.htm
\item produce example of mckenzie slab temperature
\item write about impose bc on el matrix
\item constraints
\item discontinuous galerkin
\item formatting of code style
\item navier-stokes ? (LUKAS) use dohu matlab code
\item nonlinear poiseuille
\item Finish nonlinear cavity case5.
\item write about mappings 
\item write about stream functions 
\item free surface 
\item zaleski disk advection
\item better yet simple matrix storage ?
\item write Scott about matching compressible2 setup with his paper
\item deal with large matrices. 
\item compositions, marker chain
\item free-slip bc on annulus and sphere . See for example p540 Gresho and Sani book.
\item non-linear rheologies (two layer brick spmw16, tosn15) 
\item Picard vs Newton
\item including phase changes (w. R. Myhill)
\item compute strainrate in middle of element or at quad point for punch?
\item GEO1442 code 
\item GEO1442 indenter setup in plane ?
\item in/out flow on sides for lith modelling
\item Fehlberg RK advection
\item redo puth17 2 layer experiment
\item create stone for layeredflow (see folder one up)
\item SIMPLE a la p667 \cite{john16} 
\item implement mms5 \ref{mms5}
\item implement mms7 \ref{mms7}
\item implement/monitor div v
\item shape fct, trial fct, basis fct vs test fct doc
\item Delaunay triangulation, Voronoi, stripack
\item symmetric vs gradient formulation of Stokes
\item write/draw the whole FEM process for a 4x3 grid for compgeo
\item Sphere drag in a visco-plastic fluid \cite{bemj04}
%\item Flow over a backward-facing step \cite{grdn}
%\item Flow past an obstacle \cite{grdn}
%\item Dam break problem \cite{grdn,hini81,moeb99,basd08}
%\item Zalesak disc \cite{basd08}
\item mention Lattice-Boltzmann in geosciences \cite{hupc08}
\item lukas' 2D and 3D benchmark
\item ROTATING disc
\item cylindrical footing on (elasto)-viscous medium - analytical solution, Haskell, etc ...
\end{itemize}

 <Cells>
      <DataArray type=”Int32” Name=”connectivity” .../>
      <DataArray type=”Int32” Name=”offsets” .../>
      <DataArray type=”UInt8” Name=”types” .../>
    </Cells>


Why do I have to promise where I am going while I am not there yet?

You can't google something you don't know exists.

You can be correct or you can get stuff done

open questions:
what does it mean to have a negative pressure ? should we threshold it when computing yield strength ? 
