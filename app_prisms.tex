
{\sl This appendix was written by Sverre Hassing as part of his Bachelor thesis.} \index{contributors}{S. Hassing}
Although the final formula are definitely correct, the derivations below may still contain a typo. 

\vspace{.4cm}

The derivations for prisms have been published in the early 50's \cite{made51}. 
However, to the best of our knowledge the full derivation has not been carried out in English in full detail. 
The derivations are based on those of Mader (1951) \cite{made51} and of Nagy \etal (2000) \cite{napb00,napb02}. 
Mader provided the derivations in some detail, while Nagy \etal interpreted the results in a more modern style. 


%----------------------------
\subsection{Basic formulas}

The derivations for prisms are a lot more complicated than that for the point masses. 
We start with the following two integral equations which are integral part of the derivations:

\begin{eqnarray}
\int \frac{x^2 dx}{x^2 + z^2} &=& x - z  \arctan{\frac{x}{z}} \label{eq:basic1} \\
\int \frac{dx}{\sqrt{x^2+y^2+z^2}} &=& \ln{\left( x + \sqrt{x^2+y^2+z^2} \right)} \label{eq:basic2}
\end{eqnarray}
Another equation that will come back multiple times is of the form:
\begin{equation}
\int \frac{du}{(v^2+w^2)\sqrt{u^2+v^2+w^2}}
\end{equation}
This can be solved with a trigonometric substitution, where $u = \sqrt{v^2+w^2} \tan{\phi}$. This means that $du = \frac{\sqrt{v^2+w^2}}{cos^2{\phi}} d\phi$.
\begin{eqnarray}
\int \frac{du}{(v^2+w^2)\sqrt{u^2+v^2+w^2}} 
&=& 
\int \frac{\sqrt{v^2+w^2}}{cos^2{\phi}} 
\frac{1}{(v^2+w^2)\tan^2{\phi}+v^2} 
\frac{d\phi}{\sqrt{v^2+w^2+(v^2+w^2)\tan^2{\phi})}} \nonumber\\
&=& 
\int \frac{\sqrt{v^2+w^2}}{cos{\phi}^2} 
\frac{1}{v^2(\tan^2{\phi}+1) + w^2 \tan^2{\phi}} 
\frac{d\phi}{\sqrt{(v^2+w^2)(\tan^2{\phi}+1)}} \nonumber\\
&=&
\int \frac{\sqrt{v^2+w^2}}{cos{\phi}^2} 
\frac{1}{\frac{v^2+w^2 \sin^2{\phi}}{cos^2{\phi}}} 
\frac{d\phi}{\frac{\sqrt{v^2+w^2}}{cos{\phi}}} \nonumber\\
&=& 
\int \frac{\cos{\phi} d\phi}{v^2+w^2\sin^2{\phi}}
\end{eqnarray}
A second substitution is needed where $t = \frac{w}{v}\sin{\phi}$ and $dt = \frac{v}{w} \cos{\phi} d\phi$:
\begin{eqnarray}
\int \frac{\cos{\phi} d\phi}{v^2+w^2\sin^2{\phi}} 
&=& \int \frac{v dt}{w (v^2+v^2t^2)} \nonumber\\
&=& \frac{1}{vw} \int \frac{dt}{1+t^2} \nonumber\\
&=& \frac{1}{vw} \arctan{t} \nonumber\\
&=& \frac{1}{vw} \arctan{\frac{w \sin{\phi}}{v}} \label{eq:prismbe1}
\end{eqnarray}
Now the $\sin{\phi}$ needs to be converted back to $u,v,w$. If it is known that $\tan{\phi} = \frac{u}{\sqrt{v^2+w^2}}$, then it follows that $\sin{\phi}=\frac{u}{\sqrt{u^2+v^2+w^2}}$.
Eq.\eqref{eq:prismbe1} then becomes
\begin{equation}
{\frac{1}{vw} \arctan \frac{w \sin{\phi}}{v}} = \frac{1}{vw} \arctan{\frac{u w}{v \sqrt{u^2+v^2+w^2}}}
\end{equation}
and finally 
\begin{equation}
\boxed{
\int \frac{du}{(v^2+w^2)\sqrt{u^2+v^2+w^2}} = \frac{1}{vw} \arctan{\frac{u w}{v \sqrt{u^2+v^2+w^2}}} \label{eq:basic3}
}
\end{equation}

\subsection{The gravitational potential}

Each prism is assumed to have constant density $\rho$. The gravitational potential is integrated over the whole volume of the prism:
\begin{equation}
U(P) =  -{\cal G}  \rho \underbrace{\int_{x_1}^{x_2} \int_{y_1}^{y_2} \int_{z_1}^{z_2} \frac{dx dy dz}{\sqrt{x^2+y^2+z^2}}}_{I} \label{eq:der_U_1}
\end{equation}

In what follows we work out the exact form for the triple integral term. Elementary Eq.~\eqref{eq:basic2} can be applied to the integral for $dx$ 
in Eq.~\eqref{eq:der_U_1}.
\begin{eqnarray}
I
&=& \iiint \frac{dx dy dz}{\sqrt{x^2+y^2+z^2}}  \nn\\
&=& \iint \left(\int \frac{dx}{\sqrt{x^2+y^2+z^2}} \right) dydz \nonumber\\
&=& \iint \ln{\left( x+\sqrt{x^2+y^2+z^2} \right)} dy dz
\end{eqnarray}
We further proceed with the integration with respect to $y$. We define
\[
\begin{array}{c|c}
  f = \int \ln{\left( x+\sqrt{x^2+y^2+z^2} \right)} dz 
  & g' = dy \\ 
  \hline
  f' = \frac{y}{\left( x+\sqrt{x^2+y^2+z^2} \right) \sqrt{x^2+y^2+z^2}} 
  & g=y
 \end{array}
\]
and using $\int f g'  = f g - \int f g'$ we have 
\begin{equation}
I= 
\underbrace{y \int \ln{ \left( x + \sqrt{x^2+y^2+z^2} \right) } dz}_{A} 
\underbrace{- \iint  \frac{y^2 dz}{\left( x+\sqrt{x^2+y^2+z^2}\right) \sqrt{x^2+y^2+z^2}} dy}_{B} 
\end{equation}

The calculation of $I$ is then split into two large integrals denoted
$A$ and $B$, calculated in the following subsections.
Note that we have not made use of the integral bounds yet.  

%..................................................................
\subsubsection{The calculation of $A$} \label{subsection:calc_A}

The first step in calculating $A$ is to carry out a similar partial integration as seen before. 

\[
\begin{array}{c|c}
  f=\ln{ \left( x + \sqrt{x^2+y^2+z^2} \right)} & 
  g'=dz \\ 
  \hline
  f' = \frac{\partial f}{\partial z} = \frac{z}{\left( x+\sqrt{x^2+y^2+z^2} \right) \sqrt{x^2+y^2+z^2}} 
  & g = z
 \end{array}
\]

\begin{eqnarray}
A 
&=& 
y \left(z \ln{ \left( x + \sqrt{x^2+y^2+z^2} \right)} - 
\int \frac{z^2 dz}{\left( x+\sqrt{x^2+y^2+z^2} \right) \sqrt{x^2+y^2+z^2}} \right) \nonumber\\
&=& 
\underbrace{yz \ln{ \left( x + \sqrt{x^2+y^2+z^2} \right)} }_{A_0} 
- 
\underbrace{y\int \frac{z^2 dz}{\left( x+\sqrt{x^2+y^2+z^2} \right) \sqrt{x^2+y^2+z^2}}  }_{A_1}
\end{eqnarray}

We now focus on the $A_1$ integral. 
We first multiply the numerator and denominator by $-x+\sqrt{x^2+y^2+z^2} $. The last step uses Eqs.~\eqref{eq:basic2}, \eqref{eq:basic3} and \eqref{eq:basic1} respectively for each term.
\begin{eqnarray}
A_1 
&=& \int \frac{z^2 dz}{\left( x+\sqrt{x^2+y^2+z^2} \right) \sqrt{x^2+y^2+z^2}}
\frac{-x+\sqrt{x^2+y^2+z^2}}{-x+\sqrt{x^2+y^2+z^2}} \nonumber\\
&=& \int \frac{(-x z^2 + z^2 \sqrt{x^2+y^2+z^2} )dz}{\left( x^2+y^2+z^2 - x^2 \right) \sqrt{x^2+y^2+z^2}} \label{eq:der_U_2} \nonumber\\
&=& 
\int \frac{-x z^2 dz}{\left( y^2+z^2  \right) \sqrt{x^2+y^2+z^2}}  
+ \int \frac{z^2 \sqrt{x^2+y^2+z^2} dz}{\left( y^2+z^2  \right) \sqrt{x^2+y^2+z^2}} \nonumber\\
&=& \int \frac{-x \left( z^2 + y^2 -y^2 \right) dz}{(y^2+z^2)\sqrt{x^2+y^2+z^2}} + 
\int \frac{z^2dz}{y^2+z^2} \nonumber\\
&=& \int \frac{-x dz}{\sqrt{x^2+y^2+z^2}} + 
\int \frac{x y^2 dz}{(y^2+z^2)\sqrt{x^2+y^2+z^2}} + 
\int \frac{z^2dz}{y^2+z^2} \nonumber\\
&=& -x \int \frac{ dz}{\sqrt{x^2+y^2+z^2}} + 
x y^2 \int \frac{ dz}{(y^2+z^2)\sqrt{x^2+y^2+z^2}} + 
\int \frac{z^2dz}{y^2+z^2} \nonumber\\
&=& -x \ln{ \left( z + \sqrt{x^2+y^2+z^2} \right)} + 
y \arctan{\frac{x z}{y \sqrt{x^2+y^2+z^2}}} + 
z -  y \arctan{\frac{z}{y}}
\end{eqnarray}
This can be combined to get the final expression for A:
\begin{equation}
A = 
y \left( z \ln{ \left( x + \sqrt{x^2+y^2+z^2} \right)} + 
x \ln{ \left( z + \sqrt{x^2+y^2+z^2} \right)} -  
y \arctan{\frac{x z}{y \sqrt{x^2+y^2+z^2}}} - 
z + 
y \arctan{\frac{z}{y}} \right)
\end{equation}
The last two terms can be left out because they will cancel out when computing the integration boundaries from $x_1$ to $x_2$, because these terms do not contain the variable $x$. 
Finally we arrive at the following expression for $A$:
\begin{equation}
A = 
yz \ln{\left( x + \sqrt{x^2+y^2+z^2} \right)} + 
xy \ln{\left( z + \sqrt{x^2+y^2+z^2} \right)} - 
y^2 \arctan{\frac{xz}{y\sqrt{x^2+y^2+z^2}}}
\end{equation}



%.......................................
\subsubsection{The calculation of $B$}

The inner integral can be simplified similarly to how $A_1$ was simplified in Eq.~\eqref{eq:der_U_2}, 
by multiplying both numerator and denominator with $-x+\sqrt{x^2+y^2+z^2}$. The last step uses Eqs.~\eqref{eq:basic3} and \eqref{eq:basic1}:
\begin{eqnarray}
B 
&=& 
-\int y^2 \int \frac{dz}{\left( x + \sqrt{x^2+y^2+z^2} \right) \sqrt{x^2+y^2+z^2}} 
\frac{-x + \sqrt{x^2+y^2+z^2}}{-x + \sqrt{x^2+y^2+z^2}} dy \nonumber\\
&=& - \int y^2 \int \frac{-x + \sqrt{x^2+y^2+z^2} }{\left( x^2 + y^2+z^2 -x^2 \right) \sqrt{x^2+y^2+z^2}}dz dy \nonumber\\
&=& - \int y^2 \left( 
- \int \frac{x dz}{(y^2+z^2)\sqrt{x^2+y^2+z^2}} +
\int \frac{dz}{y^2+z^2} 
\right) dy \nonumber\\
&=& - \int y^2 \left( -\frac{1}{y} \arctan{\frac{xz}{y\sqrt{x^2+y^2+z^2}}} + \frac{1}{y} \arctan{\frac{z}{y}}\right) dy \nn\\
&=& \int y \arctan{\frac{xz}{y\sqrt{x^2+y^2+z^2}}} dy
\end{eqnarray}
Again the second term can be left out, because it does not contain the variable $x$. The next step is to apply a partial integration to $B$.  

\[
\begin{array}{c|c}
  f=\arctan{\frac{xz}{y\sqrt{x^2+y^2+z^2}}} & g'=y \\ 
  \hline
  f' = -xz \frac{\frac{1}{y^2 \sqrt{x^2+y^2+z^2}}+\frac{1}{(x^2+y^2+z^2)^\frac{3}{2}}}{\frac{x^2z^2}{y^2(x^2+y^2+z^2)}+1} & g = \frac{y^2}{2}
 \end{array}
\]

\begin{equation}
B = 
\frac{y^2}{2} \arctan{\frac{xz}{y\sqrt{x^2+y^2+z^2}}} +
\underbrace{
\frac{xz}{2} \int y^2 \frac{\frac{1}{y^2\sqrt{x^2+y^2+z^2}}+\frac{1}{(x^2+y^2+z^2)^\frac{3}{2}}}{\frac{x^2z^2}{y^2(x^2+y^2+z^2)}+1}}_{B_1}
\end{equation}

Let us finish by calculating the integral $B_1$:

\begin{eqnarray}
B_1 
&=& 
\frac{xz}{2} \int y^2 
\frac{\frac{1}{y^2\sqrt{x^2+y^2+z^2}}+\frac{1}{(x^2+y^2+z^2)^{3/2}}}{\frac{x^2z^2}{y^2(x^2+y^2+z^2)}+1} 
dy \nonumber\\
&=& 
\frac{xz}{2} \int y^2 
\frac{\frac{x^2+y^2+z^2}{y^2(x^2+y^2+z^2)^{3/2}}+\frac{y^2}{y^2(x^2+y^2+z^2)^{3/2}}}{\frac{x^2z^2+y^2(x^2+y^2+z^2)}{y^2(x^2+y^2+z^2)}} 
dy \nonumber\\
&=& 
\frac{xz}{2} \int y^2 
\frac{\frac{x^2+2y^2+z^2}{y^2(x^2+y^2+z^2)^{3/2}}}{\frac{x^2z^2+y^2(x^2+y^2+z^2)}{y^2(x^2+y^2+z^2)}} 
dy \nonumber\\
&=& 
\frac{xz}{2} \int y^2 
\frac{x^2+2y^2+z^2}{\sqrt{x^2+y^2+z^2}(x^2z^2+y^2(x^2+y^2+z^2))} 
dy \nonumber\\
&=& 
\frac{xz}{2} 
\int y^2 \frac{x^2+2y^2+z^2}{\sqrt{x^2+y^2+z^2}(x^2+y^2)(z^2+y^2)} 
dy \nonumber\\
&=& 
\frac{xz}{2} 
\left( 
\int \frac{2dy}{\sqrt{x^2+y^2+z^2}} + 
\int \frac{-(x^2+z^2)y^2-2x^2z^2}{(x^2+y^2)(y^2+z^2)\sqrt{x^2+y^2+z^2}} 
dy \right) \nonumber\\
&=& 
xz \ln{\left( y + \sqrt{x^2+y^2+z^2} \right)} - 
\frac{xz}{2} 
\int \frac{(x^2+z^2)y^2+2x^2z^2}{(x^2+y^2)(y^2+z^2)\sqrt{x^2+y^2+z^2}} 
dy \nonumber\\
&=& 
xz \ln{\left( y + \sqrt{x^2+y^2+z^2} \right)} - 
\frac{xz}{2} \int \frac{x^2y^2+y^2z^2+2x^2z^2}{(x^2+y^2)(y^2+z^2)\sqrt{x^2+y^2+z^2}} 
dy \nonumber\\
&=& 
xz \ln{\left( y + \sqrt{x^2+y^2+z^2} \right)} - 
\frac{xz}{2} \int \frac{x^2(y^2+z^2)+z^2(x^2+y^2)}{(x^2+y^2)(y^2+z^2)\sqrt{x^2+y^2+z^2}} 
dy \nonumber\\
&=& 
xz \ln{\left( y + \sqrt{x^2+y^2+z^2} \right)} - 
\frac{xz}{2} \int \frac{x^2}{(x^2+y^2)\sqrt{x^2+y^2+z^2}} dy - 
\frac{xz}{2} \int \frac{z^2}{(y^2+z^2)\sqrt{x^2+y^2+z^2}} 
dy \nonumber\\
&=& 
xz \ln{\left( y + \sqrt{x^2+y^2+z^2} \right)} - 
\frac{xz}{2} \frac{x^2\arctan{\frac{yz}{x\sqrt{x^2+y^2+z^2}}}}{xz} - 
\frac{xz}{2} \frac{z^2\arctan{\frac{xy}{z\sqrt{x^2+y^2+z^2}}}}{xz} 
\nonumber\\
&=& 
xz \ln{\left( y + \sqrt{x^2+y^2+z^2} \right)} - 
\frac{x^2}{2} \arctan{\frac{yz}{x\sqrt{x^2+y^2+z^2}}} - 
\frac{z^2}{2} \arctan{\frac{xy}{z\sqrt{x^2+y^2+z^2}}}
\end{eqnarray}
This can be combined to get the full expression for $B$:
\[
B = 
xz \ln{\left( \sqrt{x^2+y^2+z^2}+y \right)} - 
\frac{x^2}{2} \arctan{\frac{zy}{x\sqrt{x^2+y^2+z^2}}} + 
\frac{y^2}{2} \arctan{\frac{xz}{y\sqrt{x^2+y^2+z^2}}} - 
\frac{z^2}{2} \arctan{\frac{xy}{x\sqrt{z^2+y^2+z^2}}}
\]

%..............................................
\subsubsection{Combining $A$ and $B$}
Now A and B can be combined to get the expression of $I$ 

\begin{eqnarray}
I&=& A + B \nonumber\\
&=& yz \ln{\left( x + \sqrt{x^2+y^2+z^2} \right)} + 
xy \ln{\left( z + \sqrt{x^2+y^2+z^2} \right)} - 
y^2 \arctan{\frac{xz}{y\sqrt{x^2+y^2+z^2}}} \nonumber\\ 
&+& 
xz \ln{\left( \sqrt{x^2+y^2+z^2}+y \right)} - 
\frac{x^2}{2} \arctan{\frac{zy}{x\sqrt{x^2+y^2+z^2}}} + 
\frac{y^2}{2} \arctan{\frac{xz}{y\sqrt{x^2+y^2+z^2}}} - 
\frac{z^2}{2} \arctan{\frac{xy}{x\sqrt{z^2+y^2+z^2}}} \nonumber\\
&=&
yz \ln{\left( x + \sqrt{x^2+y^2+z^2} \right)} + 
xy \ln{\left( z + \sqrt{x^2+y^2+z^2} \right)} + 
xz \ln{\left( y+\sqrt{x^2+y^2+z^2} \right)} \nonumber\\ 
&-& 
\frac{x^2}{2} \arctan{\frac{zy}{x\sqrt{x^2+y^2+z^2}}} - 
\frac{y^2}{2} \arctan{\frac{xz}{y\sqrt{x^2+y^2+z^2}}} - 
\frac{z^2}{2} \arctan{\frac{xy}{x\sqrt{z^2+y^2+z^2}}} 
\end{eqnarray}

The boundaries for the volume from Eq.~\eqref{eq:def_points} need to be applied to the result of the integration. 
The boundary conditions are computed by plugging the upper value into the equation and subtracting the equation with the lower 
value plugged in. When the upper and lower values are respectively $x_2$ and $x_1$ for some function $f(x)$, this is $f(x_2) - f(x_1)$. 
This can be represented more efficiently with a summation over the subscript. Something needs to be added to still keep the subtraction in there. 
This can be done by adding a factor of $-1^i$, where i is the summation index. This will be positive when $i$ is even and negative when $i$ is odd. 
The new way of showing the result would be $\displaystyle\sum_{i=1}^{2} -1^i f(x_i)$. This is especially useful when there are three different 
integration boundaries to resolve. $r$ will be used instead of $\sqrt{x^2+y^2+z^2}$. 

\begin{eqnarray}
I
&=& \left| \left| \left| 
yz \ln{\left( x + r \right)} + 
xy \ln{\left( z + r \right)} + 
xz \ln{\left( y + r \right)} - 
\frac{x^2}{2} \arctan{\frac{zy}{xr}} - 
\frac{y^2}{2} \arctan{\frac{xz}{yr}} - 
\frac{z^2}{2} \arctan{\frac{xy}{xr}} 
\right|_{x_1}^{x_2} \right|_{y_1}^{y_2} \right|_{z_1}^{z_2} 
\nonumber\\
&=& \sum_{i=1}^{2} \sum_{j=1}^{2} \sum_{k=1}^{2} (-1)^{i+j+k} 
\Bigg(
y_j z_k \ln{\left( x_i + r_{ijk} \right)} + 
x_i y_j \ln{\left( z_k + r_{ijk} \right)} + 
x_i z_k \ln{\left( y_j+r_{ijk} \right)}  \nonumber\\
&& \qquad\qquad\qquad\qquad\qquad \left. - 
\frac{x_i^2}{2} \arctan{\frac{z_k y_j}{x_i r_{ijk}}} - 
\frac{y_j^2}{2} \arctan{\frac{x_i z_k}{y_j r_{ijk}}} - 
\frac{z_k ^2}{2} \arctan{\frac{x_i y_j}{x_i r_{ijk}}} 
\right)
\end{eqnarray} 

\todo[inline]{There is probably a mistake in eq above and below, last term, most likely should contain zk in denominator?}

Finally,
\begin{mdframed}[backgroundcolor=blue!5]
\begin{eqnarray}
U(\vec{r}) &=& {\cal G}  \rho 
\sum_{i=1}^{2} \sum_{j=1}^{2} \sum_{k=1}^{2} (-1)^{i+j+k} \Bigg( 
y_j z_k \ln{\left( x_i + r_{ijk} \right)} + 
x_i y_j \ln{\left( z_k + r_{ijk} \right)} + 
x_i z_k \ln{\left( y_j+r_{ijk} \right)}   \nn\\ 
&&  \qquad \qquad - 
\frac{x_i^2}{2} \arctan{\frac{z_k y_j}{x_i r_{ijk}}} - 
\frac{y_j^2}{2} \arctan{\frac{x_i z_k}{y_j r_{ijk}}} - 
\frac{z_k^2}{2} \arctan{\frac{x_i y_j}{x_i r_{ijk}}} 
\Bigg) 
\end{eqnarray}
\end{mdframed}








%------------------------------------------
\subsection{The gravity vector $\vec{g}$}

In 3D Cartesian coordinates the gravity vector is expressed as
\begin{equation}
\vec{g} = -\vec\nabla U  = 
\left( 
\begin{array}{c} 
-\frac{\partial U}{\partial x} \\ \\
-\frac{\partial U}{\partial y} \\ \\
-\frac{\partial U}{\partial z} 
\end{array} \right)
\end{equation}
The easiest way to calculate this is by including the partial derivatives in the original integral \eqref{eq:der_U_1}.
\begin{eqnarray}
I_x(\vec{r}) 
&=& \iiint \frac{\partial}{\partial x} \frac{dx dy dz}{\sqrt{x^2+y^2+z^2}} \nonumber\\
&=& -\iiint \frac{x dx dy dz}{(\sqrt{x^2+y^2+z^2})^3} \nonumber\\
&=& \iint \frac{dy dz}{\sqrt{x^2+y^2+z^2}}
\end{eqnarray}

The integral \eqref{eq:basic2} can be used, followed by the calculation of $A$ as seen in Section~\ref{subsection:calc_A} 
without the multiplication with $y$:
\begin{eqnarray}
I_x(\vec{r})
&=& \int \ln{\left( x + \sqrt{x^2+y^2+z^2} \right)} dz \nonumber\\
&=& 
z \ln{\left( y + \sqrt{x^2+y^2+z^2} \right)} + 
y \ln{\left( z + \sqrt{x^2+y^2+z^2} \right)} - 
x \arctan{\frac{yz}{x \sqrt{x^2+y^2+z^2}}}
\end{eqnarray}
The integration boundaries can be applied. Multiplication with ${\cal G}$ and $\rho$ is the final step in deriving the element of 
the gravity vector component ($g_x$). 

\begin{eqnarray}
g_x = 
{\cal G}  \rho \sum_{i,j,k=1}^{2} (-1)^{i+j+k}  \left( 
z_k \ln{\left( y_j + \sqrt{x_i^2+y_j^2+z_k^2} \right)} + 
y_j \ln{\left( z_j + \sqrt{x_i^2+y_j^2+z_k^2} \right)} - 
x_i \arctan{\frac{y_j z_k}{x_i \sqrt{x_i^2+y_j^2+z_k^2}}} \right) \nn
\end{eqnarray}

The same can be done for the $y-$ and $z-$components and in the end we obtain
\begin{mdframed}[backgroundcolor=blue!5]
\begin{eqnarray}
g_x &=& {\cal G}  \rho \displaystyle\sum_{i=1}^{2} \displaystyle\sum_{j=1}^{2} \displaystyle\sum_{k=1}^{2} (-1)^{i+j+k} 
\left( 
z_k \ln{\left( y_j + r_{ijk} \right)} + 
y_j \ln{\left( z_j + r_{ijk} \right)} - 
x_i \arctan{\frac{y_j z_k}{x_i r_{ijk}}} 
\right) \nonumber\\
g_y &=& 
{\cal G}  \rho \displaystyle\sum_{i=1}^{2} \displaystyle\sum_{j=1}^{2} \displaystyle\sum_{k=1}^{2} (-1)^{i+j+k} 
\left( 
z_k \ln{\left( x_i + r_{ijk} \right)} + 
x_i \ln{\left( z_j + r_{ijk} \right)} - 
y_j \arctan{\frac{x_i z_k}{y_j r_{ijk}}} 
\right) \nonumber\\
g_z &=& 
{\cal G}  \rho \displaystyle\sum_{i=1}^{2} \displaystyle\sum_{j=1}^{2} \displaystyle\sum_{k=1}^{2} (-1)^{i+j+k} 
\left( 
x_i \ln{\left( y_j + r_{ijk} \right)} + 
y_j \ln{\left( x_i + r_{ijk} \right)} - 
z_k \arctan{\frac{x_i y_j}{z_k r_{ijk}}} 
\right) \nonumber
\end{eqnarray}
\end{mdframed}

These equations can be found in various other papers such as Eq.~(6) in Heck and Seitz, 2007 \cite{hese07}, Eqs.~(8,11,12) in Nagy \etal, 2000 \cite{napb00} 
(note that there is a mistake there that is later fixed in \cite{napb02}), 
appendix A in Couder-Castaneda \etal., 2015 \cite{cooo15} 
and the derivation between (14) and (15) in Mader, 1951 \cite{made51}.

%-------------------------------------------------------
\subsection{The gravity gradient tensor} % (${\bm T}$)}

The different elements of the gravity gradient tensor can be determined by partially differentiating each component of the 
gravity vector with respect to each space coordinate. As will be shown later, ${\bm T}$ should be a symmetric matrix and its trace should equal zero.

%...............................
\subsubsection{The diagonal terms}

\begin{eqnarray}
T_{xx} 
&=& \frac{\partial}{\partial x} g_x 
= -\frac{\partial^2}{\partial x^2} U(\vec{r}) 
= {\cal G} \rho \frac{\partial^2}{\partial x^2} \left( -I(\vec{r}) \right)
\end{eqnarray}

\begin{eqnarray}
I_{xx}(\vec{r}) 
&=& \iiint \frac{\partial^2}{\partial x^2} \frac{dx dy dz}{\sqrt{x^2+y^2+z^2}} \nonumber\\
&=& \iint \frac{\partial}{\partial x} \frac{dy dz}{\sqrt{x^2+y^2+z^2}} \nonumber\\
&=& - \iint \frac{x dy dz}{\sqrt{x^2+y^2+z^2}^3}
\end{eqnarray}

A trigonometric substitution is applied to solve this integral. 
This uses $y = \sqrt{x^2+z^2}\tan{\phi}$ and $dy = \frac{\sqrt{x^2+y^2}}{\cos^2{\phi}}d\phi$.

\begin{eqnarray}
I_{xx} 
&=& -x \iint \frac{\sqrt{x^2+y^2}}{\cos^2{\phi}} 
\frac{d\phi dz}{\sqrt{(x^2+z^2)\tan^2{\phi}+x^2+z^2}^3} \nonumber\\
&=& -x \iint \frac{\sqrt{x^2+y^2}}{\cos^2{\phi}} 
\frac{d\phi dz}{\sqrt{(x^2+z^2)(\tan^2{\phi}+1)}^3} \nonumber\\
&=& -x \iint \frac{\sqrt{x^2+y^2}}{\cos^2{\phi}} 
\frac{d\phi dz}{\left( \frac{\sqrt{(x^2+z^2)}}{\cos{\phi}} \right)^3} \nonumber\\
&=& -x \int \frac{1}{x^2+z^2} \int \cos{\phi} \; d\phi dz \nonumber\\
&=& -x \int \frac{1}{x^2+z^2} \sin{\phi} dz
\end{eqnarray}
Now the substitution needs to be undone. If $\tan{\phi} = \frac{y}{\sqrt{x^2+z^2}}$, then $\sin{\phi} = \frac{y}{\sqrt{x^2+y^2+z^2}}$ and then
\begin{equation}
I_{xx} = -xy \int \frac{dz}{(x^2+z^2)\sqrt{x^2+y^2+z^2}}
\end{equation}
This can be solved by applying equation \eqref{eq:basic3}.
\begin{eqnarray}
I_{xx} 
&=& -\frac{xy}{xy} \arctan{\frac{yz}{x\sqrt{x^2+y^2+z^2}}} \nonumber\\
&=& -\arctan{\frac{y z}{x \sqrt{x^2+y^2+z^2}}} \nn
\end{eqnarray}

The tensor element $T_{xx}$ is then formulated as follows 
(the other elements of the diagonal are found by cyclic permutation of $x$, $y$ and $z$):

\begin{mdframed}[backgroundcolor=blue!5]
\begin{eqnarray}
T_{xx} &=& {\cal G} \rho \sum_{i=1}^{2} \sum_{j=1}^{2} \sum_{k=1}^{2} (-1)^{i+j+k} 
\left( - \arctan{\frac{y_j z_k}{x_i r_{ijk}}} \right) \nonumber\\
T_{yy} &=& {\cal G} \rho \sum_{i=1}^{2} \sum_{j=1}^{2} \sum_{k=1}^{2} (-1)^{i+j+k} 
\left( - \arctan{\frac{x_i z_k}{y_j r_{ijk}}} \right) \nonumber\\
T_{zz} &=& {\cal G} \rho \sum_{i=1}^{2} \sum_{j=1}^{2} \sum_{k=1}^{2} (-1)^{i+j+k} 
\left( - \arctan{\frac{x_i y_j}{z_k r_{ijk}}} \right) \nonumber
\end{eqnarray}
\end{mdframed}

%....................................................
\subsubsection{The off-diagonal terms of the tensor}

The other elements are easier to calculate, because the partial derivatives cancel out the integrals:

\begin{eqnarray}
I_{xy} 
&=& \iiint \frac{\partial^2}{\partial x \partial y} \frac{dx dy dz}{\sqrt{x^2+y^2+z^2}} \nonumber\\
&=& \iint \frac{\partial}{\partial y} \frac{dy dz}{\sqrt{x^2+y^2+z^2}} \nonumber\\
&=& \int \frac{dz}{\sqrt{x^2+y^2+z^2}} \nonumber\\
&=& \ln{\left(z + \sqrt{x^2+y^2+z^2} \right)}
\end{eqnarray}

\begin{eqnarray}
I_{xz} 
&=& \iiint \frac{\partial^2}{\partial x \partial z} \frac{dx dy dz}{\sqrt{x^2+y^2+z^2}} \nonumber\\
&=& \iint \frac{\partial}{\partial z} \frac{dy dz}{\sqrt{x^2+y^2+z^2}} \nonumber\\
&=& \int \frac{dy}{\sqrt{x^2+y^2+z^2}} \nonumber\\
&=& \ln{\left(y + \sqrt{x^2+y^2+z^2} \right)}
\end{eqnarray}

\begin{eqnarray}
I_{yz} 
&=& \iiint \frac{\partial^2}{\partial y \partial z} \frac{dx dy dz}{\sqrt{x^2+y^2+z^2}} \nonumber\\
&=& \iint \frac{\partial}{\partial z} \frac{dx dz}{\sqrt{x^2+y^2+z^2}} \nonumber\\
&=& \int \frac{dx}{\sqrt{x^2+y^2+z^2}} \nonumber\\
&=& \ln{\left(x + \sqrt{x^2+y^2+z^2} \right)}
\end{eqnarray}

From these calculations it should be obvious why ${\bm T}$ is a symmetric tensor. 
When applying the second partial derivatives, their order does not matter: 
\begin{equation}
I_{xy} 
=\iiint \frac{\partial^2}{\partial x \partial y} \frac{dx dy dz}{\sqrt{x^2+y^2+z^2}} 
=\iiint \frac{\partial^2}{\partial y \partial x} \frac{dx dy dz}{\sqrt{x^2+y^2+z^2}} 
= I_{yx}
\end{equation}

The tensor elements following from this are:
\begin{mdframed}[backgroundcolor=blue!5]
\begin{eqnarray}
T_{xy} = T_{yx} 
&=& 
{\cal G}  \rho \sum_{i=1}^{2} \sum_{j=1}^{2} \sum_{k=1}^{2} (-1)^{i+j+k} 
\left( \ln{\left( z_k + r_{ijk} \right)} \right) \nonumber\\
T_{xz} = T_{zx} 
&=& 
{\cal G}  \rho \sum_{i=1}^{2} \sum_{j=1}^{2} \sum_{k=1}^{2} (-1)^{i+j+k} 
\left( \ln{\left( y_j + r_{ijk} \right)} \right) \nonumber\\
T_{yz} = T_{zy} 
&=& 
{\cal G}  \rho \sum_{i=1}^{2} \sum_{j=1}^{2} \sum_{k=1}^{2} (-1)^{i+j+k} 
\left(\ln{\left( x_i + r_{ijk} \right)} \right) \nonumber
\end{eqnarray}
\end{mdframed}


%-------------------------------------------
\subsection{Revisiting Poisson's equation}

The gravitational potential Poisson equation is $\nabla^2 U = 4\pi {\cal G} \rho$. 
This can and should be verified for the derived equations for prisms. 
Inside the prism, the density has an assigned constant value. 
Outside of the prism, the density is zero, so the result is $\nabla^2 U = 0$. 
These cases will be treated separately.

%................................
\subsubsection{Outside the prism}  

$\nabla^2 U = 0$ can be written 
$\frac{\partial^2 U}{\partial x^2} + \frac{\partial^2 U}{\partial y^2}+\frac{\partial^2 U}{\partial z^2}=0$. 
which is the trace of ${\bm T}$. 
We first add the terms and then the boundary conditions are applied.

We will need the formula to add arctangents together:

\begin{equation}
\arctan{a} + \arctan{b} = \arctan{\frac{a+b}{1-ab}}
\end{equation}
We start by adding the terms $T_{xx}$ and $T_{yy}$ together:
\begin{eqnarray}
T_{xx} + T_{yy} 
&=& (-1)^{i+j+k} \arctan{\frac{y z}{x r}} + (-1)^{i+j+k} \arctan{\frac{x z}{y r}} \nonumber\\
&=& (-1)^{i+j+k} \arctan{\frac{\frac{yz}{xr}+\frac{xz}{yr}}{1-\frac{yz}{xr}\frac{xz}{yr}}} \nonumber\\
&=& (-1)^{i+j+k} \arctan{\frac{\frac{y^2z}{xyr}+\frac{x^2z}{xyr}}{1-\frac{xyz^2}{xyr^2}}} \nonumber\\
&=& (-1)^{i+j+k} \arctan{\frac{\frac{z(x^2+y^2)}{xyr}}{\frac{xy(r^2-z^2)}{xyr^2}}} \nonumber\\
&=& (-1)^{i+j+k} \arctan{\frac{xyzr^2(x^2+y^2)}{x^2y^2r(r^2-z^2)}} \nonumber\\
&=& (-1)^{i+j+k} \arctan{\frac{zr(x^2+y^2)}{xy(x^2+y^2+z^2-z^2)}} \nonumber\\
&=& (-1)^{i+j+k} \arctan{\frac{zr}{xy}}
\end{eqnarray}
By considering a right triangle with sides 1 and $x$, it easy to prove that:
\begin{equation}
\arctan{x} + \arctan{\frac{1}{x}} = \frac\pi2
\end{equation}
This can be used to transform the $\arctan$ to one that is similar to $T_{zz}$.

\begin{equation}
T_{xx} + T_{yy} 
= (-1)^{i+j+k} \arctan{\frac{zr}{xy}} 
= (-1)^{i+j+k} \left( \frac{\pi}{2} - \arctan{\frac{xy}{zr}} \right)
\end{equation}
The last step is to add the term $T_{zz}$:
\begin{equation}
\nabla^2 U 
= T_{xx} + T_{yy} + T_{zz} 
= (-1)^{i+j+k} \left( \frac{\pi}{2} - \arctan{\frac{xy}{zr}} + \arctan{\frac{xy}{zr}}\right) 
= (-1)^{i+j+k} \frac{\pi}{2}
\end{equation}
The end result is a single value. 
When the boundary conditions are applied this single value will be 
subtracted from itself resulting in zero, so $\nabla^2 U(\vec{r})=0$.

%.................................
\subsubsection{Inside the prism}

We can simply put the observation point at the centre of the prism. The coordinates of the prism are now such that $-x_1 = x_2$, $-y_1 = y_2$ and $-z_1 = z_2$.
All eight terms for these conditions results give $\frac{\pi}{2}$, so the result is:
\begin{equation}
    \nabla^2U={\cal G} \rho8\frac{\pi}{2} = 4\pi {\cal G} \rho 
\end{equation}

%--------------------------------------
\subsection{Better numerical stability}

Heck and Seitz (2007) \cite{hese07} modify the standard formulae for 
the prism to get a better numerical stability in the logarithms. 
This is done by dividing the inside of the logs by an extra factor:
\[
\ln{\left( z_k + r_{ijk} \right)} \to \ln{\frac{z_k + r_{ijk}}{\sqrt{x_i^2+y_j^2}}}
\]

This extra factor disappears when applying the boundary conditions in the $z$ direction, 
so that the results remain identical: 

\begin{eqnarray}
\left| 
\ln{\frac{z_k + r_{ijk}}{\sqrt{x_i^2+y_j^2}}} 
\right|^{z_2}_{z_1} 
&=& 
\left| 
\ln{\left(z_k + r_{ijk}\right)} - 
\ln{\sqrt{x_i^2+y_j^2}} 
\right|^{z_2}_{z_1} \nonumber\\
&=& 
\ln{\left(z_2 + r_{ijk}\right)} - 
\ln{\sqrt{x_i^2+y_j^2}} - 
\ln{\left(z_1 + r_{ijk}\right)} + 
\ln{\sqrt{x_i^2+y_j^2}} \nonumber\\
&=& 
\ln{\left(z_2 + r_{ijk}\right)} - 
\ln{\left(z_1 + r_{ijk}\right)}
\end{eqnarray}

