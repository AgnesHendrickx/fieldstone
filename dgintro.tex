\begin{flushright} {\tiny {\color{gray} dgintro.tex}} \end{flushright}
%~~~~~~~~~~~~~~~~~~~~~~~~~~~~~~~~~~~~~~~~~~~~~~~~~~~~~~~~~~~~~~~~~~~~~~~~~~~~~~~~~~~~~~~~~~~~~~~~~~

\paragraph{What is DG?}

\begin{itemize}
\item it is a variant of the SG ("Standard Galerkin FEM")\footnote{Some authors use the acronym 
CG for Continuous Galerkin but since the Conjugate Gradient solver acronym CG is very much present in FE codes it can 
be confusing so we use here SG instead.}
\item SG-FEM requires continuity of the solution along element interfaces (edges).
\item DG-FEM does not require continuity of the solution along edges.
\item DG methods have more degrees of freedom than SG methods.
\item DG-FEM shares some properties with FVM
\end{itemize}

\paragraph{Various books about DG-FEM}

\begin{itemize}
\item {\it Discontinuous Galerkin Methods. Theory, Computation and Applications} by
Cockburn, Karniadakis and Shu \cite{cockburn00}
\item {\it Mathematical Aspects of Discontinuous Galerkin Methods} by Di Pietro and Ern 
\cite{dipietro_ern12}
\item {\it Discontinuous Galerkin Methods. Analysis and Applications to Compressible Flow} by 
Dolejsi and Feistauer \cite{dolejsi_feistauer15}
\item {\it Discontinuous Galerkin Methods for Solving Elliptic and Parabolic Equations} by Rivi{\'e}re
\cite{riviere08}
\item {\it Discontinuous finite elements in fluid dynamics and heat transfer} by Li \cite{li06}
\item {\it Nodal Discontinuous Galerkin Methods. Algorithms, Analysis, and Applications} by 
Hesthaven \& Warbuton \cite{hewa08}
\end{itemize}

\paragraph{DG flavors}
There are many different flavours of the Discontinuous Galerkin Finite Element Method:
\begin{itemize}
\item {\bf HDG}: Hybridizable DG \cite{cogo09,conp10,ngpc10,ngpc11,ngpe12}
\item {\bf IPG}: Interior Penalty G  \cite{mofh08,mofp10}
\item {\bf IIPG}: Incomplete Interior Penalty G  \cite{dole08}
\item {\bf SIPG}: Symmetric Interior Penalty G  \cite{bodi11,sclu17a}
\item {\bf LDG}: Local DG \cite{cacp02,coks02,cacs05,coks05}  
\end{itemize}

\paragraph{Pro and cons for DG-FEM versus SG-FEM}

\begin{itemize}
\item Assembly of stiffness matrix is easier to implement (ref?).
\item Refinement of triangles is easier to implement (ref?). 
\item DG methods can easily handle adaptivity strategies since refinement or
unrefinement of the grid can be achieved without taking into account
the continuity restrictions typical of conforming finite element methods. 
Moreover, the degree of the approximating polynomial can be easily
changed from one element to the other. \cite{coks00}
\item DG methods can support high order local approximations that can vary nonuniformly over the mesh.
\item DG methods are readily parallelizable. Since the elements are discontinuous, 
the mass matrix is block diagonal and since the size of the blocks
is equal to the number of degrees of freedom inside the corresponding
elements, the blocks can be inverted by hand once and for all.\cite{coks00}
\end{itemize}

\paragraph{The DG-FEM in geodynamics} 
This method has not been used extensively in geodynamics with (so-far) two 
noticeable exceptions:

\begin{itemize}
\item Lehmann \etal, Comparison of continuous and discontinuous Galerkin approaches
for variable-viscosity Stokes flow (2015) \cite{lelk15} 
\item He \etal, A discontinuous Galerkin method with a bound preserving limiter for
the advection of non-diffusive fields in solid Earth geodynamics (2017) \cite{hepb17}
\item Puckett \etal, New numerical approaches for modeling thermochemical convection in a
compositionally stratified fluid (2018) \cite{puth18}
\end{itemize}


