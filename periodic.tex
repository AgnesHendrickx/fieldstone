
This type of boundary conditions can be handy in some specific cases such 
as infinite domains. The idea is simple: when material leaves the domain 
through a boundary it comes back in through the opposite boundary (which 
of course presupposes a certain topology of the domain). 

For instance, if one wants to model a gas a the molecular level and wishes 
to avoid interactions of the molecules with the walls of the container, 
such boundary conditions can be used, mimicking an infinite domain in all 
directions. 

Let us consider the small mesh depicted hereunder:

We wish to implement horizontal boundary conditions so that 
\[
u_5=u_1
\quad\quad
u_{10}=u_6
\quad\quad
u_{15}=u_{11}
\quad\quad
u_{20}=u_{16}
\]
One could of course rewrite these conditions as constraints and extend the Stokes 
matrix but this approach turns out to be not practical at all. 

Instead, the method is rather simple: replace in the connectivity array the dofs on the right side
(nodes 5, 10, 15, 20) by the dofs on the left side. In essence, we wrap the system upon itself 
in the horizontal direction so that elements 4, 8 and 12 'see' and are 'made of' the nodes 1, 6, 11 and 16.
In fact, this is only necessary during the assembly. Everywhere in the loops nodes 5, 10, 15 and 20 appear 
one must replace them by their left pendants 1, 6, 11 and 16. This autmatically generates a matrix 
with lines and columns corresponding to the $u_5$, $u_{10}$, $u_{15}$ and $u_{20}$ being exactly zero. 
The Stokes matrix is the same size, the blocks are the same size and the symmetric character of the matrix 
is respected. However, there is a remaining problem. There are zeros on the diagonal 
of the above mentioned lines and columns. One must then place there 1 or a value more
appropriate.

Another way of seeing this is as follows: let us assume we have built and assembled
the Stokes matrix, and we want to impose periodic b.c. so that dof $j$ and $i$ are the same. 
The algorithm is composed of four steps:
\begin{enumerate} 
\item add col $j$ to col $i$
\item add row $j$ to row $i$ (including rhs)
\item zero out row $j$, col $j$
\item put average diagonal value on diagonal ($j,j$)
\end{enumerate} 

\begin{remark}
Unfortunately the non-zero pattern of the matrix with periodic b.c. is not the same 
as the matrix without periodic b.c.
\end{remark}


