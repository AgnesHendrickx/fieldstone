\index{general}{Static Condensation}
\begin{flushright} {\tiny {\color{gray} static\_condensation.tex}} \end{flushright}

The idea behind static condensation is quite simple: in some cases, there are dofs 
belonging to an element which only belong to that element. For instance, the so-called MINI 
element ($P_1^+ \times P_1$) showcases a bubble function in the middle (see section \ref{ss:pair}). 
In the following, $\vec{\cal V}^\star$ corresponds to the list of such dofs inside an element.
The discretised Stokes equations on any element looks like:

\begin{equation}
\left(
\begin{array}{ccc}
\K   & L & \G \\
L^T & \K^\star  & H \\
\G^T & H^T & 0
\end{array}
\right)_e
\left(
\begin{array}{c}
\vec{\cal V} \\ \vec{\cal V}^\star \\ \vec{\cal P}
\end{array}
\right)_e
=
\left(
\begin{array}{c}
\vec{f} \\ \vec{f}^\star \\ \vec{h}
\end{array}
\right)_e
\end{equation}
This is only a re-writing of the elemental Stokes matrix where the matrix $\K$ has been 
split in four parts.
Note that the matrix $\K^\star$ is diagonal.\todo{check}

This can also be re-written in non-matrix form:
\begin{eqnarray}
\K \cdot \vec{\cal V} + L \cdot \vec{\cal V}^\star + \G \cdot \vec{\cal P} &=& \vec{f} \\
L^T V + K^\star \cdot  \vec{\cal V}^\star + H \cdot \vec{\cal P} &=& \vec{f}^\star \\
\G^T \cdot \vec{\cal V} + H^T \vec{\cal V}^\star &=& \vec{h}
\end{eqnarray}
The $\vec{\cal V}^\star$ in the second equation can be isolated:
\[
\vec{\cal V}^\star = \K^{-\star} \cdot ( \vec{f}^\star - L^T \cdot \vec{\cal V} - H \cdot \vec{\cal P})
\]
and inserted in the first and third equations:
\begin{eqnarray}
\K \cdot \vec{\cal V} + L \left[ \K^{-\star} ( \vec{f}^\star - L^T \cdot \vec{\cal V} - H \cdot \vec{\cal P} )  \right] + \G \cdot \vec{\cal P} &=& \vec{f} \\
\G^T \cdot \vec{\cal V} + H^T \left[  \K^{-\star} ( \vec{f}^\star - L^T \cdot \vec{\cal V} - H \cdot \vec{\cal P}) \right]  &=& \vec{h}
\end{eqnarray}
or,
\begin{eqnarray}
(\K-L\cdot \K^{-\star} \cdot L^T)\cdot \vec{\cal V} + (G-L\cdot \K^{-\star} \cdot H) \cdot \vec{\cal P} &=& \vec{f}-L\cdot \K^{-\star} \cdot \vec{f}^\star \\
(G^T -H^T\cdot \K^{-\star}\cdot  L^T ) \cdot \vec{\cal V}  - 
(H^T \cdot \K^{-\star} \cdot H )\cdot \vec{\cal P}   &=& \vec{h} -H^T\cdot \K^{-\star}\cdot \vec{f}^\star
\end{eqnarray}
i.e.
\begin{eqnarray}
\underline{\K} \cdot \vec{\cal V} + \underline{\G}\cdot \vec{\cal P} &=& \underline{\vec{f}} \\
\underline{\G}^T \cdot \vec{\cal V} - \underline{\C} \cdot \vec{\cal P} &=& \underline{\vec{h}}
\end{eqnarray}
with
\begin{eqnarray}
\underline{\K}&=& K-L\cdot \K^{-\star} \cdot L^T \\
\underline{\G}&=& G-L\cdot \K^{-\star} \cdot H \\
\underline{\C}&=& H^T \cdot \K^{-\star} \cdot H \\
\underline{\vec{f}}&=& \vec{f}-L\cdot \K^{-\star} \cdot \vec{f}^\star \\
\underline{\vec{h}}&=& \vec{h} -H^T\cdot \K^{-\star}\cdot \vec{f}^\star
\end{eqnarray}
Note that $\underline{\K}$ is symmetric, and so is the Stokes matrix.


For instance, in the case of the MINI element, the dofs corresponding to the bubble 
could be eliminated at the elemental level, which would make the Stokes matrix smaller
(see book by Braess \cite{braess}). 
However, it is then important to note that static condensation introduces a 
pressure-pressure term which was not there in the original formulation.









