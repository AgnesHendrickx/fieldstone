\index{general}{pressure scaling}
\begin{flushright} {\tiny {\color{gray} pressure\_scaling.tex}} \end{flushright}
%~~~~~~~~~~~~~~~~~~~~~~~~~~~~~~~~~~~~~~~~~~~~~~~~~~~~~~~~~~~~~~~~~~~~~~~~~~~~~~~~~~~~~~~~~~~~~~~~~~

As nicely explained in the 
step 32 of deal.ii\footnote{\url{https://www.dealii.org/9.0.0/doxygen/deal.II/step\_32.html}},
we often need to scale the $\G$ block since it is many orders of magnitude smaller than $\K$, 
which introduces large inaccuracies in the solving process to the point that the solution is nonsensical. 
This scaling coefficient is $\eta/L$ where $\eta$ and $L$ are representative viscosities and lengths. 
We start from 
\[
\left(
\begin{array}{cc}
\K & \G \\ \G^T & -\C 
\end{array}
\right)
\cdot
\left(
\begin{array}{c}
\vec{\cal V} \\ \vec{\cal P}
\end{array}
\right)
=
\left(
\begin{array}{c}
\vec{f} \\ \vec{h}
\end{array}
\right)
\]
and introduce the scaling coefficient as follows (which in fact does not alter the solution at all):
\[
\left(
\begin{array}{cc}
\K & \frac{\eta}{L}\G \\ \frac{\eta}{L}\G^T & - \frac{\eta^2}{L^2} \C 
\end{array}
\right)
\cdot
\left(
\begin{array}{c}
\vec{\cal V} \\\frac{L}{\eta} \vec{\cal P}
\end{array}
\right)
=
\left(
\begin{array}{c}
 \vec{f} \\ \frac{\eta}{L} \vec{h}
\end{array}
\right)
\]
We then end up with the modified Stokes system:
\[
\left(
\begin{array}{cc}
\K & \underline{\G} \\ \underline{\G}^T & \underline{\C} 
\end{array}
\right)
\cdot
\left(
\begin{array}{c}
\vec{\cal V} \\ \underline{\vec{\cal P}}
\end{array}
\right)
=
\left(
\begin{array}{c}
\vec{f} \\ \underline{\vec{h}}
\end{array}
\right)
\]
where 
\[
\underline{\G}=\frac{\eta}{L}\G
\quad\quad
\quad\quad
\underline{\vec{\cal P}}=\frac{L}{\eta} \vec{\cal P}
\quad\quad
\quad\quad
\underline{\C}=\frac{\eta^2}{L^2} \C
\quad\quad
\quad\quad
\underline{\vec{h}}=\frac{\eta}{L}\vec{h}
\]
After the solve phase, we recover the real pressure with $\vec{\cal P}=\frac{\eta}{L}\underline{\vec{\cal P}}$.





