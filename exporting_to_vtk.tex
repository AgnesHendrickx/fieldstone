
This format seems to be the universally accepted format for 2D and 3D visualisation in 
Computational Geodynamics. Such files can be opened with free softwares such as 
Paraview \footnote{https://www.paraview.org/}, MayaVi \footnote{https://docs.enthought.com/mayavi/mayavi/}
or Visit \footnote{https://wci.llnl.gov/simulation/computer-codes/visit/}.

Unfortunately it is my experience that no simple tutorial exists about how to build 
such files. There is an official document which describes the vtk 
format\footnote{https://www.vtk.org/wp-content/uploads/2015/04/file-formats.pdf}
but it delivers the information in a convoluted way. I therefore describe hereafter 
how \fieldstone{} builds the vtk files. 

I hereunder show vtk file corresponding to the 3x2 grid presented earlier \ref{subsection_meshes}.
In this particular example there are:
\begin{itemize}
\item 12 nodes and 6 elements
\item 1 elemental field: the pressure {\tt p})
\item 2 nodal fields: 1 scalar (the smoothed pressure {\tt q}), 1 vector (the velocity field {\tt u,v,0})
\end{itemize}
Note that vtk files are inherently 3D so that even in the case of a 2D simulation the $z$-coordinate 
of the points and for instance their $z$-velocity component must be provided.
The file, usually called {\sl solution.vtu} starts with a header:

\lstinputlisting[language=python,firstline=1,lastline=3]{images/vtk/solution.vtu}

We then proceed to write the node coordinates as follows:

\lstinputlisting[language=python,firstline=4,lastline=19]{images/vtk/solution.vtu}

These are followed by the elemental field(s):

\lstinputlisting[language=python,firstline=20,lastline=29]{images/vtk/solution.vtu}

Nodal quantities are written next:

\lstinputlisting[language=python,firstline=30,lastline=59]{images/vtk/solution.vtu}

To these informations we must append 3 more datasets. The first one is the connectivity, 
the second one is the offsets and the third one is the type. The first one is trivial
since said connectivity is needed for the Finite Elements. The second must be understood as follows:
when reading the connectivity information in a linear manner the offset values 
indicate the beginning of each element (omitting the zero value). The third simply is the type of element 
as given in the vtk format document (9 corresponds to a generic quadrilateral with an 
internal numbering consistent with ours). 

\lstinputlisting[language=python,firstline=60,lastline=85]{images/vtk/solution.vtu}

The file is then closed with

\lstinputlisting[language=python,firstline=86,lastline=88]{images/vtk/solution.vtu}

The {\sl solution.vtu} file can then be opened with ParaView, MayaVi or Visit and the reader 
is advised to find tutorials online on how to install and use these softwares. 



