Scalars such as temperature, density, pressure, etc ... are simply 
obtained in \LaTeX{} by using the math mode, e.g. $T$, $\rho$, $p$.
Although it is common to lump vectors and matrices/tensors together
by using bold fonts, I have decided in the interest of clarity to 
distinguish between those: vectors are denoted by an arrow 
atop the quantity, e.g. $\vec \upnu$, $\vec g$, while matrices 
and tensors are in bold $\bm M$, $\bm \sigma$, etc ...

Also I use the $\cdot$ notation between two vectors to denote a 
dot product $\vec u \cdot \vec v = u_iv_i$ or a matrix-vector
multiplication ${\bm M}\cdot \vec a = M_{ij}a_j$. If there is no
$\cdot$ between vectors, it means that the result 
$\vec a \vec b = a_ib_j$ is a matrix (it is a dyadic 
product\footnote{\url{https://en.wikipedia.org/wiki/Dyadics}}).
Case in point, $\vec\nabla\cdot\vec\upnu$ is the velocity divergence
while $\vec\nabla\vec\upnu$ is the velocity gradient tensor.
