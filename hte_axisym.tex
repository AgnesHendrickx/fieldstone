We start from
\[
\rho C_p \left(\frac{\partial T}{\partial t} 
+\vec\upnu\cdot\vec\nabla T \right)
= k \Delta T
\]
The temperature gradient in cylindrical coordinates is 
\[
\vec\nabla T = 
\left(
\begin{array}{c}
\partial_r T \\
\frac{1}{r}\partial_\theta T \\
\partial_z T 
\end{array}
\right)
\]
Since $\upnu_\theta=0$ and also $\partial_\theta T=0$ then 
\[
\vec\upnu\cdot\vec\nabla T = \upnu_r \frac{\partial T}{\partial r}
+ \upnu_z \frac{\partial T}{\partial z}
\]
and we have the Laplace operator (terms in $\partial_\theta$ have 
been left out):
\[
\Delta T = \frac{1}{r} \frac{\partial }{\partial r}
\left(r \frac{\partial T}{\partial r} \right)
+\frac{\partial^2 T }{\partial z^2}
\]
However for the FE formulation we will formulate the equation as
\[
\rho C_p \left(\frac{\partial T}{\partial t} 
+\vec\upnu\cdot\vec\nabla T \right)
= \vec\nabla \cdot (k \vec\nabla T)
\]
After multiplying this equation by a test function and integrating over the domain, the diffusion term is integrated by parts (surface terms are per usual discarded), and we finally obtain
\[
{\bm M} \cdot \frac{\partial \vec{\cal T}}{\partial t}
+
({\bm K}_a + {\bm K}_d ) \cdot \vec{\cal T} = \vec{0}
\]
with
\begin{eqnarray}
{\bm K}_a &=& \int \rho C_p \vec    {N}^T (\vec\upnu\cdot {\bm B}) \; dV \\
{\bm K}_d &=& \int k {\bm B}^T \cdot {\bm B} \; dV 
\end{eqnarray}
where the matrix ${\bm B}$ is identical in this case to the 2D Cartesian one.

It looks like switching from 2D Cartesian to 3D cylindrical axisymmetric does not introduce any change in the formulation.
A bit too good to be true ? 
