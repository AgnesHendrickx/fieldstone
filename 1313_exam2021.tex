
%---------------------------------
\subsection*{Exercise 1}


Let us consider a spherically symmetric body of radius $R$.

\begin{enumerate}
\item (1 pt) use symmetry considerations with regards to the 3 axis
to arrive at the moment of inertia $I$:
\begin{equation}
I=\frac{8\pi}{3} \int_0^R \rho(r) r^4 dr 
\end{equation}
\item (1 pt) The density inside the sphere is given by
\[
\rho(r)=a \frac{r}{R} +b
\]
Compute the total mass $M$ of the planet using this expression for the density.
\item (1 pt) Compute the moment of inertia $I$ also as a function of $a$ and $b$.
\item (1/2 pt) Can $I$ be written $I=fMR^2$? if so, give $f$.
\item (1/2 pt) What are the dimensions of $a$, $b$, $\rho$, $I$ and $M$?
\item (1/2 pt) Set $a=0$ and $b=\rho_0$ and look again at $M$ and $I$. Conclude.
\end{enumerate}

%-----------------------------------------
\subsection*{Exercise 1 - answer}

The first question is answered in Problem 1, see Section~\ref{sect_scalarmomint}.

The mass of the planet is given by
\begin{eqnarray}
M
&=& \iiint_V \rho(r) dV \nn\\
&=& \iiint_V \rho(r) r^2 \sin\theta dr d\theta d\phi \nn\\
&=& \iiint_V \left(a\frac{r}{R}+b\right) r^2 \sin\theta dr d\theta d\phi \nn\\
&=& 4\pi \int_0^R \left(a\frac{r}{R}+b\right) r^2 dr \nn\\
&=& 4\pi \left[ \frac{a}{R} \int_0^R r^3 dr +b\int_0^R  r^2 dr \right]\nn\\
&=& 4\pi \left[ \frac{a}{R} \frac14 R^4  + b\frac13 R^3 \right]\nn\\
&=& 4 \pi R^3 \left( \frac{a}{4} + \frac{b}{3} \right)
\end{eqnarray}


The moment of inertia of the planet is given by
\begin{eqnarray}
I
&=& \frac{8\pi}{3} \iiint_0^R \rho(r) r^4 dV \nn\\
&=& \frac{8\pi}{3} \iiint_0^R \left(a\frac{r}{R}+b\right) r^4 dV \nn\\
&=& \frac{8\pi}{3} \left( \frac{a}{6} + \frac{b}{5} \right) R^5 \nn\\
&=& \underbrace{(4 \pi R^3 ) \left( \frac{a}{4} + \frac{b}{3} \right)}_{M}
\left( \frac{a}{4} + \frac{b}{3} \right)^{-1} 
R^2  \frac23\left( \frac{a}{6} + \frac{b}{5} \right) \nn\\
&=& \underbrace{\left( \frac{a}{4} + \frac{b}{3} \right)^{-1}
\frac23 \left( \frac{a}{6} + \frac{b}{5} \right)}_{f} M R^2
\end{eqnarray}

The dimensions of $a$, $b$, $I$ and $M$ are as follows:
\[
[a]=[b]=ML^{-3}
\qquad
[I]= ML^2
\qquad
[M]=M
\]

When $a=0$ and $b=\rho_0$ we find the standard mass of a constant density sphere 
$M=\frac43 \pi R^3 \rho_0$ and $f=2/5$.


%---------------------------------
\subsection*{Exercise 2}





Let us consider the same sphere as in the previous exercise, and the same density profile $\rho(r)$.
The gravitational potential satisfies the Poisson equation:
\begin{equation}
\Delta U = 4 \pi {\cal G} \rho({\vec r}) \label{eqlap}
\end{equation}
and we have the following relationship between the gravitational acceleration 
vector and the potential: ${\vec g}=-{\vec \nabla} U$.

\begin{enumerate}
\item (1/2 pt) Write explicitely Eq.(\ref{eqlap}) for a point inside the sphere and a point outside the sphere.
\item (1 pt) Compute $g(r)$ and $U(r)$ for a point inside the sphere as a function of $r$. Use
$\lim_{r\rightarrow 0} g(r) \neq \infty$
to get rid of an integration constant.
\item (1 pt) Compute $g(r)$ and $U(r)$ for a point outside the sphere as a function of $r$.
Use $\lim_{r\rightarrow\infty}U(r)=0$ to get rid of another integraion constant.
\item (1 pt) Use the continuity of $g(r)$ and $U(r)$ at $r=R$
to compute the last remaining integration constants.
\item (1/2 pt) Set $a=0$ and $b=\rho_0$ and sketch the obtained fields.
\end{enumerate}



%---------------------------------
\subsection*{Exercise 2 - answer}

outside $\Delta U=0$. Inside $\Delta U= 4\pi {\cal G} (a\frac{r}{R}+b)$

\begin{equation}
\frac{1}{r^2} \frac{\partial}{\partial r} \left( r^2 \frac{\partial U}{\partial r} \right) = 4 \pi {\cal G} 
\left( a \frac{r}{R} + b \right) \nn
\end{equation}

\begin{equation}
\Rightarrow \qquad
\frac{\partial}{\partial r} \left( r^2 \frac{\partial U}{\partial r} \right) = 4 \pi {\cal G} 
 \left( a \frac{r^3}{R} + b r^2 \right) \nn
\end{equation}

\begin{equation}
\Rightarrow \qquad
r^2 \frac{\partial U}{\partial r} = 4 \pi {\cal G} \left(a \frac{r^4}{4R} + b \frac{r^3}{3} \right) + A  \nn
\end{equation}

\begin{equation}
\Rightarrow \qquad
\frac{\partial U}{\partial r} 
= 4 \pi {\cal G} \left( a\frac{r^2}{4R} + b \frac{r}{3} \right) + \frac{A}{r^2} \nn
\end{equation}
so that 
\begin{equation}
g(r)=-\frac{\partial U}{\partial r} = -4 \pi {\cal G} \left[ a \frac{r^2}{4R} + b \frac{r}{3} \right] + \frac{A}{r^2} \nn
\end{equation}
We use $\lim_{r\rightarrow 0} g(r) \neq \infty$ to arrive at $A=0$.
Finally 
\begin{equation}
g_{in}(r)= -4 \pi {\cal G} \left( a \frac{r^2}{4R} + b \frac{r}{3} \right)
\end{equation}
and 
\[
U_{in}= - \int g(r) dr = 4 \pi {\cal G} \left( a \frac{r^3}{12R} + b \frac{r^2}{6} \right) + B
\]
Outside the sphere we must solve
\begin{equation}
\frac{1}{r^2} \frac{\partial}{\partial r} \left( r^2 \frac{\partial U}{\partial r} \right) = 0 \nn
\end{equation}

\begin{equation}
\Rightarrow \qquad
r^2 \frac{\partial U}{\partial r}  = C \nn 
\end{equation}
\begin{equation}
\Rightarrow \qquad
\frac{\partial U}{\partial r}  = \frac{C}{r^2} \nn
\end{equation}
\begin{equation}
\Rightarrow \qquad
U_{out}(r) = -\frac{C}{r} + D  \nn
\end{equation}
We use $\lim_{r\rightarrow\infty}U(r)=0$ to arrive at $D=0$ so that 
\begin{eqnarray}
g_{out}(r) &=& -\frac{C}{r^2}  \nn\\
U_{out}(r) &=& -\frac{C}{r} \nn
\end{eqnarray}
Both fields should match at $r=R$:
\begin{eqnarray}
g_{in}(r=R) &=& g_{out}(r=R) \nn\\
U_{in}(r=R) &=& U_{out}(r=R) \nn
\end{eqnarray}
i.e.
\begin{equation}
 -4 \pi {\cal G} \left( a \frac{R^2}{4R} + b \frac{R}{3} \right) 
= -\frac{C}{R^2} 
\end{equation}
\[
\Rightarrow \qquad
C = 4 \pi {\cal G} R^3 \left( \frac{a}{4}+ \frac{b}{3} \right) = M {\cal G}
\]
so unsurprisingly (!):
\begin{eqnarray}
g_{out}(r) &=& -\frac{M{\cal G}}{r^2}  \nn\\
U_{out}(r) &=& -\frac{M{\cal G}}{r} \nn
\end{eqnarray}


and 
\[
4 \pi {\cal G} \left( a \frac{R^3}{12R} + b \frac{R^2}{6} \right) + B 
= -\frac{M{\cal G}}{R}
\]
\[
4 \pi {\cal G}  R^2 \left( a \frac{1}{12} + b \frac{1}{6} \right) + B 
= -\frac{M {\cal G} }{R}
\]
\[
B = 
-\frac{M {\cal G} }{R}
- 4 \pi {\cal G}  R^2 \left( a \frac{1}{12} + b \frac{1}{6} \right) 
\]
so 
\[
U_{in}= 4 \pi {\cal G} \left( a \frac{r^3}{12R} + b \frac{r^2}{6} \right) 
-\frac{M {\cal G} }{R}
- 4 \pi {\cal G}   R^2 \left( a \frac{1}{12} + b \frac{1}{6} \right) 
\]

If $a=0$ and $b=\rho_0$, then 
\[
g_{in}(r) = -4 \pi {\cal G} \rho_0 \frac{r}{3} 
\]
and 
\[
U_{in}= 4 \pi {\cal G} \left( \rho_0 \frac{r^2}{6} \right) 
-\frac{M {\cal G} }{R}
- 4 \pi {\cal G}   R^2 \left( \rho_0 \frac{1}{6} \right) 
= \frac{2 \pi}{3} {\cal G} \rho_0 r^2 
-\frac32 \frac{M {\cal G} }{R}
\]
which is the solution to problem 14. 




