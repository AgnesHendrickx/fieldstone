
\index{general}{Maxwell Time}
\begin{flushright} {\tiny {\color{gray} evrheo.tex}} \end{flushright}
%~~~~~~~~~~~~~~~~~~~~~~~~~~~~~~~~~~~~~~~~~~~~~~~~~~~~~~~~~~~~~~~~~~~~~~~~~~~~~~~~~~~~~~~~~~~~~~~~~~

A viscoelastic material can behave both elastically and viscously. Its response to 
an applied stress is dependent on the material properties and can be found using 
the Maxwell time ($t_M$) of said material. 
This material constant is defined as the ratio of the material viscosity and shear modulus
\[
t_M = \frac{\eta}{\mu}
\]
where $\eta$ is the viscosity and $\mu$ the shear modulus.

The total deviatoric strainrate tensor can be decomposed into an elastic component and a viscous component:
\[
\dot{{\bm \varepsilon}}^d(\vec\upnu) = 
\dot{{\bm \varepsilon}}^d_e(\vec\upnu)  + \dot{{\bm \varepsilon}}^d_v(\vec\upnu)= 
\frac{\tilde{\dot{\bm \tau}}}{2\mu}
+\frac{\bm \tau}{2 \eta}
\]
%\begin{remark}
%The deviatoric strain rate tensor should be denoted by $\dot{\bm \varepsilon}^d$. However, 
%because of the time superscripts that enter the equations a bit later one, we temporarily 
%choose to denote it by $\dot{\underline{\bm \varepsilon}}^d$ in this section. 
%\end{remark}

[From wikipedia] In continuum mechanics, objective stress rates are time derivatives of 
stress that do not depend on the frame of reference. 
Many constitutive equations are designed in the form of a relation between a 
stress-rate and a strain-rate (or the rate of deformation tensor). i
The mechanical response of a material should not depend on the frame of reference. 
In other words, material constitutive equations should be frame indifferent (objective). 
If the stress and strain measures are material quantities then objectivity is automatically 
satisfied. However, if the quantities are spatial, then the objectivity of the stress-rate 
is not guaranteed even if the strain-rate is objective.

There are numerous objective stress rates in continuum mechanics - 
all of which can be shown to be special forms of Lie derivatives. 
Some of the widely used objective stress rates are \cite{holm20}:
a) the Truesdell rate of the Cauchy stress tensor,
b) the Green–Naghdi rate of the Cauchy stress, and
c) the Jaumann rate of the Cauchy stress.

The Jaumann rate of the Cauchy stress is a further specialization of 
the Lie derivative (Truesdell rate). This rate has the form
\[
\tilde{\dot{\bm \tau}}^{t+\delta t} 
= \frac{D {\bm \tau}}{Dt}
- ( \dot{\bm \omega}(\vec\upnu^t)\cdot {\bm \tau}^t - {\bm \tau}^t \cdot \dot{\bm \omega}(\vec\upnu^t)   )
= \frac{ {\bm \tau}^{t+\delta t} - {\bm \tau}^t }{ \delta t} 
- ( \dot{\bm \omega}(\vec\upnu^t)\cdot {\bm \tau}^t - {\bm \tau}^t \cdot \dot{\bm \omega}(\vec\upnu^t)   )
\]
where $D/Dt$ is the material derivative and $\dot{\bm \omega}$ is the rotation rate -also called spin tensor- which is anti-symmetric and has zero trace - see Section~\ref{ss:srst}:
\[
\dot{\bm \omega}(\vec\upnu) = \frac{1}{2}\left( \vec\nabla\vec \upnu - (\vec\nabla \vec\upnu)^T \right)
\]
In the case of a Lagrangian description, we have \cite{vosc15} 
\[
\tilde{\dot{\bm \tau}}^{t+\delta t} 
= \frac{ {\bm \tau}^{t+\delta t} - {\bm \tau}^t }{ \delta t} 
- ( \dot{\bm \omega}(\vec\upnu^t)\cdot {\bm \tau}^t - {\bm \tau}^t \cdot \dot{\bm \omega}(\vec\upnu^t)   )
\]
so that 
\[
\dot{{\bm \varepsilon}}^d(\vec\upnu^{t+\delta t}) = 
\frac{\tilde{\dot{\bm \tau}}^{t+\delta t}}{2\mu}
+\frac{\bm \tau ^{t+\delta t}}{2 \eta}
=
\frac{1}{2\mu} \left[ \frac{ {\bm \tau}^{t+\delta t} - {\bm \tau}^t }{ \delta t} 
- ( \dot{\bm \omega}(\vec\upnu^t) \cdot {\bm \tau}^t - {\bm \tau}^t \cdot \dot{\bm \omega}(\vec\upnu^t)   )  \right]
+\frac{\bm \tau ^{t+\delta t}}{2 \eta}
\]

Let us multiply this by $2\mu \delta t$ and transform the equations until a satisfying 
formulation is found:

\[
2\mu \delta t\dot{\bm \varepsilon}^d(\vec\upnu^{t+\delta t}) 
=
{\bm \tau}^{t+\delta t} - {\bm \tau}^t 
- \delta t ( \dot{\bm \omega}(\vec\upnu^t) \cdot {\bm \tau}^t - {\bm \tau}^t \cdot \dot{\bm \omega}(\vec\upnu^t)   ) 
+
\frac{\mu \delta t }{ \eta}
{\bm \tau ^{t+\delta t}}   
\]

\[
2\mu \delta t\dot{\bm \varepsilon}^d(\vec\upnu^{t+\delta t}) 
=
\left( 1 + \frac{\mu \delta t }{ \eta}   \right) {\bm \tau}^{t+\delta t} - {\bm \tau}^t 
- \delta t ( \dot{\bm \omega}(\vec\upnu^t) \cdot {\bm \tau}^t - {\bm \tau}^t \cdot \dot{\bm \omega}(\vec\upnu^t)   ) 
\]

\[
\left( 1 + \frac{\mu \delta t }{ \eta}   \right) {\bm \tau}^{t+\delta t} 
=
2\mu \delta t\dot{\bm \varepsilon}^d(\vec\upnu^{t+\delta t}) 
+ {\bm \tau}^t + \delta t ( \dot{\bm \omega}(\vec\upnu^t) \cdot {\bm \tau}^t - {\bm \tau}^t \cdot 
\dot{\bm \omega}(\vec\upnu^t)   ) 
\]

\[
{\bm \tau}^{t+\delta t} 
=
2\frac{\mu \delta t}{\left( 1 + \frac{\mu \delta t }{ \eta}   \right)}  
\dot{\bm \varepsilon}^d(\vec\upnu^{t+\delta t}) 
+ \frac{1}{\left( 1 + \frac{\mu \delta t }{ \eta}   \right)}{\bm \tau}^t 
+ \frac{\delta t}{\left( 1 + \frac{\mu \delta t }{ \eta}   \right)}
 ( \dot{\bm \omega}(\vec\upnu^t) \cdot {\bm \tau}^t - {\bm \tau}^t \cdot \dot{\bm \omega}(\vec\upnu^t)   ) 
\]
We define:
\begin{equation}
\boxed{
\eta_{eff}=\frac{\mu \delta t}{\left( 1 + \frac{\mu \delta t }{ \eta}   \right)} = 
\frac{ \eta \delta t}{\delta t + \eta/\mu} =
 \frac{\eta}{1+ t_M/\delta t}
}
\label{eq:evetaeff}
\end{equation}

\[
\boxed{
Z=\frac{\eta_{eff}}{\mu \delta t} = \frac{\eta}{\mu \delta t + \eta}
}
\]

\[
\boxed{
{\bm J}^t=  \dot{\bm \omega}(\vec\upnu^t) \cdot {\bm \tau}^t - {\bm \tau}^t \cdot \dot{\bm \omega}(\vec\upnu^t)  
}
\]

so that we can write 
\[
{\bm \tau}^{t+\delta t} 
=
2 \eta_{eff}   \dot{\bm \varepsilon}^d(\vec\upnu^{t+\delta t}) 
+ Z {\bm \tau}^t 
+ Z \delta t {\bm J}^t
\]

or,  
\[
{\bm \tau}^{t+\delta t} 
= 2 \eta_{eff}   \dot{\bm \varepsilon}^d(\vec\upnu^{t+\delta t}) +  \underline{\bm \tau}^{t}
\qquad
\text{with}
\qquad
\underline{\bm \tau}^{t} = 
 Z  {\bm \tau}^t + Z \delta t {\bm J}^t 
\]


The total stress tensor is then 
\begin{equation}
\boxed{
{\bm \sigma}^{t+\delta t} 
= -p^{t+\delta t} {\bm 1} + {\bm \tau}^{t+\delta t} 
= - p^{t+\delta t} {\bm 1} + 2 \eta_{eff}   \dot{\bm \varepsilon}^d(\vec\upnu^{t+\delta t}) 
+  Z  {\bm \tau}^t + Z \delta t {\bm J}^t
}
\label{eq:sigmaev}
\end{equation}

\begin{remark}
When $\mu\rightarrow \infty$ we have
$\eta_{eff} \rightarrow \eta$ and $Z \rightarrow 0$ and 
we recover the Stokes equation for a purely viscous fluid.
\end{remark}

%..................................
\subsubsection{Strong form}

Let us now turn to the momentum conservation equation:
\begin{eqnarray}
&&{\vec \nabla}\cdot {\bm \sigma}^{t+\delta t} + \rho^{t+\delta t} {\vec g} = \vec{0} \nn\\
&\Rightarrow&
{\vec \nabla}\cdot (-p^{t+\delta t} {\bm 1}+  {\bm \tau}^{t+\delta t})+\rho^{t+\delta t}{\vec g}= \vec{0} \nn\\
&\Rightarrow&
- {\vec \nabla}p^{t+\delta t} +  {\vec \nabla}\cdot  {\bm \tau}^{t+\delta t} + \rho^{t+\delta t} {\vec g} = \vec{0} \nn\\
&\Rightarrow&
- {\vec \nabla}p^{t+\delta t} +  {\vec \nabla}\cdot  
\left(
2 \eta_{eff}   \dot{\bm \varepsilon}^d(\vec\upnu^{t+\delta t}) 
+  \underline{\bm \tau}^{t}
\right)
+ \rho^{t+\delta t} {\vec g} = \vec{0} \nn
\end{eqnarray}
and finally
\[
\boxed{
- {\vec \nabla}p^{t+\delta t} +  {\vec \nabla}\cdot  
2 \eta_{eff}   \dot{\bm \varepsilon}^d(\vec\upnu^{t+\delta t}) 
 = - \rho^{t+\delta t} {\vec g}
-{\vec \nabla}\cdot  \underline{\bm \tau}^{t}
}
\]

%..................................
\subsubsection{Weak form}

The mass conservation equation is still $\vec\nabla\cdot\vec\upnu=0$ so we need not look into 
it since its weak form is in Section~\ref{sec:mixed}.

For the $N_i^\upnu$'s we can write:
\begin{eqnarray}
\int_{\Omega_e} N_i^\upnu {\vec \nabla}\cdot {\bm \sigma}^{t+\delta t} d\Omega 
+ \int_{\Omega_e} N_i^\upnu  \rho {\vec g} \; d\Omega 
&=& \vec 0 
\end{eqnarray}
We can integrate by parts and drop the surface term\footnote{We will come back to this at a later stage}
REVISIT and use Eq.~(\ref{eq:sigmaev}):
\begin{eqnarray}
\int_{\Omega_e} {\vec \nabla } N_i^\upnu \cdot {\bm \sigma}^{t+\delta t} \;  d\Omega 
&=& \int_{\Omega_e} N_i^\upnu  \rho {\vec g} \; d\Omega \nn\\ 
\int_{\Omega_e} {\vec \nabla } N_i^\upnu \cdot 
\left[ - p^{t+\delta t} {\bm 1} + 2 \eta_{eff} \dot{\bm \varepsilon}^d(\vec\upnu^{t+\delta t}) +
 \underline{\bm \tau}^{t}  \right] \;  d\Omega 
&=& \int_{\Omega_e} N_i^\upnu   \rho {\vec g}    \; d\Omega  \\
\int_{\Omega_e} {\vec \nabla } N_i^\upnu \cdot 
\left[ - p^{t+\delta t} {\bm 1} + 2 \eta_{eff} \dot{\bm \varepsilon}^d(\vec\upnu^{t+\delta t}) 
\right] \;  d\Omega 
&=& \int_{\Omega_e} N_i^\upnu   \rho {\vec g} \; d\Omega
- \int_{\Omega_e}  {\vec \nabla } N_i^\upnu \cdot \underline{\bm \tau}^{t}   \; d\Omega  
\end{eqnarray}
We see that the left hand term is virtually identical to the one in Section~\ref{sec:mixed}, although
the viscosity has been replaced with the effective viscosity of Eq.~(\ref{eq:evetaeff}).
The headache will come from the right hand side term $ \underline{\bm \tau}^{t}$, as we will see.

\paragraph{In two dimensions - Cartesian coordinates}. 
The rotation rate tensor is given by:
\[
\dot{\bm \omega}(\vec\upnu) = \frac{1}{2}\left( \vec\nabla\vec \upnu - (\vec\nabla \vec\upnu)^T \right)
= 
\left( \begin{array}{cc}
0 & \dot{\omega}_{xy} \\
-\dot{\omega}_{xy} & 0
\end{array}\right)
=\frac{1}{2}
\left( \begin{array}{cc}
0 & \frac{\partial v}{\partial x} - \frac{\partial u}{\partial y} \\ 
\frac{\partial v}{\partial x} - \frac{\partial u}{\partial y} & 0 
\end{array}\right)
\]
so that the tensor ${\bm J}$ can be computed explicitely: 
\begin{eqnarray}
{\bm J}^t 
&=& \dot{\bm \omega}(\vec\upnu^t) \cdot {\bm \tau}^t - {\bm \tau}^t \cdot \dot{\bm \omega}(\vec\upnu^t)  \nn\\
&=&
\left( \begin{array}{cc}
0 & \dot{\omega}_{xy}(\vec\upnu^t) \\
-\dot{\omega}_{xy}(\vec\upnu^t) & 0
\end{array}\right)
\cdot
\left( \begin{array}{cc}
\tau^t_{xx} & \tau^t_{xy} \\
\tau^t_{xy} & \tau^t_{yy}
\end{array}\right)
-
\left( \begin{array}{cc}
\tau^t_{xx} & \tau^t_{xy} \\
\tau^t_{xy} & \tau^t_{yy}
\end{array}\right)
\cdot
\left( \begin{array}{cc}
0 & \dot{\omega}_{xy}(\vec\upnu^t) \\
-\dot{\omega}_{xy}(\vec\upnu^t) & 0
\end{array}\right)\nn\\
&=&
\dot{\omega}_{xy}(\vec\upnu^t)
\left( \begin{array}{cc}
0 & 1\\ 
-1 & 0
\end{array}\right)
\cdot
\left( \begin{array}{cc}
\tau^t_{xx} & \tau^t_{xy} \\
\tau^t_{xy} & \tau^t_{yy}
\end{array}\right)
-
\dot{\omega}_{xy}(\vec\upnu^t)
\left( \begin{array}{cc}
\tau^t_{xx} & \tau^t_{xy} \\
\tau^t_{xy} & \tau^t_{yy}
\end{array}\right)
\cdot
\left( \begin{array}{cc}
0 & 1 \\ 
-1 & 0
\end{array}\right) \nn\\
&=&
\dot{\omega}_{xy}(\vec\upnu^t)
\left( \begin{array}{cc}
\tau^t_{xy} & \tau^t_{yy} \\
-\tau^t_{xx} & -\tau^t_{xy}
\end{array}\right)
-
\dot{\omega}_{xy}(\vec\upnu^t)
\left( \begin{array}{cc}
-\tau^t_{xy} & \tau^t_{xx} \\
-\tau^t_{yy} & \tau^t_{xy}
\end{array}\right) \nn\\
&=&
\dot{\omega}_{xy}(\vec\upnu^t)
\left( \begin{array}{cc}
2\tau^t_{xy} & \tau_{yy}^t-\tau_{xx}^t \\
\tau^t_{yy}-\tau_{xx}^t & -2\tau_{xy}^t
\end{array}\right)
\end{eqnarray}

so that the tensor equation $\underline{\bm \tau}= Z {\bm \tau} + Z \delta t {\bm J}$ can be 
reformulated as follows in a vector form: 
\begin{eqnarray}
\left(
\begin{array}{c}
\underline{\tau}^t_{xx}\\
\underline{\tau}^t_{yy}\\
\underline{\tau}^t_{xy}
\end{array}
\right)
=
Z 
\left(
\begin{array}{c}
{\tau}_{xx}^t\\
{\tau}_{yy}^t\\
{\tau}_{xy}^t
\end{array}
\right)
+ Z \delta t
\left(
\begin{array}{c}
J^t_{xx}\\
J^t_{yy}\\
J^t_{xy}
\end{array}
\right)
=
Z 
\left(
\begin{array}{c}
{\tau}^t_{xx}\\
{\tau}^t_{yy}\\
{\tau}^t_{xy}
\end{array}
\right)
+ Z \delta t
\dot{\omega}^t_{xy}
\left(
\begin{array}{c}
 2\tau^t_{xy} \\
-2\tau^t_{xy} \\
\tau^t_{yy}-\tau^t_{xx} 
\end{array}
\right)
\end{eqnarray}




or, 
\begin{eqnarray}
\left(
\begin{array}{c}
{ \sigma}_{xx}^{t+\delta t}\\
{ \sigma}_{yy}^{t+\delta t}\\
{ \sigma}_{xy}^{t+\delta t}
\end{array}
\right)
&=&
\left(
\begin{array}{c}
-p^{t+\delta t} \\ 
-p^{t+\delta t} \\ 
0
\end{array}
\right)
+
2 \eta_{eff}
\left(
\begin{array}{c}
\dot{\varepsilon}_{xx}(\vec\upnu^{t+\delta t})\\
\dot{\varepsilon}_{yy}(\vec\upnu^{t+\delta t})\\
\dot{\varepsilon}_{xy}(\vec\upnu^{t+\delta t})
\end{array}
\right)
+
\left(
\begin{array}{c}
\underline{ \tau}_{xx}^t\\
\underline{ \tau}_{yy}^t\\
\underline{ \tau}_{xy}^t
\end{array}
\right) \nn\\
&=&
\left(
\begin{array}{c}
-p^{t+\delta t} \\ 
-p^{t+\delta t} \\ 
0
\end{array}
\right)
+
2 \eta_{eff}
\left(
\begin{array}{c}
\dot{\varepsilon}_{xx}(\vec\upnu^{t+\delta t})\\
\dot{\varepsilon}_{yy}(\vec\upnu^{t+\delta t})\\
\dot{\varepsilon}_{xy}(\vec\upnu^{t+\delta t})
\end{array}
\right)
+
Z
\left(
\begin{array}{c}
{\tau}_{xx}^t\\
{\tau}_{yy}^t\\
{\tau}_{xy}^t
\end{array}
\right)
+
Z \delta t \dot{\omega}_{xy}
\left(
\begin{array}{c}
 2\tau_{xy}^t \\
-2\tau_{xy}^t \\
\tau_{yy}^t-\tau_{xx}^t 
\end{array}
\right)
\end{eqnarray}



...


\begin{equation}
\int_{\Omega_e} {\bm B}^T \cdot 
\left(
\begin{array}{c}
\sigma_{xx}^{t+\delta t}\\
\sigma_{yy}^{t+\delta t}\\
\sigma_{xy}^{t+\delta t}
\end{array}
\right)
d\Omega
=
\int_{\Omega_e} {\vec N}_b d\Omega 
\end{equation}


\begin{equation}
\int_{\Omega_e} {\bm B}^T \cdot 
\left[
\left(
\begin{array}{c}
-p^{t+\delta t} \\ -p^{t+\delta t} \\ 0
\end{array}
\right)
+
2 \eta_{eff}
\left(
\begin{array}{c}
\dot{\varepsilon}^d_{xx}(\vec\upnu^{t+\delta t})\\
\dot{\varepsilon}^d_{yy}(\vec\upnu^{t+\delta t})\\
\dot{\varepsilon}^d_{xy}(\vec\upnu^{t+\delta t})
\end{array}
\right)
\right]
d\Omega
=
\int_{\Omega_e} {\vec N}_b d\Omega
-
\int_{\Omega_e} {\bm B}^T \cdot 
\left[
Z
\left(
\begin{array}{c}
{\tau}_{xx}^t\\
{\tau}_{yy}^t\\
{\tau}_{xy}^t
\end{array}
\right)
+
Z \delta t \dot{\omega}_{xy}(\vec\upnu^t)
\left(
\begin{array}{c}
 2\tau_{xy}^t \\
-2\tau_{xy}^t \\
\tau_{yy}^t-\tau^t_{xx} 
\end{array}
\right)
\right]
\; d\Omega \nn 
\end{equation}

As seen in Section~\ref{sec:mixed} the left hand side terms yield $\K \cdot \vec{V} + \G \cdot \vec{P}$.
The buoyancy term in the rhs is also standard and yields $\vec{f}$.
The discretised momentum equation then writes
\[
\K \cdot \vec{V} + \G \cdot \vec{P} = \vec{f} + \vec{f}_{el}
\]
and the last rhs term is 
\[
\boxed{
\vec{f}_{el} = 
-
\int_{\Omega_e} {\bm B}^T \cdot 
\left[
Z
\left(
\begin{array}{c}
{\tau}_{xx}^t\\
{\tau}_{yy}^t\\
{\tau}_{xy}^t
\end{array}
\right)
+
Z \delta t \dot{\omega}_{xy}(\vec\upnu^t)
\left(
\begin{array}{c}
 2\tau_{xy}^t \\
-2\tau_{xy}^t \\
\tau_{yy}^t-\tau^t_{xx} 
\end{array}
\right)
\right]
\; d\Omega 
}
\]
with the matrix ${\bm B}$ being given by
\[
{\bm B}=
\left(
\begin{array}{ccccccccccc}
\frac{\partial N_1^\upnu}{\partial x} & 0 & 0 &  \cdots  & \frac{\partial N_{m_v}^\upnu}{\partial x} & 0 & 0 \\ \\
0 & \frac{\partial N_1^\upnu}{\partial y} & 0 & \cdots & 0 & \frac{\partial N_{m_v}^\upnu}{\partial y} & 0 \\ \\
\frac{\partial N_1^\upnu}{\partial y} &  \frac{\partial N_1^\upnu}{\partial x} &  
0 & \cdots  &\frac{\partial N_{m_v}^\upnu}{\partial x} 
& \frac{\partial N_{m_v}^\upnu}{\partial x} & 0 \\ \\
\end{array}
\right) 
\]






%..........................................................
\paragraph{In three dimensions - Cartesian coordinates}
The spin tensor is given by 
\[
\dot{\bm \omega}(\vec\upnu) 
= \frac{1}{2}\left( \vec\nabla\vec \upnu - (\vec\nabla \vec\upnu)^T \right)
= 
\left( \begin{array}{ccc}
0 & \dot{\omega}_{xy} & \dot{\omega}_{xz}\\
-\dot{\omega}_{xy} & 0 & \dot{\omega}_{yz} \\
-\dot{\omega}_{xz} & -\dot{\omega}_{yz} & 0
\end{array}\right)
\]
so that 
\begin{eqnarray}
{\bm J}^t 
&=& \dot{\bm \omega}(\vec\upnu^t) \cdot {\bm \tau}^t - {\bm \tau}^t \cdot \dot{\bm \omega}(\vec\upnu^t)  \nn\\
&=&
\left( \begin{array}{ccc}
0 & \dot{\omega}_{xy} & \dot{\omega}_{xz}\\
-\dot{\omega}_{xy} & 0 & \dot{\omega}_{yz} \\
-\dot{\omega}_{xz} & -\dot{\omega}_{yz} & 0
\end{array}\right)
\cdot
\left( \begin{array}{ccc}
\tau_{xx} & \tau_{xy} & \tau_{xz}\\
\tau_{xy} & \tau_{yy} & \tau_{yz}\\
\tau_{xz} & \tau_{yz} & \tau_{zz}
\end{array}\right)
-
\left( \begin{array}{ccc}
\tau_{xx} & \tau_{xy} & \tau_{xz}\\
\tau_{xy} & \tau_{yy} & \tau_{yz}\\
\tau_{xz} & \tau_{yz} & \tau_{zz}
\end{array}\right)
\cdot
\left( \begin{array}{ccc}
0 & \dot{\omega}_{xy} & \dot{\omega}_{xz}\\
-\dot{\omega}_{xy} & 0 & \dot{\omega}_{yz} \\
-\dot{\omega}_{xz} & -\dot{\omega}_{yz} & 0
\end{array}\right)
\nn\\
&=&
\end{eqnarray}

FINISH!!!


check appendix A of Loes' GR

\newpage
%..................................
\subsubsection{Derivation of Jaumann derivative from stress rotation formula}

What follows is taken from Appendix D of Beuchert \& Podlachikov (2001) \cite{bepo10}.

The Jaumann co-rotational derivative is shown to be a truncated Taylor series expansion 
of simple stress rotation formula. These can be
obtained from a state of stress analysis based on geometric 
considerations (Biot 1965; Turcotte \& Schubert 2002). 
The (deviatoric) stress tensor ${\bm \tau}$ can be transformed according to
\[
{\bm \tau}' = {\bm R}\cdot {\bm \tau} \cdot {\bm R}^{T}
\]
with prime denoting the transformed stress tensor and ${\bm R}$ 
denoting the finite solid body rotation matrix. In a 2D Cartesian reference frame,
counter-clockwise rotation is given by\footnote{the authors got the matrix wrong!}
\[
{\bm R} = \left(\begin{array}{cc} \cos\theta & -\sin\theta \\ \sin\theta & \cos\theta  \end{array}\right)
\]
where $\theta = \int_t \dot{\omega} dt$ is a finite rotation angle (or $\theta=\dot{\omega} \delta t$ 
for finite time steps).
As we have seen in Section~\ref{sec:princ_stress}, we obtain 
\begin{eqnarray}
{\bm \tau}' 
&=& \left(
\begin{array}{cc}
\cos\theta & -\sin\theta \\
\sin\theta & \cos\theta
\end{array}
\right)
\cdot
\left(
\begin{array}{cc}
\tau_{xx} & \tau_{xy} \\
\tau_{xy} & \tau_{yy} 
\end{array}
\right)
\cdot
\left(
\begin{array}{cc}
\cos\theta & \sin\theta \\
-\sin\theta & \cos\theta
\end{array}
\right) \nn\\
&=&
\left(
\begin{array}{cc}
\tau_{xx} \cos^2\theta + \tau_{yy} \sin^2 \theta + \tau_{xy}\sin2\theta &
\frac{1}{2}(\tau_{yy}-\tau_{xx})\sin2\theta + \tau_{xy} \cos2\theta \\
\frac{1}{2}(\tau_{yy}-\tau_{xx})\sin2\theta + \tau_{xy} \cos2\theta &
\tau_{xx} \sin^2\theta + \tau_{yy} \cos^2 \theta - \tau_{xy}\sin2\theta 
\end{array}
\right)
\end{eqnarray}
\todo[inline]{check for minus sign!!}
or, 
\begin{eqnarray}
\tau_{xx}' &=&  \tau_{xx} \cos^2\theta + \tau_{yy} \sin^2 \theta + \tau_{xy}\sin2\theta \nn\\
\tau_{yy}' &=&  \tau_{xx} \sin^2\theta + \tau_{yy} \cos^2 \theta - \tau_{xy}\sin2\theta \nn\\
\tau_{xy}' &=&  \frac{1}{2}(\tau_{yy}-\tau_{xx})\sin2\theta + \tau_{xy} \cos2\theta \nn
\end{eqnarray}
For small rotation angles $\theta$ then 
\[
\cos \theta \rightarrow 1 
\qquad
\cos 2\theta \rightarrow 1 
\qquad
\sin \theta \rightarrow \theta 
\qquad
\sin 2\theta \rightarrow 2 \theta 
\]
and 
\begin{eqnarray}
\tau_{xx}' &\simeq&  \tau_{xx}  + 2 \tau_{xy} \theta \nn\\
\tau_{yy}' &\simeq&  \tau_{yy}  - 2 \tau_{xy} \theta \nn\\
\tau_{xy}' &\simeq&  (\tau_{yy}-\tau_{xx}) \theta + \tau_{xy} \nn
\end{eqnarray}
Given $\theta= \dot{\omega} \delta t$ we obtain 
\begin{eqnarray}
\tau_{xx}' &\simeq&  \tau_{xx}  + 2 \tau_{xy} \dot{\omega} \delta t  \nn\\
\tau_{yy}' &\simeq&  \tau_{yy}  - 2 \tau_{xy}  \dot{\omega} \delta t \nn\\
\tau_{xy}' &\simeq&  \tau_{xy}  + (\tau_{yy}-\tau_{xx}) \dot{\omega} \delta t \nn
\end{eqnarray}







