
\index{isoparametric}
The name isoparametric derives from the fact that the same ('iso') 
functions are used as basis functions and for the mapping to the reference element.


%...........................................
\subsubsection{Linear mapping on a triangle}

\begin{verbatim}
2
|\     s
| \    |_r
|  \
3===1
\end{verbatim}

Let us assume that the coordinates of the vertices are 
$(x_1,y_1)$,  
$(x_2,y_2)$, and 
$(x_3,y_3)$.
The coordinates inside the reference element are $(r,s)$. We then simply have the 
following relationship, i.e. any point of the reference element 
can be mapped to the physical triangle as follows:
\begin{eqnarray}
x&=& r x_1 + s x_2 + (1-r-s) x_3 \\
y&=& r y_1 + s y_2 + (1-r-s) y_3 
\end{eqnarray} 
There is also an inverse map, which is easily computed:
\begin{eqnarray}
r&=& \frac{(y_2-y_3)(x-x_3)-(x_2-x_3)(y-y_3)}{(x_1-x_3)(y_2-y_3)-(y_1-y_3)(x_2-x_3)} \\
s&=& \frac{-(y_1-y_3)(x-x_3)+(x_1-x_3)(y-y_3)}{(x_1-x_3)(y_2-y_3)-(y_1-y_3)(x_2-x_3)} 
\end{eqnarray} 
\begin{remark}
The denominator will not vanish, because it is a multiple of the area of the triangle.
\end{remark}

%................................................
\subsubsection{Linear mapping on a quadrilateral}

\begin{verbatim}
4====3
|    |  s
|    |  |_r
|    |
1====2
\end{verbatim}

The coordinates of the vertices are 
$(x_1,y_1)$, $(x_2,y_2)$, $(x_3,y_3)$ and $(x_4,y_4)$.
The coordinates inside the reference element are $(r,s)$. We then simply have the 
following relationship, i.e. any point of the reference element 
can be mapped to the physical quadrilateral as follows:
\begin{eqnarray}
x&=& N_1(r,s) x_1 + N_2(r,s) x_2 + N_3(r,s) x_3 + N_4(r,s) x_4 \\
y&=& N_1(r,s) y_1 + N_2(r,s) y_2 + N_3(r,s) y_3 + N_4(r,s) y_4 
\end{eqnarray} 
where the shape functions $N_i(r,s)$ are defined in section \ref{elements1D}.
There is also an inverse map, which is not so easily computed (see section \ref{sec:amiin}).

However, if the quadrilateral in the $(x,y)$ space is a rectangle of size $(h_x,h_y)$, 
the inverse mapping is trivial:
\begin{eqnarray}
r&=&\frac{x-x_1}{x_2-x_1} \\
s&=&\frac{y-y_1}{y_4-y_1} 
\end{eqnarray}
Also in this case the shape functions can easily be written as functions of $(x,y)$:
\begin{eqnarray}
N_1(x,y) &=& \left( \frac{x_3 -x }{h_x}  \right) \left( \frac{y_3 -y }{h_y}  \right) \nn\\
N_2(x,y) &=& \left( \frac{x - x_1}{h_x}  \right) \left( \frac{y_3 -y }{h_y}  \right) \nn\\
N_3(x,y) &=& \left( \frac{x - x_1}{h_x}  \right) \left( \frac{y - y_1}{h_y}  \right) \nn\\
N_4(x,y) &=& \left( \frac{x_3 -x }{h_x}  \right) \left( \frac{y - y_1}{h_y}  \right) \nn 
\end{eqnarray}







 
