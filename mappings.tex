
\index{isoparametric}
The name isoparametric derives from the fact that the same ('iso') 
functions are used as basis functions and for the mapping to the reference element.


%.............................
\subsubsection{On a triangle}

\begin{verbatim}
2
|\     s
| \    |_r
|  \
3===1
\end{verbatim}

Let us assume that the coordinates of the vertices are 
$(x_1,y_1)$,  
$(x_2,y_2)$, and 
$(x_3,y_3)$.
The coordinates inside the reference element are $(r,s)$. We then simply have the 
following relationship, i.e. any point of the reference element 
can be mapped to the physical triangle as follows:
\begin{eqnarray}
x&=& r x_1 + s x_2 + (1-r-s) x_3 \\
y&=& r y_1 + s y_2 + (1-r-s) y_3 
\end{eqnarray} 
There is also an inverse map, which is easily computed:
\begin{eqnarray}
r&=& \frac{(y_2-y_3)(x-x_3)-(x_2-x_3)(y-y_3)}{(x_1-x_3)(y_2-y_3)-(y_1-y_3)(x_2-x_3)} \\
s&=& \frac{-(y_1-y_3)(x-x_3)+(x_1-x_3)(y-y_3)}{(x_1-x_3)(y_2-y_3)-(y_1-y_3)(x_2-x_3)} 
\end{eqnarray} 
\begin{remark}
The denominator will not vanish, because it is a multiple of the area of the triangle.
\end{remark}


 
