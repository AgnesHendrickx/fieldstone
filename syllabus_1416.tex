Note: 
This handout was written by W. Spakman and 
is for a large part based on a syllabus by Dr. A.P. van den Berg and Prof. N.J. Vlaar
and on material from the book “Mantle convection in the Earth and Planets” by Schubert, Turcotte,
and Olson, Cambridge University Press, 2002.

%---------------------------------------------------------------
\subsection{Review of some essentials of continuum mechanics}


Newtonian mechanics deals with particles and rigid (undeformable) bodies on which
forces are acting. The application of Newtonian mechanics to realistic media (gases,
fluids, solids) is undoable simply because of the many particles (atoms, molecules)
involved. Continuum mechanics tackles this problem by assuming that physical fields
(e.g. density, temperature, velocity) can be viewed as (piece-wise) continuous functions
defined on the time and space coordinates involved in the description of macroscopic
matter. The idea is essentially that a tiny cube with sides of, 
say, $10^{-8}~\si{\meter}$ already contains
a sufficient number of atoms (millions) which allows for establishing physically
meaningful descriptions of quantities as temperature and density of the cube. In
continuum mechanics we are mostly interested in material behavior on much larger scales
than $10^{-8}~\si{\meter}$ for which is assumed that physical quantities 
are smooth functions of time and spatial coordinates.


The forces involved in the deformation of a continuum are postulated to be 
{\bf body forces}
$\vec{b}$ [\si{\newton\per\cubic\meter}], 
such that $\vec{b}dV$ is the force acting on the infinitesimal 
volume $dV$, and surface 
{\bf tractions} $\vec{t}^{\vec n}$ [\si{\newton\per\square\meter}]
(e.g. internal friction, applied surface tractions), 
such that $\vec{t}^{\vec n} dS$ is the
force acting on the infinitesimal surface $dS$. 
It is usual to write $\vec{b} = \rho \vec{g}$, 
with $\rho$ [\si{\kg\per\cubic\meter}] the
mass density and with $\vec{g}$ [\si{\meter\per\square\second}]
the acceleration due to the body force, which in
mantle dynamics is gravity. 

$\vec{b}$ and $\vec{t}^{\vec n}$
are force densities which after integration over a
volume or a surface, respectively, lead to net forces acting on 
the volume or surface.
The traction (or stress vector) is defined as
\[
\vec{t}^{\vec n} = \lim_{\Delta S \rightarrow 0} 
\frac{\sum\limits_i f_i^{\vec n}}{\Delta S},
\]
which expresses the force
per unit area working on a tiny surface $\Delta S$ 
with unit normal $\vec{n}$ (defining the orientation of
the surface). 
The forces $f_i^{\vec n}$  can be viewed as the atomic 
forces [\si{\newton}] that are applied at the
$\vec{n}$-side of $\Delta S$ to atoms at the other side of the surface. 
To maintain equilibrium, by the
third law of Newton, the traction applied to the $-\vec{n}-side$ of the surface 
is $\vec{t}^{-{\vec n}} = -\vec{t}^{\vec n}$.
Tractions depend on the orientation of the surface. In principle, one can draw an
infinite number of oriented surfaces through one point, each associated with a different
traction.

%%% page 3


Tractions are usually separated into the thermodynamic pressure force $p \vec{n}$ and the traction $\vec{\tau}$
related to mechanical deformation: $\vec{t}^{\vec n} = -p \vec{n} + \vec{\tau} \quad (p>0)$. The thermodynamic pressure
(a traction always acting perpendicular to any surface) is obtained from the equation of
state $f(\rho,p,T)$ relating thermodynamic quantities density, pressure, and temperature of a
continuum. The sign convention in continuum mechanics is that compression is negative
and tension is positive (in geology this is usually the other way around).


{\bf Stress} is a second order tensor quantity, which is defined by the following steps:
\begin{enumerate}
\item 
Assume a Cartesian coordinate frame in a point of interest for which the coordinate
axis are spanned by three unit orthogonal vectors $\vec{e}_i \quad (i = 1,2,3)$,
\item Imagine a tiny cube centered about the origin and with its faces parallel to the
coordinate planes,
\item Consider the 3 tractions $\vec{t}^{\vec{e}_i}$ that are acting on the three positive 
faces of the cube (i.e. the faces which have the normal vectors $\vec{e}_i$),
\item Lastly, define the components $\sigma_{ij}$ of the stress tensor ${\bm \sigma}$ as
\begin{equation}
\sigma_{ij} = \vec{t}^{\vec{e}_i} \cdot \vec{e}_j
\end{equation}
\end{enumerate}
When the stress tensor is visualized as a matrix, the three rows are 
the tractions on the
positive faces of the cube. The diagonal elements of the stress tensor 
${\bm \sigma}$ are called
normal stresses and the off-diagonal elements are the shear stresses.

Stress is a physical quantity, independent of coordinate frame, but the actual values of
components $\sigma_{ij}$ of the stress tensor can only be computed in 
a coordinate system. These
numbers are dependent on the frame adopted like the components of a flow vector (a first
order tensor) are frame dependent. Second order tensors follow (by definition) the
coordinate transformation rules of $3 \times 3$ matrices. From an analysis of force and force-
moment balance it can be demonstrated that the stress tensor is symmetric: $\sigma_{ij}=\sigma_{ji}$. 

An eigenvalue-eigenvector analysis leads to the three principal stresses $\sigma_k$ (eigenvalues) and
the corresponding three corresponding principal directions $\vec{q}_k$ (eigenvectors of unit
length). The latter span three mutually orthogonal (Cartesian) axes. In the principal-axes
frame the stress tensor is diagonal with the three principal stresses as diagonal elements.
In the principal-axes frame the tensor components are the maximal normal stresses
(tractions perpendicular to the faces of a tiny cube oriented along the principal coordinate
planes) compared to the normal stresses in any other coordinate system.

%%%%page 4

An important relation (the Cauchy relation) exists between the local state of stress (i.e.
the stress tensor ${\bm \sigma}$) 
and the traction $\vec{t}^{\vec n}$  acting on 
an (arbitrarily) oriented tiny surface $\Delta S$
with unit normal $\vec{n}$:
\begin{equation}
{\bm \sigma} \cdot \vec{n} = \vec{t}^{\vec n}
\label{chap10_tract1}
\end{equation}
or in components $\sigma_{ij} n_j = t_i^{\vec n}$ (summation convention implied).
This relation states how traction can be computed from the local stress and conversely,
that from known tractions on independently oriented surfaces the stress tensor can be
constructed by solving Eq.~\eqref{chap10_tract1}.
Note that it is required that $\vec{n}$ is a unit normal, i.e. $\vec{n}\cdot\vec{n}=1$.

\vspace{0.5cm}
\fbox{
\begin{minipage}{0.9\textwidth}
\begin{problem}
{\small \it 
Determine the tractions acting on the negative faces of a tiny cube (of which
the faces are aligned with the local coordinate axes) when the stress is given.
}
\end{problem}
\end{minipage}
}
\vspace{0.5cm}

{\bf Force balance equation of a continuum at rest}: Assume a 
continuum (gas, liquid, solid)
at rest. In this situation the net force acting on the entire 
continuum is $\vec{0}$ (Newton). The
sum of body forces and applied surface tractions cancel in some way. 
This also holds for
any sub-volume $V$. An internal stress field may still exist 
as a result of the applied forces
and surface tractions. The relation between the body force, tractions, 
and the internal
stress field is derived as follows: 
Consider an arbitrary sub-volume $V$ with boundary $S$.
Internal tractions act on the boundary $S$ 
(e.g. to be determined with equation \eqref{chap10_tract1} from the
internal stress field at $S$). 
The following equation postulates that the total sum of body
forces
\begin{equation}
\int_V \rho \vec{g} dV + \int_S \vec{t}^{\; \vec n} dS = \vec{0}
\label{chap10_three}
\end{equation}
acting on $V$ and of tractions on $S$ leads to a zero net
force acting on V.

Substituting \eqref{chap10_tract1} in the surface integral and next 
applying the Divergence 
theorem\footnote{\url{https://en.wikipedia.org/wiki/Divergence_theorem}}
one arrives at
\[
\int_V \rho \vec{g} dV + \int_V \vec\nabla \cdot {\bm\sigma} dV = \vec{0}
\]
Because $V$ is an
arbitrary volume the integrant must equal 0, which leads to the 
{\bf equilibrium equation}:
\begin{equation}
\vec\nabla \cdot {\bm\sigma} +  \rho \vec{g} = \vec{0} 
\qquad
\text{or,}
\qquad
\frac{\partial \sigma_ij}{\partial x_j} + \rho g_i = 0
\label{chap10_balance}
\end{equation}

This equation holds for any point in the interior of the continuum. 
Any traction applied at the boundary of the continuum relates to 
the (local) stress through equation \eqref{chap10_tract1}. Equation
\eqref{chap10_balance} states that body forces are in equilibrium 
with the divergence of the stress tensor.
Note: Gravity implies spatial variation in stress.

A similar analysis for the equilibrium of torques leads to the symmetry of the stress
tensor. In this case the equilibrium equation is
\[
\int_V \rho \vec{r} \times \vec{g}  dV 
+ 
\int_S \vec{r} \times \vec{t}^{\vec n} dS = \vec{0},
\]
where $\vec{r}$ is the position vector 
$\vec{r}=(x_1,x_2,x_3)^T$. The cross products can be written in
index notation using the permutation symbol $\epsilon_{ijk}$
which equals zero if at least two indices
have the same value (e.g. $\epsilon_{121}=0$), 
equals +1 if $ijk$ is an even permutation of $123$, 
and
equals -1 if $ijk$ is an odd permutation of $123$. 
This leads to the following notation of the
cross product of two vectors 
$\vec{a}\times \vec{b}= \epsilon_{ijk} \vec{e}_i a_j b_k$
and per component 
$(\vec{a}\times \vec{b})=\epsilon_{ijk} a_j b_k$.


\vspace{0.5cm}
\fbox{
\begin{minipage}{0.9\textwidth}
\begin{problem}
{\small \it
\noindent 
a) Derive equation \eqref{chap10_balance}\\
b) Using a similar approach, prove the symmetry of the stress tensor from the balance of
torques (assuming that no internally applied tractions exist)\\
c) Derive \eqref{chap10_balance} by considering the force balance of a tiny cube
}
\end{problem}
\end{minipage}
}
\vspace{0.5cm}

{\bf The material derivative}:
{\color{red} (for a mathematical intro see Appendix A)}
Consider a continuum with a three-dimensional flow 
field $\vec{v}(x_1,x_2,x_3,t)$ dependent on
the 3 spatial coordinates $x_j$ and time $t$. 
For any differentiable scalar function $T$ defined on
these 4 parameters we can write the total differential
\footnote{See any basic textbook on Calculus.}
\begin{equation}
dT = \frac{\partial T}{\partial t} dt + 
\sum_i \frac{\partial T}{\partial x_j} dx_j
\label{chap10_der1}
\end{equation}
This equation can be interpreted as follows: 
Consider a certain point $(\vec{r}, t)$ in which the
scalar function $T(\vec{r}, t)$ has continuous partial 
derivatives. The infinitesimal change $dT$,
which results from going 
from $(\vec{r}, t)$ to 
$(\vec{r} + d\vec{r}, t + dt)$ in de domain of $T$ 
is given by Eq.~\eqref{chap10_der1}. 
Importantly, $d\vec{r}$ and $dt$ can be arbitrarily 
chosen (including 0 values). The total
differential is at the basis of the definition of 
the so-called directional derivative.
Differentiable functions of more than 1 variable 
can be differentiated in arbitrary
directions (in their domain) to find their rate of change 
in this direction with respect to a
specified parameter. Particularly, we can consider the 
rate of change of $T$ with the time
parameter $t$ and (spatially) in the direction of 
the velocity field $\vec{v}(\vec{r}, t)$. This time-
derivative is easily obtained from \eqref{chap10_der1}
by coupling $dt$ and the spatial increment $d\vec{r}$ such
that $d\vec{r} = \vec{v}dt$. 
Substitution in \eqref{chap10_der1} leads to the time 
derivative of $T$ in the direction of the
velocity vector:
\begin{equation}
\frac{DT}{Dt} = 
\frac{\partial T}{\partial t} 
+\vec{v} \cdot \vec\nabla T
=
\frac{\partial T}{\partial t} 
+\sum_j v_j \cdot \frac{\partial T}{\partial x_j} 
\qquad
\text{with}
\qquad
\vec{v} =\frac{d \vec{r}}{dt}
\label{chap10_matder}
\end{equation}
which is called the material derivative of $T$ 
giving the rate of change of $T$ with time in
the direction of the flow $\vec{v}(\vec{r},t)$ 
at a certain point $(\vec{r}, t)$. 
The lhs of \eqref{chap10_matder} treats $T$ as a
function of $t$ only in the point $(\vec{r}(t), t)$ 
whereas the rhs of \eqref{chap10_matder} shows how this can be
computed from the partial derivatives of $T(\vec{r}, t)$
and the local velocity $\vec{v}(\vec{r}, t)$. The partial
derivative $\partial T/\partial t$
gives the temporal rate of change at fixed position 
$\vec{r}$, while the second term
gives the spatial contribution at fixed time, i.e
$\vec{v} \cdot \vec\nabla T$
expresses the advective contribution
(carried with the flow) to $DT/Dt$.

Without reference to a particular scalar function the material 
derivative (operator) is:
\[
\frac{D}{Dt} = \frac{\partial }{\partial t}
+\vec{v} \cdot \vec\nabla 
\]
The material derivative holds for any scalar function, 
particularly, for the components $v_i$
of the velocity field leading to the particle acceleration:
\[
\frac{D \vec{v}}{Dt} 
= \frac{\partial \vec{v}}{\partial t}
+ \vec\nabla \vec{v} \cdot \vec{v}
\]
Note that if the velocity field is time-stationary, i.e. 
$\partial_t=0$, there is still acceleration. In
this case the velocity vector field does not change with time. 
But, there can still be a
spatial variation which gives rise to a stationary acceleration 
field and material velocity
still changes in space, although the velocity is a constant vector in each point.

\vspace{0.5cm}
\fbox{
\begin{minipage}{0.9\textwidth}
\begin{problem}
{\small \it 
Assume 2-D space. Let the temperature field $T$ be given by
\begin{equation}
T(\vec{r},t)= T_0 \frac{1}{r} \exp(-t) \qquad t>0, r>0
\end{equation}
The temperature field belongs to a flow field given
by $\vec{\upnu}(\vec{r},t)=\frac{1}{r^2}\exp(-t) \vec{r}$.\\
a) Determine the divergence of the velocity field.
b) Compute the acceleration.
c) Compute the material derivate of $T$ at any position and time.
}
\end{problem}
\end{minipage}
}
\vspace{0.5cm}

{\bf The material derivative of a material volume integral}: 
Let $V$ be a material volume, i.e. a
volume that encompasses for all $t$ the same flow particles. 
This volume is following the
flow, possibly being deformed, while there is no material exchange 
with the region outside $V$. Let $T$ be again a scalar function 
of the space and time coordinates. The
material derivative of the (material) volume integral 
of $T$ is {\color{red} (Appendix B)}:
\begin{equation}
\frac{D}{Dt}
\left[
\int_{V(t)} T(\vec{r},t) dV
\right]
=
\int_{V(t)} \frac{DT}{Dt} + T \vec\nabla\cdot \vec\upnu dV
=
\int_{V(t)} \frac{DT}{Dt} + T \frac{\partial v_k}{\partial x_k} dV
\label{chap10_matder2}
\end{equation}


\vspace{0.5cm}
\fbox{
\begin{minipage}{0.9\textwidth}
\begin{problem}
{\small \it
\noindent 
Derive from \eqref{chap10_matder2} the alternative formula
\[
\frac{D}{Dt}
\left[
\int_{V(t)} T(\vec{r},t) dV
\right]
=
\int_{V(t)} \frac{\partial T}{\partial t} + \frac{\partial T v_k}{\partial x_k} dV
\]
}
\end{problem}
\end{minipage}
}
\vspace{0.5cm}


{\bf Diffusion processes} are in many cases described by (empirical) 
laws of the form $\vec{a} = -{\bm D} \cdot \vec\nabla H$
where $\vec{a}$ is a vector and ${\bm D}$ 
is the (anisotropic) diffusion (coefficient) tensor,
and $H$ a scalar field. 
Examples are Fourier’s (isotropic) heat flow vector 
$\vec{q} = -k \vec\nabla T$
where $k$ is thermal conductivity and $T$ temperature, 
or the isotropic diffusion of matter
(atoms) given by the mass density flow vector $\vec{J} = -D \vec\nabla c$
where $D$ is the diffusion
coefficient and $c$ the concentration of the substance.

%%page8

{\bf The continuity equation} (Conservation of mass; Reynold’s transport theorem):
Consider an arbitrary material volume $V$ within a continuum. 
By definition of a material
volume, the mass it contains is conserved, $M$=constant, or
$DM/Dt=0$.
The mass is given by
\[
M=\int_{V(t)} \rho(\vec{r},t) dV
\]
Applying \eqref{chap10_matder2} to $DM/Dt=0$ we find 
(for arbitrary V) the continuity
equation as a local expression of mass conservation:
\begin{equation}
\frac{D\rho}{Dt} + \rho \frac{\partial v_k}{\partial x_k} =0
\label{chap10_massconv}
\end{equation}
or, when substituting the material derivative:
\[
\frac{\partial \rho}{\partial t} + \frac{\partial (\rho v_k)}{\partial x_k} =0
\]
In incompressible fluids the density cannot change: $D\rho/Dt=0$.
Consequently, mass
conservation requires that the divergence of the velocity is 0, i.e.
\[
\frac{\partial v_k}{\partial x_k} = \vec\nabla \cdot \vec\upnu = 0
\]
Note that an equation like 
\eqref{chap10_massconv}
can also be derived for any other quantity that is
conserved by a material volume.


\vspace{0.5cm}
\fbox{
\begin{minipage}{0.9\textwidth}
\begin{problem}
{\small \it 
Prove that in a flowing medium with density $\rho$ 
the following relation holds
\begin{equation}
\frac{D}{Dt} \left[
\int_{V(t)} \rho T dV
\right]
=\int_{V(t)} \rho \frac{DT}{Dt} dV
\label{chap10_ex5}
\end{equation}
for any differentiable scalar function $T$ and material volume $V$. 
This formula will be
frequently used.
}
\end{problem}
\end{minipage}
}
\vspace{0.5cm}

\fbox{
\begin{minipage}{0.9\textwidth}
\begin{problem}
{\small \it 
\noindent
a) Prove that in an incompressible fluid the cubic-meter content of a material volume
does not change (hence, only the shape of boundary of the material volume is allowed to
change).\\
b) Prove that in an incompressible fluid the boundary $S$ of a material 
volume obeys the
following integral $\int_S \upnu_k n_k dS=0$. (Interpret this integral)
}
\end{problem}
\end{minipage}
}
\vspace{0.5cm}



\fbox{
\begin{minipage}{0.9\textwidth}
\begin{problem}
{\small \it 
\noindent
Let $V$ be an imaginary volume fixed in space. 
Derive the alternative mass
conservation law: 
\[
\int_V \frac{\partial \rho}{\partial t} dV
=
-\int_S \vec{J} \cdot \vec{n} dS
\]
where $\vec{J} = \rho\vec{\upnu}$ is the mass-density flow.
(Interpret this equation).
}
\end{problem}
\end{minipage}
}
\vspace{0.5cm}

%%%%% exercise 8

\vspace{0.5cm}
\fbox{
\begin{minipage}{0.9\textwidth}
\begin{problem}
{\small \it 
We wish to describe the transport of a polluting substance $X$ carried by a
fluid flow. We assume the substance is chemically non-reactive (passive) 
and dissolved in
the fluid. The spatial distribution of $X$ is given by the concentration function
$c(\vec{r},t)$ [\si{\kg\per\cubic\meter}]. 
Pollutant is also being produced/destroyed according to the function
$H(\vec{r},t)$ [\si{\kg\per\second\per\cubic\meter}].

a) Assume for the moment that mass diffusion of $X$ can be ignored. 
Derive a conservation
law in the form of a differential equation for the concentration 
function $c(\vec{r},t)$. 
[Hint:
start with computing the total mass of $X$ (integral form) contained 
in a material volume $V$. Next consider the (material) time derivative 
of this integral. Is equal to what?]\\
b) Now assume that mass diffusion of the pollutant is important. 
This implies material diffusion (not controlled by the fluid flow) 
across the boundary $S$ of the control volume $V$.
Assume that the mass flow density vector $\vec{J}$ 
[\si{\kg\per\second\per\square\meter}] is given by $\vec{J}=-D \vec\nabla c$ 
where $D$ is the diffusion coefficient. Extend the answer obtained at 
a) for this situation.
}
\end{problem}
\end{minipage}
}
\vspace{0.5cm}

{\bf The general equation of motion} (momentum equation):
The second law of Newton postulates that the sum of applied forces equals the rate of
change of the linear momentum of a particle with mass $m$:
\[
\sum_i \vec{F}_i = \frac{d\vec{p}}{dt},
\qquad
\vec{p}=m \vec{\upnu}.
\]
To arrive at a similar postulate for continuum mechanics, the total linear momentum of
an arbitrary material volume is defined as: $\int_{V(t)}\rho \vec{\upnu} dV$.

Newton’s second law leads to the following postulate of continuum mechanics:
\[
\frac{D}{Dt} \int_{V(t)} \rho \vec\upnu dV
=
\int_{V} \rho \vec{g} dV + \int_S \vec{t}^{\; \vec{n}} dS
\]
Using \eqref{chap10_ex5} to evaluate the left side of this equation 
for each velocity component and
applying the derivation following equation 
\eqref{chap10_three}
to the 
right side we find (as V is arbitrarily
chosen):
\begin{equation}
\rho \frac{D\vec\upnu}{Dt} = \vec\nabla\cdot {\bm \sigma} + \rho \vec{g}
\qquad
\text{or}
\qquad
\rho\frac{D \upnu_i}{Dt} = \frac{\partial \sigma_{ij}}{\partial j} + \rho {g}_i
\label{chap10_motion}
\end{equation}
which is the general equation of motion.
Recall that
\[
\frac{D\vec\upnu}{Dt} = \frac{\partial \vec\upnu}{\partial t}
+
\vec\nabla \vec\upnu \cdot \vec\upnu
\]
is the material derivative of velocity which renders \eqref{chap10_motion} to
be a non-linear equation in the unknown velocity field.

{\bf Velocity gradient, strain rate, and rotation rate}
The velocity gradient tensor is $\vec\nabla\vec\upnu = \partial \upnu_i/\partial x_j$
and can be separated in a symmetric part, the strain rate tensor 
\[
\dot{\bm \varepsilon}(\vec\upnu) = \frac12 \left(  
\vec\nabla\vec\upnu + (\vec\nabla\vec\upnu)^T 
\right)
\qquad
\text{or}
\qquad
\dot\varepsilon_{ij}=\frac12 \left(
\frac{\partial \upnu_i}{\partial x_j}
+ \frac{\partial \upnu_j}{\partial x_i}
\right)
\]
and in an anti-symmetric part called the rotation
rate tensor or spin-rate tensor
\[
\dot{\bm\omega}=  \frac12 \left(  
\vec\nabla\vec\upnu - (\vec\nabla\vec\upnu)^T 
\]
Note that $\dot\omega_{11}=\dot\omega_{22}=\dot\omega_{33}=0$.
The strain rate tensor is associated with the rate of
deformation (rates of relative length and volume changes and shear) 
while the spin rate
tensor describes an increment of uniform rotation in a continuum 
(i.e. without internal deformation). 
Note that  $\vec\nabla\cdot\vec\upnu= \partial \upnu_k/\partial x_k = 
\dot\varepsilon_{kk}$ gives the rate of relative volume change
during deformation.




















