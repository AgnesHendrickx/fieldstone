%6.3 of donea and huerta

In the case of an incompressible flow, we have seen that the continuity (mass conservation)
equation takes the simple form ${\vec \nabla}\cdot{\vec \upnu}=0$. In other words flow takes place 
under the constraint that the divergence of its velocity field is exactly zero eveywhere 
(solenoidal constraint), i.e. it is divergence free. 
\index{general}{Divergence-free Flow} 
\index{general}{Solenoidal Field}

We see that the pressure in the momentum equation is then a degree of freedom which is needed 
to satisfy the incompressibilty constraint (and it is not related to any constitutive equation)
(see for example \textcite{dohu03}). In other words the pressure is acting as a Lagrange multiplier of the incompressibility
constraint. 

Various approaches have been proposed in the literature to deal with the 
incompressibility constraint but we will only focus on the penalty method 
(section \ref{sec:penalty}) and the so-called mixed finite element method
\ref{sec:mixed}.
