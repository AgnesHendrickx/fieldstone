\begin{flushright} {\tiny {\color{gray} dgfem1D\_diff.tex}} \end{flushright}
%~~~~~~~~~~~~~~~~~~~~~~~~~~~~~~~~~~~~~~~~~~~~~~~~~~~~~~~~~~~~~~~~~~~~~~~~~~~~~~~~~~~~~~~~~~~~~~~~~~

Starting from the simple transient 1-D heat conduction problem similar to the 
steady state heat conduction problem only with added time dependence:

\begin{equation}
\frac{\partial T}{\partial t}=\frac{\partial^2T}{\partial x^2} \qquad T(x=0)=0 \qquad T(x=1)=1 \qquad \text{on} \quad x\in[0,1]
\end{equation}

Once again we split this system into two seperate first order equations:
\begin{eqnarray}
\frac{\partial T}{\partial t}-\frac{\partial q}{\partial x}&=&0 \nn\\
\frac{\partial T}{\partial x} -q &=& 0
\end{eqnarray}

We apply the standard approach to establish the weak forms of these two first-order PDEs, and we do so 
on an element $e$ bound by nodes $k$ and $k+1$ with coordinates $x_k$ and $x_{k+1}$
\begin{eqnarray}
-\int_{x_k}^{x_{k+1}} \left( \frac{\partial T}{\partial t}- \frac{dq}{dx} \right) \tilde{f}(x) dx =
\int_{x_k}^{x_{k+1}} \frac{\partial T}{\partial t} \tilde{f}(x)dx
 -\left[q \tilde{f} \right]_{x_k}^{x_{k+1}} 
+ \int_{x_k}^{x_{k+1}} \frac{d\tilde{f}}{dx} q(x) dx &=& 0
\label{eq:dg1a}\\
\int_{x_k}^{x_{k+1}}  \left( q-\frac{dT}{dx} \right) \overline{f}(x) dx
=
\int_{x_k}^{x_{k+1}}  q(x) \overline{f}(x) dx
-\left[ T \overline{f}  \right]_{x_k}^{x_{k+1}} + \int_{x_k}^{x_{k+1}} \frac{d\overline{f}}{dx} T(x) dx 
&=& 0
\label{eq:dg2a}
\end{eqnarray}
where $\tilde{f}$ and $\overline{f}$ are test functions.

In what follows we coin $\dot{T}=\partial T/\partial t$ (for convenience of notation). 
We once again recover Equations (\ref{eq:dgq1}) and (\ref{eq:dgT1}), although with 
an additional time derivative term. 

Filling this into equations (\ref{eq:dgq1}) and (\ref{eq:dgT1}), gives 
\begin{eqnarray}
{\bm K}^e \cdot 
\left( 
\begin{array}{cc}
    {\color{red}q_k^+}  \\
    {\color{red}q_{k+1}^-}
\end{array}
\right)
+
{\bm M}^e \cdot 
\left(
\begin{array}{c}
{\color{red}\dot{T}_k^+}  \\
{\color{red}\dot{T}_{k+1}^-} 
\end{array}
\right) 
+ 
\left(
\begin{array}{cc}
     ({\cal C}+\frac{1}{2})  {\color{red}q_k^+}  \\
     ({\cal C}-\frac{1}{2})  {\color{red}q_{k+1}^-} 
\end{array}
\right)+
\left(
\begin{array}{c}
     {\cal E}    {\color{red}T_k^+}  \\
     {\cal E}    {\color{red}T_{k+1}^-} 
\end{array}
\right) 
&=& 
\left(
\begin{array}{cc}
     ({\cal C}-\frac{1}{2}) q_k^-  \\
     ({\cal C}+\frac{1}{2}) q_{k+1}^+ 
\end{array}
\right)
+ \left(
\begin{array}{cc}
     {\cal E}   T_k^-  \\
     {\cal E}   T_{k+1}^+
\end{array}
\right)  
\nn
\\
{\bm M}^e \cdot
\left(
\begin{array}{cc}
    {\color{red}q_k^+}  \\
    {\color{red}q_{k+1}^-}
\end{array}
\right)+
{\bm K}^e \cdot
\left(
\begin{array}{cc}
 {\color{red}T_k^+}  \\
{\color{red}T_{k+1}^-} 
\end{array}
\right) 
+ \left(
\begin{array}{cc}
     (\frac{1}{2}-{\cal C}) {\color{red}T_k^+}  \\
     -({\cal C}+\frac{1}{2}){\color{red}T_{k+1}^-} 
\end{array}
\right)
&=& \left(
\begin{array}{cc}
     -({\cal C}+\frac{1}{2})  T_k^- \\
     (\frac{1}{2}-{\cal C})  T_{k+1}^+ 
\end{array}
\right) 
\end{eqnarray}

In what follows we set ${\cal E}=0$ so that we have
\begin{eqnarray}
{\bm K}^e \cdot 
\left( 
\begin{array}{cc}
    {\color{red}q_k^+}  \\
    {\color{red}q_{k+1}^-}
\end{array}
\right)
+
{\bm M}^e \cdot 
\left(
\begin{array}{c}
{\color{red}\dot{T}_k^+}  \\
{\color{red}\dot{T}_{k+1}^-} 
\end{array}
\right) 
+ 
\left(
\begin{array}{cc}
     ({\cal C}+\frac{1}{2})  {\color{red}q_k^+}  \\
     ({\cal C}-\frac{1}{2})  {\color{red}q_{k+1}^-} 
\end{array}
\right)
&=& 
\left(
\begin{array}{cc}
     ({\cal C}-\frac{1}{2}) q_k^-  \\
     ({\cal C}+\frac{1}{2}) q_{k+1}^+ 
\end{array}
\right)
\nn
\\
{\bm M}^e \cdot
\left(
\begin{array}{cc}
    {\color{red}q_k^+}  \\
    {\color{red}q_{k+1}^-}
\end{array}
\right)+
{\bm K}^e \cdot
\left(
\begin{array}{cc}
 {\color{red}T_k^+}  \\
{\color{red}T_{k+1}^-} 
\end{array}
\right) 
+ \left(
\begin{array}{cc}
     (\frac{1}{2}-{\cal C}) {\color{red}T_k^+}  \\
     -({\cal C}+\frac{1}{2}){\color{red}T_{k+1}^-} 
\end{array}
\right)
&=& \left(
\begin{array}{cc}
     -({\cal C}+\frac{1}{2})  T_k^- \\
     (\frac{1}{2}-{\cal C})  T_{k+1}^+ 
\end{array}
\right) 
\end{eqnarray}

Using the expressions for ${\bm M}^e$ and ${\bm K}^e$ 
obtained in Appendix~\ref{app:mm} for 1D linear elements we arrive at
\begin{eqnarray}
\frac{1}{2}
\left(
\begin{array}{cc}
-1  & -1 \\
1 & 1
\end{array}
\right)
\cdot 
\left( 
\begin{array}{cc}
    {\color{red}q_k^+}  \\
    {\color{red}q_{k+1}^-}
\end{array}
\right)
+
\frac{h}{6}
\left(
\begin{array}{cc}
2  & 1 \\
1 & 2
\end{array}
\right)
\cdot 
\left(
\begin{array}{c}
{\color{red}\dot{T}_k^+}  \\
{\color{red}\dot{T}_{k+1}^-} 
\end{array}
\right) 
+ 
\left(
\begin{array}{cc}
     ({\cal C}+\frac{1}{2})  {\color{red}q_k^+}  \\
     ({\cal C}-\frac{1}{2})  {\color{red}q_{k+1}^-} 
\end{array}
\right)
&=& 
\left(
\begin{array}{cc}
     ({\cal C}-\frac{1}{2}) q_k^-  \\
     ({\cal C}+\frac{1}{2}) q_{k+1}^+ 
\end{array}
\right)
\nn
\\
\frac{h}{6}
\left(
\begin{array}{cc}
2  & 1 \\
1 & 2
\end{array}
\right)
\cdot
\left(
\begin{array}{cc}
    {\color{red}q_k^+}  \\
    {\color{red}q_{k+1}^-}
\end{array}
\right)+
\frac{1}{2}
\left(
\begin{array}{cc}
-1  & -1 \\
1 & 1
\end{array}
\right)
\cdot
\left(
\begin{array}{cc}
 {\color{red}T_k^+}  \\
{\color{red}T_{k+1}^-} 
\end{array}
\right) 
+ \left(
\begin{array}{cc}
     (\frac{1}{2}-{\cal C}) {\color{red}T_k^+}  \\
     -({\cal C}+\frac{1}{2}){\color{red}T_{k+1}^-} 
\end{array}
\right)
&=& \left(
\begin{array}{cc}
     -({\cal C}+\frac{1}{2})  T_k^- \\
     (\frac{1}{2}-{\cal C})  T_{k+1}^+ 
\end{array}
\right) 
\end{eqnarray}

which simplifies to 
\begin{eqnarray}
\left(
\begin{array}{cc}
C  & -1/2 \\
1/2 & C 
\end{array}
\right)
\cdot 
\left( 
\begin{array}{cc}
    {\color{red}q_k^+}  \\
    {\color{red}q_{k+1}^-}
\end{array}
\right)
+
\left(
\begin{array}{cc}
h/3 & h/6 \\
h/6 & h/3
\end{array}
\right)
\cdot 
\left(
\begin{array}{c}
{\color{red}\dot{T}_k^+}  \\
{\color{red}\dot{T}_{k+1}^-} 
\end{array}
\right) 
&=& 
\left(
\begin{array}{cc}
     ({\cal C}-\frac{1}{2}) q_k^-  \\
     ({\cal C}+\frac{1}{2}) q_{k+1}^+ 
\end{array}
\right)
\nn
\\
\left(
\begin{array}{cc}
h/3 & h/6 \\
h/6 & h/3
\end{array}
\right)
\cdot
\left(
\begin{array}{cc}
    {\color{red}q_k^+}  \\
    {\color{red}q_{k+1}^-}
\end{array}
\right)+
\left(
\begin{array}{cc}
-C  & -1/2 \\
1/2 & -C
\end{array}
\right)
\cdot
\left(
\begin{array}{cc}
 {\color{red}T_k^+}  \\
{\color{red}T_{k+1}^-} 
\end{array}
\right) 
&=& \left(
\begin{array}{cc}
     -({\cal C}+\frac{1}{2})  T_k^- \\
     (\frac{1}{2}-{\cal C})  T_{k+1}^+ 
\end{array}
\right) 
\end{eqnarray}
or, 
\[
{\bm C}_1 \vec{q} +  {\bm M} \vec{\dot{T}} = \vec{f}  
\]
\[
{\bm M} \vec{q} + {\bm C}_2 \vec{T} = \vec{g}
\]
so 
\[
 \vec{q} = {\bm M}^{-1}   (\vec{g} -  {\bm C}_2 \vec{T} )
\]
and then 
\[
{\bm C}_1 [  {\bm M}^{-1}   (\vec{g} -  {\bm C}_2 \vec{T} )   ]    +  {\bm M} \vec{\dot{T}} = \vec{f}  
\]


NOT REALLY FINISHED...










































































%\end{document}


