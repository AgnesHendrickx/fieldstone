Two relevant papers: 
\begin{itemize}
\item Cockburn et al (2002) \cite{coks02} - LDG
\item Cockburn et al (2010) \cite{conp10} - HDG
\end{itemize}

Let us start with the dimensionless Stokes system \cite{coks02}:
\begin{eqnarray}
- \eta \Delta \vec\upnu + \vec\nabla p &=& \vec{f}  \qquad \textrm{in } \Omega\\
\vec\nabla\cdot\vec\upnu &=& 0 \qquad \textrm{in } \Omega\\
\vec{\upnu} &=& \vec{\upnu}_D \qquad \textrm{on } \Gamma
\end{eqnarray}
where $\Omega$ is a bounded domain of $\mathbb{R}^d$ and the Dirichlet boundary conditions are
such that they satisfy the compatibility condition
\[
\int_\Gamma \vec\upnu_D \cdot \vec{n} =0
\]
where $\vec{n}$ is the outward unit normal. 

\index{general}{Gradient-Based Formulation (DG-FEM)}
\paragraph{Gradient-based formulation} In order to obtain the LDG methods we first rewrite this system as the following collection of conservation 
laws \cite{coks02}:
\begin{eqnarray}
{\bm L} &=& \vec\nabla \vec\upnu  \qquad \textrm{in } \Omega\\
\vec\nabla\cdot (-2\eta {\bm L} + p {\bm 1}) &=& \vec{f}  \qquad \textrm{in } \Omega\\
\vec\nabla\cdot\vec\upnu &=& 0 \qquad \textrm{in } \Omega\\
\vec{\upnu} &=& \vec{\upnu}_D \qquad \textrm{on } \Gamma
\end{eqnarray}
supplemented by
\[
\int_\Omega p =0
\]
where ${\bm L}$ is the gradient tensor, ${\bm 1}$ is the unit tensor. \index{general}{Gradient Tensor}
\begin{remark}
It may appear counter-intuitive at first to define ${\bm L}$ as being the gradient
of the velocity instead of the strain rate tensor but under the assumption
of incompressibility $\partial_x u + \partial_y v =0$ (and constant viscosity) we can write:
\[
\vec\nabla\cdot (2 \eta {\bm L}) = 
2\eta
\left(
\begin{array}{c}
\partial_x^2 u + \frac{1}{2}\partial_x\partial_y v + \frac{1}{2}\partial_y^2 u \\
\frac{1}{2}\partial_x^2 v + \frac{1}{2} \partial_y\partial_x u + \partial_y^2 v
\end{array}
\right)
=
2\eta
\left(
\begin{array}{c}
\partial_x^2 u + \frac{1}{2}\partial_x(-\partial_x u) + \frac{1}{2}\partial_y^2 u \\
\frac{1}{2}\partial_x^2 v + \frac{1}{2} \partial_y(-\partial_y v) + \partial_y^2 v
\end{array}
\right)
=
\eta
\left(
\begin{array}{c}
\partial_x^2 u + \partial_y^2 u \\
\partial_x^2 v + \partial_y^2 v
\end{array}
\right)
=
\eta \Delta \vec\upnu
\]
\end{remark}


\begin{remark}
Cockburn et al (2010) \cite{conp10} also introduce the vorticity-based formulation and the stress-based 
formulation but we will not explore these in what follows.
\end{remark}

RETYPE section 2.1 of \cite{coks02}

