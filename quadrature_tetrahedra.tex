\begin{flushright} {\tiny {\color{gray} quadrature\_tetrahedra.tex}} \end{flushright}
%~~~~~~~~~~~~~~~~~~~~~~~~~~~~~~~~~~~~~~~~~~~~~~~~~~~~~~~~~~~~~~~~~~~~~~~~~~~~~~~~~~~~~~~~~~~~~~~~~~

\begin{remark}
In what follows the coefficients in the tables are not the reduced coordinates
of the quadratue points but the coefficients corresponding to the 4 nodes.
\end{remark}

Quadrature rules on tetrahedra take the form:
\[
\int\int\int_{el} f(x,y,z) dxdydz = V_{el} \sum_{iq=1}^{nqel} 
w_{iq} f(\xi^{iq}_1,\xi^{iq}_2,\xi^{iq}_3,\xi^{iq}_4) 
\]
or, that is to say:
\[
\int\int\int_{el} f(x,y,z) dxdydz = \sum_{iq=1}^{nqel} 
(w_{iq}V_{el}) f(\xi^{iq}_1,\xi^{iq}_2,\xi^{iq}_3,\xi^{iq}_4) 
\]
with in our case $V_{el}=1/6$.

In the literature it can be found that a one point quadrature is characterised by 
\[
w_{iq}=1 \quad\quad\quad \xi^{iq}_1=\xi^{iq}_2=\xi^{iq}_3=\xi^{iq}_4=0.25
\]
i.e, the coordinates of the single point are given by:
\[
x_{iq}=\sum_{i=1}^4 \xi_i^{iq} x_i = \frac{1}{4} (x_1+x_2+x_3+x_4)
\]
Same for $y$ and $z$ coordinates. 

A four-point quadrature rule is characterised by $w_{iq}=V_{el}*0.25=1/24\simeq 04166666666666667$ and 

\begin{tabular}{lcccc}
\hline
 & $\xi_1$ & $\xi_2$ & $\xi_3$ & $\xi_4$ \\
\hline\hline
iq=1 & 0.585410196624969 & 0.138196601125011 & 0.138196601125011 & 0.138196601125011 \\
iq=2 & 0.138196601125011 & 0.585410196624969 & 0.138196601125011 & 0.138196601125011 \\
iq=3 & 0.138196601125011 & 0.138196601125011 & 0.585410196624969 & 0.138196601125011 \\
iq=4 & 0.138196601125011 & 0.138196601125011 & 0.138196601125011 & 0.585410196624969 \\
\hline
\end{tabular}

We then have:
\[
r_{iq}=\sum_{i=1}^4 \xi_i^{iq} x_i 
= (\xi_1^{iq},\xi_2^{iq},\xi_3^{iq},\xi_4^{iq})\cdot(r_1,r_2,r_3,r_4) 
= (\xi_1^{iq},\xi_2^{iq},\xi_3^{iq},\xi_4^{iq})\cdot(0,1,0,0) 
= \xi_2^{iq}
\]
\[
s_{iq}=\sum_{i=1}^4 \xi_i^{iq} y_i 
= (\xi_1^{iq},\xi_2^{iq},\xi_3^{iq},\xi_4^{iq})\cdot(s_1,s_2,s_3,s_4) 
= (\xi_1^{iq},\xi_2^{iq},\xi_3^{iq},\xi_4^{iq})\cdot(0,0,1,0) 
= \xi_3^{iq}
\]
\[
t_{iq}=\sum_{i=1}^4 \xi_i^{iq} z_i 
= (\xi_1^{iq},\xi_2^{iq},\xi_3^{iq},\xi_4^{iq})\cdot(t_1,t_2,t_3,t_4) 
= (\xi_1^{iq},\xi_2^{iq},\xi_3^{iq},\xi_4^{iq})\cdot(0,0,0,1) 
= \xi_4^{iq}
\]
Finally:

\begin{tabular}{ccccc}
\hline
     & $r_q$ & $s_q$ & $t_q$  & $w_q$ \\
\hline
\hline
iq=1 & 0.138196601125011 & 0.138196601125011 & 0.138196601125011 & 0.04166666666666667\\
iq=2 & 0.585410196624969 & 0.138196601125011 & 0.138196601125011 & 0.04166666666666667\\
iq=3 & 0.138196601125011 & 0.585410196624969 & 0.138196601125011 & 0.04166666666666667\\
iq=4 & 0.138196601125011 & 0.138196601125011 & 0.585410196624969 & 0.04166666666666667\\
\hline
\end{tabular}



