\index{general}{Nonlinear PDE} 
\index{general}{Picard Iterations} 
\index{general}{Relaxation}

\todo[inline]{explain why our eqs are nonlinear}

\Literature Quasi Newton methods \cite{ensb81}

%--------------------------------
\subsubsection{Picard iterations} \label{ss:picard}

Let us consider the following system of nonlinear algebraic equations:
\[
\mathbb{A}(\vec X) \cdot \vec X = \vec b(\vec X)
\]
Both matrix and right hand side depend on the solution vector $\vec X$.

For many mildly nonlinear problems, a simple successive substitution 
iteration scheme (also called Picard method) will converge to the solution
and it is given by the simple relationship:
\[
\mathbb{A}(\vec X^n) \cdot \vec X^{n+1} = \vec b(\vec X^n)
\]
where $n$ is the iteration number. 
It is easy to implement:
\begin{enumerate}
\item guess $\vec X^0$ or use the solution from previous time step
\item compute $\mathbb{A}$ and $\vec b$ with current solution vector $\vec X^{old}$
\item solve system, obtain $T^{new}$
\item check for convergence (are $\vec X^{old}$ and $\vec X^{new}$ close enough?)
\item $\vec X^{old} \leftarrow \vec X^{new}$
\item go back to 2.
\end{enumerate}

There are various ways to test whether iterations have converged. The simplest
one is to look at $\norm{\vec X^{old}-\vec X^{new} }$ (in the $L_1$, $L_2$ or maximum norm)
and assess whether this term is smaller than a given tolerance $\epsilon$. 
However this approach poses a problem: in geodynamics, if two consecutively obtained 
temperatures do not change by more than a thousandth of a Kelvin (say $\epsilon=10^{-3}$K )
we could consider that iterations have converged but looking now at velocities which 
are of the order of a cm/year (i.e. $\sim 3\cdot 10^{-11}$m/s) we would need a tolerance 
probably less than $10^{-13}$m/s. We see that using absolute values for a convergence 
criterion is a potentially dangerous affair, which is why one uses a relative 
formulation (thereby making $\epsilon$ a dimensionless parameter):
\[
\frac{\norm{\vec X^{old}-\vec X^{new}}}{\norm{\vec X^{new}}} < \epsilon
\]
Another convergence criterion is proposed by Reddy (section 3.7.2) \cite{reddybook2}:
\[
\left(
\frac{ (\vec X^{old}-\vec X^{new})\cdot(\vec X^{old}-\vec X^{new} ) }{ X^{new}\cdot X^{new}  } 
\right)^{1/2} < \epsilon
\]
Yet another convergence criterion is used in \cite{thie11}: the means $<\vec X^{old}>$, $<\vec X^{new}>$
as well as the variances $\sigma^{old}$ and $\sigma^{new}$ are computed, followed by the 
correlation factor $R$:
\[
R= \frac{ <  (\vec X^{old}-<\vec X^{old}>)\cdot( \vec X^{new}-<\vec X^{new}> )>  }{\sqrt{\sigma^{old}\sigma^{new}}}
\]
Since the correlation is normalised, it takes values between 0
(very dissimilar velocity fields) and 1 (very similar fields). The
following convergence criterion is then used: $1-R < \epsilon$.

\todo[inline]{write about nonlinear residual}


Note that in some instances and improvement in convergence rate can be obtained by use of a 
relaxation formula where one first solves
\[
\mathbb{A}(\vec X^n) \cdot \vec X^{\star} = \vec b(\vec X^n)
\]
and then updates $\vec X^n$ as follows:
\[
\vec X^n = \gamma \vec X^n + (1-\gamma) \vec X^\star 
\quad\quad\quad
0 < \gamma \leq 1
\]
When $\gamma=1$ we recover the standard Picard iterations formula above.

%------------------------------------------
\subsection{Defect correction formulation}
\index{general}{Defect Correction Formulation}

Work in progress. 

We start from the system to solve:
\[
{\bm A}(\vec X) \cdot \vec X = \vec b(\vec X)
\]
with the associated residual vector $\vec F$ 
\[
\vec F(\vec X) = {\bm A}(\vec X) \cdot \vec X - \vec b(\vec X)
\]
The Newton-Raphson algorithm consists of two steps:
\begin{enumerate}
\item solve $\bm J_k \cdot \delta \vec X_k = -\vec F(\vec X_k)$, or in the 
case of the incompressible Stokes equation FEM system:
\[
\left(
\begin{array}{cc}
\bm J^{{\cal V}{\cal V}}_k & \bm J^{{\cal V}{\cal P}}_k \\
\bm J^{{\cal P}{\cal V}}_k & 0
\end{array}
\right)
\cdot
\left(
\begin{array}{c}
\delta \vec {\cal V}_k \\ \delta \vec {\cal P}_k
\end{array}
\right)
=
\left(
\begin{array}{c}
- \vec F_k^{\cal V} \\ -\vec F_k^{\cal P}
\end{array}
\right)
\]

\item update $\vec X_{k+1} = \vec X_k + \alpha_k \delta \vec X_k$
\end{enumerate}
The defect correction Picard approach consists of neglecting the derivative terms present 
in the $J$ terms (Eqs. 16,17,18 of \cite{frbt19}) so that 
\[
\bm J^{{\cal V}{\cal V}}_k \simeq \K_k 
\quad\quad
\bm J^{{\cal V}{\cal P}}_k \simeq \G 
\quad\quad
\bm J^{{\cal P}{\cal V}}_k \simeq \G^T
\]
and step 1 of the above iterations become:
\[
\left(
\begin{array}{cc}
\K_k & \G \\ \G^T & 0
\end{array}
\right)
\cdot
\left(
\begin{array}{c}
\delta \vec {\cal V}_k \\ \delta \vec {\cal P}_k
\end{array}
\right)
=
\left(
\begin{array}{c}
- \vec F_k^{\cal V} \\ -\vec F_k^{\cal P}
\end{array}
\right)
\]


\mscthesis: implement a simple Newton solver and apply it to a few nonlinear benchmarks. \index{general}{MSc Thesis} 
