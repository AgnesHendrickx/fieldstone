\begin{flushright} {\tiny {\color{gray} maths.tex}} \end{flushright}
%~~~~~~~~~~~~~~~~~~~~~~~~~~~~~~~~~~~~~~~~~~~~~~~~~~~~~~~~~~~~~~~~~~~~~~~~~~~~~~~~~~~~~~~~~~~~~~~~~~

%------------------------------------------------------
\subsection{Inverse of a 3x3 matrix \label{sec:inv2x2}}

Let us assume we wish to solve the 
system $\bm A \cdot \vec X = \vec b$, with $\vec X=(x,y)$. Then the solution is given by
The solution is given by
\[
x=\frac{1}{det(\bm A)}
\left|
\begin{array}{cc}
b_1 & a_{21} \\
b_2 & a_{22}
\end{array}
\right|
\qquad
y=\frac{1}{det(\bm A)}
\left|
\begin{array}{cc}
a_{11} & b_1\\
a_{21} & b_2
\end{array}
\right|
\]



%------------------------------------------------------
\subsection{Inverse of a 3x3 matrix \label{sec:inv3x3}}

Let us consider the 3x3 matrix ${\bm M}$
\[
{\bm M}=
\left(
\begin{array}{ccc}
M_{xx} & M_{xy} & M_{xz} \\
M_{yx} & M_{yy} & M_{yz} \\
M_{zx} & M_{zy} & M_{zz} 
\end{array}
\right)
\]

\begin{enumerate}
\item
Find $det({\bm M})$, the determinant of the Matrix ${\bm M}$.
The determinant will usually show up in the denominator of the inverse. If the determinant is zero, the matrix won't have an inverse.

\item  Find ${\bm M}^T$ , the transpose of the matrix. Transposing means reflecting the matrix about the main diagonal.

\[
{\bm M}^T=
\left(
\begin{array}{ccc}
M_{xx} & M_{yx} & M_{zx} \\
M_{xy} & M_{yy} & M_{zy} \\
M_{xz} & M_{yz} & M_{zz} 
\end{array}
\right)
\]

\item  Find the determinant of each of the 2x2 minor matrices. For instance $\tilde{M}_{xx}=M_{yy}M_{zz}-M_{yz}M_{zy}$,
or $\tilde{M}_{xz}=M_{xy}M_{yz}- M_{xz}M_{yy}$.

\item assemble the $\tilde{\bm M}$ matrix:

\[
\tilde{\bm M}=
\left(
\begin{array}{ccc}
+\tilde{M}_{xx} & -\tilde{M}_{xy} & +\tilde{M}_{xz} \\
-\tilde{M}_{yx} & +\tilde{M}_{yy} & -\tilde{M}_{yz} \\
+\tilde{M}_{zx} & -\tilde{M}_{zy} & +\tilde{M}_{zz} 
\end{array}
\right)
\]

\item the inverse of ${\bm M}$ is then given by
\[
{\bm M}^{1} = \frac{1}{det({\bm M})} \tilde{\bm M}
\]

\end{enumerate}

Another approach which of course is equivalent to the above is Cramer's rule. Let us assume we wish to solve the 
system $\bm A \cdot \vec X = \vec b$, with $\vec X=(x,y,z)$. Then the solution is given by
\[
x=
\frac{1}{det(\bm M)}
\left| 
\begin{array}{ccc}
b_1 & a_{12} & a_{13} \\
b_2 & a_{22} & a_{23} \\
b_3 & a_{32} & a_{33}
\end{array}
\right|
\qquad
y=
\frac{1}{det(\bm M)}
\left| 
\begin{array}{ccc}
a_{11} & b_1 & a_{13} \\
a_{21} & b_2 & a_{23} \\
a_{31} & b_3 & a_{33} 
\end{array}
\right|
\qquad
z=
\frac{1}{det(\bm M)}
\left| 
\begin{array}{ccc}
a_{11} & a_{12} & b_1\\
a_{21} & a_{22} & b_2\\
a_{31} & a_{32} & b_3
\end{array}
\right|
\]



