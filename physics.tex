
\begin{center}
\begin{tabular}{lll}
\hline
Symbol & meaning & unit \\
\hline
\hline
$t$ & Time & s \\
$x,y,z$ & Cartesian coordinates & m \\
${\bm v}$ & velocity vector & m$\cdot$ s$^{-1}$\\
$\rho$ & mass density & kg/m$^3$ \\
$\eta$ & dynamic viscosity &  Pa$\cdot$ s \\
$\lambda$ & penalty parameter & Pa$\cdot$ s \\
$T$ & temperature & K \\
${\bm \nabla}$ & gradient operator & m$^{-1}$ \\
${\bm \nabla}\cdot$ & divergence operator & m$^{-1}$ \\
$p$ & pressure & Pa\\
$\dot{\bm \epsilon}({\bm v})$ & strain rate tensor & s$^{-1}$ \\
$\alpha$ & thermal expansion coefficient & K$^{-1}$ \\
$k$ & thermal conductivity & W/(m $\cdot$ K) \\
$C_p$ & Heat capacity & J/K \\
\hline
\end{tabular}
\end{center}

[from aspect manual]
We focus on the system of equations in a $d=2$- or $d=3$-dimensional
domain $\Omega$ that describes the motion of a highly viscous fluid driven
by differences in the gravitational force due to a density that depends on
the temperature. In the following, we largely follow the exposition of this
material in Schubert, Turcotte and Olson \cite{scto01}.

Specifically, we consider the following set of equations for velocity $\mathbf
u$, pressure $p$ and temperature $T$:
\begin{align}
  \label{eq:stokes-1}
  -\nabla \cdot \left[2\eta \left(\dot\varepsilon(\bm v)
                                  - \frac{1}{3}(\nabla \cdot \bm v)\mathbf 1\right)
                \right] + \nabla p &=
  \rho \bm g
  &
  & \textrm{in $\Omega$},
  \\
  \label{eq:stokes-2}
  \nabla \cdot (\rho \bm v) &= 0
  &
  & \textrm{in $\Omega$},
  \\
  \label{eq:temperature}
  \rho C_p \left(\frac{\partial T}{\partial t} + \bm v\cdot\nabla T\right)
  - \nabla\cdot k\nabla T
  &=
  \rho H
  \notag
  \\
  &\quad
  +
  2\eta
  \left(\dot\varepsilon(\bm v) - \frac{1}{3}(\nabla \cdot \bm v)\mathbf 1\right)
  :
  \left(\dot\varepsilon(\bm v) - \frac{1}{3}(\nabla \cdot \bm v)\mathbf 1\right)
  \\
  &\quad
  +\alpha T \left( \bm v \cdot \nabla p \right)
  \notag
  \\
  &\quad
  + \rho T \Delta S \left(\frac{\partial X}{\partial t} + \bm v\cdot\nabla X\right)
  &
  & \textrm{in $\Omega$},
  \notag
\end{align}
where $\dot\varepsilon(\mathbf u) = \frac{1}{2}(\nabla \mathbf u + \nabla\mathbf
u^T)$ is the symmetric gradient of the velocity (often called the
\textit{strain rate}).%

In this set of equations, \eqref{eq:stokes-1} and \eqref{eq:stokes-2}
represent the compressible Stokes equations in which $\mathbf v=\mathbf
v(\mathbf x,t)$ is the velocity field and $p=p(\mathbf x,t)$ the pressure
field. Both fields depend on space $\mathbf x$ and time $t$. Fluid flow is
driven by the gravity force that acts on the fluid and that is proportional to
both the density of the fluid and the strength of the gravitational pull.

Coupled to this Stokes system is equation \eqref{eq:temperature} for the
temperature field $T=T(\mathbf x,t)$ that contains heat conduction terms as
well as advection with the flow velocity $\mathbf v$. The right hand side
terms of this equation correspond to
\begin{itemize}
\item internal heat production for example due to radioactive decay;
\item friction heating;
\item adiabatic compression of material;
\item phase change.
\end{itemize}
The last term of the temperature equation corresponds to
the latent heat generated or consumed in the process of phase change of material. 
In what follows we will not assume that no phase change takes place so that we disregard this term 
altogether.

%---------------------------------
\subsection{the Boussinesq approximation: an Incompressible flow}

\index{Boussinesq}

[from aspect manual]
The Boussinesq approximation assumes that the density can be
considered constant in all occurrences in the equations with the exception of
the buoyancy term on the right hand side of \eqref{eq:stokes-1}. The primary
result of this assumption is that the continuity equation \eqref{eq:stokes-2}
will now read
\[
{\bm \nabla}\cdot{\bm v} = 0
\]
This implies that the strain rate tensor is deviatoric.
Under the Boussinesq approximation, the equations are much simplified:

\begin{align}
  \label{eq:stokes-1}
  -\nabla \cdot \left[2\eta \dot\varepsilon(\bm v)
                \right] + \nabla p &=
  \rho \bm g
  &
  & \textrm{in $\Omega$},
  \\
  \label{eq:stokes-2}
  \nabla \cdot (\rho \bm v) &= 0
  &
  & \textrm{in $\Omega$},
  \\
  \label{eq:temperature}
  \rho_0 C_p \left(\frac{\partial T}{\partial t} + \bm v\cdot\nabla T\right)
  - \nabla\cdot k\nabla T
  &=
  \rho H
  &
  & \textrm{in $\Omega$}
\end{align}
Note that all terms on the rhs of the temperature equations have disappeared, with the exception 
of the source term.





