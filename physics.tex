
\begin{center}
\begin{tabular}{lll}
\hline
Symbol & meaning & unit \\
\hline
\hline
$t$ & Time & s \\
$x,y,z$ & Cartesian coordinates & m \\
$r,\theta$ & Polar coordinates & m,-\\
$r,\theta, z$ & Cylindrical coordinates & m,-,m\\
$r,\theta,\phi$ & Spherical coordinates & m,-,- \\
${\vec \upnu}$ & velocity vector & m$\cdot$ s$^{-1}$\\
${\vec u}$ & displacement vector & m \\
$\rho$ & mass density & kg/m$^3$ \\
$\eta$ & dynamic viscosity &  Pa$\cdot$ s \\
$\lambda$ & penalty parameter & Pa$\cdot$ s \\
$T$ & temperature & K \\
${\vec \nabla}$ & gradient operator & m$^{-1}$ \\
${\vec \nabla}\cdot$ & divergence operator & m$^{-1}$ \\
$p$ & pressure & Pa\\
$\dot{\bm \varepsilon}({\vec \upnu})$ & strain rate tensor & s$^{-1}$ \\
$\alpha$ & thermal expansion coefficient & K$^{-1}$ \\
$k$ & thermal conductivity & W/(m $\cdot$ K) \\
$C_p$ & Heat capacity & J/K \\
$H$ & intrinsic specific heat production & W/kg\\
$\beta_T$ & isothermal compressibility & Pa$^{-1}$  \\
${\bm \tau}$ & deviatoric stress tensor & Pa \\
${\bm \sigma}$ & full stress tensor & Pa \\
\hline
\end{tabular}
\end{center}

%------------------------------------------------------------------------
\subsection{Strain rate and spin tensor}
\index{velocity gradient}
\index{strain rate}
\index{spin tensor}

The velocity gradient is given in Cartesian coordinates by:
\begin{equation}
\vec\nabla\vec\upnu = 
\left(
\begin{array}{ccc}
\frac{\partial u}{\partial x} & \frac{\partial v}{\partial x} & \frac{\partial w}{\partial x} \\\\
\frac{\partial u}{\partial y} & \frac{\partial v}{\partial y} & \frac{\partial w}{\partial y} \\\\
\frac{\partial u}{\partial z} & \frac{\partial v}{\partial z} & \frac{\partial w}{\partial z} 
\end{array}
\right)
\end{equation}
It can be decomposed into its symmetric and skew-symmetric parts according to:
\begin{equation}
\vec\nabla\vec\upnu = \vec\nabla^s\vec\upnu + \vec\nabla^w\vec\upnu
\end{equation}
The symmetric part is called the strain rate (or rate of deformation):
\begin{equation}
\dot{\bm \epsilon}(\vec \upnu) = \frac{1}{2}\left( \vec\nabla\vec\upnu + \vec\nabla\vec\upnu^T \right)
\end{equation}
The skew-symmetric tensor is called spin tensor (or vorticity tensor):
\begin{equation}
\dot{\bm R}(\vec \upnu) = \frac{1}{2}\left( \vec\nabla\vec\upnu - \vec\nabla\vec\upnu^T \right)
\end{equation}

%------------------------------------------------------------------------
\subsection{Stress tensor}

\index{Stress tensor} \index{Normal stress} \index{Shear stress} \index{Stress vector}

The Cauchy tensor\footnote{\url{https://en.wikipedia.org/wiki/Cauchy_stress_tensor}} 
consists of nine components $\sigma_{ij}$  that completely define the state of stress 
at a point inside a material. 
The tensor relates a unit-length direction vector $\vec{n}$ to the so-called 'stress vector' (most commonly called 'traction') $\vec{t}(\vec{n})$ across an imaginary surface perpendicular to $\vec{n}$:
\[
\vec{t}(\vec n)={\bm \sigma}\cdot {\vec n}
\]
With respect to an orthonormal basis $\{\vec{e}_x,\vec{e}_y,\vec{e}_z\}$, the Cauchy stress tensor
is given by:
\begin{equation}
{\bm \sigma}=
\left(
\begin{array}{ccc}
\sigma_{xx} & \sigma_{xy} & \sigma_{xz} \\
\sigma_{yx} & \sigma_{yy} & \sigma_{yz} \\
\sigma_{zx} & \sigma_{zy} & \sigma_{zz} 
\end{array}
\right)
\end{equation}
The three diagonal elements are called normal stresses while the off-diagonal terms 
are called shear stresses.

One can easily prove (see for instance Section 3.3.6 of \cite{grbl09}) that the balance 
of angular momentum leads reduces to the statement that the Cauchy stress tensor 
is symmetric, i.e. ${\bm \sigma}={\bm \sigma}^T$.

The SI units of both stress tensor and traction are $\text{N}/\text{m}^2$.

%.............................................
\subsubsection{Compressible Newtonian Fluid}

For the compressible case, ${\bm \sigma}$ 
depends linearly on the strain rate tensor $\dot{\bm \varepsilon}$: 
\begin{equation}
{\bm \sigma} = -p(\rho,T) {\bm 1} + \lambda \; \text{tr} (\dot{\bm \varepsilon}) {\bm 1} + 2\eta \dot{\bm \varepsilon}
\end{equation}
where $p$ is the thermodynamic pressure which is a function of the density $\rho$ and the temperature $T$ (an equation of state is then needed). $\lambda$ and $\eta$ are the coefficients of viscosity. 
We also have:
\begin{equation}
{\bm \sigma} = -p(\rho,T) {\bm 1} + \zeta \; \text{tr} (\dot{\bm \varepsilon}) {\bm 1} + 2\eta \dot{\bm \varepsilon}^d
\end{equation}
where $\zeta=\lambda+2\eta/3$ is the bulk viscosity and $\eta$ is the shear viscosity.



%.............................................
\subsubsection{Incompressible Newtonian Fluid}

In this case the stress tensor is 
\begin{equation}
{\bm \sigma}=-p {\bm 1} + {\bm \tau}
\end{equation}
where $p=-1/3 \; \text{tr} (\bm \sigma)$ and ${\bm \tau}$ is the deviatoric stress tensor:
\begin{equation}
{\bm \tau}=2\eta \dot{\bm \varepsilon}^d
\end{equation}


%------------------------------------------------------------------------
\subsection{The heat transport equation - energy conservation equation}

Let us start from the heat transport equation as shown in Schubert, Turcotte and Olson \cite{scto01}:
\begin{equation}
\rho C_p \frac{DT}{Dt} - \alpha T \frac{Dp}{Dt} = {\vec \nabla} \cdot k {\vec \nabla} T + \Phi + \rho H  
\end{equation}
with $D/Dt$ being the total derivatives so that 
\begin{equation}
\frac{DT}{Dt} = \frac{\partial T}{\partial t} + {\vec \upnu}\cdot {\vec \nabla}T
\quad\quad
\frac{Dp}{Dt} = \frac{\partial p}{\partial t} + {\vec \upnu}\cdot {\vec \nabla}p
\end{equation}
Solving for temperature, this equation is often rewritten as follows:
\begin{mdframed}[backgroundcolor=blue!5]
\begin{equation}
\rho C_p \frac{DT}{Dt} - {\vec \nabla} \cdot k {\vec \nabla} T =  \alpha T \frac{Dp}{Dt} + \Phi + \rho H  
\end{equation}
\end{mdframed}
where $\Phi$ is the shear heating \cite[p287]{reddybook2}. In many publications, $\Phi$ 
is given by $\Phi=\tau_{ij}\partial_j u_i={\bm \tau}:{\vec \nabla}{\vec \upnu}$.

\begin{eqnarray}
\Phi 
&=& \tau_{ij}\partial_j u_i \nonumber\\
&=& 2 \eta \dot{\varepsilon}_{ij}^d\partial_j u_i \nonumber\\
&=& 2 \eta \frac{1}{2}\left( \dot{\varepsilon}_{ij}^d\partial_j u_i + \dot{\varepsilon}_{ji}^d\partial_i u_j \right) \nonumber\\
&=& 2 \eta \frac{1}{2}\left( \dot{\varepsilon}_{ij}^d\partial_j u_i + \dot{\varepsilon}_{ij}^d\partial_i u_j \right) \nonumber\\
&=& 2 \eta  \dot{\varepsilon}_{ij}^d  \frac{1}{2}\left(\partial_j u_i + \partial_i u_j \right) \nonumber\\
&=& 2 \eta  \dot{\varepsilon}_{ij}^d   \dot{\varepsilon}_{ij} \nonumber\\
&=& 2 \eta  \dot{\bm \varepsilon}^d :  \dot{\bm \varepsilon} \nonumber\\
&=& 2 \eta  \dot{\bm \varepsilon}^d : \left( \dot{\bm \varepsilon}^d +\frac{1}{3} ({\vec \nabla}\cdot{\vec \upnu}) {\bm 1} \right)\nonumber\\
&=& 2 \eta  \dot{\bm \varepsilon}^d : \dot{\bm \varepsilon}^d 
+ 2 \eta  \dot{\bm \varepsilon}^d : {\bm 1} ({\vec \nabla}\cdot{\vec \upnu}) \nonumber\\ 
&=& 2 \eta  \dot{\bm \varepsilon}^d : \dot{\bm \varepsilon}^d 
\end{eqnarray}
Finally
\[
\Phi = {\bm \tau}:{\vec \nabla}{\vec \upnu} = 2 \eta  \dot{\bm \varepsilon}^d : \dot{\bm \varepsilon}^d
= 2 \eta \left( (\dot{\varepsilon}_{xx}^d)^2 + (\dot{\varepsilon}_{yy}^d)^2 + 2(\dot{\varepsilon}_{xy}^d)^2 \right)
\]

%------------------------------------------------------------------------
\subsection{The momentum conservation equations} 

Because the Prandlt number is virtually zero in Earth science applications the Navier Stokes 
equations reduce to the Stokes equation:
\begin{equation}
{\vec \nabla}\cdot {\bm \sigma} + \rho {\vec g} = \vec{0}
\end{equation}
Since 
\begin{equation}
{\bm \sigma} = -p {\bm 1} + {\bm \tau}
\end{equation}
it also writes
\begin{equation}
-{\vec \nabla}p + {\vec \nabla}\cdot {\bm \tau} + \rho {\vec g} = \vec{0}
\end{equation}
Using the relationship ${\bm \tau} = 2 \eta \dot{\bm \varepsilon}^d$ we arrive at 
\begin{mdframed}[backgroundcolor=blue!5]
\begin{equation}
-{\vec \nabla}p + {\vec \nabla}\cdot (2 \eta \dot{\bm \varepsilon}^d ) + \rho {\vec g} = \vec{0}
\end{equation}
\end{mdframed}

%------------------------------------------------------------------------
\subsection{The mass conservation equations} 
\index{Solenoidal field} \index{Divergence-free}

The mass conservation equation is given by
\[
\frac{D\rho}{Dt} + \rho {\vec \nabla}\cdot{\vec \upnu} = 0
\]
or, 
\begin{mdframed}[backgroundcolor=blue!5]
\[
\frac{\partial \rho}{\partial t} + {\vec \nabla}\cdot(\rho {\vec \upnu}) = 0
\]
\end{mdframed}
In the case of an incompressible flow, then $\partial \rho/\partial t=0$ and 
${\vec \nabla}\rho=0$, i.e. $D\rho/Dt=0$ and the remaining equation is simply:
\[
{\vec \nabla}\cdot{\vec \upnu} = 0
\]
A vector field that is divergence-free is also called 
solenoidal\footnote{\url{https://en.wikipedia.org/wiki/Solenoidal_vector_field}}.


\subsection{The equations in ASPECT manual}
The following is lifted off the ASPECT manual.
We focus on the system of equations in a $d=2$- or $d=3$-dimensional
domain $\Omega$ that describes the motion of a highly viscous fluid driven
by differences in the gravitational force due to a density that depends on
the temperature. In the following, we largely follow the exposition of this
material in Schubert, Turcotte and Olson \cite{scto01}.

Specifically, we consider the following set of equations for velocity $\mathbf
u$, pressure $p$ and temperature $T$:
\begin{align}
  \label{eq:stokes-1}
  -\vec\nabla \cdot \left[2\eta \left(\dot{\bm \varepsilon}(\vec \upnu)
                                  - \frac{1}{3}(\vec\nabla \cdot \vec \upnu)\mathbf 1\right)
                \right] + \vec\nabla p &=
  \rho \vec g
  &
  & \textrm{in $\Omega$},
  \\
  \label{eq:stokes-2}
  \vec\nabla \cdot (\rho \vec v) &= 0
  &
  & \textrm{in $\Omega$},
  \\
  \label{eq:temperature}
  \rho C_p \left(\frac{\partial T}{\partial t} + \vec \upnu\cdot\vec\nabla T\right)
  - \vec\nabla\cdot k\vec\nabla T
  &=
  \rho H
  \notag
  \\
  &\quad
  +
  2\eta
  \left(\dot\varepsilon(\bm v) - \frac{1}{3}(\vec\nabla \cdot \vec \upnu)\mathbf 1\right)
  :
  \left(\dot\varepsilon(\bm v) - \frac{1}{3}(\vec\nabla \cdot \vec \upnu)\mathbf 1\right)
  \\
  &\quad
  +\alpha T \left( \bm v \cdot \vec\nabla p \right)
  \notag
  \\
  &\quad
  &
  & \textrm{in $\Omega$},
  \notag
\end{align}
where $\dot{\bm \varepsilon}(\vec\upnu) = \frac{1}{2}(\vec\nabla \vec\upnu + \vec\nabla \vec\upnu^T)$ is the symmetric gradient of the velocity (often called the
\textit{strain rate}).%

In this set of equations, \eqref{eq:stokes-1} and \eqref{eq:stokes-2}
represent the compressible Stokes equations in which $\mathbf v=\mathbf
v(\mathbf x,t)$ is the velocity field and $p=p(\mathbf x,t)$ the pressure
field. Both fields depend on space $\mathbf x$ and time $t$. Fluid flow is
driven by the gravity force that acts on the fluid and that is proportional to
both the density of the fluid and the strength of the gravitational pull.

Coupled to this Stokes system is equation \eqref{eq:temperature} for the
temperature field $T=T(\mathbf x,t)$ that contains heat conduction terms as
well as advection with the flow velocity $\mathbf v$. The right hand side
terms of this equation correspond to
\begin{itemize}
\item internal heat production for example due to radioactive decay;
\item friction (shear) heating;
\item adiabatic compression of material;
\end{itemize}

In order to arrive at the set of equations that ASPECT solves, 
we need to 
\begin{itemize}
\item neglect the $\partial p/\partial t$. {\color{red}WHY?}
\item neglect the $\partial \rho / \partial t$ . {\color{red}WHY?}
\end{itemize}
from equations above. 

----------------------------------------

Also, their definition of the shear heating term $\Phi$ is:
\[
\Phi = k_B ({\bm \nabla}\cdot{\bm v})^2 + 2\eta \dot{\bm \varepsilon}^d:\dot{\bm \varepsilon}^d
\]
For many fluids the bulk viscosity $k_B$ is very small and is often taken to be zero, an assumption known
as the Stokes assumption: $k_B=\lambda+2\eta/3=0$. \index{bulk viscosity}
Note that $\eta$ is the dynamic viscosity and $\lambda$ the second viscosity. \index{dynamic viscosity}
\index{second viscosity}
Also, 
\[
{\bm \tau}=2\eta \dot{\bm \varepsilon} + \lambda ({\bm \nabla}\cdot{\bm v}) {\bm 1}
\]
but since $k_B=\lambda+2\eta/3=0$, then $\lambda=-2\eta/3$ so 
\[
{\bm \tau}=2\eta \dot{\bm \varepsilon} -\frac{2}{3}\eta ({\bm \nabla}\cdot{\bm v}) {\bm 1} = 2\eta \dot{\bm \varepsilon}^d
\]







\newpage
%---------------------------------
\subsection{The Boussinesq approximation: an Incompressible flow}

\index{Boussinesq}

[from \aspect{} manual]
The Boussinesq approximation assumes that the density can be
considered constant in all occurrences in the equations with the exception of
the buoyancy term on the right hand side of \eqref{eq:stokes-1}. The primary
result of this assumption is that the continuity equation \eqref{eq:stokes-2}
will now read
\[
{\bm \nabla}\cdot{\bm v} = 0
\]
This implies that the strain rate tensor is deviatoric.
Under the Boussinesq approximation, the equations are much simplified:

\begin{align}
  \label{eq:stokes-1}
  -\nabla \cdot \left[2\eta \dot{\bm \varepsilon}(\bm v)
                \right] + \nabla p &=
  \rho \bm g
  &
  & \textrm{in $\Omega$},
  \\
  \label{eq:stokes-2}
  \nabla \cdot (\rho \bm v) &= 0
  &
  & \textrm{in $\Omega$},
  \\
  \label{eq:temperature}
  \rho_0 C_p \left(\frac{\partial T}{\partial t} + \bm v\cdot\nabla T\right)
  - \nabla\cdot k\nabla T
  &=
  \rho H
  &
  & \textrm{in $\Omega$}
\end{align}
Note that all terms on the rhs of the temperature equations have disappeared, with the exception 
of the source term.


\newpage
%%%%%%%%%%%%%%%%%%%%%%%%%%%%%%%%%%%%%%%%%%%%%%%%%%%%%%%%%%%%%%%%%%%%%%%%%%%%%%%%%%%%%%%%%%55
\subsection{Stokes equation for elastic medium}

What follows is mostly borrowed from Becker \& Kaus lecture notes.

%\begin{tabular}{|l|l|l|}
%\hline
%${\bm u}       $ & displacement vector &   \\
%${\bm \sigma}  $ & full stress tensor  & Pa\\
%${\bm \epsilon}$ & strain tensor       &   \\
%${\bm 1}       $ & unit tensor         &   \\
%${\bm f}       $ & body forces         &   \\
%\hline
%\end{tabular}

The strong form of the PDE that governs force balance in a medium is given by
\[
\vec{\nabla}\cdot{\bm \sigma}  + \vec{f} = \vec{0}
\]
where ${\bm \sigma}$ is the stress tensor and $\vec{f}$ is a body force.

The stress tensor is related to the strain tensor through the generalised 
Hooke's law\footnote{\url{https://en.wikipedia.org/wiki/Hooke's_law}}:
\begin{equation}
\sigma_{ij}=\sum_{kl}C_{ijkl}\epsilon_{kl} 
\qquad
\text{or}
\qquad
{\bm \sigma} = {\bm C} : {\bm \epsilon}
\label{eq:one}
\end{equation}
where ${\bm C}$ is the fourth-order elastic tensor.

Due to the inherent symmetries of ${\bm \sigma}$, ${\bm \epsilon}$, and ${\bm C}$, 
only 21 elastic coefficients of the latter are independent. 
For isotropic media (which have the same physical properties in any direction), ${\bm C}$ 
can be reduced to only two independent numbers (for example the bulk modulus $K$ and the shear modulus $G$ 
that quantify the material's resistance to changes in volume and to shearing deformations, respectively). 

One often then write Eq.~\label{eq:one} as follows:
\begin{equation}
\sigma_{ij}=\lambda \epsilon_{kk} \delta_{ij} + 2\mu \epsilon_{ij}
\quad\quad
or, 
\quad\quad
{\bm \sigma} = \lambda (\vec{\nabla}\cdot\vec{u}) {\bm 1} + 2\mu {\bm \epsilon}   \label{eq:two}
\end{equation}
where $\lambda$ is the Lam\'e parameter and $\mu$ is the shear modulus\footnote{It is also sometimes written $G$}.
The term $\vec{\nabla}\cdot\vec{u}$ is the isotropic dilation.

\index{Lam\'e parameter} \index{shear modulus}

The strain tensor is related to the displacement as follows: \index{strain tensor}
\[
{\bm \epsilon} = \frac{1}{2}(\vec{\nabla}\vec{u} + \vec{\nabla}\vec{u}^T)
\]

The incompressibility (bulk modulus), $K$, is defined as $p=-K \vec{\nabla}\cdot\vec{u}$ 
where $p$ is the pressure with \index{bulk modulus}
\begin{eqnarray}
p&=&-\frac{1}{3} \text{tr}({\bm \sigma}) \nonumber\\
 &=& -\frac{1}{3} [ \lambda (\vec{\nabla}\cdot\vec{u}) \text{tr}[{\bm 1}] + 2 \mu \text{tr}[{\bm \epsilon}]] \nonumber\\
 &=& -\frac{1}{3} [ \lambda (\vec{\nabla}\cdot\vec{u})  3  + 2 \mu  (\vec{\nabla}\cdot\vec{u}) ] \nonumber\\
 &=& -\left[ \lambda + \frac{2}{3} \mu \right] (\vec{\nabla}\cdot\vec{u})  
\end{eqnarray}
so that $K=\lambda+\frac{2}{3}\mu$.

%or
%\[
%\mu=\frac{3K(1-2\nu)}{2(1+\nu)}
%\]


\begin{remark}
Eq. (\ref{eq:one}) and (\ref{eq:two}) are analogous to the ones that one has to solve
in the context of viscous flow using the penalty method. In this case $\lambda$ is the penalty coefficient, 
${\bm u}$ is the velocity, and $\mu$ is then the dynamic viscosity.
\end{remark}

%\begin{center}
%\includegraphics[width=15cm]{images/coeffs}\\
%{\small Homogeneous isotropic linear elastic materials have their elastic properties uniquely determined by any two moduli among these; thus, given any two, any other of the elastic moduli can be calculated according to these formulas.}
%\end{center}

The Lam\'e parameter and the shear modulus are also linked to $\nu$ the poisson ratio, 
and $E$, Young's modulus: \index{Poisson ratio} \index{Young's modulus}
\[
\lambda=\mu\frac{2\nu}{1-2\nu}
=\frac{\nu E}{(1+\nu)(1-2\nu)}
\quad\quad
{\rm with}
\quad\quad
E=2\mu(1+\nu)
\]
The shear modulus, expressed often in GPa, describes the material's response to shear stress.
The poisson ratio describes the response in the direction orthogonal to uniaxial stress.
The Young modulus, expressed in GPa, describes the material's strain response to uniaxial stress in the 
direction of this stress.


%%%%%%%%%%%%%%%%%%%%%%%%%%%%%%%%%%%%%%%%%%%%%%%%%%%%%%%%%%%%%%%%%%55
\newpage
\subsection{The strain rate tensor in all coordinate systems}

The strain rate tensor $\dot{\bm\varepsilon}$ is given by
\begin{equation}
\dot{\bm \varepsilon} = \frac{1}{2}( {\vec \nabla}{\vec \upnu}+ {\vec \nabla}{\vec \upnu}^T) 
\end{equation}

%.....................................
\subsubsection{Cartesian coordinates}
\begin{eqnarray}
\dot\varepsilon_{xx} &=& \frac{\partial u}{\partial x} \\
\dot\varepsilon_{yy} &=& \frac{\partial v}{\partial y} \\
\dot\varepsilon_{zz} &=& \frac{\partial w}{\partial z} \\
\dot\varepsilon_{yx} =
\dot\varepsilon_{xy} &=& \frac{1}{2} \left( \frac{\partial u}{\partial y} + \frac{\partial v}{\partial x}  \right)\\
\dot\varepsilon_{zx} =
\dot\varepsilon_{xz} &=& \frac{1}{2} \left( \frac{\partial u}{\partial z} + \frac{\partial w}{\partial x}  \right)\\
\dot\varepsilon_{zy} =
\dot\varepsilon_{yz} &=& \frac{1}{2} \left( \frac{\partial v}{\partial z} + \frac{\partial w}{\partial y}  \right)
\end{eqnarray}

%.....................................
\subsubsection{Polar coordinates}

\begin{eqnarray}
\dot\varepsilon_{rr} &=& \frac{\partial v_r}{\partial r} \\
\dot\varepsilon_{\theta\theta} &=& \frac{v_r}{r} + \frac{1}{r} \frac{\partial v_\theta}{\partial \theta}  \\
\dot\varepsilon_{\theta r} =
\dot\varepsilon_{r\theta} &=& \frac{1}{2} \left(   \frac{\partial v_\theta}{\partial r} - \frac{v_\theta}{r} 
+\frac{1}{r} \frac{\partial v_r}{\partial \theta}  \right) 
\end{eqnarray}

%.....................................
\subsubsection{Cylindrical coordinates}

\begin{eqnarray}
\dot\varepsilon_{rr} &=& \frac{\partial v_r}{\partial r} \\
\dot\varepsilon_{\theta\theta} &=& \frac{v_r}{r} + \frac{1}{r} \frac{\partial v_\theta}{\partial \theta}  \\
\dot\varepsilon_{\theta r} =
\dot\varepsilon_{r\theta} &=& \frac{1}{2} \left(   \frac{\partial v_\theta}{\partial r} - \frac{v_\theta}{r} 
+\frac{1}{r} \frac{\partial v_r}{\partial \theta}  \right)\\
\dot\varepsilon_{zz} &=& \frac{\partial w}{\partial z} \\
\dot{varepsilon}_{rz} = \dot{varepsilon}_{zr} &=& \frac{1}{2}\left(   \right) \\
\dot{varepsilon}_{\theta z} = \dot{varepsilon}_{z \theta} &=& \frac{1}{2}\left( \frac{1}{r} \frac{\partial w}{\partial \theta} + \frac{\partial v_\theta}{\partial z}  \right) \\
\end{eqnarray}
 
CHECK AND HOMOGENIZE NOTATIONS



http://eml.ou.edu/equation/FLUIDS/STRAIN/STRAIN.HTM

%.....................................
\subsubsection{Spherical coordinates}

\begin{eqnarray}
\dot\varepsilon_{rr} &=& \frac{\partial v_r}{\partial r} \\
\dot\varepsilon_{\theta\theta} &=& \frac{v_r}{r} + \frac{1}{r} \frac{\partial v_\theta}{\partial \theta}  \\
\dot\varepsilon_{\phi\phi} &=& \frac{1}{r \sin\theta} \frac{\partial v_\phi}{\partial \phi} \\
\dot\varepsilon_{\theta r} =
\dot\varepsilon_{r\theta}   &=& \frac{1}{2} \left( r \frac{\partial}{\partial r} (\frac{v_\theta}{r} ) 
+\frac{1}{r} \frac{\partial v_r}{\partial \theta} \right) \\
\dot\varepsilon_{\phi r} =
\dot\varepsilon_{r\phi}      &=&  \frac{1}{2} \left(  \frac{1}{r \sin\theta} \frac{\partial v_r}{\partial \phi} 
+ r \frac{\partial }{\partial r} (\frac{v_\phi}{r}) \right)  \\
\dot\varepsilon_{\phi \theta} =
\dot\varepsilon_{\theta\phi} &=& \frac{1}{2} \left( \frac{\sin \theta}{r} \frac{\partial }{\partial \theta} (\frac{v_\phi}{\sin\theta}) + \frac{1}{r \sin\theta} \frac{\partial v_\theta}{\partial \phi}    \right) 
\end{eqnarray}



\newpage
%-------------------------------
\subsection{Boundary conditions}

%wiki
In mathematics, the Dirichlet (or first-type) 
boundary condition is a type of boundary condition, named after Peter Gustav Lejeune Dirichlet.
When imposed on an ODE or PDE, it specifies the values that a solution needs 
to take on along the boundary of the domain.
Note that a Dirichlet boundary condition may also be referred to as a fixed boundary condition. 

The Neumann (or second-type) boundary condition is a type of boundary condition, 
named after Carl Neumann. When imposed on an ordinary or a partial differential equation, 
the condition specifies the values in which the derivative of a solution is 
applied within the boundary of the domain.

It is possible to describe the problem using other boundary conditions: 
a Dirichlet boundary condition specifies the values of the solution itself 
(as opposed to its derivative) on the boundary, whereas the Cauchy boundary condition, mixed boundary condition and Robin boundary condition are all different types of combinations of the Neumann and Dirichlet boundary conditions.

\index{Dirichlet boundary condition}
\index{Neumann boundary condition}





%....................................
\subsubsection{The Stokes equations}

You may find the following terms in the computational geodynamics literature:

\begin{itemize}
\item { free surface}: this means that no force is acting on the surface, i.e. ${\bm \sigma}\cdot {\vec n}={\vec 0}$. It is usually used on the top boundary of the domain and allows for topography evolution.
\item { free slip}: ${\vec \upnu}\cdot \vec n = 0$ and $({\bm \sigma}\cdot{\vec n})\times {\vec n}={\vec 0}$. This condition ensures a frictionless flow parallel to the boundary where it is prescribed.
\item { no slip}: this means that the velocity (or displacement) is exactly zero on the boundary, i.e. ${\vec \upnu}={\vec 0}$.
\item { prescribed velocity}: ${\vec \upnu}={\vec \upnu}_{bc}$
\item stress b.c.: 
\item open .b.c.: see fieldstone 29. 
\end{itemize}

%....................................
\subsubsection{The heat transport equation}

There are two types of boundary conditions for this equation: temperature boundary conditions (Dirichlet boundary conditions) and heat flux boundary conditions (Neumann boundary conditions). 

\newpage
%------------------------------------------
\subsection{Meaningful physical quantities}

\begin{itemize}
\item Velocity $\vec \upnu (\text{m/s})$: This is a vector quantity and both magnitude and direction are needed to define it. It is the rate of change of position with respect to a frame of reference.
\item Root mean square velocity $\upnu_{rms} (\text{m/s})$: 
\begin{equation}
\upnu_{rms} = \left ( \frac{\int_\Omega |{\vec \upnu}|^2 \;  d \Omega}{\int_\Omega d\Omega }  \right )^{1/2}
=\left ( \frac{1}{V_\Omega} \int_\Omega |{\vec \upnu}|^2 \;  d \Omega \right )^{1/2} \label{eqVrms}
\end{equation}
\begin{remark}
$V_\Omega$ is usually computed numerically at the same time $\upnu_{vrms}$ is computed.
\end{remark}
In Cartesian coordinates, for a cuboid domain of size $Lx\times L_y \times Lz$, 
the $\upnu_{rms}$ is simply given by:
\begin{equation}
\upnu_{rms}  = \left ( \frac{1}{L_xL_yL_z} \int_0^{L_x}\int_0^{L_y}\int_0^{L_z} 
(u^2 + v^2 + w^2) dxdydz  \right )^{1/2}
\end{equation}
In the case of an annulus domain, although calculations are carried out 
in Cartesian coordinates, it makes sense
to look at the radial velocity component $\upnu_r$ and the tangential velocity 
component $\upnu_\theta$, and their respective
root mean square averages:
\begin{equation}
\upnu_r|_{rms}  =\left ( \frac{1}{V_\Omega} \int_\Omega v_r^2 \;  d \Omega \right )^{1/2} \label{eqVrVrms}
\end{equation}
\begin{equation}
\upnu_\theta|_{rms}  = \left ( \frac{1}{V_\Omega} \int_\Omega v_\theta^2 \;  d \Omega \right )^{1/2} \label{eqThetaVrms}
\end{equation}


\item Pressure $p (\text{Pa})$:
\item Stress tensor ${\bm \sigma}$ (Pa): \index{Stress Tensor}
\item Strain tensor ${\bm \varepsilon}$ (dimensionless): \index{Strain tensor}
\item Strain rate tensor $ \dot{\bm \varepsilon} (\text{s}^{-1}$): \index{Strain rate tensor}
\item Rayleigh number $Ra$ (X): \index{Rayleigh Number}
\item Prandtl number $Pr$ (X): \index{Prandtl Number} It is named after the German physicist 
Ludwig Prandtl and is defined as the ratio of momentum diffusivity to thermal diffusivity. It is given as: 
\[
Pr = \frac{\text{momentum diffusivity}}{\text{thermal diffusivity}} = \frac{\eta/\rho}{k/(\rho C_p)}= \frac{\eta C_p}{k}
\]
For Earth materials, we have $Pr \sim (10^{21} 1000)/3 >> 1$, which means that momentum diffusivity dominates.


\item Nusselt number $N_u$ (X): \index{Nusselt Number}  the Nusselt number (Nu) is the ratio of convective to conductive heat transfer across (normal to) the boundary. The conductive component is measured under the same conditions as the heat convection but with a (hypothetically) stagnant (or motionless) fluid.

In practice the Nusselt number Nu of a layer (typically the mantle of a planet) is defined as follows:
\begin{equation}
\text{Nu} = \frac{q}{q_c}
\end{equation} 
where $q$ is the heat transferred by convection while $q_c=k \Delta T /D$ 
is the amount of heat that would be conducted through a layer of
thickness $D$ with a temperature difference $\Delta T$ across it with 
$k$ being the thermal conductivity.

For 2D Cartesian systems of size ($L_x$,$L_y$) the Nu is computed \cite{blbc89}
\[
\text{Nu} = 
 \frac{\frac{1}{L_x}\int_{0}^{L_x} k \frac{\partial T}{\partial y}(x,y=L_y) dx }
{-\frac{1}{L_x}\int_0^{L_x} k T(x,y=0) /L_y dx}
=-L_y \frac{\int_{0}^{L_x} \frac{\partial T}{\partial y}(x,y=L_y) dx }{\int_0^{L_x} T(x,y=0) dx}
\]
i.e. it is the mean surface temperature gradient
over the mean bottom temperature.

\todo[inline]{finish. not happy with definition. Look at literature}

Note that in the case when no convection takes place then the measured heat flux at the top is 
the one obtained from a purely conductive profile which yields Nu=1.

Note that a relationship Ra $\propto$ Nu$^\alpha $ exists between the Rayleigh number Ra and the Nusselt number Nu in convective systems, see \cite{wodd09} and references therein. 

Turning now to cylindrical geometries with inner radius $R_1$ and outer radius $R_2$, we define $f=R_1/R_2$. A small value of $f$ corresponds to a high
degree of curvature. We assume now that $R_2-R_1=1$, so that $R_2=1/(1-f)$ and $R_1=f/(1-f)$. 
Following \cite{jarv93}, the Nusselt number at the inner and outer boundaries are:
\begin{equation}
\boxed{
\text{Nu}_{inner} = \frac{f \ln f}{1-f} \frac{1}{2\pi} \int_0^{2\pi} \left( \frac{\partial T}{\partial r} \right)_{r=R_1} d\theta
}
\label{eqNuAnnIn}
\end{equation}
\begin{equation}
\boxed{
\text{Nu}_{outer} = \frac{\ln f}{1-f} \frac{1}{2\pi} \int_0^{2\pi} \left( \frac{\partial T}{\partial r} \right)_{r=R_2} d\theta
}
\label{eqNuAnnOut}
\end{equation}

Note that a conductive geotherm in such an annulus between temperatures $T_1$ and $T_2$ is given by 
\[
T_c(r)=\frac{\ln (r/R_2)}{\ln(R_1/R_2)} = \frac{\ln(r(1-f))}{\ln f}
\]
so that 
\[
\frac{\partial T_c}{\partial r} = \frac{1}{r}\frac{1}{\ln f} 
\]
We then find:
\begin{eqnarray}
\text{Nu}_{inner} 
&=& \frac{f \ln f}{1-f} \frac{1}{2\pi} \int_0^{2\pi} \left( \frac{\partial T_c}{\partial r} \right)_{r=R_1} d\theta
= \frac{f \ln f}{1-f} \frac{1}{R_1}\frac{1}{\ln f} 
= 1 \\
\text{Nu}_{outer} 
&=& \frac{\ln f}{1-f} \frac{1}{2\pi} \int_0^{2\pi} \left( \frac{\partial T_c}{\partial r} \right)_{r=R_2} d\theta 
= \frac{\ln f}{1-f} \frac{1}{R_2}\frac{1}{\ln f} = 1 
\end{eqnarray}
As expected, the recovered Nusselt number at both boundaries is exactly 1 when the temperature field is
given by a steady state conductive geotherm.

\todo[inline]{derive formula for Earth size R1 and R2}

 
\item Temperature (K):
\item Viscosity (Pa.s):
\item Density (kg/m$^3$):
\item Heat capacity $C_p$ ($J.K^{-1}$): It is the measure of the heat energy required to increase the 
temperature of a unit quantity of a substance by unit degree. Note that the specific heat capacity $c_P$ of a 
substance is the heat capacity of a sample of the substance divided by the mass of the sample, with units $J\cdot K\cdot kg^{-1}$.
\item Heat conductivity, or thermal conductivity $k$ ($W.m^{-1}.K^{-1}$). It is the property of a material that indicates its ability to conduct     heat. It appears primarily in Fourier's Law for heat conduction.
Note that it is a function of temperature, especially in mantle convection settings \cite{mika13}.

\item Heat diffusivity: $\kappa=k/(\rho C_p)$ ($m^2.s^{-1}$). Substances with high thermal diffusivity rapidly adjust their temperature to that of their surroundings, because they 
conduct heat quickly in comparison to their volumetric heat capacity or 'thermal bulk'.
\item thermal expansion $\alpha$ (K$^{-1}$): it is the tendency of a matter to change in volume in response to a change in temperature. Note that it is a function of temperature, especially in mantle convection settings \cite{mika13}.

\end{itemize}


\todo[inline]{check aspect manual The 2D cylindrical shell benchmarks by Davies et al. 5.4.12}


\newpage
%--------------------------------------------
\subsection{The need for numerical modelling}

The gouverning equations we have seen in this chapter require the use 
of numerical solution techniques for three main reasons:
\begin{itemize}
\item the advection term in the energy equation couples velocity and temperature;
\item the constitutive law (the relationship between stress and strain rate) 
often depends on velocity (or rather, strain rate), temperature, pressure, ...
\item Even when the coefficients of the PDE's are linear, often their spatial
variability, coupled to potentially complex domain geometries prevent 
arriving at the analytical solution.
\end{itemize}




