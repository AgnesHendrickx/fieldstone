\begin{flushright} {\tiny {\color{gray} pair\_chen.tex}} \end{flushright}
%~~~~~~~~~~~~~~~~~~~~~~~~~~~~~~~~~~~~~~~~~~~~~~~~~~~~~~~~~~~~~~~~~~~~~~~~~~~~~~~~~~~~~~~~~~~~~~~~~~

What follows is tentative!

This space is proposed in \textcite{chen93} (1993), albeit not in the 
context of the Stokes equations.

It is based on the mid-point variant of the RT basis functions, 
\begin{eqnarray}
\bN_1(r,s) &=& \frac{1}{4} (1-2s-(r^2-s^2)) \nonumber\\
\bN_2(r,s) &=& \frac{1}{4} (1+2r+(r^2-s^2)) \nonumber\\
\bN_3(r,s) &=& \frac{1}{4} (1+2s-(r^2-s^2)) \nonumber\\
\bN_4(r,s) &=& \frac{1}{4} (1-2r+(r^2-s^2)) \nonumber
\end{eqnarray}
to which a $P_2$ bubble is added
\[
\phi(r,s) = 1-\frac34(r^2+s^2)
\]
Note thath this function is zero at locations $\pm 1/\sqrt{3}$ 
on all four edges and exactly 1 in the middle. 

A field $f$ is represented inside the element by 
\[
f^h(r,s)=a \bN_1(r,s)
+b \bN_2(r,s)
+c \bN_3(r,s)
+d \bN_4(r,s)
+e \phi(r,s)
\]
We immediately see that this space is not interpolatory, i.e. the basis function $\phi(r,s)$ cannot be 1 in the middle and 0 at the other four nodes. 

\textcite{chen} also extends this to 3D in the paper. 

This space is used for velocity and a $Q_0$ space is used for 
pressure in \stone~120 (only because the basis functions above are
based on the Rannacher-Turek ones).
