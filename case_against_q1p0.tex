What follows was written by Dave May and sent to me by email in May 2014. 
It captures so well the problem at hand that I have decided to reproduce it hereunder.

\hspace{4mm}

In the case of the incompressible Stokes equations, we would like to solve
\[
\left(
\begin{array}{cc}
\K   & \G\\
\G^T & 0
\end{array}
\right)
\left(
\begin{array}{c}
\vec{\cal V}\\
\vec{\cal P}
\end{array}
\right)
= 
\left(
\begin{array}{c}
\vec f \\0
\end{array}
\right)
\]
with an iterative method which is algorithmically scalable and optimal. 
Scalable here would mean
that the number of iterations doesn't grow as the mesh is refined. Optimal means the solution 
time varies linearly with the total number of unknowns. 
When using a stable element,
If we right precondition the above system with
\[
P=
\left(
\begin{array}{cc}
\K   & \G\\
0   & -\SSS
\end{array}
\right)
\]
then convergence will occur in 2 iterations,
however this requires an exact solve on $\K$ 
and on $\SSS = \G^T\cdot \K^{-1}\cdot \G$ ($\SSS$ is the pressure schur complement).
In practice, people relax the ideal "two iteration" scenario by first replacing
$\SSS$ via 
$\SSS^* = \int \eta^{-1} \vec N^T \vec N \, dv$ 
(e.g. the pressure mass matrix scaled by the local inverse of viscosity).
\[
P^*=
\left(
\begin{array}{cc}
\K   & \G\\
0   & -\SSS^*
\end{array}
\right)
\]

Using $P^*$, we obtain iteration counts which are larger than 2, but likely
less than 10 - {\it however}, the number of iterations is independent of the mesh size.
Replacing the exact $\K$ solve in $P^*$ again increases the iterations required to solve Stokes,
but it's still independent of the number of elements. When you have this behaviour,
we say the preconditioner ($P^*$) is spectrally equivalent to the operator (which here is Stokes)

The problem with $Q_1\times P_0$ is that there are no approximations for
$\SSS$ which can be generated that ensure a spectrally equivalent $P^*$. Thus, as you refine
the mesh using $Q_1 \times P_0$ elements, the iteration count ALWAYS grows. I worked on this problem
during my thesis, making some improvements to the situation - however the problem still remains,
it cannot be completely fixed and stems entirely from using unstable elements.

Citcom solvers works like this:
\begin{enumerate}
\item Solve $\SSS \cdot {\cal P} = \vec f'$  for pressure
\item Solve $\K \cdot {\cal V} = \vec f - \G \cdot {\cal P}$ for velocity
\end{enumerate}
To obtain a scalable method, we need the number of iterations performed in (1) and (2)
to be independent of the mesh. This means we need a spectrally equivalent preconditioner
for $\SSS$ and $\K$. Thus, we have the same issue as when you iterate on the full stokes system.

When we don't have a scalable method, it means increasing the resolution requires 
more cpu time in a manner which cannot be predicted. The increase in iteration counts 
as the mesh is refined can be dramatic.

If we can bound the number of iterations, AND ensure that the cost per iteration is 
linearly related to the number of unknowns, then we have a good method which can 
run on any mesh resolution with a predictable cpu time. Obtaining scalable and 
optimal preconditioners for $\K$ is somewhat easier. Multi-grid will provide us with this.

The reason citcom doesn't run with $400^3$ elements is exactly due to this issue.
I've added petsc support in citcom (when i was young and naive) - but the root cause of the
non-scalable solve is directly caused by the element choice. Note that many of the high resolution
citcom jobs are single time step calculations--- there is a reason for that.

For many lithosphere dynamics problems, we need a reasonable resolution (at least $200^3$ and realistically $400^3$ to $800^3$). Given the increase in cost which occurs when using Q1P0, this is not achievable, as the citcom code has demonstrated.
Note that citcom is 20 years old now and for its time, it was great, 
but we know much more now and we know how to improve on it.
As a result of this realization, I dumped all my old Q1P0 codes (and Q1Q1 codes, but for other reasons) 
in the trash and started from scratch. The only way to make something like $800^3$ tractable is via iterative, scalable and optimal methods and
that mandates stable elements. I can actually run at something like $1000^3$ (nodal points) these days because of such design choices.


