
This element is shown in Table~3.13-2 of Gresho \& Sani's book \cite{grsa}, 
and discussed in Section~3.13.6b of the book too. It is {\it not} LBB stable
and has one chequerboard pressure mode.

Used in \textcite{grsu02} (2002) and compared with $Q_1\times P_0$, $Q_2\times P_{-1}$ and 
$Q_2\times Q_1$ for thermal cavity problem with NS equations.

It is used (alongside many other element pairs) in \textcite{chgs02} (2002) in the context of 
a flow benchmark in a 2D box. The authors conclude that ''[...] the Q2-Q-1 element fared
slightly better than the Q2-P-1 . Most surprising, though, were the good results obtained with
the 'old' Taylor–Hood element, Q2-Q1 .''

It is also used in \textcite{grsu02} (2002) on a similar benchmark setup (8:1 thermal 
cavity problem) along with Q1Q0, Q2Q1 and Q2P-1. The authors state that Q2Q-1 has div- stability problems
but ``produces excellent results and is still useful in general.''
They also state ``If the pesky-mode instability could be eciently dealt with, then the Q2xQ-1 element
should be employed over the Q2xP-1 -especially in 3D (we believe).''
Authors mention that it was also used in \textcite{dejo83} and that it ``performed
EXTREMELY WELL.''

 
