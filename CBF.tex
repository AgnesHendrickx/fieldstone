The Consistent Boundary Flux technique was devised to 
alleviate the problem of the accuracy of primary variables 
derivatives (mainly velocity and temperature) on boundaries, 
where basis function (nodal) derivatives do not exist.
These derivatives are important since they are needed to compute
the heat flux (and therefore the NUsselt number) or 
dynamic topography and geoid. 


The idea was first introduced in \cite{mizu86} and later used 
in geodynamics \cite{zhgh93}. It was finally implemented 
in the CitcomS code \cite{zhmt08} and more recently
in the ASPECT code (dynamic topography postprocessor).
Note that the CBF should be seen as a post-processor step 
as it does not alter the primary variables values.

The CBF method is implemented and used in \ref{f_XX}.

%---------------------------------------------------------------
\subsubsection{applied to the Stokes equation}
We start from the strong form:
\[
{\bm \nabla}\cdot{\bm \sigma} = {\bm b}
\]
and then write the weak form:
\[
\int_\Omega N {\bm \nabla}\cdot{\bm \sigma} dV = \int_\Omega N {\bm b} dV
\]
where $N$ is any test function. We then use the two equations:
\[
\bm \nabla \cdot ( N  \bm \sigma ) = N \bm \nabla \cdot \bm \sigma + \bm \nabla N \cdot  \bm \sigma  
\quad\quad \text{(chain rule)}
\]
\[
\int_\Omega (\bm \nabla \cdot \bm f )\; dV = \int_\Gamma \bm f\cdot \bm n \; dS
\quad\quad \text{(divergence theorem)}
\]
Integrating the first equation over $\Omega$ and using the second, we can write:
\[
\int_\Gamma N {\bm \sigma}\cdot {\bm n} \; dS 
-  \int_\Omega {\nabla N} \cdot{\bm \sigma} \; dV 
= \int_\Omega N {\bm b} dV
\]
On $\Gamma$, the traction vector is given by ${\bm t}={\bm \sigma}\cdot {\bm n}$: 
\[
\int_\Gamma N {\bm t} dS =  \int_\Omega {\nabla N} \cdot{\bm \sigma} dV + \int_\Omega N {\bm b} dV
\]
Considering the traction vector as an unknown living on the nodes on the boundary, 
we can expand (for $Q_1$ elements)
\[
t_x = \sum_{i=1}^2 t_{x|i} N_i 
\quad\quad
t_y = \sum_{i=1}^2 t_{y|i} N_i 
\]
on the boundary so that the left hand term yields a mass matrix $M'$.
Finally, using our previous experience of discretising the weak form, we can write:
\[
M' \cdot {\cal T} = -\K {\cal V} - \G {\cal P} + f
\]
where ${\cal T}$ is the vector of assembled tractions which we want to compute 
and ${\cal V}$ and ${\cal T}$ are the solutions of the Stokes problem. 
Note that the assembly
only takes place on the elements along the boundary.

Note that the assembled mass matrix is tri-diagonal can be easily solved with 
a Conjugate Gradient method. With a trapezoidal integration rule 
(i.e. Gauss-Lobatto) the matrix can even be diagonalised and the resulting 
matrix is simply diagonal, which results in a very cheap solve \cite{zhgh93}.

%---------------------------------------------------------------
\subsubsection{applied to the heat equation}
We start from the strong form of the heat transfer equation (without the source terms for simplicity):
\[
\rho c_p
\left(\frac{\partial T}{\partial t} + {\bm v}\cdot {\bm \nabla}T\right)
=
{\bm \nabla} \cdot k{\bm \nabla T}
\]
The weak form then writes:
%\[
%\int_\Omega N
%\rho c_p
%\left(\frac{\partial T}{\partial t} + {\bm v}\cdot {\bm \nabla}T\right) dV
%=
%\int_\Omega N
%{\bm \nabla} \cdot k{\bm \nabla T} dV
%\]
\[
\int_\Omega N
\rho c_p
\frac{\partial T}{\partial t} dV 
+
\rho c_p
\int_\Omega N
 {\bm v}\cdot {\bm \nabla}T  dV
=
\int_\Omega N
{\bm \nabla} \cdot k{\bm \nabla T} dV
\]
Using once again integration by parts and divergence theorem:
\[
\int_\Omega N
\rho c_p
\frac{\partial T}{\partial t} dV 
+
\rho c_p
\int_\Omega N
 {\bm v}\cdot {\bm \nabla}T  dV
=
\int_\Gamma N k {\bm \nabla T} \cdot {\bm n} d\Gamma
-
\int_\Omega  {\bm \nabla} N \cdot k{\bm \nabla T} dV
\]
On the boundary we are interested in the heat flux ${\bm q}=-k {\bm \nabla T}$
\[
\int_\Omega N
\rho c_p
\frac{\partial T}{\partial t} dV 
+
\rho c_p
\int_\Omega N
 {\bm v}\cdot {\bm \nabla}T  dV
=
-\int_\Gamma N {\bm q} \cdot {\bm n} d\Gamma
- \int_\Omega  {\bm \nabla} N \cdot k{\bm \nabla T} dV
\]
or,
\[
\int_\Gamma N {\bm q} \cdot {\bm n} d\Gamma
=
-\int_\Omega N
\rho c_p
\frac{\partial T}{\partial t} dV 
-\rho c_p
\int_\Omega N
 {\bm v}\cdot {\bm \nabla}T  dV
- \int_\Omega  {\bm \nabla} N \cdot k{\bm \nabla T} dV
\]
Considering the normal heat flux $q_n = {\bm q} \cdot {\bm n}$ as an unknown 
living on the nodes on the boundary, 
\[
q_n = \sum_{i=1}^2 q_{n|i} N_i
\]
so that the left hand term becomes a mass matrix for the shape functions living on 
the boundary.
We have already covered the right hand side terms when building the FE system 
to solve the heat transport equation, so that in the end 
\[
M' \cdot {\cal Q}_n =
- M \cdot \frac{\partial \bm T}{\partial t} -K_a \cdot {\bm T} - K_d \cdot {\bm T} 
\]
where ${\cal Q}_n$ is the assembled vector of normal heat flux components.
Note that in all terms the assembly only takes place over the elements along the boundary.












