
Let us consider a quadrilateral element with one degree of freedom per node and let us assume that we are solving the temperature equation. The local matrix and right-hand side vector are given by 
\[
A_{el}(4\times 4) \quad\quad {\text and} \quad\quad B_{el}(4)
\]
Let us assume that we want to impose $\tilde{T}=10$ on the third node (local coordinates numbering). For instance, having built $A_{el}$ and $B_{el}$, the system looks like :
\[
\left(
\begin{array}{cccc}
3 & 1 & 6  & 9 \\
5 & 2 & 2  & 8 \\
7 & 4 & 11 & 2 \\
9 & 6 & 4  & 3
\end{array}
\right)
\left(
\begin{array}{c}
T_1 \\ T_2 \\ T_3 \\ T_4
\end{array}
\right)
=
\left(
\begin{array}{c}
4 \\ 3 \\ 1 \\ 2
\end{array}
\right)
\]



which can be rewritten 
\[
3 T_1 + T_2 + 6 T_3 + 9 T_4 = 4
\]
\[
5 T_1 + 2T_2 + 2 T_3 + 8 T_4 = 3
\]
\[
7 T_1 + 4T_2 + 11 T_3 + 2 T_4 = 1
\]
\[
9 T_1 + 6T_2 + 4 T_3 + 3 T_4 = 2
\]
or, 
\[
3 T_1 + T_2 + \quad  + 9 T_4 = 4 - 6T_3
\]
\[
5 T_1 + 2T_2 + \quad + 8 T_4 = 3 - 2T_3
\]
\[
7 T_1 + 4T_2 + 11T_3 + 2 T_4 = 1 
\]
\[
9 T_1 + 6T_2 + \quad + 3 T_4 = 2 - 4T_3
\]


\begin{itemize}

\item \underline{Technique 1:} Replace the hereabove system by
\[
\left(
\begin{array}{cccc}
3 & 1 & 6  & 9 \\
5 & 2 & 2  & 8 \\
7 & 4 & 11 +  10^{12} & 2 \\
9 & 6 & 4  & 3
\end{array}
\right)
\left(
\begin{array}{c}
T_1 \\ T_2 \\ T_3 \\ T_4
\end{array}
\right)
=
\left(
\begin{array}{c}
4 \\ 3 \\ \tilde{T}\times (11 + 10^{12}) \\ 2
\end{array}
\right)
\]




\item \underline{Technique 2:} One can choose not to solve for $T_3$ anymore, i.e. not to consider it as a degree of freedom and therefore write:

\[
3 T_1 + T_2 + 9 T_4 = 4 - 6T_3
\]
\[
5 T_1 + 2T_2 + 8 T_4 = 3 - 2T_3
\]
\[
9 T_1 + 6T_2 +  3 T_4 = 2 - 4T_3
\]


\item \underline{Technique 3:} Since we want to impose $T_3=10$, then we can write 
\[
3 T_1 + T_2 + \quad  + 9 T_4 = 4 - 6T_3
\]
\[
5 T_1 + 2T_2 + \quad + 8 T_4 = 3 - 2T_3
\]
\[
0 + 0 + T_3 + 0 = 10
\]
\[
9 T_1 + 6T_2 + \quad + 3 T_4 = 2 - 4T_3
\]
and in matrix form :
\[
\left(
\begin{array}{cccc}
3 & 1 & 0  & 9 \\
5 & 2 & 0  & 8 \\
0 & 0 & 1 & 0 \\
9 & 6 & 0  & 3
\end{array}
\right)
\left(
\begin{array}{c}
T_1 \\ T_2 \\ T_3 \\ T_4
\end{array}
\right)
=
\left(
\begin{array}{c}
4 - A_{13} T_3\\ 3 - A_{23}T_3 \\ 10 \\ 2-A_{43} T_3
\end{array}
\right)
\]

\end{itemize}

The first technique is not a good idea in practice as it introduces very large 
values and will likely derail the solver. The second option is somewhat difficult
to implement as it means that elemental matrix and rhs sizes will change from 
element to element and it therefore requires more book-keeping.
The third technique is the one adopted throughout this document. 

As shown in \cite{wuxl08}, it is better to replace the 1 on the diagonal 
by the former diagonal term as it reduces the condition number of the matrix. 
The rhs must then be modified accordingly.
 
