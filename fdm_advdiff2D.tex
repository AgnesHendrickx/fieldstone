
So far, we have mainly focused on the diffusion equation in a non-moving domain 
(relevant for the case of a dike intrusion cooling off 
or for a lithosphere which remains undeformed). 

We now want to consider problems where material moves during the time period under 
consideration and takes temperature anomalies with it (e.g. a plume rising 
through a convecting mantle). 
If the numerical grid remains fixed in the background, the hot temperatures should 
be moved to different grid points at each time step. 

We start again from the heat transport equation of Section~\ref{ss:hte}:
\begin{equation}
\rho C_p \left( \frac{\partial T}{\partial t} + \vec\upnu \cdot \vec\nabla T  \right)=
\vec\nabla \cdot k \vec\nabla T + Q 
\end{equation}
In one-dimensional Cartesian coordinates:
\begin{equation}
\rho C_p \left( \frac{\partial T}{\partial t}  
+ u \frac{\partial T}{\partial x} \right)= 
\frac{\partial }{\partial x} \left(  k  \frac{\partial T}{\partial x} \right)+ Q
\end{equation}
and in 2D
\begin{equation}
\rho C_p \left( \frac{\partial T}{\partial t}   
+ u \frac{\partial T}{\partial x}  
+ v \frac{\partial T}{\partial y} \right) 
=
\frac{\partial }{\partial x} \left(  k  \frac{\partial T}{\partial x} \right)
+
\frac{\partial }{\partial y} \left(  k  \frac{\partial T}{\partial y} \right)
+Q
\end{equation}
and in the case where $k$ is constant in space:
\begin{equation}
\frac{\partial T}{\partial t}   
+ u \frac{\partial T}{\partial x}  
+ v \frac{\partial T}{\partial y} 
=
\kappa \left( 
\frac{\partial^2 T}{\partial x^2} 
+ \frac{\partial^2 T}{\partial y^2} \right) +Q
\end{equation}
Since we have already seen how to deal with 'pure' diffusion equations in the 
previous section, let us now turn to 'pure' advection equations:

\begin{equation}
\frac{\partial T}{\partial t}  + u \frac{\partial T}{\partial x} = 0
\end{equation}
or
\begin{equation}
\frac{\partial T}{\partial t}  + u \frac{\partial T}{\partial x} + v \frac{\partial T}{\partial y}= 0
\end{equation}
where we assume $\vec\upnu=(u,v)$ known. 

Even though the equations appear simple, it is quite tricky to solve them accurately, 
more so than for the diffusion problem. 
This is particularly the case if there are large gradients in the quantity that is to be advected. 





