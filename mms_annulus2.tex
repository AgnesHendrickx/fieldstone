
We seek an exact solution to the incompressible Stokes equations for an isoviscous, isothermal fluid in an  annulus.
Given the geometry of the problem, we work in polar coordinates.
We denote the orthonormal basis vectors by $\vec{e}_r$ and $\vec{e}_\theta$, the inner radius of the annulus by $R_1$ and the outer radius by $R_2$.
Further, we assume that the viscosity $\eta$ is constant, 
which we set to $\eta = 1$, we set the gravity vector to $\vec{g} = -g_r \, \vec{e}_r$ with $g_r = 1$. 

Given these assumptions, the incompressible Stokes equations in the annulus are (see Schubert, Turcotte \& Olson ~\cite{scto01}):
\begin{eqnarray}
A_r =     \frac{\partial^2 \upnu_r}{\partial r^2} + \frac{1}{r} \frac{\partial \upnu_r}{\partial r} +   
      \frac{1}{r^2} \frac{\partial^2 \upnu_r}{\partial \theta^2}
    - \frac{\upnu_r}{r^2} - \frac{2}{r^2} \frac{\partial \upnu_\theta}{\partial \theta} 
-\frac{\partial p}{\partial r}  &=& \rho g_r \label{f35:eq1} \\
A_\theta=
\frac{\partial^2 \upnu_\theta}{\partial r^2} + \frac{1}{r} \frac{\partial \upnu_\theta}{\partial r} + \frac{1}{r^2} \frac{\partial^2 \upnu_\theta}{\partial \theta^2}
+\frac{2}{r^2} \frac{\partial \upnu_r}{\partial \theta} - \frac{\upnu_\theta}{r^2} 
-\frac{1}{r}\frac{\partial p}{\partial \theta} &=& 0
\label{f35:eq2} \\
\frac{1}{r} \frac{\partial (r\upnu_r)}{\partial r} + \frac{1}{r} \frac{\partial \upnu_\theta}{\partial \theta} &=&0 \label{f35:eq3}
\end{eqnarray}
Equations (\ref{f35:eq1}) and (\ref{f35:eq2}) are the momentum equations in polar coordinates while
Equation (\ref{f35:eq3}) is the mass conservation equation (also called continuity equation).
The components of the velocity are obtained from the stream function $\Psi$ as follows:
\[
\upnu_r = \frac{1}{r}\frac{\partial \Psi}{\partial \theta}
\quad\quad
\upnu_\theta = - \frac{\partial \Psi}{\partial r}
\]
where $\upnu_r$ is the radial component and $\upnu_\theta$ is the tangential component of the velocity vector.

The stream function is defined for incompressible (divergence-free) 
flows in 2D (as well as in 3D with axisymmetry).
The stream function can be used to plot streamlines, 
which represent the trajectories of particles in a steady flow.
From calculus it is known that the gradient vector $\nabla \Psi$
is normal to the curve $\Psi =C$. 
It can be shown that everywhere ${\vec{\upnu}}\cdot \nabla \Psi =0$ 
using the formula for $\vec{u}$ in terms of 
$\Psi$ which proves that level curves of $\Psi$ are streamlines:
\[
{\vec \upnu}\cdot \nabla \Psi 
= \upnu_r \frac{\partial \Psi}{\partial r} + \upnu_\theta \frac{1}{r} \frac{\partial \Psi}{\partial \theta} 
= \frac{1}{r}\frac{\partial \Psi}{\partial \theta} \frac{\partial \Psi}{\partial r} 
- \frac{\partial \Psi}{\partial r} \frac{1}{r} \frac{\partial \Psi}{\partial \theta} 
=0
\] 
In polar coordinates the curl of a vector ${\vec A}$ is
\footnote{\url{https://en.wikipedia.org/wiki/Del_in_cylindrical_and_spherical_coordinates}}:
\[
{\vec \nabla}\times {\vec A}
=
\frac{1}{r}\left(  
\frac{\partial (r A_\theta)}{\partial r}
-
\frac{\partial A_r}{\partial \theta}
\right) \vec{e}_z
\]
Taking the curl of vector ${\vec A}$ (see Eqs.~\eqref{f35:eq1} and \eqref{f35:eq2}) yields:
\[
\frac{1}{r}\left(  
\frac{\partial (r A_\theta)}{\partial r}
- \frac{\partial A_r}{\partial \theta}
\right)
=
\frac{1}{r}\left(  
- \frac{\partial (\rho g_r)}{\partial \theta}
\right)
\]
Multiplying on each side by $r$ 
\[
\frac{\partial (r A_\theta)}{\partial r}
- \frac{\partial A_r}{\partial \theta}
=
- \frac{\partial \rho g_r}{\partial \theta}
\]
If we now replace $A_r$ and $A_\theta$ by their expressions as a function of velocity and pressure, 
we will see that the pressure terms cancel out (which is one of the advantages of working with stream line formulations).

Let us assume the following separation of variables $\boxed{\Psi(r,\theta)=\phi(r)\xi(\theta)}$.
Then 
\[
\upnu_r = \frac{1}{r}\frac{\partial \Psi}{\partial \theta} = \frac{\phi \xi'}{r}
\quad\quad
\upnu_\theta = - \frac{\partial \Psi}{\partial r} = -\phi' \xi
\]
Let us first express $A_r$ and $A_\theta$ as functions of $\phi$ and $\xi$:
\begin{eqnarray}
A_r 
&=& 
\frac{\partial^2 \upnu_r}{\partial r^2} 
+ \frac{1}{r} \frac{\partial \upnu_r}{\partial r} +   
\frac{1}{r^2} \frac{\partial^2 \upnu_r}{\partial \theta^2}
- \frac{\upnu_r}{r^2} 
- \frac{2}{r^2} \frac{\partial \upnu_\theta}{\partial \theta} -\frac{\partial p}{\partial r}\nn\\ 
&=& 
\frac{\partial^2 }{\partial r^2}  \left(\frac{\phi \xi'}{r} \right)
+ \frac{1}{r} \frac{\partial }{\partial r}  \left(\frac{\phi \xi'}{r}\right)  
+\frac{1}{r^2} \frac{\partial^2 }{\partial \theta^2}
\left(\frac{\phi \xi'}{r}\right)
- \frac{1}{r^2} \left(\frac{\phi \xi'}{r}\right)
- \frac{2}{r^2} \frac{\partial }{\partial \theta} 
\left(-\phi' \xi \right) -\frac{\partial p}{\partial r}\nn\\
&=& 
\left(\frac{\phi''}{r} -2\frac{\phi'}{r^2} + 2\frac{\phi}{r^3} \right)\xi'
+ \left(\frac{\phi'}{r^2} - \frac{\phi}{r^3} \right ) \xi' 
+\frac{\phi }{r^3} \xi'''  
-  \frac{\phi \xi'}{r^3}
+ \frac{2}{r^2} \phi' \xi' -\frac{\partial p}{\partial r}\nn\\
&=&
\frac{\phi'' \xi'}{r} + \frac{\phi' \xi'}{r^2} + \frac{\phi\xi''' }{r^3}
-\frac{\partial p}{\partial r}
\nn\\
\frac{\partial A_r}{\partial \theta} &=& 
\frac{\phi'' \xi''}{r} + \frac{\phi' \xi''}{r^2} + \frac{\phi\xi'''' }{r^3}
-\frac{\partial^2 p}{\partial r \partial \theta}
\end{eqnarray}

\begin{eqnarray}
A_\theta 
&=&
\frac{\partial^2 \upnu_\theta}{\partial r^2} 
+ \frac{1}{r} \frac{\partial \upnu_\theta}{\partial r} 
+ \frac{1}{r^2} \frac{\partial^2 \upnu_\theta}{\partial \theta^2}
+\frac{2}{r^2} \frac{\partial \upnu_r}{\partial \theta} 
- \frac{\upnu_\theta}{r^2} 
-\frac{1}{r}\frac{\partial p}{\partial \theta}
\nn\\
&=&
\frac{\partial^2 }{\partial r^2}(-\phi' \xi) 
+ \frac{1}{r} \frac{\partial }{\partial r} (-\phi' \xi)
+ \frac{1}{r^2} \frac{\partial^2 }{\partial \theta^2}(-\phi' \xi)
+\frac{2}{r^2} \frac{\partial }{\partial \theta} 
\left(\frac{\phi \xi'}{r} \right)
- \frac{1}{r^2} (-\phi' \xi) -\frac{1}{r}\frac{\partial p}{\partial \theta} \nn\\
&=& - \phi''' \xi 
- \frac{\phi'' \xi}{r} 
+ \frac{\phi' (\xi -\xi'')}{r^2} 
+\frac{2 \phi \xi''}{r^3} 
-\frac{1}{r}\frac{\partial p}{\partial \theta} \nn\\
r A_\theta &=& 
- \phi''' \xi r
- \phi'' \xi
+ \frac{\phi' (\xi -\xi'')}{r} 
+\frac{2 \phi \xi''}{r^2} 
  -\frac{\partial p}{\partial \theta}\nn\\
\frac{\partial (r A_\theta)}{\partial r} 
&=& 
- \left( \phi''' \xi + \phi'''' \xi r \right)
- \phi''' \xi 
+ (\xi -\xi'') \left( \frac{\phi'' }{r} - \frac{\phi' }{r^2} \right)
+ \left(\frac{2 \phi' \xi''}{r^2} - 2 \frac{2 \phi \xi''}{r^3}\right)
-\frac{\partial^2 p}{\partial \theta \partial r}\nn\\
&=& -2\phi''' \xi - \phi'''' \xi r
+ \frac{1}{r  } \phi'' (\xi -\xi'')
+ \frac{1}{r^2} ( -\phi' (\xi-\xi'') + 2 \phi' \xi'' )
+ \frac{1}{r^3} (-4 \phi \xi''  )
-\frac{\partial^2 p}{\partial \theta \partial r} \nn\\
&=& -2\phi''' \xi - \phi'''' \xi r
+ \frac{\phi''}{r  }  (\xi -\xi'')
+ \frac{\phi'}{r^2}  ( - \xi +3\xi''  )
+ \frac{\phi}{r^3} (-4  \xi''  )
-\frac{\partial^2 p}{\partial \theta \partial r} 
\end{eqnarray}




%%%%%%%%%%%%%%%%%%%%%%%%%%%%%%%%%%%%%%%%%%%%%%%%%%5
\subsubsection*{No slip boundary conditions}

No-slip boundary conditions inside and outside impose that all components of the velocity
must be zero on both boundaries, i.e.
\[
\vec{\upnu}(r=R_1,\theta)=\vec{\upnu}(r=R_2,\theta)=\vec{0}
\]
Due to the separation of variables, and choosing 
$\boxed{\xi(\theta)=\cos(k\theta)}$ we have
\begin{eqnarray}
\upnu_r(r,\theta) 
&=& \frac{1}{r}\frac{\partial \Psi}{\partial \theta} 
=\frac{\phi \xi'}{r}  
=-\frac{1}{r} \phi(r) k \sin(k \theta)  \\
\upnu_\theta(r,\theta) 
&=& - \frac{\partial \Psi}{\partial r} 
= - \phi'(r) \xi (\theta)
= - \phi'(r) \cos(k\theta)
\end{eqnarray}
The velocity divergence is given by
\[
\vec\nabla\cdot\vec\upnu = 
\frac{1}{r} \frac{\partial (r\upnu_r)}{\partial r} + \frac{1}{r} \frac{\partial \upnu_\theta}{\partial \theta} 
=
\frac{1}{r}
\left(
- \phi'(r) k \sin(k\theta) + \phi'(r) k \sin(k\theta) 
\right)=0
\]
so the flow is indeed incompressible.

Since $\xi$ is a function of $\theta$ it is obvious that the only way to insure no-slip boundary conditions for any $\theta$ value is to have all the following four conditions satisfied
\begin{eqnarray}
\phi(R_1)=\phi'(R_1) &=& 0 \\
\phi(R_2)=\phi'(R_2) &=& 0 
\end{eqnarray}
We could then choose
\begin{eqnarray}
\phi(r)&=&(r-R_1)^2(r-R_2)^2 {\cal F}(r) \\
\phi'(r)&=&2(r-R_1)(r-R_2)^2 {\cal F}(r)  +2(r-R_1)^2(r-R_2) {\cal F}(r) +(r-R_1)^2(r-R_2)^2 {\cal F}'(r)
\end{eqnarray}
which are indeed identically zero on both boundaries. Here ${\cal F}(r)$ is any (smooth enough) function of $r$.
A generic form for $\Psi$ could then be
\[
\boxed{
\Psi(r,\theta) = (r-R_1)^2(r-R_2)^2 {\cal F}(r) \cos(k\theta)
}
\]
In what follows I take ${\cal F}(r)=1$ for simplicity.
Then 
\begin{eqnarray}
\phi(r)&=&(r-R_1)^2(r-R_2)^2  \nn\\
&=& (r^2-2rR_1 + R_1^2)(r^2-2rR_2+R_2^2)\nn\\
&=& \underbrace{1}_{a} r^4 + \underbrace{(-2R_1 -2R_2)}_{b}r^3 + \underbrace{(R_1^2+R_2^2+4R_1R_2)}_{c}r^2 
+ \underbrace{(-2R_1R_2^2-2R_1^2R_2)}_{d}r + \underbrace{R_1^2R_2^2}_{e} \nn\\
&=& ar^4 + br^3 + cr^2 + dr + e
\end{eqnarray}    
and then
\begin{eqnarray}
\phi'(r)&=&2(r-R_1)(r-R_2)^2 +2(r-R_1)^2(r-R_2) \\
        &=&2(r-R_1)(r-R_2) (r-R_2+r-R_1) \\
        &=&4(r-R_1)(r-R_2) \left(r-\frac{R_1+R_2}{2}\right) \\
\phi''(r) &=& 8   \left(r-\frac{R_1+R_2}{2}\right)^2  + 4(r-R_1)(r-R_2)\\
\phi'''(r) &=& 24 \left(r-\frac{R_1+R_2}{2}\right) \\
\phi''''(r) &=& 24 \\
\upnu_r(r,\theta) &=& -\frac{1}{r} (r-R_1)^2(r-R_2)^2 \;  k \sin(k \theta)  \\
\upnu_\theta(r,\theta) 
&=&  - 4(r-R_1)(r-R_2) \left(r-\frac{R_1+R_2}{2}\right) \cos(k\theta)
\end{eqnarray}


\noindent In the end the functions $\phi$ and $\xi$ are of the form:
\begin{eqnarray}
\xi(\theta) &=& \cos (k\theta) \\
\phi(r) &=& {\color{orange}ar^4}+{\color{blue}br^3}+
{\color{teal}cr^2}+{\color{red} dr}+ {\color{purple}e}
\end{eqnarray}
with
\begin{eqnarray}
\xi'(\theta) &=& -k \sin (k\theta) \nn\\
\xi''(\theta) &=& -k^2 \cos (k\theta) = -k^2 \xi(\theta)\nn\\
\xi'''(\theta) &=& k^3 \sin (k\theta)\nn\\
\xi''''(\theta) &=& k^4\cos (k\theta) = k^4 \xi(\theta)\nn\\
\phi'(r) &=& 4ar^3+3br^2 +2cr + d \nn\\
\phi''(r) &=& 12 ar^2 + 6 br + 2c \nn\\
\phi'''(r) &=& 24 ar + 6b \nn\\
\phi''''(r) &=& 24a \nn
\end{eqnarray}


\subsubsection*{Finding the pressure and density fields}
We start from the relationship
\[
A_\theta=
\frac{\partial^2 \upnu_\theta}{\partial r^2} + \frac{1}{r} \frac{\partial \upnu_\theta}{\partial r} + \frac{1}{r^2} \frac{\partial^2 \upnu_\theta}{\partial \theta^2}
+\frac{2}{r^2} \frac{\partial \upnu_r}{\partial \theta} - \frac{\upnu_\theta}{r^2} 
-\frac{1}{r}\frac{\partial p}{\partial \theta} 
=
- \phi''' \xi 
- \frac{\phi'' \xi}{r} 
+ \frac{\phi' (\xi -\xi'')}{r^2} 
+\frac{2 \phi \xi''}{r^3} 
-\frac{1}{r}\frac{\partial p}{\partial \theta} 
= 0 
\]
Then, after multiplying all by $r^3$ we have
\begin{eqnarray}
&&r^2\frac{\partial p}{\partial \theta}\\
&=&- r^3\phi''' \xi 
- r^2 \phi'' \xi
+ r \phi' (\xi -\xi'') 
+ 2 \phi \xi''
\\
%&=& - r^3(24 ar + 6b )\cos (k\theta)
%- r^2 (12 ar^2 + 6 br + 2c)  \cos (k\theta) \\
%&&+ r (4ar^3+3br^2 +2cr + d ) (\cos (k\theta) +k^2 %\cos (k\theta))
%+ 2 ( ar^4+br^3+cr^2+dr+e) (-k^2 \cos (k\theta) ) %\\
&=& \cos(k\theta) 
\left[ - r^3(24 ar + 6b )
- r^2 (12 ar^2 + 6 br + 2c)  
+ r (4ar^3+3br^2 +2cr + d ) (1 +k^2 )
+ 2 ( ar^4+br^3+cr^2+dr+e) (-k^2  ) \right]\\
&=& \cos(k\theta) \left[  
-{\color{orange}24 ar^4} - {\color{blue}6br^3} 
-  {\color{orange}12 ar^4} - {\color{blue}6 br^3} 
- {\color{teal} 2cr^2 }
+  ({\color{orange}4ar^4}+{\color{blue}3br^3} + {\color{teal}2cr^2} + {\color{red}dr} )(1+k^2)
- 2k^2({\color{orange}ar^4}+ {\color{blue}br^3}+{\color{teal}cr^2}
+{\color{red}dr}+{\color{purple}e})
\right] \nn\\
&=&  \cos(k\theta) \left[
(-24-12+4+4k^2-2k^2) {\color{orange} ar^4}
+(-6-6+3+3k^2-2k^2) {\color{blue}br^3}
+(-2+2+2k^2-2k^2) {\color{teal}cr^2}
+(1+k^2-2k^2) {\color{red}dr}
+(-2k^2){\color{purple}e}
\right] \nn\\
&=& \cos(k\theta) \left[
2(k^2-16){\color{orange} ar^4}
+(k^2-9){\color{blue}br^3}
+(1-k^2){\color{red}dr}
-2k^2{\color{purple}e}
\right]
\end{eqnarray}
or, 
\[
\frac{\partial p}{\partial \theta}\\
= \cos(k\theta) 
\left[
2(k^2-16)ar^2
+(k^2-9)br
+(1-k^2)\frac{d}{r}
-2k^2\frac{e}{r^2}
\right]
\]
i.e. after integration with respect to $\theta$:
\[
p(r,\theta) = 
\frac{1}{k}\sin(k\theta) 
\left[
2(k^2-16)ar^2
+(k^2-9)br
+(1-k^2)\frac{d}{r}
-2k^2\frac{e}{r^2}
\right]
+f(r)
\]
For simplicity we set $f(r)=0$ and then
\[
\frac{\partial p}{\partial r}
=
\frac{1}{k}\sin(k\theta) 
\left[
4(k^2-16)ar
+(k^2-9)b
-(1-k^2)\frac{d}{r^2}
+4k^2\frac{e}{r^3}
\right]
%+f'(r)
\]
and since we will need it later:
\[
r^3 \frac{\partial p}{\partial r}
=
\frac{1}{k}\sin(k\theta) 
\left[
4(k^2-16) {\color{orange}ar^4}
+(k^2-9) {\color{blue} b r^3}
-(1-k^2) {\color{red} dr}
+4k^2 {\color{purple} e}
\right]
%+r^3 f'(r)
\]
We now turn to
\[
A_r= \frac{\phi'' \xi'}{r} + \frac{\phi' \xi'}{r^2} + \frac{\phi\xi''' }{r^3}
-\frac{\partial p}{\partial r} = \rho g_r
\]
or, after multiplying both sides by $r^3$:
\begin{eqnarray}
r^2  \phi'' \xi' 
+ r \phi' \xi'  
+ \phi\xi''' 
- r^3\frac{\partial p}{\partial r} &=& r^3\rho g_r \nn\\
r^2( 12 ar^2 + 6 br + 2c) (-k \sin (k\theta))
+ r (4ar^3+3br^2 +2cr + d) (-k \sin (k\theta))
+ ( ar^4+br^3+cr^2+dr+e) k^3 \sin (k\theta)  && \nn\\
-\frac{1}{k}\sin(k\theta) 
\left[
4(k^2-16)ar^4
+(k^2-9)b r^3
-(1-k^2)dr
+4k^2e
\right]
%-r^3 f'(r)
 &=& r^3\rho g_r \nn\\
\sin(k\theta)
\left[-k^2( 12 ar^4 + 6 br^3 + 2cr^2)
-k^2  (4ar^4+3br^3 +2cr^2 + dr) 
+k^4 ( ar^4+br^3+cr^2+dr+e)  \right]  && \nn\\
-\sin(k\theta) 
\left[
4(k^2-16)ar^4
+(k^2-9)b r^3
-(1-k^2)dr
+4k^2e
\right]
%-k r^3 f'(r)
&=& k r^3\rho g_r \nn\\
\sin(k\theta)
\left[
(-12k^2-4k^2+k^4-4(k^2-16) )ar^4+
(-6k^2-3k^2+k^4 -(k^2-9))br^3  \right. && \nn\\
\left.
(-2k^2-2k^2+k^4 )cr^2+
(-k^2+k^4+(1-k^2) )dr+
(k^4-4k^2e )
\right] &=& k r^3\rho g_r \nn\\
\sin(k\theta)
\left[
(k^4-20k^2+64) {\color{orange} ar^4}+
(k^4-10k^2+9) {\color{blue} br^3}  
+(k^4-4k^2) {\color{teal}c r^2}
+(k^4-2k^2+1) {\color{red}dr}+
k^2(k^2-4) {\color{purple}e}
\right] &=& k r^3\rho g_r \nn
%\sin(k\theta)
%\left[
%(k^2-4)(k^2-16)a r^4+
%(k^2-1)(k^2-1)br^3  
%+k^2(k^2-4)c r^2
%+(k^2-1)(k^2-1)dr+
%k^2(k^2-4) e
%\right] &=& k r^3\rho g_r \nn\\
\end{eqnarray}
So, assuming $g_r=1$, 
\[
\rho(r,\theta)=\frac{\sin(k\theta)}{k}
\frac{
A{\color{orange} ar^4}
+B{\color{blue} br^3}
+C {\color{teal}c r^2}
+D {\color{red}dr}
+E {\color{purple}e}
}{r^3}
\]
with 
\begin{eqnarray}
A &=& (k^2-4)(k^2-16)\nn \\
B &=& k^4-10k^2+9\nn    \\
C &=& k^2(k^2-4)  \nn    \\
D &=& (k^2-1)(k^2-1) \nn     \\
E &=& k^2(k^2-4)
\end{eqnarray}

In the end:
\begin{eqnarray}
\upnu_r (r,\theta)&=& -\frac{1}{r} (
{\color{orange} ar^4}
+{\color{blue} br^3}
+{\color{teal}c r^2}
+ {\color{red}dr}
+{\color{purple}e}
)k\sin(k\theta)\\
\upnu_\theta(r,\theta)&=& -(4ar^3+3br^2 +2cr + d ) \cos(k\theta) \\
p(r,\theta) &= &
\frac{1}{k}\sin(k\theta) 
\left[
2(k^2-16)ar^2
+(k^2-9)br
+(1-k^2)\frac{d}{r}
-2k^2\frac{e}{r^2}
\right] \\
\rho(r,\theta)&=&\frac{\sin(k\theta)}{k}
\frac{
A{\color{orange} ar^4}+
B{\color{blue} br^3}+
C{\color{teal}c r^2} +
D {\color{red}dr}+
E{\color{purple}e}
}{r^3}
\end{eqnarray}
%with
%\begin{eqnarray}
%a &=& 1  \nn\\
%b &=& -2R_1-2R_2 \nn\\
%c &=& R_1^2+R_2^2+4R_1R_2 \nn\\
%d &=& -2R_1R_2^2-2R_1^2R_2 \nn\\
%e &=& R_1^2R_2^2 \nn\\
%A &=& (k^2-4)(k^2-16)\nn \\
%B &=& k^4-10k^2+9\nn    \\
%C &=& k^2(k^2-4)  \nn    \\
%D &=& (k^2-1)(k^2-1) \nn     \\
%E &=& k^2(k^2-4) \nn
%\end{eqnarray}

\subsubsection*{Finding the density field - alternate \& easier take}
We start this time from 
\[
\frac{\partial (r A_\theta)}{\partial r}
- \frac{\partial A_r}{\partial \theta}
= - \frac{\partial \rho g_r}{\partial \theta}
\]
or, with $g_r=1$,
\begin{eqnarray}
 -2\phi''' \xi - \phi'''' \xi r
+ \frac{\phi''}{r  }  (\xi -\xi'')
+ \frac{\phi'}{r^2}  ( - \xi +3\xi''  )
+ \frac{\phi}{r^3} (-4  \xi''  )
-\frac{\partial^2 p}{\partial \theta \partial r} 
-\frac{\phi'' \xi''}{r} - \frac{\phi' \xi''}{r^2} - \frac{\phi\xi'''' }{r^3}
+\frac{\partial^2 p}{\partial r \partial \theta}
&=&
- \frac{\partial \rho }{\partial \theta} 
\nn\\
 -2\phi''' \xi - \phi'''' \xi r
+ \frac{\phi''}{r  }  (\xi -\xi'')
+ \frac{\phi'}{r^2}  ( - \xi +3\xi''  )
+ \frac{\phi}{r^3} (-4  \xi''  )
-\frac{\phi'' \xi''}{r} - \frac{\phi' \xi''}{r^2} - \frac{\phi\xi'''' }{r^3}
&=&
- \frac{\partial \rho }{\partial \theta} \nn
\end{eqnarray}

Then we note that $\xi''=-k^2 \xi$ and $\xi''''=k^4 \xi$ so that
\begin{eqnarray}
 -2\phi''' \xi - \phi'''' \xi r
+ \frac{\phi''}{r  }  (\xi +k^2\xi)
+ \frac{\phi'}{r^2}  ( - \xi -3k^2\xi  )
+ \frac{\phi}{r^3} (4k^2  \xi  )
-\frac{- k^2\phi'' \xi}{r} - \frac{- k^2 \phi' \xi}{r^2} 
- \frac{k ^4 \phi\xi }{r^3}
&=&
- \frac{\partial \rho }{\partial \theta} \nn\\
\xi \left[ 
-2\phi'''  - \phi''''  r
+ \frac{\phi''}{r  }  (1+k^2)
+ \frac{\phi'}{r^2}  ( - 1 -3k^2  )
+ \frac{\phi}{r^3} (4k^2    )
-\frac{- k^2\phi'' }{r} - \frac{- k^2 \phi' }{r^2} 
- \frac{k ^4 \phi }{r^3}
\right]
&=& 
- \frac{\partial \rho }{\partial \theta} 
\nn\\
\xi(\theta) \left[ 
-2\phi'''  - \phi''''  r
+ \frac{\phi''}{r  }  (1+2k^2)
+ \frac{\phi'}{r^2}  ( - 1 -2k^2  )
+ \frac{\phi}{r^3} (4k^2 -k^4)
\right]
&=& 
- \frac{\partial \rho }{\partial \theta} \nn 
\end{eqnarray}
i.e.
\[
\rho(r,\theta) = -\frac{1}{k}\sin (k\theta) \frac{{\cal G}(r)}{r^3}
\]
with 
\begin{eqnarray}
{\cal G}(r)  
&=&  
- \phi''''  r^4
-2\phi''' r^3 
+ \phi'' r^2  (1+2k^2)
+ \phi' r  ( - 1 -2k^2  )
+ \phi (4k^2 -k^4) \nn\\
&=& - 24a r^4
-2 (24 ar + 6b) r^3
+ ( 12 ar^2 + 6 br + 2c) r^2 (1+2k^2) \nn\\
&&+  (4ar^3+3br^2 +2cr + d) r ( - 1 -2k^2  )
+ (ar^4+br^3+cr^2+dr+e) (4k^2 -k^4) \nn\\
&=& - 24{\color{orange} ar^4}
-2 (24 {\color{orange} ar^4} + 6{\color{blue} br^3})
+ ( 12 {\color{orange} ar^4} + 6 {\color{blue} br^3} + 2{\color{teal}c r^2}) (1+2k^2) \nn\\
&&+  (4{\color{orange} ar^4}+3{\color{blue} br^3} +2{\color{teal}c r^2} + {\color{red}dr})  ( - 1 -2k^2  )
+ ({\color{orange} ar^4}+{\color{blue} br^3}+{\color{teal}c r^2}+{\color{red}dr}+{\color{purple}e}) (4k^2 -k^4) \nn\\
&=& -A{\color{orange} ar^4} - B{\color{blue} br^3} - 
C{\color{teal}c r^2} - D{\color{red}dr} -E{\color{purple}e} \nn
\end{eqnarray}
with 
\begin{eqnarray}
-A &=& -24 -48 +12(1+2k^2) + 4(- 1 -2k^2)+(4k^2 -k^4) \nn\\
  &=& -24 -48 +12 +24k^2 -4 -8k^2 +4k^2 - k^4 \nn\\
  &=& -64 +20k^2 -k^4 \nn\\
  &=& -(k^2-4)(k^2-16) \nn\\
-B &=& -12 + 6(1+2k^2) +3( - 1 -2k^2  ) + (4k^2 -k^4) \nn\\
  &=& -12 +6 + 12k^2 -3-6k^2 + 4k^2 - k^4 \nn\\
  &=& -(k^4 -10k^2 +9)  \nn\\
-C &=& 2(1+2k^2) + 2( - 1 -2k^2  ) + (4k^2 -k^4)\nn \\
  &=& 2 + 4k^2 -2 -4k^2 + 4k^2 -k^4 \nn\\
  &=& 4k^2 -k^4 \nn\\
  &=& -k^2(k^2-4) \nn\\
-D &=& ( - 1 -2k^2  ) + (4k^2 -k^4) \nn\\
  &=& -1 -2k^2 +4k^2 -k^4 \nn\\
  &=& -1 +2k^2 -k^4 \nn\\
  &=& -(k^2-1)(k^2-1) \nn\\
-E &=&  4k^2 -k^4 \nn\\
  &=& -k^2 (k^2-4) \nn
\end{eqnarray}
so in the end we recover
\begin{eqnarray}
\rho(r,\theta) &=& \frac{1}{k}\sin (k\theta) 
\frac{
A{\color{orange} ar^4}+
B{\color{blue} br^3}+
C{\color{teal}c r^2} +
D {\color{red}dr}+
E{\color{purple}e}
}{r^3}
\nn\\
A &=& (k^2-4)(k^2-16)\nn \\
B &=& k^4-10k^2+9\nn    \\
C &=& k^2(k^2-4)  \nn    \\
D &=& (k^2-1)(k^2-1) \nn     \\
E &=& k^2(k^2-4) \nn
\end{eqnarray}
The pressure is obtained as presented before.


\subsubsection*{Root mean square velocity}

\begin{eqnarray}
\upnu_{rms}^2 
&=&  \iint (\upnu_r^2 + \upnu_\theta^2) r drd\theta \nn\\
&=&  \iint \left[ 
(\frac{\phi \xi'}{r})^2 + ( -\phi' \xi)^2
\right] r drd\theta \nn\\
&=&  \iint (\frac{\phi \xi'}{r})^2 rdrd\theta
+ \iint (-\phi' \xi)^2 r drd\theta \nn\\
&=&  \underbrace{\int_{R_1}^{R_2} (\frac{\phi }{r})^2 rdr}_{I_1}
\underbrace{\int_0^{2\pi} (\xi')^2  d\theta}_{I_2}
+
\underbrace{\int_{R_1}^{R_2} (\phi') ^2 r dr}_{I_3} 
\underbrace{\int_0^{2\pi} \xi^2  d\theta}_{I_4} \nn
\end{eqnarray}

\begin{eqnarray}
I_1 &=&     \nn\\
I_2 &=&     \nn\\
I_3 &=&     \nn\\
I_4 &=&     \nn
\end{eqnarray}

Unfinished! Also compute strain rate tensor, stress, total mass, etc ...





%%%%%%%%%%%%%%%%%%%%%%%%%%%%%%%%%%%%%%%%%%%%%%%%%%5
\subsubsection*{Free slip boundary conditions}

\todo[inline]{what follows needs to be checked!!!}

Before postulating the form of $\phi(r)$, let us now turn to the boundary conditions that the flow must fulfill, i.e. free-slip on both boundaries.
Two conditions must be met:

\begin{itemize}
\item ${\bm v} \cdot {\bm n}=0$ (no flow through the boundaries) which yields $u(r=R_1)=0$ and $u(r=R_2)=0$, :
\[
\frac{1}{r}\frac{\partial \Psi}{\partial \theta} (r=R_1,R_2)=0   \quad\quad \forall \theta
\]
which gives us the first constraint since $\Psi(r,\theta)=\phi(r)\xi(\theta)$:
\[
\phi(r=R_1)=\phi(r=R_2)=0  
\]
\item $({\bm \sigma} \cdot {\bm n}) \times {\bm n} = {\bm 0} $  (the tangential stress at the boundary is zero)
which imposes: $\sigma_{\theta r}=0$, with
\[
\sigma_{\theta r}=
2 \eta \cdot \frac{1}{2} \left( \frac{\partial v}{\partial r} - \frac{v}{r} + \frac{1}{r} \frac{\partial u}{\partial \theta}    \right)
= \eta \left( \frac{\partial }{\partial r} (- \frac{\partial \Psi}{\partial r}) -
\frac{1}{r} (- \frac{\partial \Psi}{\partial r}) + \frac{1}{r} \frac{\partial }{\partial \theta} (\frac{1}{r}\frac{\partial \Psi}{\partial \theta})    \right)
\]
Finally $\Psi$ must fulfill (on the boundaries!):
\[
-\frac{\partial^2 \Psi}{\partial r^2} + \frac{1}{r}  \frac{\partial \Psi}{\partial r} + \frac{1}{r^2} \frac{\partial^2 \Psi}{\partial \theta^2}=0
\]
\[
- \phi'' \xi + \frac{1}{r} \phi' \xi +  \frac{1}{r^2}  \phi \xi'' = 0
\]
or, 
\[
- \phi''  + \frac{1}{r} \phi'  -k^2  \frac{1}{r^2}  \phi  = 0
\]
Note that this equation is a so-called Euler Differential 
Equation\footnote{http://mathworld.wolfram.com/EulerDifferentialEquation.html}.
Since we are looking for a solution $\phi$ such that $\phi(R_1)=\phi(R_2)=0 $ then 
the 3rd term of the equation above is by definition zero on the boundaries.
We have to ensure the following equality on the boundary:
\[
- \phi''  + \frac{1}{r} \phi'   = 0\quad\quad \text{for} \quad r=R_1,R_2
\]
The solution of this ODE is of the form $\phi(r)=ar^2+b$ and it becomes 
evident that it cannot satisfy $\phi(r=R_1)=\phi(r=R_2)=0$.




\end{itemize}

{\color{red} Separation of variables leads to solutions which cannot fulfill the free slip 
boundary conditions}








