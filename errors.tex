\index{$L_1$ norm}
\index{$L_2$ norm}
\index{$H^1$ norm}

What follows is written in the case of a two-dimensional model. Generalisation to
3D is trivial. What follows is mostly borrowed from \cite{thmk14}.

When measuring the order of accuracy of the primitive variables $\vec{v}$ and $p$,
it is standard to report errors in both the $L_1$ and the $L_2$ norm.
For a scalar quantity $\Psi$, the $L_1$ and $L_2$ norms are computed as
\[
\norm{\Psi}_1 = \int_V |\Psi| dV
\quad\quad
\quad\quad
\norm{\Psi}_2 = \sqrt{ \int_V \Psi^2 dV }
\]
For a vector quantity $\vec{k}=(k_x,k_y)$ in a two-dimensional space,
the $L_1$ and $L_2$ norms are defined as:
\[
\norm{\vec{k}}_1 = \int_V (|k_x|+|k_y|) dV
\quad\quad
\quad\quad
\norm{\vec{k}}_2 = \sqrt{ \int_V (k_x^2+k_y^2) dV }
\]
To compute the respective norms
the integrals in the above norms can be approximated by splitting them
into their element-wise contributions. The element volume integral can then
be easily computed by numerical integration using Gauss-Legendre quadrature.

The respective $L_1$ and $L_2$ norms for the pressure error can be evaluated via
\[
e_p^h|_1 = \sum_{i=1}^{n_e} \sum_{q=1}^{n_q} |e_p^h(\vec{r}_q)| w_q |J_q|
\quad\quad
\quad\quad
e_p^h|_2=\sqrt{ \sum_{i=1}^{n_e} \sum_{q=1}^{n_q} |e_p^h(\vec{r}_q)|^2 w_q |J_q| }
\]
where $e_p^h(\vec{r}_q)=p^h(\vec{r}_q) - p(\vec{r}_q)$ 
is the pressure error evaluated at the $q$-th quadrature associated with
the $i$th element. $n_e$ and $n_q$ refer to the number of elements and
the number of quadrature points per element.
$w_q$ and $J_q$ are the quadrature weight of the Jacobian associated with
point $q$.

The velocity error $e_{\vec v}^h$ is evaluated using the following two norms
\[
e_{\vec{v}}^h|_1 = \sum_{i=1}^{n_e} \sum_{q=1}^{n_q} [ |e_u^h(\vec{r}_q)| + |e_v^h(\vec{r}_q)| ]    w_q |J_q|
\quad\quad
\quad\quad
e_{\vec v}^h|_2=\sqrt{ \sum_{i=1}^{n_e} \sum_{q=1}^{n_q} \left[ |e_u^h({\bm r}_q)|^2 +  e_v^h({\bm r}_q)|^2 \right] w_q |J_q| }
\]
where $e_u^h(\vec{r}_q)=u^h(\vec{r}_q) - u(\vec{r}_q)$ and $e_v^h(\vec{r}_q)=v^h(\vec{r}_q)-v(\vec{r}_q)$.




\index{$H^1(\Omega)$ space} \index{$H^1$ norm} \index{$H^1$ semi-norm}
Another norm is very varely used in the geodynamics literature but is preferred in the 
Finite Element literature: the so-called $H^1$ norm. The mathematical basis for this
norm and the nature of the $H^1(\Omega)$ Hilbert space is to be found in many FE books \cite{dohu03,john16,hugh}.
This norm is expression is expressed as follows for a function $f$ such that $f,|\nabla f|\in L^2(\Omega)$
\footnote{\url{https://en.wikipedia.org/wiki/Sobolev_space}}
\[
\norm{f}_{H^1} = \left( \int_\Omega ( |f|^2 + |\nabla f|^2  ) d\Omega   \right)^{1/2}
\]
We then have 
\[
e_{\vec v}^h|_{H^1} = \norm{\vec{v}^h-\vec{v}}_{H^1} = \sqrt{
\sum\limits_{i=1}^d 
\int_\Omega  
\left[
({v}_i^h-{v}_i)^2
+
\vec\nabla(v_i^h-v_i)\cdot\vec\nabla(v_i^h-v_i) 
\right] d\Omega   
}
\]
where $d$ is the number of dimensions.
Note that sometimes the following semi-norm is used \cite{dobo04,bodg06}:
\[
e_{\vec v}^h|_{H^1} = \norm{\vec{v}^h-\vec{v}}_{H^1} = \sqrt{
\sum\limits_{i=1}^d 
\int_\Omega  
\left[
\vec\nabla(v_i^h-v_i)\cdot\vec\nabla(v_i^h-v_i) 
\right] d\Omega   
}
\]
 

When computing the different error norms for $e_p$ and $e_{\vec v}$ for a set of numerical experiments with
varying resolution $h$ we expect the error norms to follow the following relationships:
\[
e_{\vec v}^h|_1 = C h^{rvL_1} 
\quad\quad\quad\quad
e_{\vec v}^h|_2 = C h^{rvL_2} 
\quad\quad\quad\quad 
e_{\vec v}^h|_{H^1} = C h^{rvH^1}
\]
\[
e_p^h|_1 = C h^{rpL_1} 
\quad\quad\quad 
e_p^h|_2 = C h^{rpL_2}
\]
where $C$ is a resolution-independent constant
and $rpXX$ and $rvXX$ are the convergence rates for
pressure and velocity in various norms, respectively. 
Using linear regression on the logarithm of the respective error norm and the resolution $h$,
one can compute the convergence rates of the numerical solutions.

As mentioned in \cite{dobo04}, when finite element solutions converge at
the same rates as the interpolants we say that the method is optimal, i.e.:
\index{optimal rate}

\[
e_{\vec v}^h|_{L_2} = {\cal O}(h^3)
\quad\quad\quad\quad
e_{\vec v}^h|_{H^1} = {\cal O}(h^2)
\quad\quad\quad\quad
e_{p}^h|_{L_2} = {\cal O}(h^2)
\]

%\begin{itemize}
%\item For $Q_1P_0$, the theoretical lower bound for $r_v'$ is 2 and for $r_p'$ it is 1
%\item For $Q_2P_{-1}$, the theoretical lower bound for $r_v'$ is 3 and for $r_p'$ it is 2
%\end{itemize}
We note that when using discontinuous pressure space
(e.g., $P_0$, $P_{-1}$), these bounds remain valid even
when the viscosity is discontinuous provided that the element boundaries conform to the discontinuity.

 
\subsubsection{About extrapolation}
\index{Extrapolation}

{\it Section contributed by W. Bangerth and part of Thieulot \& Bangerth [in prep.]}

In a number of numerical benchmarks we
want to estimate the error $X_h-X^\ast$ between a quantity $X_h$ computed
from the numerical solution $\vec{u}_h,p_h$ and the corresponding value
$X$ computed from the exact solution $\vec{u},p$. Examples of such quantities
$X$ are the root mean square velocity $v_{rms}$, but it could also be a mass flux
across a boundary, an average horizontal velocity at the top boundary, or
any other scalar quantity.

If the exact solution is known, then one can of course compute $X$ from it.
On the other hand, we would of course like to assess convergence also in
cases where the exact solution is not known. In that case, one can compute
an \textit{estimate} $X^\ast$ for $X$ by way of \textit{extrapolation}.
To this end, we make the assumption that asymptotically, $X_h$ converges to
$X$ at a fixed (but unknown) rate $r$, so that
\begin{equation}
  \label{eq:extrapolation-1}
  e_h=|X_h-X| \approx C h^r.
\end{equation}
Here, $X$, $C$ and $r$ are all unknown constants to be determined, although
we are not really interested in $C$.
We can evaluate $X_h$ from the numerical solution
on successively refined meshes with mesh sizes $h$, $h/2$, and $h/4$. Then,
in addition to \eqref{eq:extrapolation-1} we also have
\begin{eqnarray}
  \label{eq:extrapolation-2}
  e_{h/2}=|X_{h/2}-X| \approx C \left(\frac h2\right)^r,
  \\
  \label{eq:extrapolation-3}
  e_{h/4} =|X_{h/4}-X| \approx C \left(\frac h4\right)^r.
\end{eqnarray}
Taking ratios of equations \eqref{eq:extrapolation-1}--\eqref{eq:extrapolation-3},
and replacing the unknown $X$ by an \textit{estimate} $X^\ast$, we then
arrive at the following equation:
\begin{equation*}
\frac{|X_h-X^\star|}{|X_{h/2}-X^\star|}
=
\frac{|X_{h/2}-X^\star|}{|X_{h/4}-X^\star|}=2^r.
\end{equation*}
If one assumes that $X_h$ converges to $X$ uniformly either from above or
below (rather than oscillate around $X$), then this equation allows us
to solve for $X^\ast$ and $r$:
\begin{equation*}
X^\star = \frac{X_h X_{h/2}-X_{h/2}^2}{X_h - 2 X_{h/2} + X_{h/4}}, \qquad\qquad
r = \log_2 \frac{X_{h/2}-X^\star}{X_{h/4}-X^\star}.
\end{equation*}
In the determination of $r$, we could also have used $X_h$ and $X_{h/2}$,
but using $X_{h/2}$ and $X_{h/4}$ is generally more reliable because
the higher order terms we have omitted in \eqref{eq:extrapolation-1} are less
visible on finer meshes.

