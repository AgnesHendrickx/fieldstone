\begin{flushright} {\tiny {\color{gray} pair\_p2p1.tex}} \end{flushright}
%~~~~~~~~~~~~~~~~~~~~~~~~~~~~~~~~~~~~~~~~~~~~~~~~~~~~~~~~~~~~~~~~~~~~~~~~~~~~~~~~~~~~~~~~~~~~~~~~~~

\noindent
\begin{minipage}{0.54\textwidth}
From Segal \cite{segal}: \say{Taylor-Hood elements \cite{taho73} 
are characterized by the fact that the pressure is continuous in the region $\Omega$. 
A typical example is the quadratic triangle (${\bm P}_2\times P_1$ element).
In this element the velocity is approximated by a quadratic polynomial and the pressure by a
linear polynomial. One can easily verify that both approximations are continuous over 
the element boundaries.}

It can be shown, Segal (1979), that this element is admissible if at least 3 elements 
are used. The quadrilateral counterpart of this triangle is the ${\bm Q}_2\times Q_1$ element.
Reddy and Gartling \cite[p179]{reddybook2} also report this element to be LBB stable.
It is also mentioned in \textcite{nath93}.

\Literature: Schubert \& Anderson \cite{scan85}, Leng \etal \cite{lejx14}, Cuffaro \etal \cite{cump20}
\end{minipage}
\hfill
\begin{minipage}{0.42\textwidth}
\begin{center}
\begin{flushright} {\tiny {\color{gray} (tikz\_p2p1.tex)}} \end{flushright}
%~~~~~~~~~~~~~~~~~~~~~~~~~~~~~~~~~~~~~~~~~~~~~~~~~~~~~~~~~~~~~~~~~~~~~~~~~~~~~~~~~~~~~~~~~~~~~~~~~~

%\begin{center}
\begin{tikzpicture}
%\draw[fill=gray!23,gray!23](0,0) rectangle (5,5);
%\draw[step=0.5cm,gray,very thin] (0,0) grid (5,3.5); %background grid
\draw[thick] (1,0.5) -- (4,1)  -- (3,3) -- cycle; %1-9-2-6-5

%pressure nodes
\draw[violet] (1,0.5) circle (4pt); % 0 
\draw[violet] (4,1) circle (4pt); % 1 
\draw[violet] (3,3) circle (4pt); % 2 

%velocity nodes
\draw[black,fill=teal] (1,0.5)   circle (2pt);
\draw[black,fill=teal] (4,1)   circle (2pt);
\draw[black,fill=teal] (3,3)   circle (2pt);

\draw[black,fill=teal] (2.5,0.75)   circle (2pt);
\draw[black,fill=teal] (2,1.75)   circle (2pt);
\draw[black,fill=teal] (3.5,2)   circle (2pt);

% legend
\draw[black,fill=teal] (3.1,0.2) circle (2pt); \node[] at (3.4,0.2) {$\vec\upnu$};
\draw[violet] (4.1,0.2) circle (4pt); 
\node[] at (4.4,0.2) {$p$};
\node[] at (2.5,3.75) {6 vel. nodes, 3 press. nodes};
\end{tikzpicture}
%\end{center}


\end{center}
\end{minipage}




