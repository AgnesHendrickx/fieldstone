
We introduce here some terminology for efficient element descriptions \cite{grsa}:
\begin{itemize}
\item For triangles/tetrahedra, the designation 
$P_m \times P_n$ \index{$P_m \times P_n$}
means that each component of the velocity
is approximated by continuous piecewise \index{piecewise} complete Polynomials of degree $m$ and
pressure by continuous piecewise complete Polynomials of degree  $n$.
For example $P_2 \times P_1$ means 
\[
u \sim a_1 + a_2 x + a_3 y + a_4 xy + a_5 x^2 + a_6 y^2
\]
with similar approximations for $v$, and 
\[
p \sim b_1 + b_2x + b_3 y
\]
Both velocity and pressure are continuous across element boundaries, 
and each triangular element contains 6 velocity nodes and three pressure nodes.

\item For the same families, \index{$P_m \times P_{-n}$} 
$P_m \times P_{-n}$
is as above, except that pressure is approximated via 
piecewise {\sl discontinuous} polynomials of degree $n$. For instance, $P_2 \times P_{-1}$ is the same 
as $P_2P_1$ except that pressure is now an independent linear function in each element and therefore 
discontinuous at element boundaries.

\item For quadrilaterals/hexahedra, the designation 
\index{$Q_m \times Q_n$}  $Q_m \times Q_n$
means that each component of the velocity
is approximated by a continuous piecewise polynomial of degree $m$ {\sl in each direction} on the quadrilateral
and likewise for pressure, except that the polynomial is of degree $n$.
For instance,  $Q_2 \times Q_1$ \index{$Q_2 \times Q_1$} means
\[
u \sim a_1 + a_2 x + a_3 y + a_4 xy + a_5 x^2 + a_6 y^2 + a_7 x^2y + a_8 xy^2 + a_9 x^2y^2
\]
and 
\[
p \sim b_1 + b_2x + b_3 y + b_4 xy
\]
\item For these same families, $Q_m \times Q_{-n}$ is as above, except that the pressure approximation 
is not continuous at element boundaries. \index{$Q_m \times Q_{-n}$}

\item Again for the same families, \index{$Q_m \times P_{-n}$} $Q_m \times P_{-n}$
 indicates the same velocity approximation 
with a pressure approximation that is a discontinuous complete piecewise polynomial of degree $n$
(not of degree $n$ in each direction !)

\item The designation $P_m^+$ or $Q_m^+$ means that some sort of bubble function \index{Bubble Function}
was added to the polynomial approximation for the velocity. You may also find the term 'enriched element'
in the literature.

\item Finally, for $n=0$, we have piecewise-constant pressure, and we omit the minus sign for simplicity.
\end{itemize}

Another point which needs to be clarified is the use of so-called 'conforming elements' 
(or 'non-conforming elements'). \index{conforming element} \index{non-conforming element}
Following again \cite{grsa}, conforming velocity elements are those for which the basis functions for a subset 
of $H^1$ for the continuous problem (the first derivatives and their squares are integrable in $\Omega$).
For instance, the rotated $Q_1 \times P_0$ element of Rannacher and Turek (see section \ref{pair}) is such that 
the velocity is discontinous across element edges, so that the derivative does not exist there. Another
typical example of non-conforming element is the Crouzeix-Raviart element \cite{crra73}.

 


