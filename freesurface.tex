
TOTAL WORK IN PROGRESS!!!

When carrying out global models, i.e.  mantle convection in our case, the effect of the free surface
is often neglected/negligeable: topography ranges from $\sim$ 10km depth to $\sim$ 10km height, which 
is very small to the depth of the mantle ($\sim$ 3000km). 

However, it has long been regognised that there is a feedback between topography and crust/lithosphere
deformation: the surface of the Earth reflects the deeper processes, from orogeny, back-arc basins, 
rifts, mid-ocean ridges, etc ...

Free surface flows are not unique to Earth sciences, and their modelling has given rise to many studies 
and textbooks. A typical free-surface flow problem in the CFD literature is the so-called 'dam break' 
problem \cite{moeb99,bacp07,liir07,lemx08,homa09,anco09}. Other occurrences involve 
sea waves, flow over structures, flow around ships, mould filling, flow with bubbles \cite{liir07}.

 
What distinguishes geodynamics free surface modelling from its engineering 
counterpart is the absence of surface tension, the fact that the fluids under consideration are
Stokesian, and that their rheology is complex (the elastic and plastic components can be 
dominant at the surface).

There are to main modelling approaches employed in Computational Geodynamics: the so-called 
'sticky air' approach and the Arbitrary-Lagrangian approach.

%.......................................
\subsubsection{The fully Lagrangian approach}

the mesh is deformed with the velocity (or displacement) computed on its nodes. 

yields potentially highly deformed elements, low accuracy. or even bow-tied. 

large deformation requires re-meshing , 2D ok, not 3D. 

remeshing vs no remeshing gtecton



%.......................................
\subsubsection{The Eulerian approach: The Sticky Air}

Sticky air is the default option for numerical methods which mesh 
cannot be deformed (typically the finite difference method).
In this case, the air above the crust/sediments is modelled as a zero-density fluid with 
very low viscosity. One problem quickly arises when one realises that the viscosity of the 
air is almost 25-30 orders of magnitude lower than the (effective) viscosity of Earth materials. 
Real air viscosity cannot therefore be used because of 1) round-off errors, 2) extremely 
poorly-conditioned matrices. Low viscosities around $10^{16}-10^{19}$PS$\cdot$s are then 
commonly used as they are still negligible next to those of the (plastic) crust, and the 
flow of air parallel to Earth materials only generates extremely small shear and normal stress values
(thereby approaching the true nature of a free surface). 
This approach is the one employed in all the papers based on the I2/I3(EL)VIS code (see ~\ref{app:codes})
and has been benchmarked in Crameri et al. \cite{crsg12}.

This approach has pros and cons:

1) it is simple to implement 
2) it is compatible with all the main numerical methods (FEM, FDM,FVM)
3) it avoids complicated remeshing

but it also has drawbacks:

1) it increases the size of the computational domain, thereby adding more unknowns to the linear system;
2) it negatively impacts the condition number of the matrix;
3) unless special methods are put into place, it requires the use of averaging all along the free-surface
where very large viscosity contrasts are present
4) it can showcase air entrainment;
5) it is not clear how thick the air layer must be
6) it often requires to ascribe thermal parameters to the air;
7) it makes the implementation of Dirichlet or Neuman boundary conditions for temperature at the surface less
obvious.
8) it makes the coupling with surface processes codes less straightforward.
4) its accuracy depends on the method used to track materials in the rest of the code (markers, level sets, ...)

%..........................................................
\subsubsection{The Arbitrary Lagrangian Eulerian approach}

It is a very widely used approach in FEM-based geodynamics codes. Although it does 
not originate 





thibault paper \cite{dumy16} 
\cite{anmp15}
\cite{krwd12}


