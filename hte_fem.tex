We start from the 'bare-bones' heat transport equation (source terms are omitted): 
\begin{equation}
\rho C_p \left( \frac{\partial T}{\partial t} + {\vec \upnu}\cdot {\vec\nabla T} \right)
= {\vec \nabla} \cdot k \vec\nabla T 
\end{equation}
In what follows we assume that the velocity vield $\vec \upnu$ is known so that temperature is the 
only unknwon.
Let $N^\uptheta$ be the temperature basis functions so that the temperature inside an element is 
given by\footnote{the $\uptheta$ superscript has been chosen to denote temperature so as to avoid confusion
with the transpose operator}:
\begin{equation}
T^h({\vec r}) = \sum_{i=1}^{m_T} N^\uptheta ({\vec r}) T_i = \vec N^\uptheta \cdot \vec T
\end{equation}
where $\vec T$ is a vector of length $m_T$
The weak form is then 
\begin{equation}
\int_\Omega N^\uptheta_i \left[ 
\rho C_p \left( \frac{\partial T}{\partial t} + {\vec \upnu}\cdot {\vec\nabla T} \right) \right] d\Omega
= \int_\Omega  N^\uptheta_i {\vec \nabla} \cdot k \vec\nabla T  d\Omega
\end{equation}

\[
\underbrace{\int_\Omega N^\uptheta_i  \rho C_p \frac{\partial T}{\partial t} d\Omega}_{I}
+ \underbrace{\int_\Omega N^\uptheta_i  \rho C_p  {\vec \upnu}\cdot {\vec\nabla T}   d\Omega}_{II}
= \underbrace{\int_\Omega  N^\uptheta_i {\vec \nabla} \cdot k \vec\nabla T d\Omega}_{III}
\quad\quad
i=1,m_T
\]

Looking at the first term:
\begin{eqnarray}
\int_\Omega N^\uptheta_i  \rho C_p \frac{\partial T}{\partial t} d\Omega
&=&  \int_\Omega N^\uptheta_i  \rho C_p \vec N^\uptheta \cdot \dot{\vec T}  d\Omega \\
\end{eqnarray}
so that when we assemble all contributions for $i=1,m_T$ we get:
\[
I 
= \int_\Omega \vec N^\uptheta  \rho C_p \vec N^\uptheta \cdot \dot{\vec T}  d\Omega
= \left( \int_\Omega \rho C_p  \vec N^\uptheta  \vec N^\uptheta  d\Omega \right) \cdot \dot{\vec T}
= {\bm M}^T \cdot \dot{\vec T}
 \]
where ${\bm M}^T$ is the mass matrix of the system of size $(m_T \times m_T)$ with 
\[
M_{ij}^T = \int_\Omega \rho C_p N_i^\uptheta N_j^\uptheta d\Omega
\]
Turning now to the second term:
\begin{eqnarray}
\int_\Omega N^\uptheta_i  \rho C_p  {\vec \upnu}\cdot {\vec\nabla T}   d\Omega
&=& \int_\Omega N^\uptheta_i  \rho C_p (u \frac{\partial T}{\partial x} +  v \frac{\partial T}{\partial y} ) d\Omega \\
&=& \int_\Omega N^\uptheta_i  \rho C_p (u \frac{\partial \vec N^\uptheta}{\partial x} +  v \frac{\partial \vec N^\uptheta}{\partial y} ) \cdot \vec T d\Omega \\
\end{eqnarray}
so that when we assemble all contributions for $i=1,m_T$ we get:
\[
II = \left(\int_\Omega \rho C_p \vec N^\uptheta (u \frac{\partial \vec N^\uptheta}{\partial x} +  v \frac{\partial \vec N^\uptheta}{\partial y} ) d\Omega \right)  \cdot \vec T = {\bm K}_a \cdot \vec T
\]
where ${\bm K}_a$ is the advection term matrix of size $(m_T \times m_T)$ with
\[
(K_a)_{ij} = \int_\Omega \rho C_p N_i^\uptheta 
\left(u \frac{\partial N_j^\uptheta}{\partial x} +  v \frac{\partial N_j^\uptheta}{\partial y} \right) d\Omega 
\]
Now looking at the third term, we carry out an integration by part and neglect the surface term for now, so that 
\begin{eqnarray}
\int_\Omega  N^\uptheta_i {\vec \nabla} \cdot k \vec\nabla T d\Omega
&=& - \int_\Omega  k \vec \nabla N^\uptheta_i \cdot \vec\nabla T d\Omega \\
&=& - \int_\Omega  k \vec \nabla N^\uptheta_i \cdot \vec\nabla (\vec N^\uptheta \cdot \vec T) d\Omega \\
\end{eqnarray}
with 
\[
\vec \nabla \vec N^\uptheta = 
\left(
\begin{array}{cccc}
\partial_x N_1^\uptheta & 
\partial_x N_2^\uptheta & \dots &
\partial_x N_{m_T}^\uptheta \\ \\
\partial_y N_1^\uptheta & 
\partial_y N_2^\uptheta & \dots &
\partial_y N_{m_T}^\uptheta 
\end{array}
\right)
\]
so that finally:
\[
III = - \left( \int_\Omega k (\vec \nabla \vec N^\uptheta)^T \cdot \vec \nabla \vec N^\uptheta d\Omega \right) \cdot \vec T
= - {\bm K}_d \cdot \vec T
\]
where ${\bm K}_d$ is the diffusion term matrix:
\[
{\bm K}_d = \int_\Omega  k (\vec \nabla \vec N^\uptheta)^T \cdot \vec \nabla \vec N^\uptheta d\Omega 
\]
 Ultimately terms $I,II,III$ together yield:
\[
\boxed{
{\bm M}^\uptheta \cdot \dot{\vec T} + ({\bm K}_a + {\bm K}_d) \cdot \vec T = \vec 0
}
\]
What now remains to be done is to address the time derivative on the temperature vector. 
The most simple approach would be to use an explicit Euler one, i.e.:
\[
\frac{\partial \vec T}{\partial t} = \frac{\vec T^{(k)} - \vec T^{(k-1)}}{\delta t}
\]
where $\vec T^{(k)}$ is the temperature field at time step $k$ and $\delta t$ is the time interval 
between two consecutive time steps.
In this case the discretised heat transport equation is:
\[
\boxed{
\left( {\bm M}^\uptheta  + ({\bm K}_a + {\bm K}_d) \delta t \right) \cdot \vec T^{(k)} =  {\bm M}^\uptheta \cdot \vec T^{(k-1)}
}
\]
\todo[inline]{add source term!!}




Let us then write this equation at times $t$ and $t+\delta t$ :
\begin{eqnarray}
{\bm M}^\uptheta(t) \cdot \dot{\vec T}(t) + {\bm K}(t) \cdot \vec T(t) &=& \vec F(t)  \label{htefem3} \\ 
{\bm M}^\uptheta(t+\delta t) \cdot \dot{\vec T}(t+\delta t) + {\bm K}(t+\delta t)\cdot {\vec T}(t+\delta t) &=& {\vec F}(t+\delta t) \label{htefem4} 
\end{eqnarray}
The time derivative of temperature can be written as follows:
\[
\frac{\vec T(t+\delta t)-\vec T(t) }{\delta t} = \alpha \dot{\vec T}(t+\delta t) + (1-\alpha) \dot{\vec T}(t)
\]
If $\alpha=0$ then this is a fully implicit scheme, if $\alpha=1$ this is a fully explicit scheme. 
If $\alpha=1/2$ this is a second order accurate, mid-point implicit scheme 
One can multiply Eq. (\ref{htefem3}) by $1-\alpha$ and Eq. (\ref{htefem4}) by $\alpha$ and sum them :
\begin{eqnarray}
(1-\alpha) {\bm M}^\uptheta(t) \cdot \dot{\vec T}(t) + (1-\alpha){\bm K}(t)\cdot {\vec T}(t)
&=& (1-\alpha){\vec F}(t)  \nn\\
+ \alpha  {\bm M}(t+\delta t) \cdot \dot{\vec T}(t+\delta t) 
+ \alpha {\bm K}(t+\delta t)\cdot {\vec T}(t+\delta t) && +  \alpha {\vec F}(t+\delta t) \nn 
\end{eqnarray}
Assuming ${\bm M}^\uptheta(t)\approx {\bm M}^\uptheta(t+\delta t)$, ${\bm K}(t+\delta t)={\bm K}(t)$  
and ${\vec F}(t)\approx {\vec F}(t+\delta t)$, then 
\begin{eqnarray}
{\bm M}(t) \cdot \frac{\vec T(t+\delta t)-\vec T(t) }{\delta t} 
+ (1-\alpha){\bm K}(t)\cdot {\vec T}(t)   
+ \alpha {\bm K}(t)\cdot {\vec T}(t+\delta t) 
&=& {\vec F}(t)  \nn
\end{eqnarray}
and finally
\[
\left[ {\bm M}^\uptheta(t) + \alpha {\bm K}(t) \; \delta t \right] \cdot  {\vec T}(t+\delta t)
= \left[ {\bm M}^\uptheta(t) - (1-\alpha) {\bm K}(t) \; \delta t \right] \cdot  {\vec T}(t) + {\vec F}(t) \; \delta t
\]
or, 
\[
{\bm A}\cdot {\vec T} = {\vec b}
\]
with 
\begin{eqnarray}
{\bm A} &=& {\bm M}^\uptheta(t) + \alpha {\bm K}(t) \; \delta t  \nn\\
{\vec b} &=& \left[ {\bm M}^\uptheta(t) - (1-\alpha) {\bm K}(t) \; \delta t \right] \cdot  {\vec T}(t) + {\vec F}(t) \; \delta t \nn
\end{eqnarray}


\todo[inline]{there's got to be a prettier way to arrive at this ...}

