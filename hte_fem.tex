We start from the 'bare-bones' heat transport equation (source terms are omitted): 
\begin{equation}
\rho C_p \left( \frac{\partial T}{\partial t} + {\vec \upnu}\cdot {\vec\nabla T} \right)
= {\vec \nabla} \cdot k \vec\nabla T 
\end{equation}
In what follows we assume that the velocity vield $\vec \upnu$ is known so that temperature is the 
only unknwon.
Let $N^\uptheta$ be the temperature basis functions so that the temperature inside an element is 
given by\footnote{the $\uptheta$ superscript has been chosen to denote temperature so as to avoid confusion
with the transpose operator}:
\begin{equation}
T^h({\vec r}) = \sum_{i=1}^{m_T} N^\uptheta ({\vec r}) T_i = \vec N^\uptheta \cdot \vec T
\end{equation}
where $\vec T$ is a vector of length $m_T$
The weak form is then 
\begin{equation}
\int_\Omega N^\uptheta_i \left[ 
\rho C_p \left( \frac{\partial T}{\partial t} + {\vec \upnu}\cdot {\vec\nabla T} \right) \right] d\Omega
= \int_\Omega  N^\uptheta_i {\vec \nabla} \cdot k \vec\nabla T  d\Omega
\end{equation}

\[
\underbrace{\int_\Omega N^\uptheta_i  \rho C_p \frac{\partial T}{\partial t} d\Omega}_{I}
+ \underbrace{\int_\Omega N^\uptheta_i  \rho C_p  {\vec \upnu}\cdot {\vec\nabla T}   d\Omega}_{II}
= \underbrace{\int_\Omega  N^\uptheta_i {\vec \nabla} \cdot k \vec\nabla T d\Omega}_{III}
\quad\quad
i=1,m_T
\]

Looking at the first term:
\begin{eqnarray}
\int_\Omega N^\uptheta_i  \rho C_p \frac{\partial T}{\partial t} d\Omega
&=&  \int_\Omega N^\uptheta_i  \rho C_p \vec N^\uptheta \cdot \dot{\vec T}  d\Omega \\
\end{eqnarray}
so that when we assemble all contributions for $i=1,m_T$ we get:
\[
I 
= \int_\Omega \vec N^\uptheta  \rho C_p \vec N^\uptheta \cdot \dot{\vec T}  d\Omega
= \left( \int_\Omega \rho C_p  \vec N^\uptheta  \vec N^\uptheta  d\Omega \right) \cdot \dot{\vec T}
= {\bm M}^T \cdot \dot{\vec T}
 \]
where ${\bm M}^T$ is the mass matrix of the system of size $(m_T \times m_T)$ with 
\[
M_{ij}^T = \int_\Omega \rho C_p N_i^\uptheta N_j^\uptheta d\Omega
\]
Turning now to the second term:
\begin{eqnarray}
\int_\Omega N^\uptheta_i  \rho C_p  {\vec \upnu}\cdot {\vec\nabla T}   d\Omega
&=& \int_\Omega N^\uptheta_i  \rho C_p (u \frac{\partial T}{\partial x} +  v \frac{\partial T}{\partial y} ) d\Omega \\
&=& \int_\Omega N^\uptheta_i  \rho C_p (u \frac{\partial \vec N^\uptheta}{\partial x} +  v \frac{\partial \vec N^\uptheta}{\partial y} ) \cdot \vec T d\Omega \\
\end{eqnarray}
so that when we assemble all contributions for $i=1,m_T$ we get:
\[
II = \left(\int_\Omega \rho C_p \vec N^\uptheta (u \frac{\partial \vec N^\uptheta}{\partial x} +  v \frac{\partial \vec N^\uptheta}{\partial y} ) d\Omega \right)  \cdot \vec T = {\bm K}_a \cdot \vec T
\]
where ${\bm K}_a$ is the advection term matrix of size $(m_T \times m_T)$ with
\[
(K_a)_{ij} = \int_\Omega \rho C_p N_i^\uptheta 
\left(u \frac{\partial N_j^\uptheta}{\partial x} +  v \frac{\partial N_j^\uptheta}{\partial y} \right) d\Omega 
\]
Now looking at the third term, we carry out an integration by part and neglect the surface term for now, so that 
\begin{eqnarray}
\int_\Omega  N^\uptheta_i {\vec \nabla} \cdot k \vec\nabla T d\Omega
&=& - \int_\Omega  k \vec \nabla N^\uptheta_i \cdot \vec\nabla T d\Omega \\
&=& - \int_\Omega  k \vec \nabla N^\uptheta_i \cdot \vec\nabla (\vec N^\uptheta \cdot \vec T) d\Omega \\
\end{eqnarray}
with 
\[
\vec \nabla \vec N^\uptheta = 
\left(
\begin{array}{cccc}
\partial_x N_1^\uptheta & 
\partial_x N_2^\uptheta & \dots &
\partial_x N_{m_T}^\uptheta \\ \\
\partial_y N_1^\uptheta & 
\partial_y N_2^\uptheta & \dots &
\partial_y N_{m_T}^\uptheta 
\end{array}
\right)
\]
so that finally:
\[
III = - \left( \int_\Omega k (\vec \nabla \vec N^\uptheta)^T \cdot \vec \nabla \vec N^\uptheta d\Omega \right) \cdot \vec T
= - {\bm K}_d \cdot \vec T
\]
where ${\bm K}_d$ is the diffusion term matrix:
\[
{\bm K}_d = \int_\Omega  k (\vec \nabla \vec N^\uptheta)^T \cdot \vec \nabla \vec N^\uptheta d\Omega 
\]
 Ultimately terms $I,II,III$ together yield:
\[
\boxed{
{\bm M}^\uptheta \cdot \dot{\vec T} + ({\bm K}_a + {\bm K}_d) \cdot \vec T = \vec 0
}
\]

%What now remains to be done is to address the time derivative on the temperature vector. 
%The most simple approach would be to use an explicit Euler one, i.e.:
%\[
%\frac{\partial \vec T}{\partial t} = \frac{\vec T^{(k)} - \vec T^{(k-1)}}{\delta t}
%\]
%where $\vec T^{(k)}$ is the temperature field at time step $k$ and $\delta t$ is the time interval 
%between two consecutive time steps.
%In this case the discretised heat transport equation is:
%\[
%\boxed{
%\left( {\bm M}^\uptheta  + ({\bm K}_a + {\bm K}_d) \delta t \right) \cdot \vec T^{(k)} =  {\bm M}^\uptheta \cdot \vec T^{(k-1)}
%}
%\]
\todo[inline]{add source term!!}

%....................................................
\subsubsection{Dealing with the time discretisation}

Essentially we have to solve a PDE of the type:
\[
\frac{\partial T}{\partial t} = {\cal F}(\vec \upnu,T,\vec\nabla T,\Delta T)
\]
with ${\cal F}=\frac{1}{\rho C_p}(-\vec\upnu\cdot\vec\nabla T + \vec\nabla\cdot k\vec\nabla T)$.

\index{Forward Euler}
\index{Backward Euler}
\index{Crank-Nicolson}

The (explicit) forward Euler method is:
\[
\frac{T^{n+1}-T^n}{\delta t} = {\cal F}^n(T,\vec\nabla T,\Delta T)
\]
The (implicit) backward Euler method is:
\[
\frac{T^{n+1}-T^n}{\delta t} = {\cal F}^{n+1}(T,\vec\nabla T,\Delta T)
\]
and the (implicit) Crank-Nicolson algorithm is:
\[
\frac{T^{n+1}-T^n}{\delta t} = 
\frac{1}{2}
\left[
{\cal F}^{n}(T,\vec\nabla T,\Delta T)
+
{\cal F}^{n+1}(T,\vec\nabla T,\Delta T)
\right]
\]
where the superscript $n$ indicates the time step.
The Crank-Nicolson is obviously based on the trapezoidal rule, with second-order convergence in time.


In what follows, I omit the superscript on the mass matrix to simplify notations: ${\bm M}^\uptheta={\bm M}$.
In terms of Finite Elements, these become:
\begin{itemize}
\item Explicit Forward euler:
\[
\frac{1}{\delta t} ({\bm M}^{n+1} \cdot \vec T^{n+1}  -{\bm M}^n \cdot \vec T^{n} )
=
-({\bm K}_a^n+{\bm K}^n_d) \cdot \vec T^{n}
\]
or, 
\[
\boxed{
{\bm M}^{n+1} \cdot \vec T^{n+1}
= \left(  {\bm M}^n  + ({\bm K}_a^n+{\bm K}_d^n) \delta t \right)\cdot \vec T^{n} 
}
\]

\item Implicit Backward euler:
\[
\frac{1}{\delta t} ({\bm M}^{n+1} \cdot \vec T^{n+1}  -{\bm M}^n \cdot \vec T^{n} )
= -({\bm K}_a^{n+1}+{\bm K}_d^{n+1}) \cdot \vec T^{n+1}
\]
or, 
\[
\boxed{
\left( {\bm M}^{n+1} +({\bm K}_a^{n+1}+{\bm K}_d^{n+1})\delta t \right) \cdot \vec T^{n+1}
=
{\bm M}^n \cdot \vec T^{n} 
}
\]

\item Crank-Nicolson

\[
\frac{1}{\delta t} \left({\bm M}^{n+1} \cdot \vec T^{n+1}  -{\bm M}^n \cdot \vec T^{n} \right)
= 
\frac{1}{2}
\left[
-({\bm K}_a^{n+1}+{\bm K}_d^{n+1}) \cdot \vec T^{n+1}
-({\bm K}_a^{n}+{\bm K}_d^{n}) \cdot \vec T^{n}
\right]
\]
or,
\[
\boxed{
\left( {\bm M}^{n+1} +({\bm K}_a^{n+1}+{\bm K}_d^{n+1})\frac{\delta t}{2} \right) \cdot \vec T^{n+1}
= \left(  {\bm M}^n  + ({\bm K}_a^n+{\bm K}_d^n) \frac{\delta t}{2} \right)\cdot \vec T^{n} 
}
\]

Note that in benchmarks where the domain/grid does not deform, the coefficients do not change in space
and the velocity field is constant in time, or in practice out of convenience, the ${\bm K}$  and ${\bm M}$ 
matrices do not change and the r.h.s. can be constructed with the same matrices as the FE matrix.

\end{itemize}




\index{BDF-2}
\paragraph{The Backward differentiation formula} (see for instance \cite{hawa91} or Wikipedia\footnote{\url{https://en.wikipedia.org/wiki/Backward_differentiation_formula}}. The second-order BDF (or BDF-2) as shown in \cite{krjb12} is as follows: it is a finite-difference 
quadratic interpolation approximation of the $\partial T/\partial t$ term which involves
$t^n$, $t^{n-1}$ and $t^{n-2}$:
\begin{equation}
\frac{\partial T}{\partial t}(t^n) =
\frac{1}{\tau_n} \left( \frac{2\tau_n + \tau_{n-1}}{\tau_n+\tau_{n-1} } T(t^n)  
- \frac{\tau_n +\tau_{n-1}}{\tau_{n-1}} T(t^{n-1})
+ \frac{\tau_n^2}{\tau_{n-1}(\tau_n+\tau_{n-1})} T(t^{n-2})
\right)
\end{equation}
where $\tau_n=t^n-t^{n-1}$.
Starting again from 
${\bm M}^\uptheta \cdot \dot{\vec T} + ({\bm K}_a + {\bm K}_d) \cdot \vec T = \vec 0$,
we write 
\[
{\bm M}^\uptheta \cdot 
\frac{1}{\tau_n} \left( \frac{2\tau_n + \tau_{n-1}}{\tau_n+\tau_{n-1} } \vec T^n  
- \frac{\tau_n +\tau_{n-1}}{\tau_{n-1}} \vec T^{n-1}
+ \frac{\tau_n^2}{\tau_{n-1}(\tau_n+\tau_{n-1})} \vec T^{n-2} \right)
+ ({\bm K}_a + {\bm K}_d) \cdot \vec T^n = \vec 0
\]
and finally:
\[
\left[
\frac{2\tau_n + \tau_{n-1}}{\tau_n+\tau_{n-1} }
{\bm M}^\uptheta
+ \tau_n({\bm K}_a + {\bm K}_d)
\right]
 \cdot \vec T^n =
 \frac{\tau_n +\tau_{n-1}}{\tau_{n-1}} {\bm M}^\uptheta \cdot \vec T^{n-1}
- \frac{\tau_n^2}{\tau_{n-1}(\tau_n+\tau_{n-1})} {\bm M}^\uptheta \cdot \vec T^{n-2}
\]
Note that if all timesteps are equal, i.e. $\tau_n=\tau_{n-1}=\delta t$, this equation becomes:
\[
\left[
\frac{3}{2}
{\bm M}^\uptheta
+ \delta t({\bm K}_a + {\bm K}_d)
\right]
 \cdot \vec T^n =
{\bm M}^\uptheta \cdot \left(2 \vec T^{n-1} - \frac{1}{2} \vec T^{n-2} \right)
\]
or, 
\[
\left[
{\bm M}^\uptheta
+ \frac{2}{3}\delta t({\bm K}_a + {\bm K}_d)
\right]
 \cdot \vec T^n =
{\bm M}^\uptheta \cdot \left( \frac{4}{3} \vec T^{n-1} - \frac{1}{3} \vec T^{n-2} \right)
\]

As mentioned before the 
backward differenciation formula (BDF) is a family of implicit methods
for the integration of ODEs. Each BDF-$s$ method achieves order $s$.
The BDF-1 is simply the backward Euler method as seen above:
\[
T^{n+1}-T^n=\delta t {\cal F}^{n+1}
\]
The BDF-2 is given by 
\[
T^{n+2} - \frac{4}{3}T^{n+1} +\frac{1}{3} T^n = \frac{2}{3} \delta t {\cal F}^{n+2}
\]
The BDF-3 is given by 
\[
T^{n+3} - \frac{18}{11}T^{n+2} +\frac{9}{11} T^{n+1} -\frac{2}{11}T^n = \frac{6}{11} \delta t {\cal F}^{n+3}
\]
The BDF-4 is given by 
\[
T^{n+4}-\frac{48}{25}T^{n+1}+\frac{36}{25}T^{n+1}-\frac{16}{25}T^{n+1}+\frac{3}{25}T^n = \frac{12}{25}\delta t {\cal F}^{n+4}
\]





%\end{document}


