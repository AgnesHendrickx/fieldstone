
Let us start with the 1D steady advection-diffusion equation:
\begin{equation}
\rho C_p u \frac{dT}{dx} - k \frac{d^2T}{dx^2} = f \qquad \text{in} \quad [0,L_x]
\label{eq:fdm1Dad}
\end{equation}
with the boundary conditions $T(x=0)=0$ and $T(x=L_x)=0$.

We have seen before (see Section~\ref{XXX}) 
that the elemental matrix ${\bm K}_a$ for the advection and 
the elemental matrix  ${\bm K}_d$ for the diffusion terms are
\[
{\bm K}_a^e = \frac{\rho C_p u}{2} 
\left(
\begin{array}{cc}
-1 & 1 \\
-1 & 1 
\end{array}
\right)
\qquad
{\bm K}_d^e=\frac{k}{h_x}
\left(
\begin{array}{cc}
1 & -1 \\ 
-1 & 1
\end{array}
\right)
\]
where $h_x$ is the distance between nodes and $e$ denotes the element number. 

Assuming that we have 5 elements (i.e. 6 nodes), the assembled $6\times 6$ 
advection and diffusion matrices 
(before boundary conditions are applied) are:
\[
{\bm K}_a
= \frac{\rho C_p u}{2}
\left(
\begin{array}{cccccc}
-1 & 1 & 0 & 0 & 0  &0\\
-1 & 0 & 1 & 0 & 0  &0\\
 0 &-1 & 0 & 1 & 0  &0\\
 0 & 0 &-1 & 0 & 1  &0\\
 0 & 0 & 0 &-1 & 0  &1\\
 0 & 0 & 0 & 0 &-1  &1\\
\end{array}
\right)
\qquad
{\bm K}_d
= \frac{k}{h_x}
\left(
\begin{array}{cccccc}
 1 &-1 & 0 & 0 & 0 &  0\\
-1 & 2 &-1 & 0 & 0 &  0\\
 0 &-1 & 2 &-1 & 0 &  0\\
 0 & 0 &-1 & 2 &-1 &  0\\
 0 & 0 & 0 &-1 & 2 & -1\\
 0 & 0 & 0 & 0 &-1 &  1\\
\end{array}
\right)
\]
The rhs is zero, so that we would have to solve $({\bm K}_a+{\bm K}_d)\cdot \vec{T}=0$ 
, or:
\[
\left[ 
\frac{\rho C_p u}{2}
\left(
\begin{array}{cccccc}
-1 & 1 & 0 & 0 & 0  &0\\
-1 & 0 & 1 & 0 & 0  &0\\
 0 &-1 & 0 & 1 & 0  &0\\
 0 & 0 &-1 & 0 & 1  &0\\
 0 & 0 & 0 &-1 & 0  &1\\
 0 & 0 & 0 & 0 &-1  &1\\
\end{array}
\right)
+
\frac{k}{h_x}
\left(
\begin{array}{cccccc}
 1 &-1 & 0 & 0 & 0 &  0\\
-1 & 2 &-1 & 0 & 0 &  0\\
 0 &-1 & 2 &-1 & 0 &  0\\
 0 & 0 &-1 & 2 &-1 &  0\\
 0 & 0 & 0 &-1 & 2 & -1\\
 0 & 0 & 0 & 0 &-1 &  1\\
\end{array}
\right)
\right]
\cdot
\left(
\begin{array}{c}
T_1 \\ T_2 \\ T_3 \\ T_4 \\ T_5 \\ T_6
\end{array}
\right)
= \vec{0}
\]
Note that boundary conditions are not applied yet. 
Therefore the algebraic equation for an interior node $i$ is 
\[
\rho C_p u
\frac{T_{i+1}-T_{i-1}}{2}
+
\frac{k}{h_x}
(-T_{i-1}+2T_i-T_{i+1}) = 0
\]
or, 
\begin{equation}
\boxed{
\rho C_p u
\frac{T_{i+1}-T_{i-1}}{2h_x}
-
\frac{k}{h_x^2}
(T_{i-1}-2T_i+T_{i+1}) = 0
} \label{eq:fdm1Ddiscr}
\end{equation}

However, we have seen in Section~\ref{fdm_basics} that the 
second order accurate central differencing based approximate first and second
derivatives written for an interior node $i$ 
of a finite difference mesh with a constant node spacing of $h$ is 
\[
\left. \frac{dT}{dx}\right|_i
\simeq \frac{T_{i+1}-T_{i-1}}{2 h_x}
\qquad
\frac{d^2T}{dx^2} 
\simeq \frac{T_{i+1}-2T_i+T_{i-1}}{h_x^2}
\]
Using these approximations, the discretised formualtion of Eq.~(\ref{eq:fdm1Dad}) is
exactly the same as Eq.~(\ref{eq:fdm1Ddiscr}).
This simple example proves that the FEM and the FDM share similarities!


It is also useful to introduce the elemental Peclet number
\[
Pe = \frac{uh}{2 \kappa} = \frac{u h \rho C_p}{2 k}
\]
\index{general}{Peclet Number}
and Eq.~(\ref{eq:fdm1Dad}) becomes:
\[
\frac{u}{2h_x}
\left[
\left(1-\frac{1}{Pe}\right) T_{i+1} + \frac{2}{Pe} T_i - \left(1+\frac{1}{Pe}\right)T_{i-1} 
\right] = f
\]
CHECK!!!













