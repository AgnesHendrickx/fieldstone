The FE matrix is the result of the assembly process of all elemental matrices. 
Its size can become quite large when the resolution is being increased (from thousands
of lines/columns to tens of millions).

One important property of the matrix is its sparsity. Typically less than 1\% of the 
matrix terms is not zero and this means that the matrix storage can and should be optimised. 
Clever storage formats were designed early on since the amount of RAM memory in computers
was the limiting factor 3 or 4 decades ago. \cite{saad}

There are several standard formats:
\begin{itemize}
\item compressed sparse row (CSR) format\index{general}{CSR} \index{general}{Compressed Sparse Row}
\item compressed sparse column format (CSC) \index{general}{CSC} \index{general}{Compressed Sparse Column}
\item the Coordinate Format (COO)
\item Skyline Storage Format
\item ...
\end{itemize}

I focus on  the CSR format in what follows. 

%..............................................................................
\subsubsection{2D domain - One degree of freedom per node}

Let us consider again the  $3\times2$ element grid which counts 12 nodes.
\begin{verbatim}
8=======9======10======11
|       |       |       |
|  (3)  |  (4)  |  (5)  |
|       |       |       |
4=======5=======6=======7
|       |       |       |
|  (0)  |  (1)  |  (2)  |
|       |       |       |
0=======1=======2=======3
\end{verbatim}

In the case there is only a single degree of freedom per node, the 
assembled FEM matrix will look like this:

\[
\left(
\begin{array}{cccccccccccc}
X & X &   &   & X & X &   &   &   &   &   &   \\
X & X & X &   & X & X & X &   &   &   &   &   \\
  & X & X & X &   & X & X & X &   &   &   &   \\
  &   & X & X &   &   & X & X &   &   &   &   \\
X & X &   &   & X & X &   &   & X & X &   &   \\
X & X & X &   & X & X & X &   & X & X & X &   \\
  & X & X & X &   & X & X & X &   & X & X & X \\
  &   & X & X &   &   & X & X &   &   & X & X \\
  &   &   &   & X & X &   &   & X & X &   &   \\
  &   &   &   & X & X & X &   & X & X & X &   \\
  &   &   &   &   & X & X & X &   & X & X & X \\
  &   &   &   &   &   & X & X &   &   & X & X \\
\end{array}
\right)
\]
where the $X$ stand for non-zero terms.
This matrix structure stems from the fact that
\begin{itemize}
\item node 0 sees nodes 0,1,4,5
\item node 1 sees nodes 0,1,2,4,5,6 
\item node 2 sees nodes 1,2,3,5,6,7 
\item ...
\item node 5 sees nodes 0,1,2,4,5,6,8,9,10 
\item ...
\item node 10 sees nodes 5,6,7,9,10,11 
\item node 11 sees nodes 6,7,10,11
\end{itemize}
In light thereof, we have
\begin{itemize}
\item 4 corner nodes which have 4 neighbours (counting themselves) 
\item 2(nnx-2) nodes which have 6 neighbours
\item 2(nny-2) nodes which have 6 neighbours
\item (nnx-2)$\times$(nny-2) nodes which have 9 neighbours
\end{itemize}
In total, the number of non-zero terms in the matrix is then:
\[
NZ=4\times4+4\times6+2\times6+2\times9=70
\]
In general, we would then have:
\[
NZ=4\times4+[2(nnx-2)+2(nny-2)]\times6 + (nnx-2)(nny-2)\times9
\]

Let us temporarily assume $nnx=nny=n$. Then the matrix size (total
number of unknowns) is $N=n^2$ and  
\[
NZ=16+24(n-2)+9(n-2)^2
\]
A full matrix array would contain $N^2=n^4$ terms. 
The ratio of $NZ$ (the actual number of reals to store)
to the full matrix size (the number of reals a full matrix contains) is then 
\[
R = \frac{16+24(n-2)+9(n-2)^2}{n^4}
\]
It is then obvious that when $n$ is large enough $R \sim 1/n^2$.

CSR stores the nonzeros of the matrix row by row, in a
single indexed array A of double precision  numbers.
Another array COLIND contains the column index of each
corresponding entry in the A array. A third integer array RWPTR
contains pointers to the beginning of each row, which an additional pointer to
the first index following the nonzeros of the matrix A.
A and COLIND have length NZ and RWPTR has length N+1.

In the case of the here-above matrix, the arrays COLIND and RWPTR will look like:
\[
COLIND=(0,1,4,5, \; 0,1,2,4,5,6, \; 1,2,3,5,6,7, ..., 6,7,10,11)
\]
\[
RWPTR=(0,4,10,16, ... )
\]

%..............................................................................
\subsubsection{2D domain - Two degrees of freedom per node}

When there are now two degrees of freedom per node, such as in the case of the Stokes equation
in two-dimensions, the size of the $\K$ matrix is given by 
\begin{lstlisting}
NfemV=nnp*ndofV  
\end{lstlisting}
In the case of the small grid above, we have {\tt NfemV=24}.
Elemental matrices are now $8\times8$ in size.

We still have
\begin{itemize}
\item 4 corner nodes which have 4 neighbours
\item 2(nnx-2) nodes which have 6 neighbours
\item 2(nny-2) nodes which have 6 neighbours
\item (nnx-2)x(nny-2) nodes which have 9 neighbours,
\end{itemize}
but now each degree of freedom from a node sees the other two
degrees of freedom of another node too.
In that case, the number of nonzeros has been multiplied by four
and the assembled FEM matrix looks like:
\begin{equation}
\left(
\begin{array}{cccccccccccccccccccccccc}
X&X & X&X &  &  &  &  & X&X & X&X &  &  &  &  &  &  &  &  &  &  &  &  \\
X&X & X&X &  &  &  &  & X&X & X&X &  &  &  &  &  &  &  &  &  &  &  &  \\
X&X & X&X & X&X &  &  & X&X & X&X & X&X &  &  &  &  &  &  &  &  &  &  \\
X&X & X&X & X&X &  &  & X&X & X&X & X&X &  &  &  &  &  &  &  &  &  &  \\
 &  & X&X & X&X & X&X &  &  & X&X & X&X & X&X &  &  &  &  &  &  &  &  \\
 &  & X&X & X&X & X&X &  &  & X&X & X&X & X&X &  &  &  &  &  &  &  &  \\
 &  &  &  & X&X & X&X &  &  &  &  & X&X & X&X &  &  &  &  &  &  &  &  \\
 &  &  &  & X&X & X&X &  &  &  &  & X&X & X&X &  &  &  &  &  &  &  &  \\
X&X & X&X &  &  &  &  & X&X & X&X &  &  &  &  & X&X & X&X &  &  &  &  \\
X&X & X&X &  &  &  &  & X&X & X&X &  &  &  &  & X&X & X&X &  &  &  &  \\
X&X & X&X & X&X &  &  & X&X & X&X & X&X &  &  & X&X & X&X & X&X &  &  \\
X&X & X&X & X&X &  &  & X&X & X&X & X&X &  &  & X&X & X&X & X&X &  &  \\
 &  & X&X & X&X & X&X &  &  & X&X & X&X & X&X &  &  & X&X & X&X & X&X \\
 &  & X&X & X&X & X&X &  &  & X&X & X&X & X&X &  &  & X&X & X&X & X&X \\
 &  &  &  & X&X & X&X &  &  &  &  & X&X & X&X &  &  &  &  & X&X & X&X \\
 &  &  &  & X&X & X&X &  &  &  &  & X&X & X&X &  &  &  &  & X&X & X&X \\
 &  &  &  &  &  &  &  & X&X & X&X &  &  &  &  & X&X & X&X &  &  &  &  \\
 &  &  &  &  &  &  &  & X&X & X&X &  &  &  &  & X&X & X&X &  &  &  &  \\
 &  &  &  &  &  &  &  & X&X & X&X & X&X &  &  & X&X & X&X & X&X &  &  \\
 &  &  &  &  &  &  &  & X&X & X&X & X&X &  &  & X&X & X&X & X&X &  &  \\
 &  &  &  &  &  &  &  &  &  & X&X & X&X & X&X &  &  & X&X & X&X & X&X \\
 &  &  &  &  &  &  &  &  &  & X&X & X&X & X&X &  &  & X&X & X&X & X&X \\
 &  &  &  &  &  &  &  &  &  &  &  & X&X & X&X &  &  &  &  & X&X & X&X \\
 &  &  &  &  &  &  &  &  &  &  &  & X&X & X&X &  &  &  &  & X&X & X&X \\
\end{array}
\right)\nonumber
\end{equation}
Note that the degrees of freedom are organised as follows: 
\[
(u_0,v_0,u_1,v_1,u_2,v_2, ... u_{11},v_{11})
\]
In general, we would then have:
\[
NZ=4 \left[4\times4+[2(nnx-2)+2(nny-2)]\times6 + (nnx-2)(nny-2)\times9 \right]
\]
and in the case of the small grid,
the number of non-zero terms in the matrix is then:
\[
NZ=4\left[4\times4+4\times6+2\times6+2\times9\right]=280
\]
In the case of the here-above matrix, the arrays COLIND and RWPTR will look like:
\[
COLIND=(0,1,2,3,8,9,10,11, \; 0,1,2,3,8,9,10,11,\; ...)
\]
\[
RWPTR=(0,8,16,28, ... )
\]

%..............................................................................
\subsubsection{Matrix Storage in fieldstone}

The majority of the codes have the FE matrix being a full array
\begin{lstlisting}
a_mat = np.zeros((Nfem,Nfem),dtype=np.float64) 
\end{lstlisting}
and it is converted to CSR format on the fly in the solve phase:
\begin{lstlisting}
sol = sps.linalg.spsolve(sps.csr_matrix(a_mat),rhs)
\end{lstlisting}

Note that linked list storages can be used (lil\_matrix). Substantial memory savings 
but much longer compute times.


%..............................................................................
\subsubsection{Sparse Matrix-Vector multiplication}
\index{general}{SpMV} \index{general}{Sparse Matrix-Vector Multiplication}

see 9.4 of Kepley's book. \cite{knepley}

\Literature \cite{krda10}


