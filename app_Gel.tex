The elemental matrix $\G_{el}$ is given by (see Section~\ref{sec:mixed}):

\[
\G_{el} = -\int_{\Omega_e} {\bm B}^T \cdot {\bm N} d\Omega
= -\int_{\Omega_e}
\left(
\begin{array}{ccc}
\partial_x N_1^\upnu & 0 & \partial_y N_1^\upnu \\
0 & \partial_y N_1^\upnu & \partial_x N_1^\upnu \\
\partial_x N_2^\upnu & 0 & \partial_y N_2^\upnu \\
0 & \partial_y N_2^\upnu & \partial_x N_2^\upnu \\
\dots & \dots & \dots \\
\dots & \dots & \dots \\
\partial_x N_{m_\upnu}^\upnu & 0 & \partial_y N_{m_\upnu}^\upnu \\
0 & \partial_y N_{m_\upnu}^\upnu & \partial_x N_{m_\upnu}^\upnu 
\end{array}
\right)
\cdot
\left(
\begin{array}{cccc}
N_1^p & N_2^p & \dots & N_{m_p}^p \\ 
N_1^p & N_2^p & \dots & N_{m_p}^p \\ 
0 & 0 & \dots & 0
\end{array}
\right)
d\Omega
\]

In what follows I set out to compute this elemental matrix for the reference element.
All of the integrals were computed with WolframAlpha since it allowed me to copy-paste 
the \LaTeX code directly into the website prompt area and obtain the value of these integrals.


%-------------------------------------------------------------------
\subsection{$Q_1\times P_0$ element - 2D}

For this element, $m_\upnu=4$ and $m_p=1$ so $\G_{el}$ is
a $8\times 1$ matrix:
\[
\G_{el} = -\int_{\Omega_e} {\bm B}^T \cdot {\bm N} d\Omega
= -\int_{\Omega_e}
\left(
\begin{array}{ccc}
\partial_r N_1^\upnu & 0 & \partial_s N_1^\upnu \\
0 & \partial_s N_1^\upnu & \partial_r N_1^\upnu \\
\partial_r N_2^\upnu & 0 & \partial_s N_2^\upnu \\
0 & \partial_s N_2^\upnu & \partial_r N_2^\upnu \\
\partial_r N_3^\upnu & 0 & \partial_s N_3^\upnu \\
0 & \partial_s N_3^\upnu & \partial_r N_3^\upnu \\
\partial_r N_4^\upnu & 0 & \partial_s N_4^\upnu \\
0 & \partial_s N_4^\upnu & \partial_r N_4^\upnu 
\end{array}
\right)
\cdot
\left(
\begin{array}{c}
N_1^p  \\ 
N_1^p  \\ 
0 
\end{array}
\right)
d\Omega
\]
also, since $N_1^p=1$ then 
\[
\G_{el} = -\int_{\Omega_e} {\bm B}^T \cdot {\bm N} d\Omega
= -\int_{\Omega_e}
\left(
\begin{array}{ccc}
\partial_r N_1^\upnu  \\
\partial_s N_1^\upnu  \\
\partial_r N_2^\upnu  \\
\partial_s N_2^\upnu  \\
\partial_r N_3^\upnu  \\
\partial_s N_3^\upnu  \\
\partial_r N_4^\upnu  \\
\partial_s N_4^\upnu  
\end{array}
\right)
d\Omega
=
\left(
\begin{array}{c}
 1\\
 1\\
-1\\
 1\\
-1\\
-1\\
 1\\
-1
\end{array}
\right)
\]

A macro element made of a single element makes no sense since if velocity 
is prescribed on all sides there is not a single velocity dof left.

We then consider the following macro-element:

\begin{verbatim}
velocity      pressure
7====8====9   .====.====.
|    |    |   |  3 |  4 |
4====5====6   .====.====.   NV=9
|    |    |   |  1 |  2 |   
1====2====3   .====.====.   NP=4
\end{verbatim}

The assembled $\G$ matrix is then $18 \times 4$:
\[
\G = 
\left(
\begin{array}{cccc}
  1  &   0 &   0 &   0  \\ 
  1  &   0 &   0 &   0  \\ 
 -1  &   1 &   0 &   0  \\ 
  1  &   1 &   0 &   0  \\ 
  0  &  -1 &   0 &   0  \\ 
  0  &   1 &   0 &   0  \\ 
  1  &   0 &   1 &   0  \\ 
 -1  &   0 &   1 &   0  \\ 
 -1  &   1 &  -1 &   1  \\ 
 -1  &  -1 &   1 &   1  \\ 
  0  &  -1 &   0 &  -1  \\ 
  0  &  -1 &   0 &   1  \\ 
  0  &   0 &   1 &   0  \\ 
  0  &   0 &  -1 &   0  \\ 
  0  &   0 &  -1 &   1  \\ 
  0  &   0 &  -1 &  -1  \\ 
  0  &   0 &   0 &  -1  \\ 
  0  &   0 &   0 &  -1  
\end{array}
\right)
\]
After applying boundary conditions on nodes 1,2,3,4,6,7,8,9:
\[
\G = 
\left(
\begin{array}{cccc}
  0  &   0 &   0 &   0  \\ 
  0  &   0 &   0 &   0  \\ 
  0  &   0 &   0 &   0  \\ 
  0  &   0 &   0 &   0  \\ 
  0  &   0 &   0 &   0  \\ 
  0  &   0 &   0 &   0  \\ 
  0  &   0 &   0 &   0  \\ 
  0  &   0 &   0 &   0  \\ 
 -1  &   1 &  -1 &   1  \\ 
 -1  &  -1 &   1 &   1  \\ 
  0  &   0 &   0 &   0  \\ 
  0  &   0 &   0 &   0  \\ 
  0  &   0 &   0 &   0  \\ 
  0  &   0 &   0 &   0  \\ 
  0  &   0 &   0 &   0  \\ 
  0  &   0 &   0 &   0  \\ 
  0  &   0 &   0 &   0  \\ 
  0  &   0 &   0 &   0  
\end{array}
\right)
\]
or, 
\[
\tilde{\G} = 
\left(
\begin{array}{cccc}
 -1  &   1 &  -1 &   1  \\ 
 -1  &  -1 &   1 &   1  
\end{array}
\right)
\]
The null space is of size two, spawn by the two vectors:
\begin{verbatim}
[[-0.1  0.7]
 [ 0.7  0.1]
 [ 0.7  0.1]
 [-0.1  0.7]]
\end{verbatim}

In the book the authors proceed to show that any such macroelement 
made of rectangles has a spurious mode. 

See code {\tt python\_codes/Gel/macro\_element\_q1p0.py}

%-------------------------------------------------------------------
\subsection{$Q_1\times P_0$ element - 3D}

For this element, $m_\upnu=8$ and $m_p=1$ so $\G_{el}$ is
a $3*8\times 1$ matrix:
\[
\G_{el} = -\int_{\Omega_e} {\bm B}^T \cdot {\bm N} d\Omega
= -\int_{\Omega_e}
\left(
\begin{array}{cccccc}
\partial_r N_1^\upnu & 0 & 0 & \partial_s N_1^\upnu  & \partial_t N_1^\upnu & 0 \\
0 & \partial_s N_1^\upnu  & 0 & \partial_r N_1^\upnu  & 0 & \partial_t N_1^\upnu \\
0 & 0 & \partial_t N_1^\upnu  & 0 & \partial_r N_1^\upnu  & \partial_s N_1^\upnu \\
\dots  & \dots & \dots  & \dots & \dots  & \dots \\
\dots  & \dots & \dots  & \dots & \dots  & \dots \\
\dots  & \dots & \dots  & \dots & \dots  & \dots \\
\partial_r N_8^\upnu & 0 & 0 & \partial_s N_8^\upnu  & \partial_t N_8^\upnu & 0 \\
0 & \partial_s N_8^\upnu  & 0 & \partial_r N_8^\upnu  & 0 & \partial_t N_8^\upnu \\
0 & 0 & \partial_t N_8^\upnu  & 0 & \partial_r N_8^\upnu  & \partial_s N_8^\upnu 
\end{array}
\right)
\cdot
\left(
\begin{array}{c}
N_1^p  \\ 
N_1^p  \\ 
N_1^p  \\ 
0 \\
0 \\
0 
\end{array}
\right)
d\Omega
\]
also, since $N_1^p=1$ then 
\[
\G_{el} = -\int_{\Omega_e} {\bm B}^T \cdot {\bm N} d\Omega
= -\int_{\Omega_e}
\left(
\begin{array}{ccc}
\partial_r N_1^\upnu  \\
\partial_s N_1^\upnu  \\
\partial_t N_1^\upnu  \\
\partial_r N_2^\upnu  \\
\partial_s N_2^\upnu  \\
\partial_t N_2^\upnu  \\
\partial_r N_3^\upnu  \\
\partial_s N_3^\upnu  \\
\partial_t N_3^\upnu  \\
\partial_r N_4^\upnu  \\
\partial_s N_4^\upnu  \\
\partial_t N_4^\upnu  \\
\partial_r N_5^\upnu  \\
\partial_s N_5^\upnu  \\
\partial_t N_5^\upnu  \\
\partial_r N_6^\upnu  \\
\partial_s N_6^\upnu  \\
\partial_t N_6^\upnu  \\
\partial_r N_7^\upnu  \\
\partial_s N_7^\upnu  \\
\partial_t N_7^\upnu  \\
\partial_r N_8^\upnu  \\
\partial_s N_8^\upnu  \\
\partial_t N_8^\upnu  
\end{array}
\right)
d\Omega
=
-
\left(
\begin{array}{c}
\int_{-1}^{+1}\int_{-1}^{+1}\int_{-1}^{+1} \frac{1}{8}(-1)(1-s)(1-t) dr ds dt \\
\int_{-1}^{+1}\int_{-1}^{+1}\int_{-1}^{+1} \frac{1}{8}(1-r)(-1)(1-t) dr ds dt \\
\int_{-1}^{+1}\int_{-1}^{+1}\int_{-1}^{+1} \frac{1}{8}(1-r)(1-s)(-1) dr ds dt \\
\int_{-1}^{+1}\int_{-1}^{+1}\int_{-1}^{+1} \frac{1}{8}(+1)(1-s)(1-t) dr ds dt \\
\int_{-1}^{+1}\int_{-1}^{+1}\int_{-1}^{+1} \frac{1}{8}(1+r)(-1)(1-t) dr ds dt \\
\int_{-1}^{+1}\int_{-1}^{+1}\int_{-1}^{+1} \frac{1}{8}(1+r)(1-s)(-1) dr ds dt \\
\int_{-1}^{+1}\int_{-1}^{+1}\int_{-1}^{+1} \frac{1}{8}(+1)(1+s)(1-t) dr ds dt \\
\int_{-1}^{+1}\int_{-1}^{+1}\int_{-1}^{+1} \frac{1}{8}(1+r)(+1)(1-t) dr ds dt \\
\int_{-1}^{+1}\int_{-1}^{+1}\int_{-1}^{+1} \frac{1}{8}(1+r)(1+s)(-1) dr ds dt \\
\int_{-1}^{+1}\int_{-1}^{+1}\int_{-1}^{+1} \frac{1}{8}(-1)(1+s)(1-t) dr ds dt \\
\int_{-1}^{+1}\int_{-1}^{+1}\int_{-1}^{+1} \frac{1}{8}(1-r)(+1)(1-t) dr ds dt \\
\int_{-1}^{+1}\int_{-1}^{+1}\int_{-1}^{+1} \frac{1}{8}(1-r)(1+s)(-1) dr ds dt \\
\int_{-1}^{+1}\int_{-1}^{+1}\int_{-1}^{+1} \frac{1}{8}(-1)(1-s)(1+t) dr ds dt \\
\int_{-1}^{+1}\int_{-1}^{+1}\int_{-1}^{+1} \frac{1}{8}(1-r)(-1)(1+t) dr ds dt \\
\int_{-1}^{+1}\int_{-1}^{+1}\int_{-1}^{+1} \frac{1}{8}(1-r)(1-s)(+1) dr ds dt \\
\int_{-1}^{+1}\int_{-1}^{+1}\int_{-1}^{+1} \frac{1}{8}(+1)(1-s)(1+t) dr ds dt \\
\int_{-1}^{+1}\int_{-1}^{+1}\int_{-1}^{+1} \frac{1}{8}(1+r)(-1)(1+t) dr ds dt \\
\int_{-1}^{+1}\int_{-1}^{+1}\int_{-1}^{+1} \frac{1}{8}(1+r)(1-s)(+1) dr ds dt \\
\int_{-1}^{+1}\int_{-1}^{+1}\int_{-1}^{+1} \frac{1}{8}(+1)(1+s)(1+t) dr ds dt \\
\int_{-1}^{+1}\int_{-1}^{+1}\int_{-1}^{+1} \frac{1}{8}(1+r)(+1)(1+t) dr ds dt \\
\int_{-1}^{+1}\int_{-1}^{+1}\int_{-1}^{+1} \frac{1}{8}(1+r)(1+s)(+1) dr ds dt \\
\int_{-1}^{+1}\int_{-1}^{+1}\int_{-1}^{+1} \frac{1}{8}(-1)(1+s)(1+t) dr ds dt \\
\int_{-1}^{+1}\int_{-1}^{+1}\int_{-1}^{+1} \frac{1}{8}(1-r)(+1)(1+t) dr ds dt \\
\int_{-1}^{+1}\int_{-1}^{+1}\int_{-1}^{+1} \frac{1}{8}(1-r)(1+s)(+1) dr ds dt 
\end{array}
\right)
=
\left(
\begin{array}{c}
 1\\
 1\\
 1\\
-1\\
 1\\
 1\\
-1\\
-1\\
 1\\
 1\\
-1\\
 1\\
 1\\
 1\\
-1\\
-1\\
 1\\
-1\\
-1\\
-1\\
-1\\
 1\\
-1\\
-1
\end{array}
\right)
\]

If we consider a macro-element 2x2x2 of size Lx=Ly=Lz=4, apply velocity b.c on the all sides
we are left with 
\[
\tilde{G} = 
\left(
\begin{array}{cccccc}
-1. -1. -1. -1.  1.  1.  1.  1.\\
-1. -1.  1.  1. -1. -1.  1.  1.\\
-1.  1. -1.  1. -1.  1. -1.  1.
\end{array}
\right)
\]

Null space has dimension 5:
\begin{verbatim}
[[-0.35355339  0.35355339  0.35355339  0.35355339  0.35355339]
 [ 0.35355339 -0.35355339 -0.35355339  0.35355339  0.35355339]
 [ 0.35355339 -0.35355339  0.35355339 -0.35355339  0.35355339]
 [ 0.41990569  0.58009431  0.19336477  0.19336477 -0.19336477]
 [ 0.58009431  0.41990569 -0.19336477 -0.19336477  0.19336477]
 [ 0.19336477 -0.19336477  0.74028293  0.03317615 -0.03317615]
 [ 0.19336477 -0.19336477  0.03317615  0.74028293 -0.03317615]
 [-0.19336477  0.19336477 -0.03317615 -0.03317615  0.74028293]]
\end{verbatim}





%-------------------------------------------------------------------
\subsection{$Q_1\times Q_1$ element}

For this element, $m_\upnu=4$ and $m_p=4$ so $\G_{el}$ is
a $8\times 4$ matrix:

\begin{eqnarray}
\G_{el} 
&=& -\int_{\Omega_e} {\bm B}^T \cdot {\bm N} d\Omega \\
&=& -\int_{\Omega_e}
\left(
\begin{array}{ccc}
\partial_r N_1^\upnu & 0 & \partial_s N_1^\upnu \\
0 & \partial_s N_1^\upnu & \partial_r N_1^\upnu \\
\partial_r N_2^\upnu & 0 & \partial_s N_2^\upnu \\
0 & \partial_s N_2^\upnu & \partial_r N_2^\upnu \\
\partial_r N_3^\upnu & 0 & \partial_s N_3^\upnu \\
0 & \partial_s N_3^\upnu & \partial_r N_3^\upnu \\
\partial_r N_4^\upnu & 0 & \partial_s N_4^\upnu \\
0 & \partial_s N_4^\upnu & \partial_r N_4^\upnu \\
\end{array}
\right)
\cdot
\left(
\begin{array}{cccc}
N_1^p & N_2^p & N_3^p & N_{4}^p \\ 
N_1^p & N_2^p & N_3^p & N_{4}^p \\ 
0 & 0 & \dots & 0
\end{array}
\right)
d\Omega \\
&=&
-\int_{\Omega_e}
\left(
\begin{array}{cccc}
N_1^p\partial_r N_1^\upnu & N_2^p\partial_r N_1^\upnu & N_3^p\partial_r N_1^\upnu & N_4^p\partial_r N_1^\upnu \\
N_1^p\partial_s N_1^\upnu & N_2^p\partial_s N_1^\upnu & N_3^p\partial_s N_1^\upnu & N_4^p\partial_s N_1^\upnu \\
N_1^p\partial_r N_2^\upnu & N_2^p\partial_r N_2^\upnu & N_3^p\partial_r N_2^\upnu & N_4^p\partial_r N_2^\upnu \\
N_1^p\partial_s N_2^\upnu & N_2^p\partial_s N_2^\upnu & N_3^p\partial_s N_2^\upnu & N_4^p\partial_s N_2^\upnu \\
N_1^p\partial_r N_3^\upnu & N_2^p\partial_r N_3^\upnu & N_3^p\partial_r N_3^\upnu & N_4^p\partial_r N_3^\upnu \\
N_1^p\partial_s N_3^\upnu & N_2^p\partial_s N_3^\upnu & N_3^p\partial_s N_3^\upnu & N_4^p\partial_s N_3^\upnu \\
N_1^p\partial_r N_4^\upnu & N_2^p\partial_r N_4^\upnu & N_3^p\partial_r N_4^\upnu & N_4^p\partial_r N_4^\upnu \\
N_1^p\partial_s N_4^\upnu & N_2^p\partial_s N_4^\upnu & N_3^p\partial_s N_4^\upnu & N_4^p\partial_s N_4^\upnu 
\end{array}
\right)
d\Omega \\
\end{eqnarray}
We have $N_i^\upnu = N_i^p $ with $i=1,2,3,4$, so we can drop the superscripts and we can write:

\begin{eqnarray}
\G_{el} 
&=&
-\int_{\Omega_e}
\left(
\begin{array}{cccc}
N_1\partial_r N_1 & N_2\partial_r N_1 & N_3\partial_r N_1 & N_4\partial_r N_1 \\
N_1\partial_s N_1 & N_2\partial_s N_1 & N_3\partial_s N_1 & N_4\partial_s N_1 \\
N_1\partial_r N_2 & N_2\partial_r N_2 & N_3\partial_r N_2 & N_4\partial_r N_2 \\
N_1\partial_s N_2 & N_2\partial_s N_2 & N_3\partial_s N_2 & N_4\partial_s N_2 \\
N_1\partial_r N_3 & N_2\partial_r N_3 & N_3\partial_r N_3 & N_4\partial_r N_3 \\
N_1\partial_s N_3 & N_2\partial_s N_3 & N_3\partial_s N_3 & N_4\partial_s N_3 \\
N_1\partial_r N_4 & N_2\partial_r N_4 & N_3\partial_r N_4 & N_4\partial_r N_4 \\
N_1\partial_s N_4 & N_2\partial_s N_4 & N_3\partial_s N_4 & N_4\partial_s N_4 
\end{array}
\right)
d\Omega 
\end{eqnarray}

\begin{tiny}

\begin{eqnarray}
\int_{\Omega_e}  N_1\partial_r N_1 d\Omega&=& \int_{-1}^{+1}\int_{-1}^{+1} \frac{1}{4}(1-r)(1-s) \frac{1}{4}(-1)(1-s) dr ds = -1/3 \\
\int_{\Omega_e}  N_1\partial_s N_1 d\Omega&=& \int_{-1}^{+1}\int_{-1}^{+1} \frac{1}{4}(1-r)(1-s) \frac{1}{4}(1-r)(-1) dr ds = -1/3 \\
\int_{\Omega_e}  N_1\partial_r N_2 d\Omega&=& \int_{-1}^{+1}\int_{-1}^{+1} \frac{1}{4}(1-r)(1-s) \frac{1}{4}(+1)(1-s) dr ds = 1/3 \\
\int_{\Omega_e}  N_1\partial_s N_2 d\Omega&=& \int_{-1}^{+1}\int_{-1}^{+1} \frac{1}{4}(1-r)(1-s) \frac{1}{4}(1+r)(-1) dr ds = -1/6 \\
\int_{\Omega_e}  N_1\partial_r N_3 d\Omega&=& \int_{-1}^{+1}\int_{-1}^{+1} \frac{1}{4}(1-r)(1-s) \frac{1}{4}(+1)(1+s) dr ds = 1/6 \\
\int_{\Omega_e}  N_1\partial_s N_3 d\Omega&=& \int_{-1}^{+1}\int_{-1}^{+1} \frac{1}{4}(1-r)(1-s) \frac{1}{4}(1+r)(+1) dr ds = 1/6 \\
\int_{\Omega_e}  N_1\partial_r N_4 d\Omega&=& \int_{-1}^{+1}\int_{-1}^{+1} \frac{1}{4}(1-r)(1-s) \frac{1}{4}(-1)(1+s) dr ds = -1/6 \\
\int_{\Omega_e}  N_1\partial_s N_4 d\Omega&=& \int_{-1}^{+1}\int_{-1}^{+1} \frac{1}{4}(1-r)(1-s) \frac{1}{4}(1-r)(+1) dr ds = 1/3 
\end{eqnarray}

\begin{eqnarray}
\int_{\Omega_e}  N_2\partial_r N_1 d\Omega&=& \int_{-1}^{+1}\int_{-1}^{+1} \frac{1}{4}(1+r)(1-s) \frac{1}{4}(-1)(1-s) dr ds = -1/3 \\
\int_{\Omega_e}  N_2\partial_s N_1 d\Omega&=& \int_{-1}^{+1}\int_{-1}^{+1} \frac{1}{4}(1+r)(1-s) \frac{1}{4}(1-r)(-1) dr ds = -1/6 \\
\int_{\Omega_e}  N_2\partial_r N_2 d\Omega&=& \int_{-1}^{+1}\int_{-1}^{+1} \frac{1}{4}(1+r)(1-s) \frac{1}{4}(+1)(1-s) dr ds = 1/3 \\
\int_{\Omega_e}  N_2\partial_s N_2 d\Omega&=& \int_{-1}^{+1}\int_{-1}^{+1} \frac{1}{4}(1+r)(1-s) \frac{1}{4}(1+r)(-1) dr ds = -1/3 \\
\int_{\Omega_e}  N_2\partial_r N_3 d\Omega&=& \int_{-1}^{+1}\int_{-1}^{+1} \frac{1}{4}(1+r)(1-s) \frac{1}{4}(+1)(1+s) dr ds = 1/6 \\
\int_{\Omega_e}  N_2\partial_s N_3 d\Omega&=& \int_{-1}^{+1}\int_{-1}^{+1} \frac{1}{4}(1+r)(1-s) \frac{1}{4}(1+r)(+1) dr ds = 1/3  \\
\int_{\Omega_e}  N_2\partial_r N_4 d\Omega&=& \int_{-1}^{+1}\int_{-1}^{+1} \frac{1}{4}(1+r)(1-s) \frac{1}{4}(-1)(1+s) dr ds = -1/6 \\
\int_{\Omega_e}  N_2\partial_s N_4 d\Omega&=& \int_{-1}^{+1}\int_{-1}^{+1} \frac{1}{4}(1+r)(1-s) \frac{1}{4}(1-r)(+1) dr ds = 1/6 
\end{eqnarray}


\begin{eqnarray}
\int_{\Omega_e}  N_3\partial_r N_1 d\Omega&=& \int_{-1}^{+1}\int_{-1}^{+1} \frac{1}{4}(1+r)(1+s) \frac{1}{4}(-1)(1-s) dr ds = -1/6 \\
\int_{\Omega_e}  N_3\partial_s N_1 d\Omega&=& \int_{-1}^{+1}\int_{-1}^{+1} \frac{1}{4}(1+r)(1+s) \frac{1}{4}(1-r)(-1) dr ds = -1/6 \\
\int_{\Omega_e}  N_3\partial_r N_2 d\Omega&=& \int_{-1}^{+1}\int_{-1}^{+1} \frac{1}{4}(1+r)(1+s) \frac{1}{4}(+1)(1-s) dr ds = 1/6\\
\int_{\Omega_e}  N_3\partial_s N_2 d\Omega&=& \int_{-1}^{+1}\int_{-1}^{+1} \frac{1}{4}(1+r)(1+s) \frac{1}{4}(1+r)(-1) dr ds = -1/3\\
\int_{\Omega_e}  N_3\partial_r N_3 d\Omega&=& \int_{-1}^{+1}\int_{-1}^{+1} \frac{1}{4}(1+r)(1+s) \frac{1}{4}(+1)(1+s) dr ds = 1/3 \\
\int_{\Omega_e}  N_3\partial_s N_3 d\Omega&=& \int_{-1}^{+1}\int_{-1}^{+1} \frac{1}{4}(1+r)(1+s) \frac{1}{4}(1+r)(+1) dr ds = 1/3  \\
\int_{\Omega_e}  N_3\partial_r N_4 d\Omega&=& \int_{-1}^{+1}\int_{-1}^{+1} \frac{1}{4}(1+r)(1+s) \frac{1}{4}(-1)(1+s) dr ds = -1/3 \\
\int_{\Omega_e}  N_3\partial_s N_4 d\Omega&=& \int_{-1}^{+1}\int_{-1}^{+1} \frac{1}{4}(1+r)(1+s) \frac{1}{4}(1-r)(+1) dr ds = 1/6 
\end{eqnarray}


\begin{eqnarray}
\int_{\Omega_e}  N_4\partial_r N_1 d\Omega&=& \int_{-1}^{+1}\int_{-1}^{+1} \frac{1}{4}(1-r)(1+s) \frac{1}{4}(-1)(1-s) dr ds = -1/6 \\
\int_{\Omega_e}  N_4\partial_s N_1 d\Omega&=& \int_{-1}^{+1}\int_{-1}^{+1} \frac{1}{4}(1-r)(1+s) \frac{1}{4}(1-r)(-1) dr ds = -1/3 \\
\int_{\Omega_e}  N_4\partial_r N_2 d\Omega&=& \int_{-1}^{+1}\int_{-1}^{+1} \frac{1}{4}(1-r)(1+s) \frac{1}{4}(+1)(1-s) dr ds = 1/6 \\
\int_{\Omega_e}  N_4\partial_s N_2 d\Omega&=& \int_{-1}^{+1}\int_{-1}^{+1} \frac{1}{4}(1-r)(1+s) \frac{1}{4}(1+r)(-1) dr ds = -1/6\\
\int_{\Omega_e}  N_4\partial_r N_3 d\Omega&=& \int_{-1}^{+1}\int_{-1}^{+1} \frac{1}{4}(1-r)(1+s) \frac{1}{4}(+1)(1+s) dr ds =  1/3 \\
\int_{\Omega_e}  N_4\partial_s N_3 d\Omega&=& \int_{-1}^{+1}\int_{-1}^{+1} \frac{1}{4}(1-r)(1+s) \frac{1}{4}(1+r)(+1) dr ds = 1/6  \\
\int_{\Omega_e}  N_4\partial_r N_4 d\Omega&=& \int_{-1}^{+1}\int_{-1}^{+1} \frac{1}{4}(1-r)(1+s) \frac{1}{4}(-1)(1+s) dr ds = -1/3 \\
\int_{\Omega_e}  N_4\partial_s N_4 d\Omega&=& \int_{-1}^{+1}\int_{-1}^{+1} \frac{1}{4}(1-r)(1+s) \frac{1}{4}(1-r)(+1) dr ds = 1/3 
\end{eqnarray}


\end{tiny}

Putting it all together:


\begin{eqnarray}
\G_{el} 
&=&
-
\left(
\begin{array}{cccc}
-1/3 & -1/3  & -1/6 &  -1/6  \\
-1/3 & -1/6  & -1/6 &  -1/3  \\
1/3  & 1/3   & 1/6  &   1/6 \\
-1/6 & -1/3  & -1/3 &  -1/6  \\
1/6 & 1/6    & 1/3  &   1/3 \\
1/6 & 1/3    & 1/3  &   1/6 \\
-1/6 & -1/6   & -1/3 &  -1/3  \\
1/3 & 1/6    & 1/6  &   1/3 \\
\end{array}
\right)
=
\frac{1}{6}
\left(
\begin{array}{cccc}
 2&  2&  1&  1\\
 2&  1&  1&  2\\
-2& -2& -1& -1\\
 1&  2&  2&  1\\
-1& -1& -2& -2\\
-1& -2& -2& -1\\
 1&  1&  2&  2\\
-2& -1& -1& -2
\end{array}
\right)
\end{eqnarray}

I have implemented a 3x3 quadrature integration to numerically compute the matrix in the file {\tt python\_codes/Gel/programQ1Q1.py}.
The code returns:
\begin{verbatim}
[[ 0.33333333  0.33333333  0.16666667  0.16666667]
 [ 0.33333333  0.16666667  0.16666667  0.33333333]
 [-0.33333333 -0.33333333 -0.16666667 -0.16666667]
 [ 0.16666667  0.33333333  0.33333333  0.16666667]
 [-0.16666667 -0.16666667 -0.33333333 -0.33333333]
 [-0.16666667 -0.33333333 -0.33333333 -0.16666667]
 [ 0.16666667  0.16666667  0.33333333  0.33333333]
 [-0.33333333 -0.16666667 -0.16666667 -0.33333333]]
\end{verbatim}
which is indeed what we have obtained above.

Similarly to the Q1P0 element, a macroelement made of a single element has 
zero left over vel dof after b.c. are applied on all sides, so we 
resort to the following macroelement:

\begin{verbatim}
velocity      pressure
7====8====9   7====8====9
|    |    |   |    |    |
4====5====6   4====5====6   NV=9
|    |    |   |    |    |   
1====2====3   1====2====3   NP=9
\end{verbatim}

After assembly we have $\G$ is a $ndofV*NV\times ndofP*NP=18*9$ matrix:
\[
\G=
\frac{1}{6}
\left(
\begin{array}{ccccccccc}
  2  &   2 &   0 &   1 &   1 &    0 &   0 &   0 &   0 \\ 
  2  &   1 &   0 &   2 &   1 &    0 &   0 &   0 &   0 \\ 
 -2  &   0 &   2 &  -1 &   0 &    1 &   0 &   0 &   0 \\ 
  1  &   4 &   1 &   1 &   4 &    1 &   0 &   0 &   0 \\ 
  0  &  -2 &  -2 &   0 &  -1 &   -1 &   0 &   0 &   0 \\ 
  0  &   1 &   2 &   0 &   1 &    2 &   0 &   0 &   0 \\ 
  1  &   1 &   0 &   4 &   4 &    0 &   1 &   1 &   0 \\ 
 -2  &  -1 &   0 &   0 &   0 &    0 &   2 &   1 &   0 \\ 
 -1  &   0 &   1 &  -4 &   0 &    4 &  -1 &   0 &   1 \\ 
 -1  &  -4 &  -1 &   0 &   0 &    0 &   1 &   4 &   1 \\ 
  0  &  -1 &  -1 &   0 &  -4 &   -4 &   0 &  -1 &  -1 \\ 
  0  &  -1 &  -2 &   0 &   0 &    0 &   0 &   1 &   2 \\ 
  0  &   0 &   0 &   1 &   1 &    0 &   2 &   2 &   0 \\ 
  0  &   0 &   0 &  -2 &  -1 &    0 &  -2 &  -1 &   0 \\ 
  0  &   0 &   0 &  -1 &   0 &    1 &  -2 &   0 &   2 \\ 
  0  &   0 &   0 &  -1 &  -4 &   -1 &  -1 &  -4 &  -1 \\ 
  0  &   0 &   0 &   0 &  -1 &   -1 &   0 &  -2 &  -2 \\ 
  0  &   0 &   0 &   0 &  -1 &   -2 &   0 &  -1 &  -2 
\end{array}
\right)
\]
and after imposing boundary conditions on nodes 1,2,3,4,6,7,8,9:
\[
\G=
\frac{1}{6}
\left(
\begin{array}{ccccccccc}
  0  &   0 &   0 &   0 &   0 &    0 &   0 &   0 &   0 \\ 
  0  &   0 &   0 &   0 &   0 &    0 &   0 &   0 &   0 \\ 
  0  &   0 &   0 &   0 &   0 &    0 &   0 &   0 &   0 \\ 
  0  &   0 &   0 &   0 &   0 &    0 &   0 &   0 &   0 \\ 
  0  &   0 &   0 &   0 &   0 &    0 &   0 &   0 &   0 \\ 
  0  &   0 &   0 &   0 &   0 &    0 &   0 &   0 &   0 \\ 
  0  &   0 &   0 &   0 &   0 &    0 &   0 &   0 &   0 \\ 
  0  &   0 &   0 &   0 &   0 &    0 &   0 &   0 &   0 \\ 
 -1  &   0 &   1 &  -4 &   0 &    4 &  -1 &   0 &   1 \\ 
 -1  &  -4 &  -1 &   0 &   0 &    0 &   1 &   4 &   1 \\ 
  0  &   0 &   0 &   0 &   0 &    0 &   0 &   0 &   0 \\ 
  0  &   0 &   0 &   0 &   0 &    0 &   0 &   0 &   0 \\ 
  0  &   0 &   0 &   0 &   0 &    0 &   0 &   0 &   0 \\ 
  0  &   0 &   0 &   0 &   0 &    0 &   0 &   0 &   0 \\ 
  0  &   0 &   0 &   0 &   0 &    0 &   0 &   0 &   0 \\ 
  0  &   0 &   0 &   0 &   0 &    0 &   0 &   0 &   0 \\ 
  0  &   0 &   0 &   0 &   0 &    0 &   0 &   0 &   0 \\ 
  0  &   0 &   0 &   0 &   0 &    0 &   0 &   0 &   0 
\end{array}
\right)
\]
or, 
\[
\G=
\frac{1}{6}
\left(
\begin{array}{ccccccccc}
 -1  &   0 &   1 &  -4 &   0 &    4 &  -1 &   0 &   1 \\ 
 -1  &  -4 &  -1 &   0 &   0 &    0 &   1 &   4 &   1 
\end{array}
\right)
\]
When passed to {\sl null\_space} as argument it returns the following nullspace:
\begin{verbatim}
[[ 1.47619e-01 -6.57142e-01 -7.40148e-18  6.57142e-01 -1.47619e-01  6.66666e-02  1.80952e-01]
 [-1.90476e-01  9.52380e-02 -7.40148e-17 -9.52380e-02  1.90476e-01  6.66666e-01  1.42857e-01]
 [ 9.57142e-01  1.04761e-01 -7.40148e-18 -1.04761e-01  4.28571e-02  6.66666e-02 -9.52380e-03]
 [ 9.52380e-02  6.19047e-01  2.34780e-34  3.80952e-01 -9.52380e-02 -2.11471e-18  9.52380e-02]
 [-8.45884e-18  4.22942e-18  1.00000e+00 -4.22942e-18  8.45884e-18  2.96059e-17  6.34413e-18]
 [-9.52380e-02  3.80952e-01  5.35591e-34  6.19047e-01  9.52380e-02 -4.82418e-18 -9.52380e-02]
 [ 4.28571e-02 -1.04761e-01  7.40148e-18  1.04761e-01  9.57142e-01 -6.66666e-02  9.52380e-03]
 [ 7.61904e-02 -3.80952e-02  2.96059e-17  3.80952e-02 -7.61904e-02  7.33333e-01 -5.71428e-02]
 [-4.76190e-03  8.57142e-02  7.40148e-18 -8.57142e-02  4.76190e-03 -6.66666e-02  9.61904e-01]]
\end{verbatim}
which is very bad: the dimension of the nullspace is 9!

Note that it does not mean that this element is unstable (see Q2Q1) since it is a sufficient 
but not necessary condition. We could test with larger macroelements (see Q2Q1) and these could 
prove to have a properly sized nullspace. 

See code {\tt python\_codes/Gel/macro\_element\_q1q1.py}

%-------------------------------------------------------------------
\subsection{$Q_1^+\times Q_1$ element}

For the quadrilateral MINI element, $m_\upnu=5$ and $m_p=4$ so $\G_{el}$ is
a $10\times 4$ matrix:
\begin{eqnarray}
\G_{el} 
&=& -\int_{\Omega_e} {\bm B}^T \cdot {\bm N} d\Omega \\
&=& -\int_{\Omega_e}
\left(
\begin{array}{ccc}
\partial_r N_1^\upnu & 0 & \partial_s N_1^\upnu \\
0 & \partial_s N_1^\upnu & \partial_r N_1^\upnu \\
\partial_r N_2^\upnu & 0 & \partial_s N_2^\upnu \\
0 & \partial_s N_2^\upnu & \partial_r N_2^\upnu \\
\partial_r N_3^\upnu & 0 & \partial_s N_3^\upnu \\
0 & \partial_s N_3^\upnu & \partial_r N_3^\upnu \\
\partial_r N_4^\upnu & 0 & \partial_s N_4^\upnu \\
0 & \partial_s N_4^\upnu & \partial_r N_4^\upnu \\
\partial_r N_5^\upnu & 0 & \partial_s N_5^\upnu \\
0 & \partial_s N_5^\upnu & \partial_r N_5^\upnu 
\end{array}
\right)
\cdot
\left(
\begin{array}{cccc}
N_1^p & N_2^p & N_3^p & N_{4}^p \\ 
N_1^p & N_2^p & N_3^p & N_{4}^p \\ 
0 & 0 & \dots & 0
\end{array}
\right)
d\Omega \\
&=&
-\int_{\Omega_e}
\left(
\begin{array}{cccc}
N_1^p\partial_r N_1^\upnu & N_2^p\partial_r N_1^\upnu & N_3^p\partial_r N_1^\upnu & N_4^p\partial_r N_1^\upnu \\
N_1^p\partial_s N_1^\upnu & N_2^p\partial_s N_1^\upnu & N_3^p\partial_s N_1^\upnu & N_4^p\partial_s N_1^\upnu \\
N_1^p\partial_r N_2^\upnu & N_2^p\partial_r N_2^\upnu & N_3^p\partial_r N_2^\upnu & N_4^p\partial_r N_2^\upnu \\
N_1^p\partial_s N_2^\upnu & N_2^p\partial_s N_2^\upnu & N_3^p\partial_s N_2^\upnu & N_4^p\partial_s N_2^\upnu \\
N_1^p\partial_r N_3^\upnu & N_2^p\partial_r N_3^\upnu & N_3^p\partial_r N_3^\upnu & N_4^p\partial_r N_3^\upnu \\
N_1^p\partial_s N_3^\upnu & N_2^p\partial_s N_3^\upnu & N_3^p\partial_s N_3^\upnu & N_4^p\partial_s N_3^\upnu \\
N_1^p\partial_r N_4^\upnu & N_2^p\partial_r N_4^\upnu & N_3^p\partial_r N_4^\upnu & N_4^p\partial_r N_4^\upnu \\
N_1^p\partial_s N_4^\upnu & N_2^p\partial_s N_4^\upnu & N_3^p\partial_s N_4^\upnu & N_4^p\partial_s N_4^\upnu \\
N_1^p\partial_r N_5^\upnu & N_2^p\partial_r N_5^\upnu & N_3^p\partial_r N_5^\upnu & N_4^p\partial_r N_5^\upnu \\
N_1^p\partial_s N_5^\upnu & N_2^p\partial_s N_5^\upnu & N_3^p\partial_s N_5^\upnu & N_4^p\partial_s N_5^\upnu 
\end{array}
\right)
d\Omega 
\end{eqnarray}

We have :
\begin{eqnarray} 
N_1^\upnu &=& N_1^p - \frac{1}{4}b(r,s)  \\
N_2^\upnu &=& N_2^p - \frac{1}{4}b(r,s)  \\
N_3^\upnu &=& N_3^p - \frac{1}{4}b(r,s)  \\
N_4^\upnu &=& N_4^p - \frac{1}{4}b(r,s)  \\
N_5^\upnu &=& b(r,s)  
\end{eqnarray}
so that (once again I drop the superscripts)
\begin{small}
\begin{eqnarray}
\G_{el} =
-\int_{\Omega_e}
\left(
\begin{array}{cccc}
N_1\partial_r N_1 & N_2\partial_r N_1 & N_3\partial_r N_1 & N_4\partial_r N_1 \\
N_1\partial_s N_1 & N_2\partial_s N_1 & N_3\partial_s N_1 & N_4\partial_s N_1 \\
N_1\partial_r N_2 & N_2\partial_r N_2 & N_3\partial_r N_2 & N_4\partial_r N_2 \\
N_1\partial_s N_2 & N_2\partial_s N_2 & N_3\partial_s N_2 & N_4\partial_s N_2 \\
N_1\partial_r N_3 & N_2\partial_r N_3 & N_3\partial_r N_3 & N_4\partial_r N_3 \\
N_1\partial_s N_3 & N_2\partial_s N_3 & N_3\partial_s N_3 & N_4\partial_s N_3 \\
N_1\partial_r N_4 & N_2\partial_r N_4 & N_3\partial_r N_4 & N_4\partial_r N_4 \\
N_1\partial_s N_4 & N_2\partial_s N_4 & N_3\partial_s N_4 & N_4\partial_s N_4 \\
0 & 0 & 0 & 0 \\
0 & 0 & 0 & 0 
\end{array}
\right)
d\Omega 
+
\frac{1}{4}\int_{\Omega_e}
\left(
\begin{array}{cccc}
N_1\partial_r b & N_2\partial_r b & N_3\partial_r b & N_4\partial_r b \\
N_1\partial_s b & N_2\partial_s b & N_3\partial_s b & N_4\partial_s b\\
N_1\partial_r b & N_2\partial_r b & N_3\partial_r b & N_4\partial_r b\\
N_1\partial_s b & N_2\partial_s b & N_3\partial_s b & N_4\partial_s b\\
N_1\partial_r b & N_2\partial_r b & N_3\partial_r b & N_4\partial_r b\\
N_1\partial_s b & N_2\partial_s b & N_3\partial_s b & N_4\partial_s b\\
N_1\partial_r b & N_2\partial_r b & N_3\partial_r b & N_4\partial_r b\\
N_1\partial_s b & N_2\partial_s b & N_3\partial_s b & N_4\partial_s b\\
-4N_1\partial_r b & -4N_2\partial_r b & -4N_3\partial_r b & -4N_4\partial_r b \\
-4N_1\partial_s b & -4N_2\partial_s b & -4N_3\partial_s b & -4N_4\partial_s b
\end{array}
\right)
d\Omega \nonumber
\end{eqnarray}
\end{small}

The matrix which only contains $N_i$ functions is in fact the $\G_{el}$ matrix for standard $Q_1\times Q_1$ elements
as we have seen in the previous section so we need not recompute it. 

\subsubsection{Bubble function 1}
Let us now assume the bubble is bubble 1:
\begin{eqnarray}
b_1(r,s)&=&(1-r)(1-s)(1-r^2)(1-s^2) \nn\\
\partial_r b_1(r,s) &=& (3r^2-2r-1)(1-s)(1-s^2) \nn\\
\partial_s b_1(r,s) &=& (3s^2-2s-1)(1-r)(1-r^2) \nn
\end{eqnarray}

\begin{tiny}
\begin{eqnarray}
\frac{1}{4} \int_{-1}^{+1}\int_{-1}^{+1} N_1 \partial_r b_1 drds &=& \frac{1}{4} \int_{-1}^{+1}\int_{-1}^{+1} \frac{1}{4}(1-r)(1-s) (3r^2-2r-1)(1-s)(1-s^2) dr ds = 2/15 \nn\\
\frac{1}{4} \int_{-1}^{+1}\int_{-1}^{+1} N_1 \partial_s b_1 drds &=& \frac{1}{4} \int_{-1}^{+1}\int_{-1}^{+1} \frac{1}{4}(1-r)(1-s) (3s^2-2s-1)(1-r)(1-r^2) dr ds = 2/15 \nn\\
\frac{1}{4} \int_{-1}^{+1}\int_{-1}^{+1} N_2 \partial_r b_1 drds &=& \frac{1}{4} \int_{-1}^{+1}\int_{-1}^{+1} \frac{1}{4}(1+r)(1-s) (3r^2-2r-1)(1-s)(1-s^2) dr ds = -2/15 \nn\\
\frac{1}{4} \int_{-1}^{+1}\int_{-1}^{+1} N_2 \partial_s b_1 drds &=& \frac{1}{4} \int_{-1}^{+1}\int_{-1}^{+1} \frac{1}{4}(1+r)(1-s) (3s^2-2s-1)(1-r)(1-r^2) dr ds = 4/45 \nn\\
\frac{1}{4} \int_{-1}^{+1}\int_{-1}^{+1} N_3 \partial_r b_1 drds &=& \frac{1}{4} \int_{-1}^{+1}\int_{-1}^{+1} \frac{1}{4}(1+r)(1+s) (3r^2-2r-1)(1-s)(1-s^2) dr ds = -4/45 \nn\\
\frac{1}{4} \int_{-1}^{+1}\int_{-1}^{+1} N_3 \partial_s b_1 drds &=& \frac{1}{4} \int_{-1}^{+1}\int_{-1}^{+1} \frac{1}{4}(1+r)(1+s) (3s^2-2s-1)(1-r)(1-r^2) dr ds = -4/45 \nn\\
\frac{1}{4} \int_{-1}^{+1}\int_{-1}^{+1} N_4 \partial_r b_1 drds &=& \frac{1}{4} \int_{-1}^{+1}\int_{-1}^{+1} \frac{1}{4}(1-r)(1+s) (3r^2-2r-1)(1-s)(1-s^2) dr ds = 4/45 \nn\\
\frac{1}{4} \int_{-1}^{+1}\int_{-1}^{+1} N_4 \partial_s b_1 drds &=& \frac{1}{4} \int_{-1}^{+1}\int_{-1}^{+1} \frac{1}{4}(1-r)(1+s) (3s^2-2s-1)(1-r)(1-r^2) dr ds = -2/15 \nn
\end{eqnarray}
\end{tiny}


\begin{eqnarray}
\G_{el} 
&=&
-
\left(
\begin{array}{cccc}
-1/3 & -1/3  & -1/6 &  -1/6  \\
-1/3 & -1/6  & -1/6 &  -1/3  \\
1/3  & 1/3   & 1/6  &   1/6 \\
-1/6 & -1/3  & -1/3 &  -1/6  \\
1/6 & 1/6    & 1/3  &   1/3 \\
1/6 & 1/3    & 1/3  &   1/6 \\
-1/6 & -1/6   & -1/3 &  -1/3  \\
1/3 & 1/6    & 1/6  &   1/3 \\
0 & 0 & 0 & 0   \\
0 & 0 & 0 & 0     
\end{array}
\right)
+
\left(
\begin{array}{cccc}
2/15 & -2/15   & -4/45   & 4/45 \\
2/15 & 4/45   & -4/45   & -2/15  \\
2/15 & -2/15   & -4/45   & 4/45  \\
2/15 & 4/45   & -4/45   & -2/15  \\
2/15 & -2/15   & -4/45   & 4/45  \\
2/15 & 4/45   & -4/45   & -2/15  \\
2/15 & -2/15   & -4/45   & 4/45  \\
2/15 & 4/45   & -4/45   & -2/15  \\
-8/15 & 8/15  & 16/45   & -16/45  \\
-8/15 & -16/45 & 16/45   & 8/15  
\end{array}
\right) 
\end{eqnarray}

I have implemented a 3x3 quadrature integration to numerically compute the matrix in the file {\tt python\_codes/Gel/programQ1pQ1.py}.
The code returns:
\begin{verbatim}
[[ 0.46666667  0.2         0.07777778  0.25555556]
 [ 0.46666667  0.25555556  0.07777778  0.2       ]
 [-0.2        -0.46666667 -0.25555556 -0.07777778]
 [ 0.3         0.42222222  0.24444444  0.03333333]
 [-0.03333333 -0.3        -0.42222222 -0.24444444]
 [-0.03333333 -0.24444444 -0.42222222 -0.3       ]
 [ 0.3         0.03333333  0.24444444  0.42222222]
 [-0.2        -0.07777778 -0.25555556 -0.46666667]
 [-0.53333333  0.53333333  0.35555556 -0.35555556]
 [-0.53333333 -0.35555556  0.35555556  0.53333333]]
\end{verbatim}
which is indeed what we have obtained above.
It can be rewritten
\[
\G_{el}=
\frac{1}{90}
\left(
\begin{array}{cccc}
 42&  18&   7&  23\\
 42&  23&   7&  18\\
-18& -42& -23&  -7\\
 27&  38&  22&   3\\
 -3& -27& -38& -22\\
 -3& -22& -38& -27\\
 27&   3&  22&  38\\
-18&  -7& -23& -42\\
-48&  48&  32& -32\\
-48& -32&  32&  48
\end{array}
\right) 
\]


Let us now build a macroelement of size LxxLy=4x4 made of 2x2 elements.
Each element has a Gel like the one above since they are of size 2x2:
\begin{verbatim}
velocity      pressure
7====8====9   7====8====9
| 12 | 13 |   |    |    |
4====5====6   4====5====6   NV=13
| 10 | 11 |   |    |    |   
1====2====3   1====2====3   NP=9
\end{verbatim}

I am here following the approach by Lamichhane \cite{lami17}
but I am not sure why he did not use a single element macro-element?
Probably because when applying bc on all four nodes of a single element, 
the left over matrix $\tilde{G}_{el}$ is composed of the last two rows of $\G_{el}$
and this has a nullspace of dimension 2.


After assembly we have $\G$ is a $ndofV*NV\times ndofP*NP=26*9$ matrix:
\[
\G=
\frac{1}{90}
\left(
\begin{array}{ccccccccc}
 42  &  18 &   0 &  23 &   7 &    0 &   0 &   0 &   0 \\ 
 42  &  23 &   0 &  18 &   7 &    0 &   0 &   0 &   0 \\ 
-18  &   0 &  18 &  -7 &   0 &    7 &   0 &   0 &   0 \\ 
 27  &  80 &  23 &   3 &  40 &    7 &   0 &   0 &   0 \\ 
  0  & -18 & -42 &   0 &  -7 &  -23 &   0 &   0 &   0 \\ 
  0  &  27 &  38 &   0 &   3 &   22 &   0 &   0 &   0 \\ 
 27  &   3 &   0 &  80 &  40 &    0 &  23 &   7 &   0 \\ 
-18  &  -7 &   0 &   0 &   0 &    0 &  18 &   7 &   0 \\ 
 -3  &   0 &   3 & -40 &   0 &   40 &  -7 &   0 &   7 \\ 
 -3  & -40 &  -7 &   0 &   0 &    0 &   3 &  40 &   7 \\ 
  0  &  -3 & -27 &   0 & -40 &  -80 &   0 &  -7 & -23 \\ 
  0  &  -3 & -22 &   0 &   0 &    0 &   0 &   3 &  22 \\ 
  0  &   0 &   0 &  27 &   3 &    0 &  38 &  22 &   0 \\ 
  0  &   0 &   0 & -18 &  -7 &    0 & -42 & -23 &   0 \\ 
  0  &   0 &   0 &  -3 &   0 &    3 & -22 &   0 &  22 \\ 
  0  &   0 &   0 &  -3 & -40 &   -7 & -27 & -80 & -23 \\ 
  0  &   0 &   0 &   0 &  -3 &  -27 &   0 & -22 & -38 \\ 
  0  &   0 &   0 &   0 &  -3 &  -22 &   0 & -27 & -38 \\ 
-48  &  48 &   0 & -32 &  32 &    0 &   0 &   0 &   0 \\ 
-48  & -32 &   0 &  48 &  32 &    0 &   0 &   0 &   0 \\ 
  0  & -48 &  48 &   0 & -32 &   32 &   0 &   0 &   0 \\ 
  0  & -48 & -32 &   0 &  48 &   32 &   0 &   0 &   0 \\ 
  0  &   0 &   0 & -48 &  48 &    0 & -32 &  32 &   0 \\ 
  0  &   0 &   0 & -48 & -32 &    0 &  48 &  32 &   0 \\ 
  0  &   0 &   0 &   0 & -48 &   48 &   0 & -32 &  32 \\ 
  0  &   0 &   0 &   0 & -48 &  -32 &   0 &  48 &  32 
\end{array}
\right)
\]

After boundary conditions on Vnodes 1,2,3,4,6,7,8,9, the matrix $G$ looks like:
\[
\G=
\frac{1}{90}
\left(
\begin{array}{ccccccccc}
  0  &   0 &   0 &   0 &   0 &    0 &   0 &   0 &   0 \\ 
  0  &   0 &   0 &   0 &   0 &    0 &   0 &   0 &   0 \\ 
  0  &   0 &   0 &   0 &   0 &    0 &   0 &   0 &   0 \\ 
  0  &   0 &   0 &   0 &   0 &    0 &   0 &   0 &   0 \\ 
  0  &   0 &   0 &   0 &   0 &    0 &   0 &   0 &   0 \\ 
  0  &   0 &   0 &   0 &   0 &    0 &   0 &   0 &   0 \\ 
  0  &   0 &   0 &   0 &   0 &    0 &   0 &   0 &   0 \\ 
  0  &   0 &   0 &   0 &   0 &    0 &   0 &   0 &   0 \\ 
 -3  &   0 &   3 & -40 &   0 &   40 &  -7 &   0 &   7 \\ 
 -3  & -40 &  -7 &   0 &   0 &    0 &   3 &  40 &   7 \\ 
  0  &   0 &   0 &   0 &   0 &    0 &   0 &   0 &   0 \\ 
  0  &   0 &   0 &   0 &   0 &    0 &   0 &   0 &   0 \\ 
  0  &   0 &   0 &   0 &   0 &    0 &   0 &   0 &   0 \\ 
  0  &   0 &   0 &   0 &   0 &    0 &   0 &   0 &   0 \\ 
  0  &   0 &   0 &   0 &   0 &    0 &   0 &   0 &   0 \\ 
  0  &   0 &   0 &   0 &   0 &    0 &   0 &   0 &   0 \\ 
  0  &   0 &   0 &   0 &   0 &    0 &   0 &   0 &   0 \\ 
  0  &   0 &   0 &   0 &   0 &    0 &   0 &   0 &   0 \\ 
-48  &  48 &   0 & -32 &  32 &    0 &   0 &   0 &   0 \\ 
-48  & -32 &   0 &  48 &  32 &    0 &   0 &   0 &   0 \\ 
  0  & -48 &  48 &   0 & -32 &   32 &   0 &   0 &   0 \\ 
  0  & -48 & -32 &   0 &  48 &   32 &   0 &   0 &   0 \\ 
  0  &   0 &   0 & -48 &  48 &    0 & -32 &  32 &   0 \\ 
  0  &   0 &   0 & -48 & -32 &    0 &  48 &  32 &   0 \\ 
  0  &   0 &   0 &   0 & -48 &   48 &   0 & -32 &  32 \\ 
  0  &   0 &   0 &   0 & -48 &  -32 &   0 &  48 &  32 
\end{array}
\right)
\]
or, removing the lines with only zeros, we arrive at a $10\times9$ matrix 
as in Lamichhane \cite{lami17}:
\[
\tilde{\G}
=
\frac{1}{90}
\left(
\begin{array}{ccccccccc}
 -3  &   0 &   3 & -40 &   0 &   40 &  -7 &   0 &   7 \\ 
 -3  & -40 &  -7 &   0 &   0 &    0 &   3 &  40 &   7 \\ 
-48  &  48 &   0 & -32 &  32 &    0 &   0 &   0 &   0 \\ 
-48  & -32 &   0 &  48 &  32 &    0 &   0 &   0 &   0 \\ 
  0  & -48 &  48 &   0 & -32 &   32 &   0 &   0 &   0 \\ 
  0  & -48 & -32 &   0 &  48 &   32 &   0 &   0 &   0 \\ 
  0  &   0 &   0 & -48 &  48 &    0 & -32 &  32 &   0 \\ 
  0  &   0 &   0 & -48 & -32 &    0 &  48 &  32 &   0 \\ 
  0  &   0 &   0 &   0 & -48 &   48 &   0 & -32 &  32 \\ 
  0  &   0 &   0 &   0 & -48 &  -32 &   0 &  48 &  32 
\end{array}
\right)
\] 

This matrix is then passed as argument to the {\sl null\_space} function 
which returns a single vector such that $ker(\tilde{\G})=(1,1,1,1,1,1,1,1,1)$.


It must be said that the matrix above contains similar values 
as the one in \cite{lami17}
as well as the same number of nonzeros. 
Similarities:  16 times $\pm 48/90=\pm 8/15 =2\cdot 4/15$  
and 16 times $\pm 32/90=\pm 16/45 = 2 \cdot 8/45$ as 
in the paper (aside from scaling factor 2). 
However the 12 remaining values differ ? 



\subsubsection{Bubble function 2}
When looking at bubble 2 with $\beta=1/4$, we get 

\[
\G_{el}=
\frac{1}{180}
\left(
\begin{array}{cccc}
 79&  41&   9&  51\\
 79&  51&   9&  41\\
-41& -79& -51&  -9\\
 49&  81&  39&  11\\
-11& -49& -81& -39\\
-11& -39& -81& -49\\
 49&  11&  39&  81\\
-41&  -9& -51& -79\\
-76&  76&  84& -84\\
-76& -84&  84&  76
\end{array}
\right)
\]

After assembly we have $\G$ is a $ndofV*NV\times ndofP*NP=26*9$ matrix:

\[
\G=
\frac{1}{180}
\left(
\begin{array}{ccccccccc}
 79  &  41 &   0 &  51 &   9 &    0 &   0 &   0 &   0 \\ 
 79  &  51 &   0 &  41 &   9 &    0 &   0 &   0 &   0 \\ 
-41  &   0 &  41 &  -9 &   0 &    9 &   0 &   0 &   0 \\ 
 49  & 160 &  51 &  11 &  80 &    9 &   0 &   0 &   0 \\ 
  0  & -41 & -79 &   0 &  -9 &  -51 &   0 &   0 &   0 \\ 
  0  &  49 &  81 &   0 &  11 &   39 &   0 &   0 &   0 \\ 
 49  &  11 &   0 & 160 &  80 &    0 &  51 &   9 &   0 \\ 
-41  &  -9 &   0 &   0 &   0 &    0 &  41 &   9 &   0 \\ 
-11  &   0 &  11 & -80 &   0 &   80 &  -9 &   0 &   9 \\ 
-11  & -80 &  -9 &   0 &   0 &    0 &  11 &  80 &   9 \\ 
  0  & -11 & -49 &   0 & -80 &  -160 &   0 &  -9 & -51 \\ 
  0  & -11 & -39 &   0 &   0 &    0 &   0 &  11 &  39 \\ 
  0  &   0 &   0 &  49 &  11 &    0 &  81 &  39 &   0 \\ 
  0  &   0 &   0 & -41 &  -9 &    0 & -79 & -51 &   0 \\ 
  0  &   0 &   0 & -11 &   0 &   11 & -39 &   0 &  39 \\ 
  0  &   0 &   0 & -11 & -80 &   -9 & -49 & -160 & -51 \\ 
  0  &   0 &   0 &   0 & -11 &  -49 &   0 & -39 & -81 \\ 
  0  &   0 &   0 &   0 & -11 &  -39 &   0 & -49 & -81 \\ 
-76  &  76 &   0 & -84 &  84 &    0 &   0 &   0 &   0 \\ 
-76  & -84 &   0 &  76 &  84 &    0 &   0 &   0 &   0 \\ 
  0  & -76 &  76 &   0 & -84 &   84 &   0 &   0 &   0 \\ 
  0  & -76 & -84 &   0 &  76 &   84 &   0 &   0 &   0 \\ 
  0  &   0 &   0 & -76 &  76 &    0 & -84 &  84 &   0 \\ 
  0  &   0 &   0 & -76 & -84 &    0 &  76 &  84 &   0 \\ 
  0  &   0 &   0 &   0 & -76 &   76 &   0 & -84 &  84 \\ 
  0  &   0 &   0 &   0 & -76 &  -84 &   0 &  76 &  84 
\end{array}
\right)
\]
After boundary conditions are applied:
\[
\G=
\frac{1}{180}
\left(
\begin{array}{ccccccccc}
 0  &   0 &   0 &   0 &   0 &    0 &   0 &   0 &   0 \\ 
  0  &   0 &   0 &   0 &   0 &    0 &   0 &   0 &   0 \\ 
  0  &   0 &   0 &   0 &   0 &    0 &   0 &   0 &   0 \\ 
  0  &   0 &   0 &   0 &   0 &    0 &   0 &   0 &   0 \\ 
  0  &   0 &   0 &   0 &   0 &    0 &   0 &   0 &   0 \\ 
  0  &   0 &   0 &   0 &   0 &    0 &   0 &   0 &   0 \\ 
  0  &   0 &   0 &   0 &   0 &    0 &   0 &   0 &   0 \\ 
  0  &   0 &   0 &   0 &   0 &    0 &   0 &   0 &   0 \\ 
-11  &   0 &  11 & -80 &   0 &   80 &  -9 &   0 &   9 \\ 
-11  & -80 &  -9 &   0 &   0 &    0 &  11 &  80 &   9 \\ 
  0  &   0 &   0 &   0 &   0 &    0 &   0 &   0 &   0 \\ 
  0  &   0 &   0 &   0 &   0 &    0 &   0 &   0 &   0 \\ 
  0  &   0 &   0 &   0 &   0 &    0 &   0 &   0 &   0 \\ 
  0  &   0 &   0 &   0 &   0 &    0 &   0 &   0 &   0 \\ 
  0  &   0 &   0 &   0 &   0 &    0 &   0 &   0 &   0 \\ 
  0  &   0 &   0 &   0 &   0 &    0 &   0 &   0 &   0 \\ 
  0  &   0 &   0 &   0 &   0 &    0 &   0 &   0 &   0 \\ 
  0  &   0 &   0 &   0 &   0 &    0 &   0 &   0 &   0 \\ 
-76  &  76 &   0 & -84 &  84 &    0 &   0 &   0 &   0 \\ 
-76  & -84 &   0 &  76 &  84 &    0 &   0 &   0 &   0 \\ 
  0  & -76 &  76 &   0 & -84 &   84 &   0 &   0 &   0 \\ 
  0  & -76 & -84 &   0 &  76 &   84 &   0 &   0 &   0 \\ 
  0  &   0 &   0 & -76 &  76 &    0 & -84 &  84 &   0 \\ 
  0  &   0 &   0 & -76 & -84 &    0 &  76 &  84 &   0 \\ 
  0  &   0 &   0 &   0 & -76 &   76 &   0 & -84 &  84 \\ 
  0  &   0 &   0 &   0 & -76 &  -84 &   0 &  76 &  84 
\end{array}
\right)
\]
or,
\[
\tilde{\G}
=
\frac{1}{180}
\left(
\begin{array}{ccccccccc}
-11  &   0 &  11 & -80 &   0 &   80 &  -9 &   0 &   9 \\ 
-11  & -80 &  -9 &   0 &   0 &    0 &  11 &  80 &   9 \\ 
-76  &  76 &   0 & -84 &  84 &    0 &   0 &   0 &   0 \\ 
-76  & -84 &   0 &  76 &  84 &    0 &   0 &   0 &   0 \\ 
  0  & -76 &  76 &   0 & -84 &   84 &   0 &   0 &   0 \\ 
  0  & -76 & -84 &   0 &  76 &   84 &   0 &   0 &   0 \\ 
  0  &   0 &   0 & -76 &  76 &    0 & -84 &  84 &   0 \\ 
  0  &   0 &   0 & -76 & -84 &    0 &  76 &  84 &   0 \\ 
  0  &   0 &   0 &   0 & -76 &   76 &   0 & -84 &  84 \\ 
  0  &   0 &   0 &   0 & -76 &  -84 &   0 &  76 &  84 
\end{array}
\right)
\]

We make the same observation as for bubble 1: 
when this matrix is passed as argument to the {\sl null\_space} function, 
it returns a single vector such that $ker(\tilde{\G})=(1,1,1,1,1,1,1,1,1)$.


\paragraph{Special case: $\beta=0$} The bubble is then 
\[
b(r,s)=(1-r^2)(1-s^2)
\]
We repeat the same process and arrive at

\[
\G_{el}=
\frac{1}{18}
\left(
\begin{array}{cccc}
 8&  4&  1&  5\\
 8&  5&  1&  4\\
-4& -8& -5& -1\\
 5&  8&  4&  1\\
-1& -5& -8& -4\\
-1& -4& -8& -5\\
 5&  1&  4&  8\\
-4& -1& -5& -8\\
-8&  8&  8& -8\\
-8& -8&  8&  8
\end{array}
\right)
\]
and 
\[
\tilde{\G}
=
\frac{1}{18}
\left(
\begin{array}{ccccccccc}
 -1  &   0 &   1 &  -8 &   0 &    8 &  -1 &   0 &   1 \\ 
 -1  &  -8 &  -1 &   0 &   0 &    0 &   1 &   8 &   1 \\ 
 -8  &   8 &   0 &  -8 &   8 &    0 &   0 &   0 &   0 \\ 
 -8  &  -8 &   0 &   8 &   8 &    0 &   0 &   0 &   0 \\ 
  0  &  -8 &   8 &   0 &  -8 &    8 &   0 &   0 &   0 \\ 
  0  &  -8 &  -8 &   0 &   8 &    8 &   0 &   0 &   0 \\ 
  0  &   0 &   0 &  -8 &   8 &    0 &  -8 &   8 &   0 \\ 
  0  &   0 &   0 &  -8 &  -8 &    0 &   8 &   8 &   0 \\ 
  0  &   0 &   0 &   0 &  -8 &    8 &   0 &  -8 &   8 \\ 
  0  &   0 &   0 &   0 &  -8 &   -8 &   0 &   8 &   8 \
\end{array}
\right)
\]




Finally the {\sl null\_space} function returns:
\begin{verbatim}
[[ 0.27870965 -0.34974409]
 [ 0.39102578  0.31160686]
 [ 0.27870965 -0.34974409]
 [ 0.39102578  0.31160686]
 [ 0.27870965 -0.34974409]
 [ 0.39102578  0.31160686]
 [ 0.27870965 -0.34974409]
 [ 0.39102578  0.31160686]
 [ 0.27870965 -0.34974409]]
\end{verbatim}
i.e. the null space has dimension 2, so that the element is then not stable. 



See code {\tt python\_codes/Gel/macro\_element\_q1pq1.py}



%-------------------------------------------------------------------
\subsection{$Q_1^+\times Q_1$ element in 3D}



For the quadrilateral MINI element, $m_\upnu=9$ and $m_p=8$ so $\G_{el}$ is
a $27\times 8$ matrix (obtained with $10^3$ quadrature points, no difference
with $6^3$ points).

\[
\G_{el} = -\int_{\Omega_e} {\bm B}^T \cdot {\bm N} d\Omega
= -\int_{\Omega_e}
\left(
\begin{array}{cccccc}
\partial_r N_1^\upnu & 0 & 0 & \partial_s N_1^\upnu  & \partial_t N_1^\upnu & 0 \\
0 & \partial_s N_1^\upnu  & 0 & \partial_r N_1^\upnu  & 0 & \partial_t N_1^\upnu \\
0 & 0 & \partial_t N_1^\upnu  & 0 & \partial_r N_1^\upnu  & \partial_s N_1^\upnu \\
\dots  & \dots & \dots  & \dots & \dots  & \dots \\
\dots  & \dots & \dots  & \dots & \dots  & \dots \\
\dots  & \dots & \dots  & \dots & \dots  & \dots \\
\partial_r N_{27}^\upnu & 0 & 0 & \partial_s N_{27}^\upnu  & \partial_t N_{27}^\upnu & 0 \\
0 & \partial_s N_{27}^\upnu  & 0 & \partial_r N_{27}^\upnu  & 0 & \partial_t N_{27}^\upnu \\
0 & 0 & \partial_t N_{27}^\upnu  & 0 & \partial_r N_{27}^\upnu  & \partial_s N_{27}^\upnu 
\end{array}
\right)
\cdot
\left(
\begin{array}{cccc}
N_1^p & N_2^p & \dots & N_{8}^p \\ 
N_1^p & N_2^p & \dots & N_{8}^p \\ 
N_1^p & N_2^p & \dots & N_{8}^p \\ 
0 & 0 & \dots & 0 \\ 
0 & 0 & \dots & 0 \\ 
0 & 0 & \dots & 0 
\end{array}
\right)
d\Omega
\]


\begin{tiny}

\[
\G_{el}=
-\int_{\Omega_e}
\left(
\begin{array}{cccccccc}
N_1^p\partial_r N_1^\upnu & N_2^p\partial_r N_1^\upnu & N_3^p\partial_r N_1^\upnu & N_4^p\partial_r N_1^\upnu & 
N_5^p\partial_r N_6^\upnu & N_2^p\partial_r N_1^\upnu & N_7^p\partial_r N_1^\upnu & N_8^p\partial_r N_1^\upnu \\ \\ 
N_1^p\partial_s N_1^\upnu & N_2^p\partial_s N_1^\upnu & N_3^p\partial_s N_1^\upnu & N_4^p\partial_s N_1^\upnu & 
N_5^p\partial_s N_6^\upnu & N_2^p\partial_s N_1^\upnu & N_7^p\partial_s N_1^\upnu & N_8^p\partial_s N_1^\upnu \\ \\
N_1^p\partial_t N_1^\upnu & N_2^p\partial_t N_1^\upnu & N_3^p\partial_t N_1^\upnu & N_4^p\partial_t N_1^\upnu &
N_5^p\partial_t N_6^\upnu & N_2^p\partial_t N_1^\upnu & N_7^p\partial_t N_1^\upnu & N_8^p\partial_t N_1^\upnu \\ 
\\
N_1^p\partial_r N_2^\upnu & N_2^p\partial_r N_2^\upnu & N_3^p\partial_r N_2^\upnu & N_4^p\partial_r N_2^\upnu &
N_5^p\partial_r N_2^\upnu & N_6^p\partial_r N_2^\upnu & N_7^p\partial_r N_2^\upnu & N_8^p\partial_r N_2^\upnu \\ \\
N_1^p\partial_s N_2^\upnu & N_2^p\partial_s N_2^\upnu & N_3^p\partial_s N_2^\upnu & N_4^p\partial_s N_2^\upnu &
N_5^p\partial_s N_2^\upnu & N_6^p\partial_s N_2^\upnu & N_7^p\partial_s N_2^\upnu & N_8^p\partial_s N_2^\upnu \\ \\
N_1^p\partial_t N_2^\upnu & N_2^p\partial_t N_2^\upnu & N_3^p\partial_t N_2^\upnu & N_4^p\partial_t N_2^\upnu &
N_5^p\partial_t N_2^\upnu & N_6^p\partial_t N_2^\upnu & N_7^p\partial_t N_2^\upnu & N_8^p\partial_t N_2^\upnu \\ 
\\
N_1^p\partial_r N_3^\upnu & N_2^p\partial_r N_3^\upnu & N_3^p\partial_r N_3^\upnu & N_4^p\partial_r N_3^\upnu &
N_5^p\partial_r N_3^\upnu & N_6^p\partial_r N_3^\upnu & N_7^p\partial_r N_3^\upnu & N_8^p\partial_r N_3^\upnu \\ \\
N_1^p\partial_s N_3^\upnu & N_2^p\partial_s N_3^\upnu & N_3^p\partial_s N_3^\upnu & N_4^p\partial_s N_3^\upnu &
N_5^p\partial_s N_3^\upnu & N_6^p\partial_s N_3^\upnu & N_7^p\partial_s N_3^\upnu & N_8^p\partial_s N_3^\upnu \\ \\
N_1^p\partial_t N_3^\upnu & N_2^p\partial_t N_3^\upnu & N_3^p\partial_t N_3^\upnu & N_4^p\partial_t N_3^\upnu &
N_5^p\partial_t N_3^\upnu & N_6^p\partial_t N_3^\upnu & N_7^p\partial_t N_3^\upnu & N_8^p\partial_t N_3^\upnu \\
\\
N_1^p\partial_r N_4^\upnu & N_2^p\partial_r N_4^\upnu & N_3^p\partial_r N_4^\upnu & N_4^p\partial_r N_4^\upnu &
N_5^p\partial_r N_4^\upnu & N_6^p\partial_r N_4^\upnu & N_7^p\partial_r N_4^\upnu & N_8^p\partial_r N_4^\upnu \\ \\
N_1^p\partial_s N_4^\upnu & N_2^p\partial_s N_4^\upnu & N_3^p\partial_s N_4^\upnu & N_4^p\partial_s N_4^\upnu &
N_5^p\partial_s N_4^\upnu & N_6^p\partial_s N_4^\upnu & N_7^p\partial_s N_4^\upnu & N_8^p\partial_s N_4^\upnu \\ \\
N_1^p\partial_t N_4^\upnu & N_2^p\partial_t N_4^\upnu & N_3^p\partial_t N_4^\upnu & N_4^p\partial_t N_4^\upnu &
N_5^p\partial_t N_4^\upnu & N_6^p\partial_t N_4^\upnu & N_7^p\partial_t N_4^\upnu & N_8^p\partial_t N_4^\upnu \\
\\
N_1^p\partial_r N_5^\upnu & N_2^p\partial_r N_5^\upnu & N_3^p\partial_r N_5^\upnu & N_4^p\partial_r N_5^\upnu \\
N_1^p\partial_s N_5^\upnu & N_2^p\partial_s N_5^\upnu & N_3^p\partial_s N_5^\upnu & N_4^p\partial_s N_5^\upnu \\
N_1^p\partial_t N_5^\upnu & N_2^p\partial_t N_5^\upnu & N_3^p\partial_t N_5^\upnu & N_4^p\partial_t N_5^\upnu \\
N_1^p\partial_r N_6^\upnu & N_2^p\partial_r N_6^\upnu & N_3^p\partial_r N_6^\upnu & N_4^p\partial_r N_6^\upnu \\
N_1^p\partial_s N_6^\upnu & N_2^p\partial_s N_6^\upnu & N_3^p\partial_s N_6^\upnu & N_4^p\partial_s N_6^\upnu \\
N_1^p\partial_t N_6^\upnu & N_2^p\partial_t N_6^\upnu & N_3^p\partial_t N_6^\upnu & N_4^p\partial_t N_6^\upnu \\
N_1^p\partial_r N_7^\upnu & N_2^p\partial_r N_7^\upnu & N_3^p\partial_r N_7^\upnu & N_4^p\partial_r N_7^\upnu \\
N_1^p\partial_s N_7^\upnu & N_2^p\partial_s N_7^\upnu & N_3^p\partial_s N_7^\upnu & N_4^p\partial_s N_7^\upnu \\
N_1^p\partial_t N_7^\upnu & N_2^p\partial_t N_7^\upnu & N_3^p\partial_t N_7^\upnu & N_4^p\partial_t N_7^\upnu \\
N_1^p\partial_r N_8^\upnu & N_2^p\partial_r N_8^\upnu & N_3^p\partial_r N_8^\upnu & N_4^p\partial_r N_8^\upnu \\
N_1^p\partial_s N_8^\upnu & N_2^p\partial_s N_8^\upnu & N_3^p\partial_s N_8^\upnu & N_4^p\partial_s N_8^\upnu \\
N_1^p\partial_t N_8^\upnu & N_2^p\partial_t N_8^\upnu & N_3^p\partial_t N_8^\upnu & N_4^p\partial_t N_8^\upnu \\
N_1^p\partial_r N_9^\upnu & N_2^p\partial_r N_9^\upnu & N_3^p\partial_r N_9^\upnu & N_4^p\partial_r N_9^\upnu \\
N_1^p\partial_s N_9^\upnu & N_2^p\partial_s N_9^\upnu & N_3^p\partial_s N_9^\upnu & N_4^p\partial_s N_9^\upnu \\
N_1^p\partial_t N_9^\upnu & N_2^p\partial_t N_9^\upnu & N_3^p\partial_t N_9^\upnu & N_4^p\partial_t N_9^\upnu 
\end{array}
\right)
d\Omega 
=- \G_{el}^{Q_1\times Q_1} + 
\]

\end{tiny}

\begin{eqnarray}
\int\int\int N_1^p\partial_r N_1^\upnu drdsdt &=&   
\int\int\int N_1\partial_r N_1 drdsdt -\frac{1}{8}  \int\int\int N_1\partial_r b\; drdsdt  \\ 
&=& \frac{8}{27} - \frac{1}{8} \frac{4}{75} = \frac{1}{1350}( 400 - 9 ) =  \frac{372}{1350} \\ 
\int\int\int N_2^p\partial_r N_1^\upnu drdsdt &=&   
\int\int\int N_2\partial_r N_1 drdsdt -\frac{1}{8}  \int\int\int N_2\partial_r b\; drdsdt  \\  
&=& \frac{4}{27} - \frac{1}{8} \\
\int\int\int N_3^p\partial_r N_1^\upnu drdsdt &=&   
\int\int\int N_3\partial_r N_1 drdsdt -\frac{1}{8}  \int\int\int N_3\partial_r b\; drdsdt  \\  
&=& \frac{2}{27} -\frac{1}{8} \\
\int\int\int N_4^p\partial_r N_1^\upnu drdsdt &=&   
\int\int\int N_4\partial_r N_1 drdsdt -\frac{1}{8}  \int\int\int N_4\partial_r b\; drdsdt  \\  
&=& \frac{4}{27} \\
\int\int\int N_5^p\partial_r N_1^\upnu drdsdt &=&   
\int\int\int N_5\partial_r N_1 drdsdt -\frac{1}{8}  \int\int\int N_5\partial_r b\; drdsdt  \\  
&=& \frac{4}{27} \\
\int\int\int N_6^p\partial_r N_1^\upnu drdsdt &=&   
\int\int\int N_6\partial_r N_1 drdsdt -\frac{1}{8}  \int\int\int N_6\partial_r b\; drdsdt  \\  
&=& \frac{2}{27} \\
\int\int\int N_7^p\partial_r N_1^\upnu drdsdt &=&   
\int\int\int N_7\partial_r N_1 drdsdt -\frac{1}{8}  \int\int\int N_7\partial_r b\; drdsdt  \\  
&=& \frac{1}{27} \\
\int\int\int N_8^p\partial_r N_1^\upnu drdsdt &=&   
\int\int\int N_8\partial_r N_1 drdsdt -\frac{1}{8}  \int\int\int N_8\partial_r b\; drdsdt  \\  
&=& \frac{2}{27} \\
\end{eqnarray}




For bubble function \#1:
\[
\G_{el}=
\frac{1}{1350}
\left(
\begin{array}{ccccccccc}
 372&  228&  102&  198&  198&  102&   43&  107\\
 372&  198&  102&  228&  198&  107&   43&  102\\
 372&  198&  107&  198&  228&  102&   43&  102\\
-228& -372& -198& -102& -102& -198& -107&  -43\\
 222&  348&  252&   78&  123&  182&  118&   27\\
 222&  348&  182&  123&   78&  252&  118&   27\\
 -78& -222& -348& -252&  -27& -123& -182& -118\\
 -78& -252& -348& -222&  -27& -118& -182& -123\\
 147&  198&  332&  198&    3&  102&  268&  102\\
 222&   78&  252&  348&  123&   27&  118&  182\\
-228& -102& -198& -372& -102&  -43& -107& -198\\
 222&  123&  182&  348&   78&   27&  118&  252\\
 222&   78&   27&  123&  348&  252&  118&  182\\
 222&  123&   27&   78&  348&  182&  118&  252\\
-228& -102&  -43& -102& -372& -198& -107& -198\\
 -78& -222& -123&  -27& -252& -348& -182& -118\\
 147&  198&  102&    3&  198&  332&  268&  102\\
 -78& -252& -118&  -27& -222& -348& -182& -123\\
  -3& -147& -198& -102& -102& -198& -332& -268\\
  -3& -102& -198& -147& -102& -268& -332& -198\\
  -3& -102& -268& -102& -147& -198& -332& -198\\
 147&    3&  102&  198&  198&  102&  268&  332\\
 -78&  -27& -123& -222& -252& -118& -182& -348\\
 -78&  -27& -118& -252& -222& -123& -182& -348\\
-576&  576&  384& -384& -384&  384&  256& -256\\
-576& -384&  384&  576& -384& -256&  256&  384\\
-576& -384& -256& -384&  576&  384&  256&  384
\end{array}
\right)
\]

Considering a 2x2x2 macroelement of size 4x4x4. 
Then NV=3*3*3+8=35, NP=3*3*3=27
Matrix $\G$ is 3*35x27=105*27

After bc are imposed on all nodes on the boundary, 
9 Vnodes are still free (8 bubble nodes and the node in the 
middle), i.e. 9*3 dofs = 27 and there are
3x3x3=27 pressure nodes. So $\tilde{G}$ is 27*27. no less. 

We get 
\[
\tilde{G} =
\frac{1}{1350}
\left(
\begin{array}{ccccccccccccccccccccccccccc}
jfg ljh hg l
\end{array}
\right)
\]

I have tried all kinds of bubbles but I usually arrive at a
null space of dimension 3 to 5... typically for bubble 1, dim=5, 
while for bubble=2 dim=3. 


%-------------------------------------------------------------------
\subsection{$Q_2 \times Q_1$ element}


\begin{eqnarray}
\G_{el} 
&=& -\int_{\Omega_e} {\bm B}^T \cdot {\bm N} d\Omega \nn\\
&=& -\int_{\Omega_e}
\left(
\begin{array}{ccc}
\partial_x N_1^\upnu & 0 & \partial_y N_1^\upnu \\
0 & \partial_y N_1^\upnu & \partial_x N_1^\upnu \\
\partial_x N_2^\upnu & 0 & \partial_y N_2^\upnu \\
0 & \partial_y N_2^\upnu & \partial_x N_2^\upnu \\
\partial_x N_3^\upnu & 0 & \partial_y N_1^\upnu \\
0 & \partial_y N_3^\upnu & \partial_x N_1^\upnu \\
\partial_x N_4^\upnu & 0 & \partial_y N_2^\upnu \\
0 & \partial_y N_4^\upnu & \partial_x N_2^\upnu \\
\partial_x N_5^\upnu & 0 & \partial_y N_1^\upnu \\
0 & \partial_y N_5^\upnu & \partial_x N_1^\upnu \\
\partial_x N_6^\upnu & 0 & \partial_y N_2^\upnu \\
0 & \partial_y N_6^\upnu & \partial_x N_2^\upnu \\
\partial_x N_7^\upnu & 0 & \partial_y N_1^\upnu \\
0 & \partial_y N_7^\upnu & \partial_x N_1^\upnu \\
\partial_x N_8^\upnu & 0 & \partial_y N_2^\upnu \\
0 & \partial_y N_8^\upnu & \partial_x N_2^\upnu \\
\partial_x N_9^\upnu & 0 & \partial_y N_1^\upnu \\
0 & \partial_y N_9^\upnu & \partial_x N_1^\upnu 
\end{array}
\right)
\cdot
\left(
\begin{array}{cccc}
N_1^p & N_2^p & N_3^p & N_4^p \\ 
N_1^p & N_2^p & N_3^p & N_4^p \\
0 & 0 & \dots & 0
\end{array}
\right)
d\Omega \nn\\
&=&
-\int_{\Omega_e}
\left(
\begin{array}{cccc}
N_1^p\partial_r N_1^\upnu & N_2^p\partial_r N_1^\upnu & N_3^p\partial_r N_1^\upnu & N_4^p\partial_r N_1^\upnu \\
N_1^p\partial_s N_1^\upnu & N_2^p\partial_s N_1^\upnu & N_3^p\partial_s N_1^\upnu & N_4^p\partial_s N_1^\upnu \\
N_1^p\partial_r N_2^\upnu & N_2^p\partial_r N_2^\upnu & N_3^p\partial_r N_2^\upnu & N_4^p\partial_r N_2^\upnu \\
N_1^p\partial_s N_2^\upnu & N_2^p\partial_s N_2^\upnu & N_3^p\partial_s N_2^\upnu & N_4^p\partial_s N_2^\upnu \\
N_1^p\partial_r N_3^\upnu & N_2^p\partial_r N_3^\upnu & N_3^p\partial_r N_3^\upnu & N_4^p\partial_r N_3^\upnu \\
N_1^p\partial_s N_3^\upnu & N_2^p\partial_s N_3^\upnu & N_3^p\partial_s N_3^\upnu & N_4^p\partial_s N_3^\upnu \\
N_1^p\partial_r N_4^\upnu & N_2^p\partial_r N_4^\upnu & N_3^p\partial_r N_4^\upnu & N_4^p\partial_r N_4^\upnu \\
N_1^p\partial_s N_4^\upnu & N_2^p\partial_s N_4^\upnu & N_3^p\partial_s N_4^\upnu & N_4^p\partial_s N_4^\upnu \\
N_1^p\partial_r N_5^\upnu & N_2^p\partial_r N_5^\upnu & N_3^p\partial_r N_5^\upnu & N_4^p\partial_r N_5^\upnu \\
N_1^p\partial_s N_5^\upnu & N_2^p\partial_s N_5^\upnu & N_3^p\partial_s N_5^\upnu & N_4^p\partial_s N_5^\upnu \\
N_1^p\partial_r N_6^\upnu & N_2^p\partial_r N_6^\upnu & N_3^p\partial_r N_6^\upnu & N_4^p\partial_r N_6^\upnu \\
N_1^p\partial_s N_6^\upnu & N_2^p\partial_s N_6^\upnu & N_3^p\partial_s N_6^\upnu & N_4^p\partial_s N_6^\upnu \\
N_1^p\partial_r N_7^\upnu & N_2^p\partial_r N_7^\upnu & N_3^p\partial_r N_7^\upnu & N_4^p\partial_r N_7^\upnu \\
N_1^p\partial_s N_7^\upnu & N_2^p\partial_s N_7^\upnu & N_3^p\partial_s N_7^\upnu & N_4^p\partial_s N_7^\upnu \\
N_1^p\partial_r N_8^\upnu & N_2^p\partial_r N_8^\upnu & N_3^p\partial_r N_8^\upnu & N_4^p\partial_r N_8^\upnu \\
N_1^p\partial_s N_8^\upnu & N_2^p\partial_s N_8^\upnu & N_3^p\partial_s N_8^\upnu & N_4^p\partial_s N_8^\upnu \\
N_1^p\partial_r N_9^\upnu & N_2^p\partial_r N_9^\upnu & N_3^p\partial_r N_9^\upnu & N_4^p\partial_r N_9^\upnu \\
N_1^p\partial_s N_9^\upnu & N_2^p\partial_s N_9^\upnu & N_3^p\partial_s N_9^\upnu & N_4^p\partial_s N_9^\upnu 
\end{array}
\right)
d\Omega 
\end{eqnarray}



\begin{tiny}

\begin{eqnarray}
\int_{\Omega_e}  N_1^p \partial_r N_1^\upnu d\Omega &=&  
\int_{-1}^{+1}\int_{-1}^{+1}\frac{1}{4}(1-r)(1-s) \frac{1}{2}(2r-1) \frac{1}{2}s(s-1) dr ds = -5/18 \\
\int_{\Omega_e}  N_2^p \partial_s N_1^\upnu d\Omega &=&  
\int_{-1}^{+1}\int_{-1}^{+1}\frac{1}{4}(1-r)(1-s) \frac{1}{2}r(r-1) \frac{1}{2}(2s-1) dr ds = -5/18 \\
\int_{\Omega_e}  N_3^p \partial_r N_1^\upnu d\Omega &=&  
\int_{-1}^{+1}\int_{-1}^{+1}\frac{1}{4}(1+r)(1-s) \frac{1}{2}(2r-1) \frac{1}{2}s(s-1) dr ds = -1/18 \\
\int_{\Omega_e}  N_4^p \partial_s N_1^\upnu d\Omega &=&  
\int_{-1}^{+1}\int_{-1}^{+1}\frac{1}{4}(1+r)(1-s) \frac{1}{2}r(r-1) \frac{1}{2}(2s-1) dr ds = 0 \\
\int_{\Omega_e}  N_1^p \partial_r N_1^\upnu d\Omega &=&  
\int_{-1}^{+1}\int_{-1}^{+1}\frac{1}{4}(1+r)(1+s) \frac{1}{2}(2r-1) \frac{1}{2}s(s-1) dr ds = 0 \\
\int_{\Omega_e}  N_2^p \partial_s N_1^\upnu d\Omega &=&  
\int_{-1}^{+1}\int_{-1}^{+1}\frac{1}{4}(1+r)(1+s) \frac{1}{2}r(r-1) \frac{1}{2}(2s-1) dr ds = 0 \\
\int_{\Omega_e}  N_3^p \partial_r N_1^\upnu d\Omega &=&  
\int_{-1}^{+1}\int_{-1}^{+1}\frac{1}{4}(1-r)(1+s) \frac{1}{2}(2r-1) \frac{1}{2}s(s-1) dr ds = 0 \\
\int_{\Omega_e}  N_4^p \partial_s N_1^\upnu d\Omega &=&  
\int_{-1}^{+1}\int_{-1}^{+1}\frac{1}{4}(1-r)(1+s) \frac{1}{2}r(r-1) \frac{1}{2}(2s-1) dr ds = -1/18
\end{eqnarray}

... same procedure for 1,2,3,4,5,6,7...

\begin{eqnarray}
\int_{\Omega_e}  N_1^p \partial_r N_9^\upnu d\Omega &=&  
\int_{-1}^{+1}\int_{-1}^{+1}\frac{1}{4}(1-r)(1-s)    (-2r)(1-s^2)  drds = 4/9 \\
\int_{\Omega_e}  N_2^p \partial_s N_9^\upnu d\Omega &=&  
\int_{-1}^{+1}\int_{-1}^{+1}\frac{1}{4}(1-r)(1-s)    (1-r^2)(-2s)  drds = 4/9 \\
\int_{\Omega_e}  N_3^p \partial_r N_9^\upnu d\Omega &=&  
\int_{-1}^{+1}\int_{-1}^{+1}\frac{1}{4}(1+r)(1-s)    (-2r)(1-s^2)  drds = -4/9 \\
\int_{\Omega_e}  N_4^p \partial_s N_9^\upnu d\Omega &=&  
\int_{-1}^{+1}\int_{-1}^{+1}\frac{1}{4}(1+r)(1-s)    (1-r^2)(-2s)  drds = 4/9 \\
\int_{\Omega_e}  N_1^p \partial_r N_9^\upnu d\Omega &=&  
\int_{-1}^{+1}\int_{-1}^{+1}\frac{1}{4}(1+r)(1+s)    (-2r)(1-s^2)  drds = -4/9 \\
\int_{\Omega_e}  N_2^p \partial_s N_9^\upnu d\Omega &=&  
\int_{-1}^{+1}\int_{-1}^{+1}\frac{1}{4}(1+r)(1+s)    (1-r^2)(-2s)  drds = -4/9  \\
\int_{\Omega_e}  N_3^p \partial_r N_9^\upnu d\Omega &=&  
\int_{-1}^{+1}\int_{-1}^{+1}\frac{1}{4}(1-r)(1+s)    (-2r)(1-s^2)  drds = 4/9 \\
\int_{\Omega_e}  N_4^p \partial_s N_9^\upnu d\Omega &=&  
\int_{-1}^{+1}\int_{-1}^{+1}\frac{1}{4}(1-r)(1+s)    (1-r^2)(-2s)  drds = -4/9 
\end{eqnarray}
\end{tiny}


We obtain 
\[
-\int_{\Omega_e}
\left(
\begin{array}{cccc}
N_1^p\partial_r N_9^\upnu & N_2^p\partial_r N_9^\upnu & N_3^p\partial_r N_9^\upnu & N_4^p\partial_r N_9^\upnu \\
N_1^p\partial_s N_9^\upnu & N_2^p\partial_s N_9^\upnu & N_3^p\partial_s N_9^\upnu & N_4^p\partial_s N_9^\upnu 
\end{array}
\right)
d\Omega 
=
- \frac{4}{9}
\left(
\begin{array}{cccc}
1 &-1 &-1 &1 \\
1 &1 &-1 &-1 
\end{array}
\right)
\]
which is identical to Eq.(3.53) of Elman \etal \cite{elsw}.


I have implemented a 3x3 quadrature integration to numerically compute the matrix 
in the file {\tt python\_codes/Gel/programQ2Q1.py}:
 
\begin{verbatim}
[[ 2.77777778e-01  5.55555556e-02  1.73472348e-18  3.68628739e-18]
 [ 2.77777778e-01 -1.29020059e-17 -3.46944695e-18  5.55555556e-02]
 [-5.55555556e-02 -2.77777778e-01 -2.77555756e-17 -3.03576608e-18]
 [-2.19008839e-17  2.77777778e-01  5.55555556e-02 -4.33680869e-18]
 [-3.46944695e-18 -1.38777878e-17 -2.77777778e-01 -5.55555556e-02]
 [ 2.60208521e-18 -5.55555556e-02 -2.77777778e-01  6.93889390e-18]
 [ 4.01154804e-18  4.33680869e-18  5.55555556e-02  2.77777778e-01]
 [-5.55555556e-02  6.07153217e-18  2.08166817e-17 -2.77777778e-01]
 [-2.22222222e-01  2.22222222e-01  0.00000000e+00 -8.67361738e-19]
 [ 5.55555556e-01  5.55555556e-01  1.11111111e-01  1.11111111e-01]
 [-1.11111111e-01 -5.55555556e-01 -5.55555556e-01 -1.11111111e-01]
 [-8.67361738e-19 -2.22222222e-01  2.22222222e-01  6.93889390e-18]
 [-8.67361738e-18 -6.93889390e-18  2.22222222e-01 -2.22222222e-01]
 [-1.11111111e-01 -1.11111111e-01 -5.55555556e-01 -5.55555556e-01]
 [ 5.55555556e-01  1.11111111e-01  1.11111111e-01  5.55555556e-01]
 [-2.22222222e-01 -5.20417043e-18  6.93889390e-18  2.22222222e-01]
 [-4.44444444e-01  4.44444444e-01  4.44444444e-01 -4.44444444e-01]
 [-4.44444444e-01 -4.44444444e-01  4.44444444e-01  4.44444444e-01]]
\end{verbatim}


or,
\[
\G_{el} 
=\frac{1}{18}
\left(
\begin{array}{cccc}
5 &1 &0 &0 \\
5 &0 &0 &1 \\
-1 &-5 &0 &0\\ 
0 &5 &1 &0\\
0 &0 &-5 &-1 \\
0 &-1 &-5 &0 \\
0 &0& 1& 5\\
-1 &0 &0& -5\\
-4 &4 &0 &0\\
10 &10& 2 &2\\
-2 &-10 &-10 &-2\\
0 &-4 &4& 0\\
0 &0 &4 &-4\\
-2 &-2& -10 &-10\\
10 &2 &2 &10\\
-4 &0 &0 &4\\
-8 &8 &8 &-8 \\
-8 &-8 &8 &8
\end{array}
\right)
\]

Reading Elman \cite{elsw} we see that this element is stable but patches of 
even and odd elements actually are needed to establish the stability of the element. 










