\begin{flushright} {\tiny {\color{gray} cvi.tex}} \end{flushright}

WORK IN PROGRESS !!!


\Literature: Pusok \etal (2016) \cite{pukp16}, McNally (2011) \cite{mcna11}, Wang \etal (2015) \cite{waav15}

%-------------------------------------------------------------
\subsubsection{In 2D with $Q_1$ basis functions - Naive approach}

Let us start directly in reduced coordinates $(r,s)\in [-1:1]^2$:
\[
u(r,s)=\sum_i N_i(r,s) u_i
\quad
\quad
v(r,s)=\sum_i N_i(r,s) v_i
\]
with 
\begin{eqnarray}
N_1&=& \frac{1}{4}(1-r)(1-s)  \nonumber\\ 
N_2&=& \frac{1}{4}(1+r)(1-s)  \nonumber\\ 
N_3&=& \frac{1}{4}(1+r)(1+s)  \nonumber\\ 
N_4&=& \frac{1}{4}(1-r)(1+s)  \nonumber
\end{eqnarray}
The incompressibility constraint imposes:
\[
\frac{\partial u}{\partial r}+
\frac{\partial v}{\partial s}=0
\]
i.e.
\[
\sum_i \left(  
\frac{\partial N_i}{\partial r} u_i+
\frac{\partial N_i}{\partial s} v_i
\right)
=0
\]
However, it is trivial to verify that the incompressibility 
condition is not and can not be verified for all values of  
$r,s \in [-1,1]^2$.
It would then make sense to think of a corrective term to the interpolation
which would add just enough degrees of freedoms so as to insure an exact
incompressibility in the element. 
Let us then write:
\begin{eqnarray}
u(r,s)&=&\sum_i N_i(r,s) u_i + (a s + b)(1-r)(1+r) \nn\\
v(r,s)&=&\sum_i N_i(r,s) v_i + (c r + d)(1-s)(1+s) \nn
\end{eqnarray}
In this case,
\begin{eqnarray}
\frac{\partial u}{\partial r}&=&\sum_i \frac{\partial N_i}{\partial r} u_i + (a s + b) (-2r) \nn\\
\frac{\partial v}{\partial s}&=&\sum_i \frac{\partial N_i}{\partial s} v_i + (c r + d)(-2s) \nn
\end{eqnarray}
We have introduced 4 coefficients  $(a,b,c,d)$ which remain to be determined. 
We start with:
\begin{eqnarray}
\sum_i \frac{\partial N_i}{\partial r} u_i 
&=& -\frac{1}{4} (1-s) u_1 + \frac{1}{4} (1-s) u_2 +\frac{1}{4} (1+s) u_3 -\frac{1}{4} (1+s) u_4 \nn\\
&=& (1-s) \frac{u_2-u_1}{4} + (1+s) \frac{u_3-u_4}{4} \nn\\
&=& (1-s) u_{21} + (1+s) u_{34} \nn\\
\nn\\
\sum_i \frac{\partial N_i}{\partial s} v_i 
&=& -\frac{1}{4} (1-r) v_1 - \frac{1}{4} (1+r) v_2 +\frac{1}{4} (1+r) v_3 +\frac{1}{4} (1-r) v_4 \nn\\
&=& (1-r) \frac{v_4-v_1}{4} + (1+r)\frac{v_3-v_2}{4} \nn\\
&=& (1-r) v_{41} + (1+r) v_{32} \nn
\end{eqnarray}
where $u_{ij}=(u_i-u_j)/4$ and $v_{ij}=(v_i-v_j)/4$, so that
\[
\frac{\partial u}{\partial r}=
(1-s) u_{21} + (1+s) u_{34} 
+ (a s + b) (-2r)
\]

\[
\frac{\partial v}{\partial s}=
(1-r) v_{41} + (1+r) v_{32}
+ (c r + d)(-2s)
\]


The incompressibility condition is now:
\[
(1-s) u_{21} + (1+s) u_{34} 
+ (a s + b) (-2r) +
(1-r) v_{41} + (1+r) v_{32}
+ (c r + d)(-2s)
=0
\]

This can be rewritten as
\[
C_0  + C_1 r + C_2 s + C_3 rs = 0
\]
where the four $C_i$ coefficients are functions of the velocities and the other coefficients.
In order for this expression to be exactly zero {\it everywhere}, each $C$ coefficient has
to be independently zero.

\begin{eqnarray}
C_0   &(.)  &  u_{21} + u_{34} + v_{41} + v_{32} =0\nn\\ 
C_1   &(r)  &  -v_{41} + v_{32} -2b =0\nn\\ 
C_2   &(s)  &  -u_{21} + u_{34} -2d =0 \nn\\ 
C_3   &(rs) &  -2a -2c =0\nn 
\end{eqnarray}

The first line is simply the incompressibility condition
expressed in the center of the element (i.e. $r=s=0$),
so we can neglect it and focus on the remaining three.
We obtain
\[
c=-a
\quad
b=\frac{1}{2}(-v_{41} + v_{32})
\quad
d=\frac{1}{2} (-u_{21} + u_{34})
\]
Since $a$ and $c$ and not otherwise constrained, we can set them to zero, and we then have:
\[
b=\frac{1}{2}(v_{14} + v_{32})
\quad\quad
d=\frac{1}{2} (u_{12} + u_{34})
\]
and finally
\begin{eqnarray}
u(r,s)
&=&\sum_i N_i(r,s) u_i + b(1-r)(1+r) \nn\\
&=&\sum_i N_i(r,s) u_i + \frac{1}{2}(v_{14} + v_{32})(1-r)(1+r) \nn\\
v(r,s)
&=&\sum_i N_i(r,s) v_i + d(1-s)(1+s) \nn\\
&=&\sum_i N_i(r,s) v_i + \frac{1}{2} (u_{12} + u_{34})(1-s)(1+s) \nn
\end{eqnarray}

\[
\boxed{
u(r,s)=\sum_i N_i(r,s) u_i + \frac{1}{2}(v_{14} + v_{32})(1-r)(1+r) 
}
\]
\[
\boxed{
v(r,s)
=\sum_i N_i(r,s) v_i + \frac{1}{2} (u_{12} + u_{34})(1-s)(1+s) 
}
\]


%-----------------------------------------------------------------
\subsubsection{In 3D with $Q_1$ basis functions - Naive approach}

Let us start directly in reduced coordinates $(r,s,t)\in [-1:1]^3$:
\[
u^h(r,s,t)=\sum_i N_i(r,s,t) u_i
\quad
\quad
v^h(r,s,t)=\sum_i N_i(r,s,t) v_i
\quad
\quad
w^h(r,s,t)=\sum_i N_i(r,s,t) w_i
\]
with
\begin{eqnarray}
N_1&=&0.125(1-r)(1-s)(1-t) \nonumber\\ 
N_2&=&0.125(1+r)(1-s)(1-t)  \nonumber\\ 
N_3&=&0.125(1+r)(1+s)(1-t)  \nonumber\\ 
N_4&=&0.125(1-r)(1+s)(1-t)  \nonumber\\ 
N_5&=&0.125(1-r)(1-s)(1+t)  \nonumber\\ 
N_6&=&0.125(1+r)(1-s)(1+t)  \nonumber\\ 
N_7&=&0.125(1+r)(1+s)(1+t)  \nonumber\\ 
N_8&=&0.125(1-r)(1+s)(1+t)  \nn
\end{eqnarray}
The incompressibility constraint imposes:
\[
\frac{\partial u^h}{\partial r}+
\frac{\partial v^h}{\partial s}+
\frac{\partial w^h}{\partial t}=0
\]
i.e.
\[
\sum_i \left(  
\frac{\partial N_i}{\partial r} u_i+
\frac{\partial N_i}{\partial s} v_i+
\frac{\partial N_i}{\partial t} w_i
\right)
=0
\]
However, once again it is trivial to verify that the incompressibility
condition is not and can not be verified for all values of
$r,s,t \in [-1,1]^3$.


It would then make sense to think of a corrective term to the interpolation
which would add just enough degrees of freedoms so as to insure an exact
incompressibility in the element.
Let us then write:
\begin{eqnarray}
u(r,s,t)&=&\sum_i N_i(r,s,t) u_i + (a s + b t +c)(1-r)(1+r) \nn\\
v(r,s,t)&=&\sum_i N_i(r,s,t) v_i + (d r + e t +f)(1-s)(1+s) \nn\\
w(r,s,t)&=&\sum_i N_i(r,s,t) w_i + (g r + h s +i)(1-t)(1+t) \nn
\end{eqnarray}
In this case,
\begin{eqnarray}
\frac{\partial u}{\partial r}&=&\sum_i \frac{\partial N_i}{\partial r} u_i + (a s + b t +c)(-2r)\nn\\
\frac{\partial v}{\partial s}&=&\sum_i \frac{\partial N_i}{\partial s} v_i + (d r + e t +f)(-2s)\nn\\
\frac{\partial w}{\partial t}&=&\sum_i \frac{\partial N_i}{\partial t} w_i + (g r + h s +i)(-2t)\nn
\end{eqnarray}
We have introduced 9 coefficients  $(a,b,c,d,e,f,g,h,i)$ which remain to be determined.
The incompressibility condition is now:
\[
\sum_i \left(  
\frac{\partial N_i}{\partial r} u_i+
\frac{\partial N_i}{\partial s} v_i+
\frac{\partial N_i}{\partial t} w_i
\right)
+ (a s + b t +c) (-2r) + (d r + e t +f)(-2s) + (g r + h s +i)(-2t) 
=0
\]
This can be rewritten as
\[
C_0  + C_1 r + C_2 s + C_3 t + C_4 rs + C_5 st + C_6 rt = 0
\]
where the seven $C_i$ coefficients are functions of the velocities and the other coefficients.
In order for this expression to be exactly zero {\it everywhere}, each $C$ coefficient has
to be independently zero.

We start with:
\begin{eqnarray}
\sum_i 8\frac{\partial N_i}{\partial r} u_i 
&=& (1-s)(1-t)(u_2-u_1)
+ (1+s)(1-t)(u_3-u_4)
+ (1-s)(1+t)(u_6-u_5)
+ (1+s)(1+t)(u_7-u_8) \nn\\
\sum_i 8\frac{\partial N_i}{\partial s} v_i 
&=& (1-r)(1-t)(v_4-v_1)
+ (1+r)(1-t)(v_3-v_2)
+ (1-r)(1+t)(v_8-v_5)
+ (1+r)(1+t)(v_7-v_6) \nn\\
\sum_i 8\frac{\partial N_i}{\partial t} w_i 
&=& (1-r)(1-s)(w_5-w_1)
+ (1+r)(1-s)(w_6-w_2)
+ (1+r)(1+s)(w_7-w_3)
+ (1-r)(1+s)(w_8-w_4) \nn
\end{eqnarray}

Let us denote $u_{ij}=(u_i-v_j)/8$ (same for $v$, $w$), so that:
\begin{eqnarray}
\sum_i \frac{\partial N_i}{\partial r} u_i 
&=& (1-s)(1-t)u_{21}
+ (1+s)(1-t)u_{34}
+ (1-s)(1+t)u_{65}
+ (1+s)(1+t)u_{78} \nn\\
\sum_i \frac{\partial N_i}{\partial s} v_i 
&=& (1-r)(1-t)v_{41}
+ (1+r)(1-t)v_{32}
+ (1-r)(1+t)v_{85}
+ (1+r)(1+t)v_{76} \nn\\
\sum_i \frac{\partial N_i}{\partial t} w_i 
&=& 
  (1-r)(1-s)w_{51}
+ (1+r)(1-s)w_{62}
+ (1+r)(1+s)w_{73}
+ (1-r)(1+s)w_{84} \nn
\end{eqnarray}
We finally arrive at:
\begin{eqnarray}
C_0   &(.)  &  u_{21} + u_{34} + u_{65} + u_{78} + v_{41} + v_{32} + v_{85} + v_{76} + w_{51} + w_{62} + w_{73} + w_{84} =0  \nn\\
C_1   &(r)  &  -v_{41} +v_{32} -v_{85} + v_{76} - w_{51} + w_{62} + w_{73} -w_{84} -2c =0\nn\\ 
C_2   &(s)  &  -u_{21} +u_{34} -u_{65} + u_{78} - w_{51} - w_{62} + w_{73} +w_{84} -2f =0 \nn\\ 
C_3   &(t)  &  -u_{21} -u_{34} +u_{65} + u_{78} - v_{41} - v_{32} + v_{85} +v_{76} -2i =0 \nn\\ 
C_4   &(rs) &  w_{51} -w_{62} +w_{73} - w_{84}  -2a -2d =0  \nn\\
C_5   &(st) &  u_{21} -u_{34} -u_{65} + u_{78}  -2e -2h =0  \nn\\
C_6   &(rt) &  v_{41} -v_{32} -v_{85} + v_{76}  -2b -2g =0  \nn
\end{eqnarray}

I unfortunately end up with 6 equations and 9 unknowns $a,b,c,d,e,f,g,h$.
Coming up with additional constraints is not trivial, so I will instead further assume 
$\alpha_r=b=a$, $\alpha_s=e=d$ and $\alpha_t=h=g$, and rename 
$\beta_r=c$, $\beta_s=f$ and $\beta_t=i$ so that
I have now six unknowns $\alpha_r,\alpha_s,\alpha_t,\beta_r,\beta_s,\beta_t$ for six equations
\begin{eqnarray}
C_1   &(r)  &  -v_{41} +v_{32} -v_{85} + v_{76} - w_{51} + w_{62} + w_{73} -w_{84} -2\beta_r \nn\\ 
C_2   &(s)  &  -u_{21} +u_{34} -u_{65} + u_{78} - w_{51} - w_{62} + w_{73} +w_{84} -2\beta_s \nn\\ 
C_3   &(t)  &  -u_{21} -u_{34} +u_{65} + u_{78} - v_{41} - v_{32} + v_{85} +v_{76} -2\beta_t \nn\\ 
C_4   &(rs) &  w_{51} -w_{62} +w_{73} - w_{84}  -2\alpha_r -2\alpha_s   \nn\\
C_5   &(st) &  u_{21} -u_{34} -u_{65} + u_{78}  -2\alpha_s -2\alpha_t   \nn\\
C_6   &(rt) &  v_{41} -v_{32} -v_{85} + v_{76}  -2\alpha_r -2\alpha_t   \nn
\end{eqnarray}


This naturally yields:
\begin{eqnarray}
\beta_r
&=& \frac{1}{2} ( -v_{41} +v_{32} -v_{85} + v_{76} - w_{51} + w_{62} + w_{73} -w_{84}  ) \nn\\
&=& \frac{1}{16} (v_1-v_2+v_3-v_4+v_5-v_6+v_7-v_8  +w_1-w_2 - w_3 + w_4 - w_5 + w_6 +w_7  - w_8    )  \nn\\
\beta_s&=& \frac{1}{2} ( -u_{21} +u_{34} -u_{65} + u_{78} - w_{51} - w_{62} + w_{73} +w_{84}  ) \nn\\
&=& \frac{1}{16} (u_1-u_2+u_3-u_4+u_5-u_6+u_7-u_8  +w_1 + w_2 - w_3 - w_4 - w_5 - w_6 +w_7 + w_8   )  \nn\\
\beta_t&=& \frac{1}{2} ( -u_{21} -u_{34} +u_{65} + u_{78} - v_{41} - v_{32} + v_{85} +v_{76}   ) \nn\\
&=& \frac{1}{16} ( u_1-u_2-u_3+u_4 -u_5 + u_6 + u_7 - u_8 +v_1 +v_2 - v_3 - v_4 - v_5 - v_6 + v_7 + v_8  )  \nn
\end{eqnarray}
and we need to solve
\begin{eqnarray}
\tilde{w} -2\alpha_r -2\alpha_s&=&0\nn\\
\tilde{u} -2\alpha_s -2\alpha_t&=&0\nn\\
\tilde{v} -2\alpha_r -2\alpha_t&=&0\nn
\end{eqnarray}
where
\begin{eqnarray}
\tilde{u} 
&=& u_{21} -u_{34} -u_{65} + u_{78} 
=\frac{1}{8}(-u_1 + u_2-u_3+u_4 + u_5-u_6 + u_7-u_8  )
\nn\\
\tilde{v} 
&=& v_{41} -v_{32} -v_{85} + v_{76}
= \frac{1}{8} (-v_1 + v_2 - v_3 + v_4 + v_5 - v_6 + v_7 - v_8    )
  \nn\\ 
\tilde{w} 
&=&  w_{51} -w_{62} +w_{73} - w_{84} 
=\frac{1}{8} (-w_1+w_2-w_3+w_4 + w_5 - w_6 + w_7 -w_8  )
\nn
\end{eqnarray}
which yields:
\[
\alpha_r=\frac{1}{4} ( -\tilde{u} + \tilde{v} + \tilde{w} ) 
\quad\quad
\alpha_s=\frac{1}{4} ( \tilde{u} - \tilde{v} + \tilde{w} ) 
\quad\quad
\alpha_t=\frac{1}{4} ( \tilde{u} + \tilde{v} - \tilde{w} ) 
\]

So finally:

\begin{eqnarray}
u(r,s,t)&=&\sum_i N_i(r,s,t) u_i + [\alpha_r (s+t) +\beta_r](1-r)(1+r) \nn\\
v(r,s,t)&=&\sum_i N_i(r,s,t) v_i + [\alpha_s (r+t) +\beta_s](1-s)(1+s) \nn\\
w(r,s,t)&=&\sum_i N_i(r,s,t) w_i + [\alpha_t (r+s) +\beta_t](1-t)(1+t) \nn
\end{eqnarray}


%------------------------------------------------------------------------
\subsubsection{In 2D with $P_1$ basis functions - what about triangles?}


The reference linear element is: 
\begin{verbatim}
s
|
3
|\
|  \
|    \
1-----2 ->r
\end{verbatim}

Shape functions are 
\begin{eqnarray}
N_1(r,s) &=& 1-r-s \nn\\
N_2(r,s) &=& r \nn\\
N_3(r,s) &=& s 
\end{eqnarray}

Velocity vector is $\vec\upnu=(u,v)$. 
\[
u_h(r,s)=\sum_i N_i(r,s) u_i
\qquad
v_h(r,s)=\sum_i N_i(r,s) v_i
\]
In the element
\[
\vec\nabla\cdot\vec\upnu_h = 
\frac{\partial u_h}{\partial r}
+
\frac{\partial v_h}{\partial s}
=(-u_1+u_2)+(-v_1+v_3)
\]
which is evidently not zero everywhere in the element.

The CVI approach consists in adding polynomial terms to the expressions of $u_h$ and $v_h$.

\paragraph{approach 1}
In what follows I assume that the additional terms are of the form (where $r$, $s$ and $1-r-s$ are the three basis functions, and we here use only two per line, similarly to the quadrilateral counterpart):

\begin{eqnarray}
u_h(r,s)&=&\sum_i N_i(r,s) u_i + f(r,s) r(1-r-s) \\
v_h(r,s)&=&\sum_i N_i(r,s) v_i + g(r,s) s(1-r-s) 
\end{eqnarray}
The velocity divergence requirement is then
\begin{eqnarray}
0=\vec\nabla\cdot\vec\upnu_h 
&=& 
  -u_1+u_2 + \partial_r f r(1-r-s) + f(r,s)(1-2r-s) \\
&&-v_1+v_3 + \partial_s g s(1-r-s) + g(r,s)(1-2s-r)
\end{eqnarray}

\begin{itemize}
\item
We start simple and postulate $f(r,s)=a$, $g(r,s)=b$, so then 
\begin{eqnarray}
0=\vec\nabla\cdot\vec\upnu_h 
&=&   -u_1+u_2 +  a(1-2r-s) -v_1+v_3 +  b(1-2s-r) \\
&=&  (-u_1+u_2-v_1+v_3 +a +b ) + (-2a-b)r + (-a-2b)s
\end{eqnarray}
It is impossible to find $a$ and $b$ such that this expression is zero everywhere inside the element.

\item
We postulate then $f(r,s)=a+br+cs$, $g(r,s)=d+er+fs$, so  
\begin{eqnarray}
0=\vec\nabla\cdot\vec\upnu_h 
&=& -u_1+u_2 + \partial_r f r(1-r-s) + f(r,s)(1-2r-s) \nn\\
&&  -v_1+v_3 + \partial_s g s(1-r-s) + g(r,s)(1-2s-r) \nn\\
&=& -u_1+u_2 + b r(1-r-s) + (a+br+cs) (1-2r-s) \nn\\
&&  -v_1+v_3 + f s(1-r-s) + (d+er+fs)(1-2s-r) \nn\\
&=& -u_1+u_2  -v_1+v_3 + a + d \nn\\
&& +(b-2a+b-d+e)r \nn\\
&& +(f-a+c-2d+f)s \nn\\
&& +(-b-2b-e)r^2 \nn\\
&& +(-f-c-2f)s^2 \nn\\
&& +(-b-f-b-2c-2e-f)rs \nn\\
&=& -u_1+u_2  -v_1+v_3 + a + d \nn\\
&& +(2b-2a-d+e)r \nn\\
&& +(2f-a+c-2d)s \nn\\
&& +(-3b-e)r^2 \nn\\
&& +(-3f-c)s^2 \nn\\
&& +(-2b-2f-2c-2e)rs  \nn
\end{eqnarray}
Immediately $e=-3b$ and $c=-3f$. Inserting these in the last line yields
$-2b-2f-2c-2e=-2b-2f+6f+6b=4b+4f=0$, i.e. $b=-f$.
Inserting these in the remaining lines:
\begin{eqnarray}
a+d &=& u_1-u_2  +v_1-v_3 \nn\\
2b-2a-d+(-3b) &=& 0 \nn\\
2(-b)-a+(3b)-2d &=& 0 \nn
\end{eqnarray}
or,
\begin{eqnarray}
a+d &=& u_1-u_2  +v_1-v_3 \nn\\
-2a-b-d &=& 0 \nn\\
-a + b-2d &=& 0 \nn
\end{eqnarray}
or, 
\[
\left(
\begin{array}{ccc}
1 &0 & 1 \\
-2 & -1 & -1 \\
-1 & 1 & -2 
\end{array}
\right)
\cdot
\left(
\begin{array}{c}
a \\ b  \\d 
\end{array}
\right)
=
\left(
\begin{array}{c}
u_1-u_2+v_1-v_3 \\
0 \\ 0 
\end{array}
\right)
\]
Determinant= 3 -2 -1 = 0. Matrix is singular ... !! Moving on...

\item We now postulate $f(r,s)=a+br+cs+hrs$, $g(r,s)=d+er+fs+krs$, so then 


\begin{eqnarray}
0=\vec\nabla\cdot\vec\upnu_h 
&=& -u_1+u_2 + \partial_r f r(1-r-s) + f(r,s)(1-2r-s) \nn\\
&&  -v_1+v_3 + \partial_s g s(1-r-s) + g(r,s)(1-2s-r) \nn\\
&=& -u_1+u_2 + (b+hs) r(1-r-s) + (a+br+cs+hrs) (1-2r-s) \nn\\
&&  -v_1+v_3 + (f+kr) s(1-r-s) + (d+er+fs+krs)(1-2s-r) \nn\\
&=& -u_1+u_2  -v_1+v_3 + a + d \nn\\
&& +(b-2a+b-d+e)r \nn\\
&& +(f-a+c-2d+f)s \nn\\
&& +(-b-2b-e)r^2 \nn\\
&& +(-f-c-2f)s^2 \nn\\
&& +(-b-f-b-2c-2e-f+2h+2k)rs \nn\\
&& +(-h-k-2k-h)rs^2 \nn\\
&& +(-h-k-2h-k)r^2s \nn\\
&=& -u_1+u_2  -v_1+v_3 + a + d \nn\\
&& +(b-2a+b-d+e)r \nn\\
&& +(f-a+c-2d+f)s \nn\\
&& +(-3b-e)r^2 \nn\\
&& +(-3f-c)s^2 \nn\\
&& +(-2b-2f-2c-2e+2h+2k)rs \nn\\
&& +(-2h-3k)rs^2 \nn\\
&& +(-3h-2k)r^2s 
\end{eqnarray}
Immediately we see that the last 2 lines yield $k=h=0$. Back to the drawing board...

I {\it could} keep adding high order terms but I suspect it is a doomed effort and even if it would work, the cost would be prohibitive.

\end{itemize}


\paragraph{approach 2} This time I include all three basis functions $r$ , $s$ and $1-r-s$, not just two. Then

\begin{eqnarray}
u_h(r,s)&=&\sum_i N_i(r,s) u_i + f(r,s) rs(1-r-s) \\
v_h(r,s)&=&\sum_i N_i(r,s) v_i + g(r,s) rs(1-r-s) 
\end{eqnarray}

\begin{eqnarray}
0=\vec\nabla\cdot\vec\upnu_h 
&=& 
  -u_1+u_2 + \partial_r f \; rs(1-r-s) + f(r,s)s(1-2r-s) \\
&&-v_1+v_3 + \partial_s g \; rs(1-r-s) + g(r,s)r(1-2s-r)
\end{eqnarray}


We postulate $f(r,s)=a$, $g(r,s)=b$, so then 
\begin{eqnarray}
0=\vec\nabla\cdot\vec\upnu_h 
&=&   -u_1+u_2 +  as(1-2r-s) -v_1+v_3 +  br(1-2s-r) \\
&=&  (-u_1+u_2 -v_1+v_3) + ...
\end{eqnarray}
it is also a dead end!

And this will not change with high order terms in $f$ and $g$. Because of the presence of all three 
basis functions in the additional terms we see that no 
coefficient will enter the parenthesis above and therefore it is doomed. 














