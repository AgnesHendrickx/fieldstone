

The system is a layer of fluid between $y=0$ and $y=h$, with boundary conditions $T(x,y=0)=T_b$ 
and $T(x,y=h)=0$, characterized by $\rho_0$, $C_p$, $k$, $\eta_0$. 

The Rayleigh number of the system is 
\[
\text{Ra}= \frac{\rho_0 g_0 \alpha \Delta T h^3}{\eta_0 \kappa}
\]

The Stokes equation is $\vec \nabla \cdot \bm \sigma + \vec b = \vec 0$ with $\vec b=\rho \vec g$. 
The components of the this equation on the $x$- and $y-$axis are:
\begin{eqnarray}
(\vec \nabla \cdot \bm \sigma)_x &=& - \rho \vec g \cdot \vec e_x = 0\\ 
(\vec \nabla \cdot \bm \sigma)_y &=& - \rho \vec g \cdot \vec e_y = \rho g_0
\end{eqnarray}
since $\vec g$ and $\vec e_y$ are in opposite directions ($\vec g = - g_0 \vec e_y$, with $g_0>0$, 
or $g_y=-g_0$).

The stream function formulation of the incompressible isoviscous Stokes equation is then
\[
\eta_0 \nabla^4 \Psi
= -\frac{\partial \rho g_y}{\partial x} + \frac{\partial \rho g_x}{\partial y}   
= -\frac{\partial \rho g_y}{\partial x} 
=  g_0 \frac{\partial \rho}{\partial x} 
\]
since $g_x=0$. 

Assuming a linearised density field with regards to temperature $\rho(T)=\rho_0 (1-\alpha T)$
we have 
\[
\frac{\partial \rho}{\partial x} 
=
-\rho_0 \alpha \frac{\partial T}{\partial x} 
\]
and then 
\begin{equation}
\boxed{
\nabla^4 \Psi= -\frac{\rho_0 g_0 \alpha}{\eta_0} \frac{\partial T}{\partial x} 
%= -Ra \frac{\partial T}{\partial x} 
}
\end{equation}
For small perturbations of the conductive state $T_c(y)=(1-y/h)T_b$ 
we define the temperature perturbation $\tilde{T}(x,y)$ such that 
\[
T(x,y)=T_c(y)+\tilde{T}(x,y)
\]
Note that the temperature perturbation $\tilde{T}$ must satisfy the homogeneous boundary 
conditions $\tilde{T}(x,y=0)=0$ and $\tilde{T}(x,y=h)=0$.

The temperature equation is
\[
\rho C_p \left( \frac{\partial T}{\partial t} + {\vec \upnu}\cdot {\vec \nabla} T \right) 
=\rho C_p \left( \frac{\partial (T_c+\tilde{T})}{\partial t} + {\vec \upnu}\cdot {\vec \nabla} 
(T_c+\tilde{T}) \right) 
= k \Delta (T_c+\tilde{T})
\]
and can be simplified as follows:
\[
\rho C_p \left( \frac{\partial \tilde{T}}{\partial t} + {\vec \upnu}\cdot {\vec \nabla} T_c \right) 
= k \Delta \tilde{T}
\]
since a) $T_c$ does not depend on time, b) $\Delta T_c=0$, c) we assume the nonlinear 
term ${\vec \upnu}\cdot {\vec \nabla} \tilde{T} $ to be second order (temperature perturbations and 
coupled velocity changes are assumed to be small).

Using the relationship between velocity and stream function
$v=-\partial_x \Psi$
and since $\vec\nabla T_c = - T_b/h \vec{e}_y$ then
\[
{\vec \upnu}\cdot {\vec \nabla} T_c =  \frac{T_b}{h}   \frac{\partial \Psi}{\partial x} 
\]
Since $\kappa =k/\rho_0 C_p=1$ we get 
\begin{equation}
\boxed{
\frac{\partial \tilde{T}}{\partial t} - \kappa \Delta \tilde{T} 
= -  \frac{T_b}{h}   \frac{\partial \Psi}{\partial x}
}
\end{equation}
%We also have [{\color{red} prove}]
%\[
%{\bm \nabla}^4 \Psi = -Ra \frac{\partial T_1}{\partial %x}
%\]
Looking at these equations, we immediately think about a separation of variables approach to solve these
equations. Both equations showcase the Laplace operator $\Delta$, and the eigenfunctions of the biharmonic operator and the Laplace operator are the same. 
We then pose that $\Psi$ and $\tilde{T}$ can be written:
\begin{eqnarray}
\Psi(x,y,t) &=& \Psi_0 \exp(pt)\exp(\pm i k_x x) \exp(\pm i k_y y) \\ %= \Psi_0 E_\psi(x,y,t) \\
\tilde{T}(x,y,t) &=& \tilde{T}_0 \exp(pt) \exp(\pm i k_x x) \exp(\pm i k_y y) %=\tilde{T}_0 E_T(x,y,t) 
\end{eqnarray}
where $C_\Psi$ and $C_T$ are constants.
%where $k_x=2\pi/L_x$ and $k_y=2\pi/L_y$ are the horizontal and vertical wave number respectively.
Note that we then have
\[
\nabla^2 \Psi = -(k_x^2+k_y^2) \Psi
\quad\quad
\nabla^2 \tilde{T} = -(k_x^2+k_y^2) \tilde{T}
\]
The boundary conditions on $\tilde{T}$, 
coupled with a choice of a real function for the $x$ dependence 
yields\footnote{We assume here that temperature is a real quantity, not a complex one.}:
\[
\tilde{T}(x,y,t) = \tilde{T}_0 \exp(pt) \cos (k_x x) \sin (n\pi y).
\]
where $n$ is an integer number. 

The velocity vector is given by:
\begin{eqnarray}
%u &=& \frac{\partial \Psi}{\partial y} \\
%  &=& \frac{\partial }{\partial y} \left[ A_\Psi \exp(pt)\exp(\pm i k_x x) \exp(\pm i k_y y) \right]  \\
%  &=& A_\Psi \exp(pt)\exp(\pm i k_x x) (\pm i k_y) \exp(\pm i k_y y)   \\
v &=& -\frac{\partial \Psi}{\partial x}  \\
  &=& -\frac{\partial }{\partial x} \left[  \Psi_0 \exp(pt)\exp(\pm i k_x x) \exp(\pm i k_y y) \right] \\
  &=& - \Psi_0 \exp(pt) (\pm i k_x) \exp(\pm i k_x x) \exp(\pm i k_y y)
\end{eqnarray}
The velocity boundary conditions are $v(x,y=0)=0$ and $v(x,y=h)=0$ 
which imposes conditions on $\partial \Psi/\partial x$ and we find that we 
can use the same $y$ dependence as for $\tilde{T}$. 
Choosing again for a real function for the $x$ dependence yields:
\[
\Psi(x,y,t) = \Psi_0 \exp(pt) \sin(k_x x) \sin(n\pi z)
\]
We then have
\begin{eqnarray}
\Psi(x,y,t) &=& \Psi_0 \exp(pt)  \sin(k_x x) \sin(n\pi y)  \\
\tilde{T}(x,y,t)  &=& \tilde{T}_0 \exp(pt)  \cos(k_x x) \sin(n\pi y)   
\end{eqnarray}
In what follows we simplify notations: $k=k_x$. Then the two framed PDEs above become:


\begin{eqnarray}
&& \nabla^4 \Psi= -\frac{\rho_0 g_0 \alpha}{\eta_0} \frac{\partial \tilde{T}}{\partial x} \\
&\Rightarrow& 
(k^2 + n \pi^2)^2 \Psi_0 \exp(pt)  \sin(k x) \sin(n\pi y) = -\frac{\rho_0 g_0 \alpha}{\eta_0} 
 k \tilde{T}_0 \exp(pt)  \cos(k x) \sin(n\pi y)  \\ 
&\Rightarrow& 
(k^2 + n \pi^2)^2 \Psi_0     = \frac{\rho_0 g_0 \alpha}{\eta_0} 
 k \tilde{T}_0   
\end{eqnarray}


\begin{eqnarray}
&& \frac{\partial \tilde{T}}{\partial t} - \kappa \Delta \tilde{T} 
= -  \frac{T_b}{h}   \frac{\partial \Psi}{\partial x} \\
&\Rightarrow & p \tilde{T} - \kappa (k^2 + n^2\pi^2) \tilde{T}   
= -  \frac{T_b}{h} k \Psi_0 \exp(pt)  \cos(k x) \sin(n\pi y) \\
&\Rightarrow & p + \kappa (k^2 + n^2\pi^2) \tilde{T}   
= -  \frac{T_b}{h} k \Psi_0  \\
\end{eqnarray}

 
These equations must then be verified for all $\Psi_0$ and $\tilde{T}_0$, 
which leads to write:
\[
\left(
\begin{array}{cc}
p + (k^2+n^2\pi^2) & -k \\
-Ra \; k & (k^2+n^2\pi^2)^2 
\end{array}
\right)
\left(
\begin{array}{c}
\tilde{T}_0 \\ \Psi_0
\end{array}
\right)
=
\left(
\begin{array}{c}
0 \\ 0
\end{array}
\right)
\]
The determinant of such system must be nul otherwise there is only a trivial solution to the problem, i.e. $A_\theta=0$ and $A_\Psi=0$ which is not helpful. CHECK/REPHRASE
\[
D= [p + (k^2+n^2\pi^2)](k^2+n^2\pi^2)^2 - Ra \; k^2 =0
\]
or, 
\[
p = \frac{Ra \; k^2 -(k^2+n^2\pi^2)^3 }{ (k^2+n^2\pi^2)^2}
\]

The coefficient $p$ determines the stability of the system: if it is negative, 
the system is stable and both $\Psi$ and $\tilde{T}$ will decay to zero (return to conductive state). 
If $p=0$, then the system is meta-stable, and if $p>0$ then the system is unstable and 
the perturbations will grow. 

The threshold is then $p=0$ and the corresponding critical Rayleigh number $\text{Ra}_c$ is:
\[
Ra_c=\frac{(k^2+n^2\pi^2)^3}{k^2}
\]

The minimum critical Rayleigh number is given by 
\[
\left. \frac{\partial Ra_c}{\partial k}\right|_{n=1}=0
\]
or, 
\[
Ra_c = \frac{27}{4}\pi^4 \simeq 657.4839
\]



