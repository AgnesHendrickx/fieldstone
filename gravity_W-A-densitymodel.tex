
In the previous section density models were based on 
assumptions about the parameterization of the bulkmodulus $K$.
The density model of Williamson and Adams (1923), (Hemley, 2006)
does not depend on a parameterized $K$. 
Instead it is defined in terms of the seismic wave velocities
$v_p$ and $v_s$ that can be determined from inversion of seismological
traveltime data as $K/\rho = v_p^2 - 4/3 v_s^2$.

The W-A model can be derived from thermodynamic principles
for a homogeneous self-compressing
layer which is in an adiabatic state.
The bulkmodulus applied in this model is expressed in the seismic
wave velocities which in turn depend on the elasticity parameters
and the density.
The elastic deformation proces in seismic wave propagation occurs
on a relatively short time scale (seconds-minutes) compared to the 
characteristic time scale of conductive heat transport in solids
(see \ref{section_energy_budget}).
Therefore (diffusive) heat exchange can be neglected and adiabatic
conditions apply in seismic wave propagation.  
This implies that the elasticity parameters determined from 
seismic data, including the bulkmodulus $K$ pertain to adiabatic
conditions (see also Appendix \ref{Appnd_adiabatic_temperature_profile}).

Other processes such as convective mantle flow that occur on
a much longer time scale may take place under more general
(non-adiabatic) conditions.

In section \ref{section_thermal-state} on the thermal state of the Earth 
it is shown that
adiabatic conditions hold for the interior of a fluid layer when heat 
transport is dominated by advection and heat diffusion by
conduction/radiation plays a minor role.
Assuming the Earth's mantle to be in a state of vigorous thermal
convection it also follows that the average temperature profile, 
the geotherm, corresponds to an adiabatic distribution.

In general the density differential can be written as,
\begin{equation}
d\rho = \left ( \frac{\partial \rho}{\partial P} \right )_S dP +
        \left ( \frac{\partial \rho}{\partial S} \right )_P dS
\label{eqn_density_differential}
\end{equation}
where the differential of the entropy $S$ is dropped in case of
adiabatic conditions and the pressure derivative is written in terms of the
adiabatic bulkmodulus $K_S$ defined in (\ref{def_incompres}),
$1/K_s =  \left ( \partial \rho/\partial P \right )_S / \rho$.


\fbox{
\begin{minipage}{0.9\textwidth}
\begin{problem}
 {\small \it
  Derive the Williamson-Adams equation for a homogeneous adiabatic layer
  from the density differential (\ref{eqn_density_differential}) and
  assumption of isentropic (adiabatic) conditions with $dS\equiv 0$,
  \begin{equation}
        \frac{d\rho}{dr} = - \frac{\rho^2 g}{K_S}
  \label{AW_density}
  \end{equation}
 }
\end{problem}
\end{minipage}
}

\vspace{0.5cm}

~\\
The density solution of the W-A equation can be expressed in terms
of the seismic parameter $\Phi = K_S/\rho$ which in turn can be obtained
from seismic velocity models:
$\Phi = v_p^2 - \frac{4}{3} v_s^2$ for $P$ and $S$ waves.
$\sqrt{\Phi}= \sqrt{K_S/\rho}$ is known as the bulkvelocity.
For a given bulkvelocity profile, 
obtained from seismic observations,
the W-A density profile is derived from
(\ref{AW_density}) as,
\begin{equation}
     \ln \left (  \frac{\rho(r)}{\rho(R)} \right ) = 
        \int_r^R \Phi^{-1}(r^{'}) g(r^{'}) dr^{'}
\label{W-A_integral}
\end{equation}

\fbox{
\begin{minipage}{0.9\textwidth}
\begin{problem}
 {\small \it
  Derive (\ref{W-A_integral}) by integration of the W-A equation
  (\ref{AW_density}).
 }
\end{problem}
\end{minipage}
}

\vspace{0.5cm}

In (\ref{W-A_integral}) the gravity acceleration $g$ depends on
the density distribution $\rho(r)$ in the lefthand side.
Therefore the density profile can not be simply obtained from 
a seismologically determined $\Phi(r)$ profile and a single 
evaluation of the integral in (\ref{W-A_integral}).
The expression represents an integral equation that can be solved
iteratively as specified in problem \ref{AW-iteration}.

\fbox{
\begin{minipage}{0.9\textwidth}
\begin{problem}
 {\small \it
  Assume that a seismic parameter profile for the mantle
  $\Phi(r)$, obtained from seismic travel times, is available.
  Investigate how (\ref{W-A_integral}) can be used to compute a 
  sequence of mantle density profiles $\rho^{(j)}(r), j=1,2,\ldots$ 
  in an iterative procedure, by succesive substitution.
  How would you define a starting profile $\rho^{(1)}(r)$ for this
  iterative procedure?
  \newline
  Hint:
  Substitute the density profile for iteration number $j$ 
  in the gravity acceleration in the righthand
  side of (\ref{W-A_integral}) for the computation of an updated
  profile $j+1$.
  This is an example of a general solution strategy for non-linear
  problems known as `succesive substitution' or Picard iteration. 
\label{AW-iteration}
 }
\end{problem}
\end{minipage}
}

\vspace{0.5cm}

~\\
Williamson and Adams (1923) \cite{wiad23} used the iterative scheme 
in problem \ref{AW-iteration}
to test the hypothesis
that the mass concentration towards the Earth's centre is 
completely explained by 
compression of a homogeneous self-gravitating sphere.
They showed that integrating (\ref{W-A_integral}) from a surface value
of $3.3\cdot 10^3~\mathrm{kg/m^3}$ results in unrealistically 
high density values for 
depths greater than the core-mantle boundary.
This way they concluded that an inhomogeneous earth with a dense,
compositionally distinct core, probably iron-nickle, was required by
the observations.
The necessary multiple integrals in the evaluation of 
(\ref{W-A_integral})
had to be computed by means of graphical approximation methods in 1923,
several decades before the advent of electronic computers.

In a later analysis Bullen (1936) showed that the assumption of 
a homogeneous selfcompressing mantle described by the 
W-A equation,
and a chemically distinct dense core,
leads to unrealistically high values of the moment of 
inertia for the core
$I_c = f M_c R_c^2$, with a prefactor value $f\sim 0.57$ greater
than the value of a core with uniform density, 0.4.    
Since this would imply a density decrease towards the centre Bullen
concluded that the applicability of the W-A model for the whole mantle 
can not be maintained and that instead a distinct mantle transition layer, 
labeled C-layer,
must be included between the upper and lower mantle proper,
related to transitions in mineral phase and/or composition
(Bullen, 1975).

\fbox{
\begin{minipage}{0.9\textwidth}
\begin{problem}
\label{problem-WA-temperature}
~
\newline
{\small \it
\begin{enumerate}
\item
  Derive the following equation for the temperature distribution of
  a W-A layer (see Appendix \ref{Appnd_adiabatic_temperature_profile}),
  \begin{equation}
     \frac{dT}{dr} = - \frac{\alpha g}{c_P} T
  \label{ode-adiabat}
  \end{equation}
  where $\alpha$ and $c_P$ are the thermal expansion coefficient and
  the specific heat at constant pressure.
  \newline
  {\it Hint:}
  Use the differential for the entropy,
  \begin{equation}
     dS = \left ( \frac{\partial S}{\partial T} \right )_P dT +
          \left ( \frac{\partial S}{\partial P} \right )_T dP
  \end{equation}
  and the thermodynamic relations:
  $\left ( \partial S/ \partial T \right )_P = c_P / T$
  and
  $\left ( \partial S/ \partial P \right )_T = -\alpha / \rho$. 
  
  ~\\
\item
  Derive the expression for the temperature profile for an adiabatic
  layer, sometimes referred to as the `adiabat', by solving equation 
  (\ref{ode-adiabat}), 
  \begin{equation}
     T(r) = T(R) \exp \left (  
                              \int_r^R \frac{\alpha g}{c_P} ~ dr^{'}
                     \right )
  \label{general_adiabat}
  \end{equation}
  The temperature extrapolated to the surface, $T_P = T(R)$ is known as
  the potential temperature of the layer. 
  The quantity $H_T = ( \alpha g / c_P)^{-1} $ 
  is known as the thermal scale height of the layer.
  
\item
  Derive an expression from (\ref{general_adiabat})
  for the special case with a constant value of the scale height
  parameter.
\end{enumerate}
} 
\end{problem}
\end{minipage}
}

\vspace{0.5cm}

The W-A equation for the density of an adiabatic layer can be
generalized introducing the Bullen parameter $\eta$ which is 
used as a measure of the departure of the actual density/temperature
profile from an adiabat. This is done by writing,
\begin{equation}
 \eta(r) = - \frac{\Phi}{\rho g} \frac{d \rho}{dr}
\end{equation}
where $\eta(r)$ has been substituted for the constant value 
$(\equiv 1)$
in the W-A equation.

\subsubsection{Current density models}
The concept of an adiabatic layer was essential when no independent
determinations for the density distribution were available and the
W-A equation was used to compute $\rho(r)$ for given values of the
seismic parameter $\Phi(r)$ determined from seismological
observations (Bullen, 1975).

During the 1970s
a radial density distribution has been obtained for the Earth from
inversion of seismological observations,
incorporating spectral analysis of the Earth's eigenvibrations,
under the constraints of the given values for $M$ and $I$. 
This, together with seismic velocities determined from bodywave
traveltimes and surfacewave dispersion, 
has resulted in the Preliminary Reference Earth Model
(PREM), (Dziewonski and Anderson, 1981 \cite{dzan81}).

Since $\rho(r)$ can be determined from analysis of the earth's 
normal modes (radial eigenvibrations) the `adiabaticity' 
of the mantle is no longer assumed.

The degree of `adiabaticity' is used in numerical modelling experiments
as a diagnostic for the dynamic state - where a high degree of 
adiabaticity indicates vigorous thermal convection and predominantly
convective heat transport
(van den Berg and Yuen, 1998, \cite{vayu98}
Matyska and Yuen, 2000, \cite{mayu00}
Bunge \etal, 2001).

Usually the outcome of such experiments shows that the upper and lower
mantle separately are approximately adiabatic - away from boundary
layers were conductive transport dominates.
In recent years models of the deep lower mantle have become
popular were a compositionally distinct dense layer occupies the bottom
30\% (roughly) of the lower mantle 
(Kellog \etal (1999) \cite{kehv99}, Albarede and van der Hilst (2002) \cite{alva02}).


