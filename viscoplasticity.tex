
\Literature: \cite{mumc03,chpe15,momu06,muso11}





















IMPLEMENTATION of plasticity ... WORK IN PROGRESS

%%%%%%%%%%%%%%%%%%%%%%%%%%%%%%%%%%%%%%%%%%%%%%%%%%%%%%%%%%%%%%%%%%%%%%%%%%%%%%%%%%%%%%%%%%%%%%%%%%%%
\subsubsection{Scalar viscoplasticity}

This formulation is quite easy to implement. It is widely used, e.g. \cite{will92,thfb08,spmw16}, and relies on the assumption that 
a scalar quantity $\eta_p$ (the 'effective plastic viscosity') exists such that the deviatoric stress tensor 
\begin{equation}
{\bm \tau}=2\eta_p \dot{\bm\varepsilon} \label{eqscpl1}
\end{equation}
is bounded by some yield stress value $Y$.
From Eq. (\ref{eqscpl1}) it follows that ${\tau}_{e}= 2\eta_p \dot{\varepsilon}_{e}=Y$ which yields
\begin{mdframed}[backgroundcolor=blue!5]
\[
\eta_p = \frac{Y}{2 \dot{\varepsilon}_{e}}
\]
\end{mdframed}
This approach has also been coined the Viscosity Rescaling Method (VRM) \cite{kacha04}. 
\index{general}{VRM} \index{general}{Viscosity Rescaling Method}

\improvement[inline]{insert here the rederivation 2.1.1 of spmw16}

It is at this stage important to realise that (i) in areas where the strainrate is low, the resulting effective viscosity will be large, and 
(ii) in areas where the strainrate is high, the resulting effective viscosity will be low. This is not without consequences since 
(effective) viscosity contrasts up to 8-10 orders of magnitude have been observed/obtained with this formulation and it makes the FE 
matrix very stiff, leading to (iterative) solver convergence issues.
In order to contain these viscosity contrasts one usually resorts to viscosity limiters $\eta_{min}$ and $\eta_{max}$ such that 
\[
\eta_{min} \leq \eta_p \leq \eta_{max}
\]
Caution must be taken when choosing both values as they may influence the final results.


%-------------------------------------------------
{Work in progress}

\Literature \cite{zico74,zigo74,zico74b,zien75,corm75,zigo75,zihl75,zijo78,vidm82,vidm84,vede84,zivt85,vimd86}
\cite{wasd97,debo88,debo01,hesd02,bewv11,mumg10,leor89,sccm13,desm93,demu92,debo91,shmv16}
\cite{modm01}\cite{baji02}\cite{slde92}
\cite{modm02}\cite{vavd99}\cite{miam13}

Note that \cite{vidm82,vidm84,vimd86,zivt85} use the following formulation which they attribute to \cite{zijo78}:
\[
\eta_{eff} = \frac{c + (\dot{\varepsilon}_e / \gamma)^{1/n}}{ \dot{\varepsilon}_e }
\] 
For a perfectly plastic flow law, $\gamma \rightarrow \infty$ and then 
\[
\eta_{eff} = \frac{c}{ \dot{\varepsilon}_e }
\] 
and when when $c=0$ then the effective viscosity is essentially of the power law type.
Also, when $n=1$ the formulation becomes identical to the v-vp formulation (when the max viscosity is infinite) and with $1/\gamma=\eta_{min}$.

fractal distribution of shear bands in \cite{pohp94}



