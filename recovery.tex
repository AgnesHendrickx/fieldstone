
write about recovering accurate strain rate components and heat flux components on the nodes.

Let $\vec g(\vec r)$  be the desired nodal 
field which we want to be the continuous $Q_1$ representation of the field $\vec \nabla f^h$.
Since the derivative of the shape function does not exist on the nodes we need to design
an algorithm do do so. This problem is well known and has been 
investigated \cite{XX.XXX}\improvement{refs!}.
The main standard techniques are listed hereafter.


%..............................
\subsubsection{Global recovery}

The global recovery approach is rather simple: we wish to find $\vec g^h$
such that it satisfies
\[
\int_\Omega \phi \vec g^h \; d\Omega  = \int_\Omega \phi \vec\nabla f^h \; d\Omega 
\quad\quad \forall \phi
\] 
We will then successively replace $\phi$ by all the shape functions $N_i$ 
and since we have $g^h=\sum_j N_i g_i$ we then obtain
\[
\sum_j \int N_i N_j d\Omega g_i = \int N_i  \vec\nabla f^h \; d\Omega 
\]
or, 
\[
\mathbb{M} \cdot \vec{\cal G} = \vec f
\]



%..................................................
\subsubsection{Local recovery - centroid average over patch}





%..................................................
\subsubsection{Local recovery - nodal average over patch}

Let $j$ be the node at which we want to compute $\vec g$.
Then 
\[
\vec g_j = \vec g(\vec r_j) = 
\frac{\sum\limits_{ e \text{ adj. to }j} |\Omega_e| (\vec\nabla f)_e(\vec r_j) }{\sum |\Omega_e|}
\]
where $|\Omega_e|$ is the volume of the element and $(\vec\nabla f^h)_e(\vec r_j)$
is the gradient of $f$ as obtained with the shape functions inside element $e$ and 
computed at location $\vec r_j$.

%........................................................
\subsubsection{Local recovery - least squares over patch}



%........................................................
\subsubsection{Link to pressure smoothing}

When the penalty method is used to solve the Stokes equation, the pressure
is then given by $p=-\lambda \vec\nabla \cdot \vec v$. As explained in 
section \ref{sec_penalty}, the velocity is first obtained and the pressure 
is recovered by using this equation as a postprocessing step. Since the divergence 
cannot be computed easily at the nodes, the pressure is traditionally computed 
in the middle of the elements, yielding an elemental pressure field (remember, 
we are talking about $Q_1P_0$ elements here -- bi/tri-linear velocity, discontinuous
constant pressure)



\improvement{tie to fieldstone 12}

