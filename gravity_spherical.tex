
Starting from the Poisson equation, 
\[
\Delta U = 4 \pi {\cal G} \rho
\]
and using Gauss' theorem (noting that $\Delta U=\vec\nabla\cdot \vec\nabla U$):
\[
\int_V \Delta U dV = \int_V \vec\nabla\cdot \vec\nabla U dV = 
\int_\Gamma \vec\nabla U \cdot \vec{n} \; dS
= \int_V 4 \pi {\cal G} \rho \; dV
= 4 \pi {\cal G} \int_V \rho \; dV
\]
where $\vec{n}$ is the outward pointing normal vector.

A uniform sphere of mass $M$ and radius $a$ (and therefore density $\rho=M/(4\pi a^3/3)$) 
has the potential
\[
U(r) =
\left\{
\begin{array}{cc}
-2\pi {\cal G} \rho (a^2-r^2/3) & r \le a \\
-{\cal G} M/r & r\ge a
\end{array}
\right.
\]
VERIFY!

Outside the sphere the potential is Keplerian, while inside it has the form of a parabola; 
both the potential and its derivative are continuous at the surface of the sphere.

A sphere with density profile 
\[
\rho(r) = \rho_0 (r/r_0)^{-2}
\]
has the potential 
\[
U(r) = 4\pi {\cal G} \rho_0 r_0^2 \ln (r/r_0)
\]
VERIFY!

As it turns out, pairs of functions related by Poisson's equation provide 
convenient building-blocks for galaxy models. 
Three such functions often used in the literature are listed here; all describe models characterized
by a total mass M and a length scale a:

\begin{itemize}
\item Plummer (1905) \cite{dejo87} 
\[
\rho(r)= \frac{3M}{4\pi a^3 }\left(1+\frac{r^2}{a^2} \right)^{-5/2}
\qquad
U(r)=-\frac{{\cal G}M}{\sqrt{r^2+a^2}}
\]
\item Hernquist (1990)
\[
\rho(r)=\frac{M}{2\pi} \frac{a}{r(r+a)^3} 
\qquad
U(r)=-\frac{{\cal G}M}{r+a}
\]
\item Jaffe (1983)
\[
\rho(r)=\frac{M}{4\pi} \frac{a}{r^2(r+a)^2} 
\qquad
U(r)=\frac{{\cal G}M}{a} \ln \frac{a}{r+a}
\]
\end{itemize}
VERIFY!

