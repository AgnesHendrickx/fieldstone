\begin{flushright} {\tiny {\color{gray} lbb.tex}} \end{flushright}
%~~~~~~~~~~~~~~~~~~~~~~~~~~~~~~~~~~~~~~~~~~~~~~~~~~~~~~~~~~~~~~~~~~~~~~~~~~~~~~~~~~~~~~~~~~~~~~~~~~

WARNING: I am not comfortable writing about this topic. What follows is a rough attempt at making sense of it.

The Lady{\v z}henskaya-Babu{\v s}ka-Brezzi (LBB\footnote{
\url{https://en.wikipedia.org/wiki/Ladyzhenskaya-Babuska-Brezzi_condition}}) condition is a sufficient 
condition for a saddle point problem to have a unique solution.
For saddle point problems coming from the Stokes equations, 
many discretizations are unstable, giving rise to artifacts such as spurious oscillations. 
The LBB condition gives criteria for when a discretization of a saddle point problem is stable. 
It also assures convergence at the optimal rate. 

Bochev \& Gunzburger \cite{bogu09} state: "
The terminology “LBB” originates from the facts that this condition was first explicitly discussed
in the finite element setting for saddle point problems by Brezzi\footnote{
\url{https://en.wikipedia.org/wiki/Franco_Brezzi}} \cite{brez74} and that it is a special case of
the general weak-coercivity condition first discussed for finite element methods by Ivo Babu{\v s}ka\footnote{
\url{https://en.wikipedia.org/wiki/Ivo_Babuska}}
\cite{babu71} and that, in the continuous setting of the Stokes equation, this condition was first proved to
hold by Olga Ladyzhenskaya\footnote{\url{https://en.wikipedia.org/wiki/Olga_Ladyzhenskaya}}; see \cite{lady69}."

Unfortunately, to quote Donea \& Huerta \cite{dohu03}: 
"In the finite element context, it is by no means easy to prove whether or not a given
velocity-pressure pair satisfies the LBB compatibility condition."
Elman \etal state: "[...] Choosing spaces for which the discrete inf-sup condition holds
and is a delicate matter, and seemingly natural choices of velocity and pressure approximation
do not work. [...] In general, care must be taken to make the velocity space 
rich enough compared to the pressure space."

The LBB condition, or inf-sup condition can be proven in different ways, 
and standard techniques have been designed
as listed in Boffi \etal (2008) \cite{bobf08}.

%p129
Elman \etal \cite{elsw} state that "The inf-sup condition is a sufficient condition 
for the pressure to be unique up
to  constant in the case of an enclosed flow." This can also be proven for other boundary conditions.
This approach, based on the macro-element technique \cite{sten90} is explored in Appendix \ref{app:Gel}.


It can be shown that, provided the kernel (null space) of matrix $\G$ is zero,
the Stokes matrix is non-singular, that is $\vec{\cal V}$ and $\vec{\cal P}$ 
are uniquely defined, and the Schur complement matrix $\SSS$ is positive definite. 
Simply put, taking $\vec{\cal V}=\vec{0}$ in the discretised Stokes system 
without body forces yields $\G \cdot \vec{\cal P}=\vec{0}$ and implies
that any pressure solution is only unique up to the null space of the matrix $\G$.

We know that the Schur complement matrix $\SSS$ is positive definite if and only if all of its eigenvalues are positive.
One could then (numerically) compute the eigenvalues of $\SSS$ and check that these are indeed strictly positive
to show that $\SSS$ is positive definite but that would prove very costly. 

Another way is to see that $\SSS$ is positive definite only if $\text{ker}(\G)=\{0\}$.
Again to quote Donea \& Huerta \cite{dohu03}: "If this is the case, the partitioned Stokes matrix  
is non-singular and delivers uniquely defined velocity and pressure fields. If this is not the case, a
stable and convergent velocity field might be obtained, but the pressure field is likely
to present spurious and oscillatory results." 
Note that in the case of the $Q_1 \times P_0$ element it has been shown that the multiple families of 
checkboard pressure modes actually lie in the kernel of $\G$. \cite{sagl81a,sagl81b}


