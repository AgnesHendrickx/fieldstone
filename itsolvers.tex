
\url{https://en.wikipedia.org/wiki/Iterative_method}

%........................................
\subsubsection{Stationary iterative methods}

Basic examples of stationary iterative methods use a splitting of the matrix ${\bm A}$ such as
\[
{\bm A}={\bm D}+{\bm L}+{\bm U}
\]
where D is only the diagonal part of A, L is the strict lower triangular part of A and
U is the strict upper triangular part of A.

For instance:
\[
{\bm A}=
\left(
\begin{array}{ccc}
1 & 5 & 8 \\
6 & 4 & 2 \\
-1 & 7 & 5
\end{array}
\right)
\qquad
\Rightarrow
\qquad
{\bm D}=
\left(
\begin{array}{ccc}
1 & 0 & 0 \\
0 & 4 & 0 \\
0 & 0 & 5
\end{array}
\right)
\quad
{\bm L}=
\left(
\begin{array}{ccc}
0 & 0 & 0 \\
6 & 0 & 0 \\
-1 & 7 & 0
\end{array}
\right)
\quad
{\bm U}=
\left(
\begin{array}{ccc}
0 & 5 & 8 \\
0 & 0 & 2 \\
0 & 0 & 0
\end{array}
\right)
\]


CHANGE with M and N
The iterative method is defined by:
\begin{equation}
{\bm D} \cdot \vec{T}^{{\color{Fuchsia}k+1}} = -({\bm L} + {\bm U}) \cdot \vec{T}^{\color{Fuchsia}k} 
+ \vec{b}
\qquad k=0,1,\dots
\end{equation}
where $\vec{T}^{\color{Fuchsia}0}$ is the initial guess (often taken to be zero).
Note that the superscript denotes the iteration number and has nothing to 
do with the time step.


\begin{itemize}
\item Richardson method: 
\item Jacobi method: 
\item Damped Jacobi method: 
\item Gauss–Seidel method: 
\item Successive over-relaxation method (SOR):
\item Symmetric successive over-relaxation (SSOR):
\end{itemize}

\index{general}{BiCG solver}
\index{general}{Jacobi solver}
\index{general}{Gauss-Seidel solver}
\index{general}{GMRES solver}
\index{general}{CG solver}
\index{general}{SSOR solver}


%........................................
\subsubsection{Krylov subspace methods}

\begin{itemize}
\item Conjugate Gradient
\footnote{\url{https://en.wikipedia.org/wiki/Conjugate_gradient_method}} 
\item Biconjugate Gradient method
\footnote{\url{https://en.wikipedia.org/wiki/Biconjugate_gradient_method}}
\item Biconjugate Gradient stabilised method
\footnote{\url{https://en.wikipedia.org/wiki/Biconjugate_gradient_stabilized_method}}
\item MINRES
\item Generalized minimal residual method (GMRES)
\footnote{\url{https://en.wikipedia.org/wiki/Generalized_minimal_residual_method}}
\end{itemize}





