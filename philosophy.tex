\begin{flushright} {\tiny {\color{gray} philosophy.tex}} \end{flushright}

This document was written with my students in mind, i.e. 3rd and 4th year 
Geology/Geophysics students at Utrecht University. 
I have chosen to use as little jargon as possible unless it is a term that is 
commonly found in the geodynamics literature (methods paper as well as 
application papers). There is no mathematical proof of any theorem that may 
be mentioned but I will try to refer to the appropriate sources, i.e.
generic Numerical Analysic, Finite Element and 
Linear Algebra books. If you find that this books lacks references
to Sobolev spaces, Hilbert spaces, and other spaces, this book is just not for you.  

The codes I provide here are by no means optimised as I have chosen code readability 
over code efficiency. I have also chosen to avoid resorting to multiple code 
files or even functions in order to favour a sequential reading of the codes. 
These codes are not designed to form the basis of a real life application:
Existing open source highly optimised codes shoud be preferred, such as 
\aspect{} \cite{krhb12,hedg17}, \citcoms \cite{zhzm00,zhmt08}, LAMEM \cite{kapb16}, 
PTATIN \cite{mabl14,mabl15}, PYLITH \cite{aakw13}, ... (see Appendix~\ref{app:codes}).

Concerning figures I have consciously decided not to place them inside {\sl figure} \LaTeX 
environments since it does not allow for complete control over where they end up. 
Instead they are inserted when they are needed in the text. 

All kinds of feedback is welcome on the text (grammar, typos, ...), on the text, the equations
or on the code(s). You will have my eternal gratitude if you wish to contribute an 
example, a benchmark, a cookbook. 

All the python scripts and tex files are freely available at 
\begin{center}
\url{https://github.com/cedrict/fieldstone}
\end{center}
This document is available at:
\begin{center}
\url{http://cedricthieulot.net/manual.pdf}  
\end{center}

{\sl Disclaimer}: there are many things in this huge document I probably do not fully understand, 
or that I am simply wrong about. I sometimes write open questions in the text about such 
things. My commitment is to revisit this document time and time again, until it is 99\% correct.
This is not a book, it has not been edited by anybody. It is not perfect in any way. 
I nevertheless hope it will be useful to many in the long run. 


