
In what follows $\vec\upnu$ now stands for the displacement vector, i.e. 
with units of length, not velocity. 
As before, the displacement inside an element is given by 
\begin{equation}
{\vec \upnu}^h({\vec r})=\sum_{i=1}^{m_v} N_i({\vec r})\;  {\vec \upnu}_i
\label{mixed01}
\end{equation}
where $N_i$ are the polynomial basis functions for the displacement.
Pressure does not appear in the equations so this is not a case of 
mixed FE as for the viscous Stokes flow. 

Other notations are sometimes used for Eqs.(\ref{mixed01}) and (\ref{mixed02}):
\begin{equation}
u^h({\vec r}) = \vec{N} \cdot \vec{u}
\quad\quad\quad\quad
v^h({\vec r}) = \vec{N} \cdot \vec{v}
\quad\quad\quad\quad
w^h({\vec r}) = \vec{N} \cdot \vec{w}
\end{equation} 
where ${\vec \upnu}=(u,v,w)$ and $\vec{N}$ 
is the vector containing all basis functions evaluated at location ${\vec r}$:
\begin{eqnarray}
\vec{N}^v &=& \left( N_1({\vec r}),  N_2({\vec r}),  N_3({\vec r}), \dots  N_{m_v}({\vec r}) \right) \\
\vec{N}^p &=& \left( N_1^p({\vec r}),  N_2^p({\vec r}),  N_3^p({\vec r}), \dots  N_{m_p}^p({\vec r}) \right)
\end{eqnarray}
and with 
\begin{eqnarray}
\vec{u} &=& \left( u_1,  u_2,  u_3, \dots  u_{m_v} \right) \\
\vec{v} &=& \left( v_1,  v_2,  v_3, \dots  v_{m_v} \right) \\
\vec{w} &=& \left( w_1,  w_2,  w_3, \dots  w_{m_v} \right) \\
\end{eqnarray}

%............................................
\paragraph{In three dimensions} We start from
\[
{\bm \sigma} = \lambda (\vec\nabla\cdot \vec\upnu) {\bm 1}+ 2\mu {\bm \varepsilon}
\]
where $\mu$ is the shear modulus and $\lambda$ the Lam{\'e} parameter.

\begin{eqnarray}
\sigma_{xx} &=& (\lambda+2\mu)  \varepsilon_{xx} + \lambda \varepsilon_{yy} + \lambda \varepsilon_{zz} \nn\\
\sigma_{yy} &=& \lambda \varepsilon_{xx} + (\lambda+2\mu)  {\varepsilon}_{yy} + \lambda \varepsilon_{zz}\nn\\
\sigma_{zz} &=& \lambda \varepsilon_{xx} + \lambda \varepsilon_{yy} + (\lambda+2\mu)  {\varepsilon}_{zz} \nn\\
\sigma_{xy} &=& 2\mu  {\varepsilon}_{xy} \nn\\
\sigma_{xz} &=& 2\mu  {\varepsilon}_{xz} \nn\\
\sigma_{yz} &=& 2\mu  {\varepsilon}_{yz} 
\end{eqnarray}
or, 
\[
\vec\sigma =
\left(
\begin{array}{c}
\sigma_{xx}\\ 
\sigma_{yy} \\
\sigma_{zz} \\
\sigma_{xy} \\
\sigma_{xz} \\
\sigma_{yz} 
\end{array}
\right)
=
\left(
\begin{array}{cccccc}
\lambda+2\mu & \lambda & \lambda & 0 & 0 & 0 \\
\lambda & \lambda+2\mu & \lambda & 0 & 0 & 0 \\
\lambda & \lambda & \lambda+2\mu & 0 & 0 & 0 \\
0 & 0 & 0 & \mu & 0 & 0\\
0 & 0 & 0 & 0 & \mu & 0\\
0 & 0 & 0 & 0 & 0 & \mu
\end{array}
\right)
\cdot
\left(
\begin{array}{c}
\varepsilon_{xx} \\
\varepsilon_{yy} \\
\varepsilon_{zz} \\
2\varepsilon_{xy} \\
2\varepsilon_{xz} \\
2\varepsilon_{yz} 
\end{array}
\right)
=\vec\varepsilon
\]
The rest of the procedure is pretty straightforward since it follows the same 
ideas as for the mixed viscous case, except that we here build the $\K$ matrix 
only as follows:
\[
\K=\int_{\Omega_e} {\bm B}^T \cdot {\bm D} \cdot {\bm B} \; dV 
\]




%............................................
\paragraph{In two dimensions} The above relationships simplify to 
\begin{eqnarray}
\sigma_{xx} &=& (\lambda+2\mu)  \varepsilon_{xx} + \lambda \varepsilon_{yy} \\
\sigma_{yy} &=& \lambda \varepsilon_{xx} + (\lambda+2\mu)  \dot{\varepsilon}_{yy} \\
\sigma_{xy} &=& 2\mu  \dot{\varepsilon}_{xy} 
\end{eqnarray}
so 

\[
\vec\sigma =
\left(
\begin{array}{c}
\sigma_{xx}\\ 
\sigma_{yy} \\
\sigma_{xy} 
\end{array}
\right)
=
\left(
\begin{array}{ccc}
\lambda+2\mu & \lambda & 0 \\ 
\lambda & \lambda+2\mu & 0 \\
0 & 0 & \mu 
\end{array}
\right)
\cdot
\left(
\begin{array}{c}
\varepsilon_{xx} \\
\varepsilon_{yy} \\
2\varepsilon_{xy} 
\end{array}
\right)
=\vec\varepsilon
\]







%%%%%%%%%%%%%%%%%%%%%%%%%%%%%%%%%%%%%%%%%%%%%%%%%%%%%%%%%%%%%%%%%%%%%%
\subsubsection{The axisymmetric case} \label{ss:fem_elast_axis}


We start from 
\begin{equation}
{\bm \sigma} = \lambda \vec\nabla\cdot\vec{u}\;  {\bm 1}
+2 \mu {\bm \varepsilon}(\vec{u})
\label{eq:elast_as}
\end{equation}
In cylindrical coordinates the velocity gradient is given by 
\[
\vec\nabla \vec{u}  =
\left(
\begin{array}{ccc}
{\partial \, u_r \over \partial \, r} &
{1 \over r} {\partial \, u_r \over \partial \, \theta} - {u_{\theta} \over r} &
{\partial \, u_r \over \partial z} \\
\\
{\partial \, u_{\theta} \over \partial \, r} &
{1 \over r} {\partial \, u_{\theta} \over \partial \, \theta} + {u_r \over r} &
{\partial \, u_{\theta} \over \partial z} \\
\\
{\partial \, u_{z} \over \partial \, r} &
{1 \over r} {\partial \, u_{z} \over \partial \, \theta} &
{\partial \, u_{z} \over \partial z}
\end{array}
\right)
\]
In the case of axisymmetry, and in this case symmetry about the $z$ axis, there is invariance with respect to the rotation around the axis so stresses and other quantities are independent of the $\theta$ coordinate, or simply put $\partial_\theta \rightarrow 0$.
The velocity gradient simplifies to:
\[
\vec\nabla \vec{u}  =
\left(
\begin{array}{ccc}
{\partial \, u_r \over \partial \, r} &
- {u_{\theta} \over r} &
{\partial \, u_r \over \partial z} \\
\\
{\partial \, u_{\theta} \over \partial \, r} &
{u_r \over r} &
{\partial \, u_{\theta} \over \partial z} \\
\\
{\partial \, u_{z} \over \partial \, r} &
0 &
{\partial \, u_{z} \over \partial z}
\end{array}
\right)
\]
Also, it follows logically that $u_\theta=0$ so that ultimately:
\[
\vec\nabla \vec{u}  =
\left(
\begin{array}{ccc}
\frac{\partial u_r}{\partial r} & 0 & {\partial  u_r \over \partial z} \\\\
0 & {u_r \over r} & 0 \\ \\
{\partial u_{z} \over \partial  r} & 0 & {\partial  u_{z} \over \partial z}
\end{array}
\right)
\]
and the strain tensor is then given by 
\begin{equation}
\label{eq:strain_as} 
{\bm \varepsilon}(\vec{u})
=\frac12\left(\vec\nabla \vec{u}+\vec\nabla \vec{u}^T\right)
=
\left(
\begin{array}{ccc}
{\partial \, u_r \over \partial \, r} &
0 &
\frac12({\partial u_{z} \over \partial r} + {\partial u_r \over \partial z}) \\ \\
0 & {u_r \over r} & 0 \\ \\
\frac12({\partial u_{z} \over \partial r} + {\partial u_r \over \partial z} ) & 0 & {\partial u_{z} \over \partial z} 
\end{array}
\right)
\end{equation}
The term $\vec\nabla \cdot \vec{u}$ is simply the trace of ${\bm \varepsilon}(\vec{u})$ so 
\[
\vec\nabla \cdot \vec{u}
= {\partial u_r \over \partial r} +{u_r \over r}
+{\partial u_{z} \over \partial z}
\]
Finally the full stress tensor is then 
\begin{eqnarray}
{\bm \sigma}
&=&
\left(
\begin{array}{ccc}
\lambda({\partial  u_r \over \partial  r}
+{u_r \over r} +{\partial  u_{z} \over \partial z}) +
2\mu {\partial  u_r \over \partial  r} &
0 & \mu({\partial u_{z} \over \partial  r} + {\partial u_r \over \partial z} ) \\
\\
0 & \lambda({\partial u_r \over \partial r}
+{u_r \over r} +{\partial u_{z} \over \partial z}) + 2\mu{u_r \over r} & 0 \\
\\
\mu({\partial u_{z} \over \partial r} + {\partial u_r \over \partial z} )&0 & \lambda({\partial u_r \over \partial r}
+{u_r \over r} +{\partial u_{z} \over \partial z}) +
2\mu{\partial  u_{z} \over \partial z}
\end{array}
\right) \nonumber\\ \nonumber\\
&=&
\left(
\begin{array}{ccc}
(\lambda+ 2\mu) {\partial u_r \over \partial r}
+\lambda ({u_r \over r} +{\partial  u_{z} \over \partial z})  &
0 &
\mu({\partial u_{z} \over \partial  r} + {\partial u_r \over \partial z} ) \\
\\
0 &
(\lambda+2\mu) \frac{u_r}{r}
+ \lambda({\partial  u_r \over \partial r}
+{\partial u_{z} \over \partial z}) &
0 \\
\\
\mu({\partial u_{z} \over \partial r} + {\partial u_r \over \partial z} ) &
0 &
(\lambda+2\mu) \frac{\partial u_z}{\partial z}
+\lambda({\partial u_r \over \partial r}
+{u_r \over r} ) 
\end{array}
\right) \nonumber
\end{eqnarray}

As we did in the 2D case, we rewrite the six independent stress terms in to a vector $\vec\sigma$ and we use Eq.~\eqref{eq:elast_as} to arrive at:
\[
\vec{\sigma}=
\left(
\begin{array}{c}
\sigma_{rr} \\
\sigma_{\theta\theta} \\
\sigma_{zz} \\
\sigma_{r\theta} \\
\sigma_{rz} \\
\sigma_{\theta z} 
\end{array}
\right)
=
\left(
\begin{array}{cccccc}
\lambda+2\mu & \lambda & \lambda & 0 & 0 & 0 \\
\lambda & \lambda+2\mu & \lambda & 0 & 0 & 0 \\
\lambda & \lambda & \lambda+2\mu & 0 & 0 & 0 \\
0 & 0 & 0 & \mu & 0 & 0\\
0 & 0 & 0 & 0 & \mu & 0\\
0 & 0 & 0 & 0 & 0 & \mu
\end{array}
\right)
\cdot
\left(
\begin{array}{c}
\varepsilon_{rr} \\
\varepsilon_{\theta\theta} \\
\varepsilon_{zz} \\
2\varepsilon_{r\theta} \\
2\varepsilon_{rz} \\
2\varepsilon_{\theta z} 
\end{array}
\right)
=\vec\varepsilon(\vec u)
\]
or $\vec\sigma = {\bm D} \cdot \vec\varepsilon(\vec u)$. Notice the similarity of matrix ${\bm D}$ with the one of Section~(XXX) in the 3D penalty formulation case.
The components of the $\vec\varepsilon$ vector are
\[
\vec\varepsilon(\vec u)
=
\left(
\begin{array}{c}
\varepsilon_{rr} \\
\varepsilon_{\theta\theta} \\
\varepsilon_{zz} \\
2\varepsilon_{r\theta} \\
2\varepsilon_{rz} \\
2\varepsilon_{\theta z} 
\end{array}
\right)
=
\left(
\begin{array}{c}
\frac{\partial u_r}{\partial r} \\ 
\frac{u_r}{r} \\ 
\frac{\partial u_z}{\partial z} \\ 
0 \\ 
\frac{\partial u_z}{\partial r}+\frac{\partial u_r}{\partial z} \\ 
0
\end{array}
\right)
\]
We see that there are two zeroes and consequently we'll find that
$\sigma_{r\theta}$ and $\sigma_{\theta z}$ are also
identically zero, so we discard these and end up with only four stress components :
\[
\vec{\sigma}=
\left(
\begin{array}{c}
\sigma_{rr} \\
\sigma_{\theta\theta} \\
\sigma_{zz} \\
\sigma_{rz} \\
\end{array}
\right)
=
\left(
\begin{array}{cccc}
\lambda+2\mu & \lambda & \lambda & 0  \\
\lambda & \lambda+2\mu & \lambda & 0  \\
\lambda & \lambda & \lambda+2\mu & 0  \\
0 & 0 & 0 & \mu 
\end{array}
\right)
\cdot
\left(
\begin{array}{c}
\varepsilon_{rr} \\
\varepsilon_{\theta\theta} \\
\varepsilon_{zz} \\
2\varepsilon_{rz} 
\end{array}
\right)
%=\vec\varepsilon(\vec u)
\]
Note that in the literature the above relationship is often written 
\[
\left(
\begin{array}{c}
\sigma_{rr} \\
\sigma_{\theta\theta} \\
\sigma_{zz} \\
\sigma_{rz} \\
\end{array}
\right)
=
\frac{E}{(1+\nu)(1-2\nu)}
\left(
\begin{array}{cccc}
1-\nu & \lambda & \nu & 0  \\
\nu & 1-\nu & \nu & 0  \\
\nu & \nu & 1-\nu & 0  \\
0 & 0 & 0 & (1-2\nu)/2
\end{array}
\right)
\cdot
\left(
\begin{array}{c}
\varepsilon_{rr} \\
\varepsilon_{\theta\theta} \\
\varepsilon_{zz} \\
2\varepsilon_{rz} 
\end{array}
\right)
\]
which is equivalent since $E=2\mu(1+\nu)$ and $\lambda=\frac{\nu E}{(1+\nu)(1-2\nu)}$ (see for instance Section~5.2.4 in \cite{zita1}).   

Only displacements in the $r$ and $z$ directions remain (note that $\varepsilon_{\theta\theta}$ is in fact equal to $u_r/r$). In what follows I rename $u=u_r$ and $u_z=w$ to simplify notations. 
Then, inside an element we have 
\begin{eqnarray}
u^h(r,z) &=& \sum_{i=1}^m N_i(r,z) u_i \nonumber\\
w^h(r,z) &=& \sum_{i=1}^m N_i(r,z) w_i
\end{eqnarray}
where $N_i$ are the basis functions attached 
to the $m$ nodes of the element.
We compute the elements of the ${\bm \varepsilon}$ tensor of Eq.~\eqref{eq:strain_as} as follows:
\begin{eqnarray}
\varepsilon_{rr} &=&
\frac{\partial u^h}{\partial r} 
= \sum_{i=1}^m \frac{\partial N_i}{\partial r}(r,z) \; u_i \\
\varepsilon_{\theta\theta} &=& \frac{u_r^h}{r} = 
\frac{1}{r}\sum_{i=1}^m N_i(r,z) \;  u_i \\
\varepsilon_{zz} &=& 
\frac{\partial w^h}{\partial z}
= \sum_{i=1}^m \frac{\partial N_i}{\partial z}(r,z) \; w_i \\
\varepsilon_{rz} &=& \frac12\frac{\partial u^h}{\partial z}
+\frac12 \frac{\partial w^h}{\partial r}
= \sum_{i=1}^m \frac{\partial N_i}{\partial z}(r,z) u_i 
+ \sum_{i=1}^m \frac{\partial N_i}{\partial r}(r,z) w_i 
\end{eqnarray}

\noindent Let us take $m=3$, i.e. linear triangles, for simplicity. Then 
the strain vector $\vec{\varepsilon}^h$ is given by
\[
\vec\varepsilon^h=
\left(
\begin{array}{c}
\frac{\partial u^h}{\partial r} \\ \\
\frac{u^h}{r} \\ \\
\frac{\partial w^h}{\partial z} \\ \\
\frac{\partial u^h}{\partial z} + \frac{\partial w^h}{\partial r} 
\end{array}
\right)
=
\underbrace{
\left(
\begin{array}{ccccccccc}
\frac{\partial N_1}{\partial r} &  0 &  
\frac{\partial N_2}{\partial r} &  0 &
\frac{\partial N_3}{\partial r} &  0 \\  \\
\frac{N_1}{r}  & 0 &  
\frac{N_2}{r}  & 0 &
\frac{N_3}{r}  & 0 \\  \\
 0 & \frac{\partial N_1}{\partial z}  &
 0 & \frac{\partial N_2}{\partial z}  &
 0 & \frac{\partial N_3}{\partial z}  \\ \\
\frac{\partial N_1}{\partial z} & \frac{\partial N_1}{\partial r}  &
\frac{\partial N_2}{\partial z} & \frac{\partial N_2}{\partial r}  &
\frac{\partial N_3}{\partial z} & \frac{\partial N_3}{\partial r}   
\end{array}
\right)
}_{\bm B (4\times 6) }
\cdot
\underbrace{
\left(
\begin{array}{c}
u1 \\  w1 \\ u2 \\  w2 \\ u3 \\ w3 
\end{array}
\right)
}_{\vec U (6\times1)}
\]
or $\vec\varepsilon^h= {\bm B} \cdot \vec{U}$
and finally 
\[
\underbrace{
\left(
\begin{array}{c}
\sigma_{rr} \\
\sigma_{\theta\theta} \\
\sigma_{zz} \\
\sigma_{rz} 
\end{array}
\right)
}_{\vec{\sigma}}
=
\underbrace{
\left(
\begin{array}{cccc}
\lambda+2\mu & \lambda & \lambda & 0  \\
\lambda & \lambda+2\mu & \lambda & 0  \\
\lambda & \lambda & \lambda+2\mu & 0  \\
0 & 0 & 0 & \mu 
\end{array}
\right)
}_{\bm D}
\!
\cdot
\!
\underbrace{
\left(
\begin{array}{ccccccccc}
\frac{\partial N_1}{\partial r} &  0 &  
\frac{\partial N_2}{\partial r} &  0 &
\frac{\partial N_3}{\partial r} &  0 \\  \\
\frac{N_1}{r}  & 0 &  
\frac{N_2}{r}  & 0 &
\frac{N_3}{r}  & 0 \\  \\
 0 & \frac{\partial N_1}{\partial z}  &
 0 & \frac{\partial N_2}{\partial z}  &
 0 & \frac{\partial N_3}{\partial z}  \\ \\
\frac{\partial N_1}{\partial z} & \frac{\partial N_1}{\partial r}  &
\frac{\partial N_2}{\partial z} & \frac{\partial N_2}{\partial r}  &
\frac{\partial N_3}{\partial z} & \frac{\partial N_3}{\partial r}   
\end{array}
\right)
}_{\bm B (4\times 6) }
\!
\cdot
\!
\underbrace{
\left(
\begin{array}{c}
u1 \\  w1 \\ u2 \\  w2 \\ u3 \\ w3 
\end{array}
\right)
}_{\vec U (6\times1)}
\]
or, 
\[
\boxed{
\vec\sigma = {\bm D} \cdot {\bm B} \cdot \vec{U}
}
\]
Note that in 2D, the matrix ${\bm D}$ is $3\times3$ and 
${\bm B}$ is $3\times 6$.

\todo[inline]{I do not know yet how to arrive at what follows}

\noindent The $6\times 6$ stiffness matrix is then 
\[
\K = \iiint {\bm B}^T \cdot {\bm D} \cdot {\bm B}\; dV
\]
with $dV= r dr d\theta dz$ in cylindrical coordinates. The integral 
over the $\theta$ coordinate yields a factor $2\pi$ so 
\[
\K = 2 \pi \iint {\bm B}^T \cdot {\bm D} \cdot {\bm B}\; {\color{red} r} drdz
\]
The integration can now be performed as simply as was the case in the plane stress problem.

\todo[inline]{write the derivation for the rhs}


Note that in practice the matrix ${\bm D}$ is computed as follows (see for example Stone~63):
\[
{\bm D}
=
\left(
\begin{array}{cccc}
\lambda+2\mu & \lambda & \lambda & 0  \\
\lambda & \lambda+2\mu & \lambda & 0  \\
\lambda & \lambda & \lambda+2\mu & 0  \\
0 & 0 & 0 & \mu 
\end{array}
\right)
=
\lambda
\left(
\begin{array}{cccc}
1 & 1 & 1 & 0  \\
1 & 1 & 1 & 0  \\
1 & 1 & 1 & 0  \\
0 & 0 & 0 & 0 
\end{array}
\right)
+
\mu
\left(
\begin{array}{cccc}
2 & 0 & 0 & 0 \\
0 & 2 & 0 & 0 \\
0 & 0 & 2 & 0 \\
0 & 0 & 0 & 1  
\end{array}
\right)
\]


The divergence of the stress tensor is given by
\begin{eqnarray}
\vec\nabla \cdot {\bm \sigma}
& = &
\left[ {1 \over r} {\partial \over \partial \, r} \left( r \, \sigma_{\!rr} \right) + 
{1 \over r} {\partial \, \sigma_{\!r\theta} \over \partial \, \theta} +
{\partial \, \sigma_{\!rz} \over \partial z} - {\sigma_{\theta \theta} \over r} \right] \vec{ e}_r \\
& + &
\left[ {1 \over r} {\partial \over \partial \, r} \left( r \, \sigma_{\!r\theta} \right) + 
{1 \over r} {\partial \, \sigma_{\!\theta\theta} \over \partial \, \theta} +
{\partial \, \sigma_{\!\theta z} \over \partial z} + {\sigma_{r \theta} \over r} \right] \vec{e}_\theta \\
& + &
\left[ {1 \over r} {\partial \over \partial \, r} \left( r \, \sigma_{\!rz} \right) + 
{1 \over r} {\partial \, \sigma_{\!\theta z} \over \partial \, \theta} +
{\partial \, \sigma_{\!zz} \over \partial z} \right] \vec{e}_z
\end{eqnarray}
Since $\sigma_{r\theta}=\sigma_{\theta r}=0$ 
and $\sigma_{z\theta}=\sigma_{\theta z}=0$
and since $\partial_\theta \rightarrow 0$
then 
\begin{eqnarray}
\vec\nabla \cdot {\bm \sigma}
& = &
\left[ {1 \over r} {\partial \over \partial \, r} \left( r \, \sigma_{\!rr} \right) + 
{\partial \, \sigma_{\!rz} \over \partial z} - {\sigma_{\theta \theta} \over r} \right] \vec{ e}_r \\
& + &
\left[ {1 \over r} {\partial \over \partial \, r} \left( r \, \sigma_{\!rz} \right) 
 +
{\partial \, \sigma_{\!zz} \over \partial z} \right] \vec{e}_z
\end{eqnarray}

Then 
\begin{eqnarray}
\vec\nabla \cdot {\bm \sigma}|_r
&=&  {1 \over r} {\partial \over \partial \, r} \left( r \, \sigma_{\!rr} \right) + 
{\partial \, \sigma_{\!rz} \over \partial z} - {\sigma_{\theta \theta} \over r} \\
&=& 
\frac{\partial \sigma_{rr}}{\partial r} + \frac1r (\sigma_{rr}-\sigma_{\theta \theta} ) + \frac{\partial \sigma_{rz}}{\partial z} \\
&=& 
\frac{\partial \sigma_{rr}}{\partial r} + \frac{2\mu}{r} 
({\partial \, u_r \over \partial \, r} - \frac{u_r}{r} ) 
+ \frac{\partial \sigma_{rz}}{\partial z} \\
\vec\nabla \cdot {\bm \sigma}|_z
&=& \frac{\partial \sigma_{rz}}{\partial r} + \frac{\sigma_{rz}}{r}
 + {\partial \, \sigma_{\!zz} \over \partial z} 
\end{eqnarray}





